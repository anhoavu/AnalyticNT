\unless\ifdefined\IsMainDocument
\documentclass[12pt]{article}
\usepackage{amsmath,amsthm,amssymb}
\newcommand{\Fhat}{\widehat{F}}
\begin{document}
\fi

\section{Abelian vs Tauberian}

Going from $f$ to Dirichlet series is an abelian process: Think of assigning the weight of $n^{-s}$ to $f(n)$. When $s = 0$, the weights are all 1 and we have $\sum f(n) = \lim_{x \rightarrow \infty} F(x) = c$. That is the simple example on page 141.

For the ``converse'', we show that if $F$ is monotone and $\Fhat(\sigma) \rightarrow c$ as $\sigma \rightarrow 0^+$ then $F(x) \rightarrow c$ as $x \rightarrow \infty$.

The intuition is clear: Just like the abelian process, we think of
$$\Fhat(\sigma) = \sum f(n) n^{-\sigma}$$
as the weighted sum of $f(n)$ with decaying weights. For a fixed $\sigma$, the weight $n^{-\sigma}$ is close to 1 (in the sense of being within $\epsilon$ error) for the first $n < N_\sigma$ terms. So the partial sum $\sum_{n < N_\sigma} f(n) n^{-\sigma}$ is close to $F(N_\sigma)$. As $\sigma$ decreases to 0, $N_\sigma$ increases to infinity and $F(N_\sigma)$ is close to $c$ i.e. bounded by $c \pm \epsilon$. Since $F$ is increasing, the limit of the subsequence $F(N_\sigma)$ is the same as the limit $F(x)$ as $x \rightarrow \infty$.

To make things precise:
\begin{itemize}
\item We can assume $F$ is increasing without loss of generality since $\widehat{c F} = c \Fhat$ for any constant $c$. Then $F_v = F$.

\item It suffices to prove that $\lim_{x \rightarrow \infty} F(x)$ exists for then we can use the forward (abelian) direction to conclude that the limit is $c$.

\item Since $F$ is increasing, this can be achieved by proving that $F(x)$ is bounded.

\item Fix an $\epsilon \in (0, 1)$ and let $\delta > 0$ be such that $|\Fhat(\sigma) - c| < \epsilon$ for all $\sigma \in (0, \delta)$ by the definition of limit ($\delta$ depends on $\epsilon$). For any such $\sigma$, let $X_\sigma > 0$ be such that $0 < 1 - x^{-\sigma} < \epsilon$ for all $x \in (1, X_\sigma)$. (Despite the notation, $X_\sigma$ depends on both $\epsilon$ and $\sigma$.) Then
\begin{align*}
c + \epsilon > \Fhat(\sigma) &= \int_{1^-}^{X_\sigma} x^{-\sigma} dF + \int_{X_\sigma}^\infty x^{-\sigma} dF\\
&\geq \int_{1^-}^{X_\sigma} (1 - \epsilon) dF + \underbrace{\int_{X_\sigma}^\infty x^{-\sigma} dF_v}_{\geq 0} &\text{as } F \uparrow\\
&\geq (1 - \epsilon) F(X_\sigma)
\end{align*}
so we have
$$F(X_\sigma) \leq \frac{c + \epsilon}{1 - \epsilon}.$$

\item It is easy to work out that
\begin{align*}
1 - x^{-\sigma} < \epsilon &\iff 1 - \epsilon < x^{-\sigma}\\
&\iff \log(1 - \epsilon) < -\sigma \log x\\
&\iff -\log(1-\epsilon) > \sigma \log x\\
&\iff x < (1 - \epsilon)^{-1/\sigma}
\end{align*}
so $X_\sigma = (1 - \epsilon)^{-1/\sigma} = \left( \frac{1}{1 - \epsilon} \right)^{1/\sigma} \rightarrow \infty$ as $\sigma \rightarrow 0^+$.

\item To complete the proof that $F$ is bounded, pick our favourite $\epsilon$, say $\epsilon = 1/2$ and we show that
$$F(x) \leq \frac{c + \epsilon}{1 - \epsilon} = 2 c + 1$$
for all $x$. Let $x$ be arbitrary and let $\sigma \in (0, \delta)$ be large enough so that $X_\sigma = 2^{1/\sigma} > x$ (this is possible since $1/\sigma \rightarrow +\infty$ as $\sigma \rightarrow 0^+$). Then the above argument shows that $F(X_\sigma) \leq 2c + 1$ and since $F$ is increasing $F(x) \leq F(X_\sigma) \leq 2c + 1$.

Of course, this argument could be used to show that $F(x) \leq \frac{c + \epsilon}{1 - \epsilon}$ for all $x$ and all $\epsilon \in (0, 1)$. So it is true that $F(x) \leq c$.
\end{itemize}

\unless\ifdefined\IsMainDocument
\end{document}
\fi
