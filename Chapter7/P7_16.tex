\unless\ifdefined\IsMainDocument
\documentclass[12pt]{article}
\usepackage{amsmath,amsthm,amssymb}

\begin{document}
\fi

\textbf{Problem 7.16}: Let $S(y)$ denote the sawtooth function and for $N = 1, 2, ...$, let $S_N(y)$ denote the $N$th partial sum of its Fourier series (7.22).
\begin{enumerate}
\item Show that $\int_0^1 \{S^2(y) - S_N^2(y)\} dy \rightarrow 0$ as $N \rightarrow \infty$.
\item By evaluating $\int_0^1 S^2(y) dy$ and $\int_0^1 S_N^2(y) dy$, give another proof of Theorem 1.5.
\end{enumerate}

\begin{proof}
\begin{enumerate}
\item Let $C$ be the uniform bound for $S_N(y)$ i.e.
$$|S_N(y)| \leq C, \qquad \text{ for all } y \in [0, 1].$$
by bounded convergence. Then
$$|S(y) + S_N(y)| \leq C + 1$$
and from the equation
$$S^2(y) - S_N^2(y) = (S(y) - S_N(y))(S(y) + S_N(y))$$
we have
$$|S^2(y) - S_N^2(y)| \leq (C + 1)^2$$
so $S_N^2(y) \rightarrow S^2(y)$ uniformly on $[\delta, 1 - \delta]$ for any $\delta \in (0, 1/2)$. This allows us to switch the limit over $N$ and the integral to get
$$\lim_{N \rightarrow \infty} \int_\delta^{1-\delta} \{S^2(y) - S_N^2(y)\} dy = \int_\delta^{1-\delta} \lim_{N \rightarrow \infty} \{S^2(y) - S_N^2(y)\} dy = 0.$$

To prove the full limit for $\int_0^1$, let $\epsilon > 0$ be arbitrary. We need to choose $N_0$ such that $|\int_0^1 ... dy| < \epsilon$ for all $N > N_0$. To do that, we first choose $\delta > 0$ such that $2 (C + 1)^2 \delta < \frac{\epsilon}{2}$ and then take $N_0$ such that
$$|S(y) - S_N(y)| < \frac{\epsilon}{2 (C + 1)}$$ 
for all $N > N_0$ and $y \in [\delta, 1-\delta]$ by uniform convergence. Note that the above inequality implies
$$|S^2(y) - S_N^2(y)| < \frac{\epsilon}{2}.$$

With this choice of $N_0$, we have for any $N > N_0$
\begin{align*}
|\int_0^1 \{S^2(y) - S_N^2(y)\} dy| &= |\int_0^\delta + \int_\delta^{1-\delta} + \int_{1-\delta}^1|\\
&\leq \left\{ \int_0^\delta + \int_{1-\delta}^1 \right\} (C + 1)^2 dy + \int_\delta^{1-\delta} \frac{\epsilon}{2} dy\\
&= 2\delta(C+1)^2 + (1 - 2\delta) \frac{\epsilon}{2}\\
&< \epsilon.
\end{align*}

\item One has
\begin{align*}
\int_0^1 S^2(y) dy &= \int_0^1 \left([y] - y + \frac12 \right)^2 dy\\
&= \int_0^1 \left(- y + \frac12 \right)^2 dy\\
&= \int_0^1 \left(y^2 - y + \frac14 \right) dy\\
&= \left.\frac{y^3}{3} - \frac{y^2}{2} + \frac14y \right|_0^1\\
&= \frac{1}{3} - \frac{1}{2} + \frac14\\
&= \frac{1}{12}
\end{align*}
and
\begin{align*}
\int_0^1 S_N^2(y) dy &= \int_0^1 \left( \sum_{n=1}^{N} \frac{1}{\pi n} \sin 2 \pi n y \right)^2 dy\\
&= \int_0^1 \left( \sum_{n=1}^{N} \sum_{m=1}^{N} \frac{1}{\pi n} \sin 2 \pi n y \frac{1}{\pi m} \sin 2 \pi m y \right) dy\\
&= \sum_{n=1}^{N} \sum_{m=1}^{N} \frac{1}{\pi^2 n m} \int_0^1 \sin (2 \pi n y) \sin(2\pi m y) \; dy\\
&= \sum_{n=1}^{N} \frac{1}{2 \pi^2 n^2}
\end{align*}
where we used the common trigonometric formula $2 \sin \alpha \sin \beta = \cos(\alpha - \beta) - \cos(\alpha + \beta)$ to resolve
\begin{align*}
\int_0^1 \sin (2 \pi n y) \sin(2\pi m y) \; dy &= \int_0^1 \frac{\cos (2 \pi (m - n) y) - \cos(2\pi (m + n) y)}{2} \; dy\\
&= \begin{cases}
1/2 &\text{if } m = n \not= 0,\\
-1/2 &\text{if } m = -n \not= 0,\\
0 &\text{otherwise.}
\end{cases}
\end{align*}
Thus, we have by part 1 that
$$\sum_{n=1}^{\infty} \frac{1}{2 \pi^2 n^2} = \frac{1}{12}$$
which is clearly the same as Theorem 1.5.
\end{enumerate}
\end{proof}

\unless\ifdefined\IsMainDocument
\end{document}
\fi
