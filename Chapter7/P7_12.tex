\unless\ifdefined\IsMainDocument
\documentclass[12pt]{article}
\usepackage{amsmath,amsthm,amssymb}
\newcommand{\A}{\mathcal{A}}
\newcommand{\Fhat}{\widehat{F}}
\newcommand{\Ghat}{\widehat{G}}
\begin{document}
\fi

\textbf{Problem 7.12}: (Generalized divisor function.) For $c$ a fixed real number, not 0 or a negative integer, define a multiplicative function $\tau_c$ by setting
$$\tau_c(p^j) = c (c + 1) \cdots (c + j - 1)/j!$$
for primes $p$ and positive integers $j$. Find an assymptotic formula for the associated summatory function. (Cf. Problem 3.26.)

\begin{proof}
Let $F_c$ be the summatory function of $\tau_c$.

\noindent \textbf{The case $c > 0$:} We consider the case $c > 0$ so that $\tau_c > 0$ and $F_c$ is monotone nondecreasing. Also observe that if $c < d$ then $\tau_c < \tau_d$ and so $F_c < F_d$. It is easy to check\footnote{Both are multiplicative and they agree on prime powers.} that $\tau_k = 1^{*k}$ for positive integer $k$ and we already have bound $F_k = O(x^{1 + \epsilon})$ for all positive integer $k$ (albeit not uniformly) thank to Problem 3.26. It follows that $F_c = O(F_{[c] + 1}) = O(x^{1 + \epsilon})$ for all $c > 0$ which implies $\sigma_c(F_c) = \sigma_a(F_c) \leq 1$. In other words, $\Fhat_c$ converges absolutely for $\sigma > 1$ and we have the Euler product there.

Notice that the function $h(x) = x^{-c}$ have higher derivatives
\begin{align*}
h_c^{(j)}(x) &= (-c) (-c - 1) ... (-c - j + 1) x^{- c - j}\\
&= (-1)^j c (c + 1) ... (c + j - 1) x^{- c - j}
\end{align*}
and thus Taylor expansion
$$h_c(x) = \sum_{j = 0}^{\infty} c (c + 1) ... (c + j - 1) \frac{(1 - x)^j}{j!}$$
which converges at least when $|x - 1| < 1$ (i.e. when $\lim_{j \rightarrow \infty} \frac{(c+j) |1-x|}{1 + j} < 1$ so ratio test applies).

Hence, we have
\begin{align*}
\Fhat_c(s) &= \prod_p \left(\sum_{j = 0}^{\infty} \frac{c (c + 1) \cdots (c + j - 1)}{j!} p^{-js}\right) & (\sigma > 1) \\
&= \prod_p h_c(1 - p^{-s})\\
&= \prod_p (1 - p^{-s})^{-c} \\
&= \zeta(s)^c \\
&= (s - 1)^{-c} \; ((s - 1)\zeta(s))^c
\end{align*}
so just like the previous problem: This formula shows that $\sigma_c(\Fhat_c) = 1$ (due to the pole of $\zeta$ and Problem 6.18) and $\Fhat_c$ satisfies the hypothesis of the generalized Wiener-Ikehara theorem with
$$\varphi(s) = ((s - 1)\zeta(s))^c$$
analytic on $\{ \sigma > 0 \}$. By the theorem, we have
$$F_c(x) = \sum_{n \leq x} \tau_c(n) \sim \frac{x (\log x)^{c - 1}}{\Gamma(c)}.$$

\textbf{Note}: If we do not want to use Problem 3.26, we can bootstrap the proof by solving this problem for $c = k$ being positive integers first: Theorem 6.2 gives us $\Fhat_k(s) = \zeta(s)^k$ for $\sigma > 1$ immediately thank to the representation $\tau_k = 1^{*k}$ even though we do not know $\sum_{n \leq x} \tau_k(n) = O(x^{1 + \epsilon})$. This implies $\sigma_c(\Fhat_k) = 1$ and we apply the generalized Wiener-Ikehara theorem just like the case for general $c > 0$.

\noindent \textbf{The case $c < 0$:} When $c < 0$, this argument does not work because the function $F_c$ is not nondecreasing as $\tau_c(n)$ can takes negative values, for instance, when $n$ is a prime number and there are infinitely many such $n$. On the other hand, we cannot even determine the right abscissa of convergence of $\zeta(s)^c$.
For example, when $c = -1$, $\zeta(s)^{-1} = \sum \mu(n) n^{-s}$ is expected to have $\sigma_c = 1/2$ and that is equivalent to the Riemann Hypothesis.
We are excluding $c$ being a negative integers here but the issue remains: It is clear that
$$|c + k| \leq |c| + k = -c + k$$
so
$$|\tau_c(p^j)| \leq \tau_{-c}(p^j)$$
which implies
$$\tau_c(n)| \leq \tau_{-c}(n)$$
for all $n$ and since we already know $F_{-c}(x) = O(x^{1 + \epsilon})$, it follows that $\Fhat_c(s)$ converes absolutely for $\sigma > 1$. With absolute convergence, the Euler product formula above still holds and we do have $\Fhat_c(s) = \zeta(s)^c$ for $\sigma > 1$ just like the case $\Fhat_{-1}(s) = \zeta(s)^{-1} = \sum \mu(n) n^{-s}$. If we write
$$\Fhat_c(s) = \left(\sum \mu(n) n^{-s}\right)^{-c}$$
then it seems to suggest that $\sigma_c(\Fhat_c) = 1/2$ (assuming the Riemann Hypothesis).

Taking the cue from Problem 7.6, let us consider the summatory function of $\tau_c + \tau_{-c}$ instead. This function is definitely $\geq 0$ thank to our proven bound $|\tau_c| \leq \tau_{-c}$ above. Then $\Fhat_c + \Fhat_{-c}$ has abscissa of convergence 1 thank to the known equation
$$(\Fhat_c + \Fhat_{-c})(s) = \zeta(s)^c + \zeta(s)^{-c}$$
valid for $\sigma > 1$ and the fact that $\zeta(s)$ has no zero on the line $\sigma = 1$ (so $\zeta(s)^c$ is continuous on $\sigma \geq 1$ and in fact is analytic on a larger half plane) and a simple pole at $s = 1$. Then we use the generalized Wiener-Ikehara theorem, this time with $\psi(s) = \zeta(s)^c$ and $\varphi(s) = ((s-1)\zeta(s))^{-c}$ (as in the $c > 0$ case) to conclude that
$$(F_c + F_{-c})(x) \sim \frac{x (\log x)^{-c - 1}}{\Gamma(-c)} \sim F_{-c}(x)$$
whence we know that
$$F_c = o\left( \frac{x (\log x)^{-c - 1}}{\Gamma(-c)} \right).$$

% THIS FOLLOWING HEURISTIC IS WRONG!
%Note that we do not expect the formula for the case $c > 0$ to hold verbatim here because the main contribution to $F_c(x)$ should come from the numbers with small total number of prime factors $\Omega(n)$, which is approximately
%$$\underbrace{c \frac{x}{\log x}}_{\Omega = 1} + \underbrace{\frac{c(c+1)}{2} \frac{\sqrt{x}}{\log \sqrt{x}} + \sum_{q < x} c^2 \pi(\min\{q, x/q\})}_{\Omega = 2} + \cdots$$
%and this has ``denominator at most $\log x$''.

This gives credence to the asymtotic formula
$$F_c(x) \sim x (\log x)^{c-1} / \Gamma(c)$$
even for negative $c$.

Observe that $\tau_c * \tau_{-c} = e$. We can see this from either $x^c x^{-c} = 1$ of Taylor series or from the fact that $\Fhat_c(s) \Fhat_{-c}(s) = 1$ and invoke Theorem 6.25. (It is not straightforward to see why this is true without the Dirichlet series.)

We shall take higher derivatives as in the proof of Wiener-Ikehara theorem: Let $N$ be the positive integer such that $0 < c + N$. Then
$$\Fhat_c^{(N)}(s) = \int x^{-s} (-\log x)^N dF_c, \qquad \sigma > 1.$$
For any function $h(s)$, one has
\begin{align*}
(h^c)^{(N)} &= (c h^{c-1} h')^{(N-1)} \\
&= (c(c-1) h^{c-2} (h')^2 + ch^{c-1} h'')^{(N-2)} \\
&= (c(c-1)(c-2) h^{c-3} (h')^3 + 2 c(c-1) h^{c-2} h' + c(c-1) h^{c-2} h' h'' + ch^{c-1} h''')^{(N-3)} \\
&= ... \\
&= \sum_{k + k_1 + \cdots + k_m = N} g^{(k)}(h) h^{(k_1)} h^{(k_2)} \cdots \qquad \text{ where } g(x) = x^c
\end{align*}
Let us consider the case $N = 1$ i.e. $-1 < c < 0$. Then
$$\Fhat_{c+1}'(s) = (\zeta(s)^{c+1})' = (c + 1) \zeta(s)^c \zeta'(s)$$
gives
$$\Fhat_c(s) = \frac{\Fhat_{c+1}'(s)}{(c + 1) \zeta'(s)}.$$
Unfortunately, this does not help us understand $\tau_c$ since the arithmetic function associated to $\zeta'(s)$ i.e. $\log n$ vanishes at $1$ so it does not have convolution inverse.

\textbf{Reference}: The asymptotic formula even for complex value of $c$ was in the paper of A. Selberg, \textit{Note on a paper by L. G. Sathe}.
\end{proof}

\unless\ifdefined\IsMainDocument
\end{document}
\fi
