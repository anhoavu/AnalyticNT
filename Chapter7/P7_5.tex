\unless\ifdefined\IsMainDocument
\documentclass[12pt]{article}
\usepackage{amsmath,amsthm,amssymb}
\newcommand{\V}{\mathcal{V}}
\newcommand{\Fhat}{\widehat{F}}
\newcommand{\cconj}[1]{\overline{#1}}
\begin{document}
\fi

\textbf{Problem 7.5}: Let $F \in \V$ be defined by
$$F(x) = \int_1^x \{1 - \cos(\lambda_0 \log u)\} du \qquad (x \geq 1)$$
for some $\lambda_0 > 0$. Find $\Fhat$, $\limsup_{x \rightarrow \infty} F(x) / x$, and $\liminf_{x \rightarrow \infty} F(x) / x$.

\begin{proof}
We can compute $F(x)$ by integration by parts but the smarter way is to observe that
$$u^{i \lambda_0} = e^{i \lambda_0 \log u} = \cos(\lambda_0 \log u) + i \sin(\lambda_0 \log u)$$
so
\begin{align*}
\int_1^x \{\cos(\lambda_0 \log u) + i \sin(\lambda_0 \log u)\} du &= \int_1^x u^{i\lambda_0} du\\
&= \left. \frac{u^{i \lambda_0 + 1}}{i \lambda_0 + 1} \right|_1^x\\
&= \frac{x^{i \lambda_0 + 1} - 1}{i \lambda_0 + 1}\\
&= \frac{(x^{i \lambda_0 + 1} - 1) (1 - i \lambda_0)}{1 + \lambda_0^2}\\
&= \frac{x e^{i (\lambda_0 \log x + \beta)} }{\sqrt{ 1 + \lambda_0^2 } } - \frac{1 - i \lambda_0}{1 + \lambda_0^2}
\end{align*}
where we have expressed the complex number
$$1 - i \lambda_0 = \sqrt{1 + \lambda_0^2} \; e^{i \beta}$$
in polar form. We can then extract $\int_1^x \cos(\lambda_0 \log u) du$ as the real part of the last expression
$$\int_1^x \cos(\lambda_0 \log u) du = \frac{x \cos (\lambda_0 \log x + \beta) }{\sqrt{ 1 + \lambda_0^2 } } - \frac{1}{1 + \lambda_0^2}$$
and so
$$\frac{F(x)}{x} = 1 - \frac{1}{x} - \frac{\cos(\beta + \lambda_0 \log x)}{\sqrt{\lambda_0^2 + 1}}  + \frac{1}{x(\lambda_0^2 + 1)}.$$
This formula gives $\limsup = 1 + \frac{1}{\sqrt{\lambda_0^2 + 1}}$ and $\liminf = 1 - \frac{1}{\sqrt{\lambda_0^2 + 1}}$ as $\cos(\beta + \lambda_0 \log x)$ infinitely oscillates between $-1$ and 1 when $x \rightarrow \infty$ given $\lambda_0 \not= 0$.

Now to compute $\Fhat(s)$, we use a similar idea
\begin{align*}
\cos(\lambda_0 \log x) &= \Re(e^{i \lambda_0 \log x})\\
&= \frac{1}{2}(e^{i \lambda_0 \log x} + e^{-i \lambda_0 \log x}) &\text{ as } \Re(z) = \frac{z + \cconj{z}}{2}\\
&= \frac{x^{i\lambda_0} + x^{-i\lambda_0}}{2} &\text{ and } \cconj{z} = z^{-1} \text{ if } |z| = 1
\end{align*}
to get
\begin{align*}
\Fhat(s) &= \int x^{-s} \{1 - \cos(\lambda_0 \log x)\} dx\\
&= \int x^{-s} \left\{1 - \frac{x^{i\lambda_0} + x^{-i\lambda_0}}{2} \right\} dx\\
&= \lim_{X \rightarrow \infty} \int_1^X \left\{x^{-s} - \frac12 x^{-s + i\lambda_0} - \frac12 x^{-s-i\lambda_0} \right\} dx\\
&= \lim_{X \rightarrow \infty} \left. \left\{\frac{x^{1-s}}{1 - s} - \frac{x^{1 - s + i\lambda_0}}{2(1 - s + i\lambda_0)}  - \frac{x^{1-s-i\lambda_0}}{2(1 - s - i\lambda_0)} \right\} \right|_1^X\\
&= \frac{1}{s - 1} - \frac{1}{2(s - i\lambda_0 - 1)}  - \frac{1}{2(s + i\lambda_0 - 1)} &\text{ for } \sigma > 1
\end{align*}
(Note that $\sigma_c(\Fhat) = 1$ because we found that the correct order of growth of $F(x) = O(x)$.)

This problem illustrates that when $\Fhat(s)$ satisfies the hypothesis of Wiener-Ikehara theorem only in the strip $|t| < \lambda_0$ and we can only conclude that $F(x) = O(x)$; in fact, the assymptotic behavior does not exist.
\end{proof}

\unless\ifdefined\IsMainDocument
\end{document}
\fi
