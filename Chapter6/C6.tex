\unless\ifdefined\IsMainDocument
\documentclass[12pt]{article}
\usepackage{amsmath,amsthm,amssymb,hyperref}
\newcommand{\Fhat}{\widehat{F}}
\newcommand{\Abs}[1]{\left| #1 \right|}
\begin{document}
\fi

\section{Uniform Convergence}

When we say $\Fhat$ converges uniformly on $U$, what we mean is: For any $\epsilon > 0$, there exists an $X_0$ such that for any $X > X_0$,
$$\Abs{\int_{1^-}^X x^{-s} dF(x) - \Fhat(s)} < \epsilon$$
for all $s \in U$.

In other words, the uniform convergence is for the $\lim_{X \rightarrow \infty}$ in the definition of $\Fhat$.

To see how this goes back to the \href{https://en.wikipedia.org/wiki/Uniform\_convergence}{classical definition of uniform convergence} of a sequence of functions $f_n \rightarrow f$, the ``sequence'' of functions in this case is
$$\Fhat_X(s) = \int_{1^-}^X x^{-s} dF(x).$$
So $\Fhat_X \rightarrow \Fhat$ uniformly, by the classical definition, means that for any $\epsilon > 0$, we can find a single $X_0(\epsilon)$ i.e. independent of $s$ such that
$$\Abs{ \Fhat_X(s) - \Fhat(s) } < \epsilon$$
as long as $X > X_0(\epsilon)$. See also the proof of Corollary 6.16.

Due to completeness of the real/complex numbers, the above definition has an equivalent Cauchy's version: $\Fhat_X \rightarrow \Fhat$ uniformly if and only if for any $\epsilon > 0$, we can find a single $X_0(\epsilon)$ such that for all $Y, Z > X_0(\epsilon)$, we have
$$\Abs{ \Fhat_Y(s) - \Fhat_Z(s) } < \epsilon.$$
This is the condition we used in the proof of Theorem 6.15. Also note in the proof of that Theorem, the reason we can write
$$\int_Y^Z x^{-s} dF(x) = \int_Y^Z x^{-(s-s_0)} d\psi(x)$$
is to first express
$$\int_Y^Z x^{-s} dF(x) = \int_Y^Z x^{-s_0} x^{-(s-s_0)} dF(x) = \int_Y^Z x^{-s_0} d\psi_0(x)$$
where 
$$\psi_0(y) = \int_{1^-}^y x^{-(s-s_0)} dF(x).$$
And then we observe that $\psi_0(y) - \psi(y) = K$ is a constant, namely $K = \int_{1^-}^{\infty} x^{-s_0} dF(x)$ by convergence assumption. This implies $\psi(y) = \psi_0(y) - K$ and so $d\psi = d\psi_0$.

\section{Weirstrass M test}

The classical statement is: Let $f_n$ be a sequence of functions $f_n: E \to \mathbb {C}$ and let $M_n$ be a sequence of positive real numbers such that $|f_n(x)|\leq M_{n}$ for all $x \in E$ and $n = 1, 2, 3, \ldots$. If $\sum M_n$ converges, then $\sum f_n$ converges uniformly on $E$.

In the proof of Lemma 6.13, we are using the obvious inequality
$$\Abs{ \int_Y^Z x^{-s} dF(x) } \leq \int_Y^Z x^{-b} dF_v$$
for all $s \in \{\sigma \geq b\}$ and the assumption $\int x^{-b} dF_v < \infty$ to deduce $\int x^{-s} dF(x)$ converges uniformly on that half plane.

\section{A consequence of Example 6.29}

We proved here that $\zeta(s) = \exp\left(\sum \kappa(n) n^{-s}\right)$ on $\{\Re s > 1\}$. This implies that $\zeta(s)$ has no zero on that halfplane.

\section{The inequality in the second proof of Lemma 6.30}

The inequality $n^\alpha \leq n^{\alpha + 1} - n^{\alpha}$ can be shown by noting that
$$n^\alpha \leq n^{\alpha + 1} - (n - 1)^{\alpha + 1} \iff n^\alpha + (n-1)^{\alpha + 1} \leq n^{\alpha + 1}$$
which easily follows from the trivial bound
$$(n-1)^{\alpha + 1} = (n-1) (n-1)^{\alpha} \leq (n-1) n^{\alpha}$$
whence
$$n^\alpha + (n-1)^{\alpha + 1} \leq n^{\alpha} + (n-1) n^{\alpha} = n^{\alpha + 1}.$$

\unless\ifdefined\IsMainDocument
\end{document}
\fi
