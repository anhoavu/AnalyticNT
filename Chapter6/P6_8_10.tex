\unless\ifdefined\IsMainDocument
\documentclass[12pt]{article}
\usepackage{amsmath,amsthm,amssymb}
\newcommand{\C}{\mathbb{C}}
\newcommand{\V}{\mathcal{V}}
\newcommand{\Fhat}{\widehat{F}}
\begin{document}
\fi

\textbf{Problem 6.8}: Assume that it is known that $\sum \mu(n) n^{-1-it}$ converges for some real $t$. Deduce from this the validity of the PNT.

\begin{proof}
By Theorem 6.9, the assumption implies $M(x) = o(x)$. Here, we are in case (1) where $\sigma = 1 > 0$. Clearly, this is equivalent to the PNT by Theorem 5.9.
\end{proof}

\textbf{Problem 6.9}: Let $F \in \V$ and assume that $\int x^{-s_0} dF(x)$ converges for some $s_0$ with $\Re s_0 > 0$. Prove that
$$\int x^{-s_0} dF(x) = s_0 \int x^{-s_0-1} F(x) dx.$$

\begin{proof}
By Theorem 6.9, we have $F(x) = o(x^\sigma)$ where $\sigma = \Re s_0$. For any $X > 1$, integration by parts gives
$$\int_{1^-}^X x^{-s_0} dF(x) = F(X) X^{-s_0} + s_0 \int_{1^-}^X x^{-s_0-1} F(x) dx$$
where taking limit as $X \rightarrow \infty$, taking into account $F(x) = o(x^{\sigma})$, yields the desired equation.
\end{proof}

\textbf{Problem 6.10}: Suppose $F(x) \sim x$. Find the precise set of points $s \in \C$ at which $\Fhat$ converges.

\begin{proof}
The example $F(x) = x$ where $\Fhat(s) = \int x^{-s} dx = \lim_{X \rightarrow \infty} \frac{X^{-s + 1}}{1 - s} - \frac{1}{1 - s}$ and $F(x) = N(x)$ where $\Fhat(s) = \zeta(s)$ shows that the set of convergence should be $\Re s > 1$.

Clearly, if $F(x) \sim x$ then $F(x) = x + o(x) = O(x)$ so Theorem 6.9 implies that $\Fhat$ converges for $\Re s > 1$.

Now assume that $\Fhat(s)$ converges and we show that $\Re s > 1$. If $\sigma \leq 0$ then either $F(x) = o(\log ex)$ or $F(x) = c + o(x^\sigma)$ by Theorem 6.9 and this clearly violates the assumption that $F(x) \sim x$. If $0 < \sigma \leq 1$ then by Theorem 6.9 $F(x) = o(x)$ once again violates the assumption $F(x) \sim x$. So $\sigma > 1$.
\end{proof}

\unless\ifdefined\IsMainDocument
\end{document}
\fi
