\unless\ifdefined\IsMainDocument
\documentclass[12pt]{article}
\usepackage{amsmath,amsthm,amssymb}
\newcommand{\Fhat}{\widehat{F}}
\newcommand{\A}{\mathcal{A}}
\newcommand{\Abs}[1]{\left| #1 \right|}
\begin{document}
\fi

\textbf{Problem 6.25}: Suppose $f \in \A$, $f(1) = 1$ and $|f| \leq 1$. Use the equation $f * f^{*-1} = e_1$ to prove that $|f^{*-1}(n)| \leq n^2$ for all $n$. This result can be used to give still another proof of Lemma 6.30.

\begin{proof}
To simplify our notation, let $g = f^{*-1}$. To get our feeling, consider the case $n = p$ is prime, the equation $f * g = e_1$ gives
$$|g(p)| = |-f(p)| \leq 1 \leq p^2$$
and by induction, if $n = p^k$ is a prime power we have
$$|g(p^k)| = \Abs{ -\sum_{j = 0}^{k-1} f(p^{k-j}) g(p^j) } \leq \sum_{j = 0}^{k - 1} |g(p^j)| \leq \sum_{j = 0}^{k - 1} p^{2j} = \frac{p^{2k}-1}{p^2 - 1} \leq p^{2k}.$$
So the inequality is quite loose.

We prove that $|g(n)| \leq n^2$ by induction. Assume that this inequality is true for all $n < m$ and we prove that it is true for $n = m$. From $f * g = e_1$, we obtain
\begin{align*}
|g(m)| &= \Abs{ -\sum_{d | m, d < m} g(d) f(n/d) }\\
&\leq \sum_{d | m, d < m} |g(d)| &\text{by assumption } |f| \leq 1\\
&\leq \sum_{d | m, d < m} d^2 &\text{by induction hypothesis}\\
&= \left( \sum_{d | m} d^2 \right) - m^2\\
&= (T^2 1 * 1)(m) - m^2 &\text{note that } (T^2 1)(n) = n^2
\end{align*}

As both $T^2 1$ and 1 are multiplicative, $T^2 1 * 1$ is also multiplicative and we therefore have
\begin{align*}
(T^2 \; 1 * 1)(m) &= \prod_{p^k || m} (T^2 \; 1 * 1)(p^k)\\
&= \prod_{p^k || m} \left( \sum_{j = 0}^{k} p^{2j} \right)\\
&= \prod_{p^k || m} \frac{p^{2k+2} - 1}{p^2 - 1}\\
&= \prod_{p^k || m} p^{2k} \frac{p^{2k+2} - 1}{p^{2k}(p^2 - 1)}\\
&= m^2 \prod_{p^k || m} \left( 1 + \frac{1 - p^{-2k}}{p^2 - 1} \right)\\
&\leq m^2 \prod_{p^k || m} \left( 1 + \frac{1}{p^2 - 1} \right)\\
&\leq m^2 \underbrace{\prod_p \left( 1 + \frac{1}{p^2 - 1} \right)}_{1.64... < 2}
\end{align*}
So it is clear that $|g(m)| \leq m^2$.

To see how this implies Lemma 6.30, let us assume that $f$ is of polynomial growth. Then $|f(n)| \leq K n^\alpha$ for some constants $K, \alpha > 0$. We can assume $K = |f(1)|$ by increasing $\alpha$: If $|f(1)| < K$ then we can simply take $K = |f(1)|$; otherwise, we replace $\alpha$ by $\alpha + b$ for any $b > 0$ such that $2^b > \frac{K}{|f(1)|} > 1$.

Consider the function $g = \frac{1}{f(1)} T^{-\alpha} f$ so $|g| \leq 1$ and $g(1) = 1$. From $f = T^{\alpha} K g$, we get
$$e_1 = T^{\alpha} \underbrace{(K g * K^{-1} g^{*-1})}_{e_1} = \underbrace{(T^{\alpha} Kg)}_{f} * (T^{\alpha} K^{-1}g^{*-1})$$
and so
$$f^{*-1} = T^{\alpha} K^{-1} g^{*-1}$$
is of polynomial growth as long as $g^{*-1}$ is and that is the case here thank to the problem.
\end{proof}

\unless\ifdefined\IsMainDocument
\end{document}
\fi
