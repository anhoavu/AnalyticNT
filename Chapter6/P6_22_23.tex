\unless\ifdefined\IsMainDocument
\documentclass[12pt]{article}
\usepackage{amsmath,amsthm,amssymb}
\newcommand{\Fhat}{\widehat{F}}
\newcommand{\Abs}[1]{\left| #1 \right|}
\newcommand{\Z}{\mathbb{Z}}
\newcommand{\C}{\mathbb{C}}
\begin{document}
\fi

\textbf{Problem 6.22}: Let $\Fhat(s) = \sum a_n n^{-s}$ be a D.s. with $\sigma_c(\Fhat) < \infty$. Let $N \in \Z^+$ and let $s_1, s_2, ...$ be a sequence from $\C$ with $\Re s_j \rightarrow +\infty$ as $j \rightarrow \infty$. Show that
$$\lim_{j \rightarrow \infty} N^{s_j} \sum_{n = N}^{\infty} a_n n^{-s_j} = a_N.$$
Use this relation prove Theorem 6.25.

\begin{proof}
Note that
$$N^{s_j} \sum_{n = N}^{\infty} a_n n^{-s_j} = \sum_{n = N}^{\infty} a_n (N/n)^{s_j}$$
so if we have uniform convergence, we could then swap the limit and the series
$$\lim_{j \rightarrow \infty} \sum_{n = N}^{\infty} a_n (N/n)^{s_j} = \sum_{n = N}^{\infty} \lim_{j \rightarrow \infty} a_n (N/n)^{s_j} = a_N$$
for if $n > N$ then $N/n < 1$ and $(N/n)^{u} \rightarrow 0$ as $\Re u \rightarrow \infty$ so the only term that survives is the one for $n = N$. Without uniform convergence, we prove it directly from the definition. Recall $\sigma_a < \infty$ so we can fix an $b > \sigma_a$. Then for all $j$ such that $\sigma_j = \Re s_j > b$, we have
\begin{align*}
\Abs{ N^{s_j} \sum_{n = N}^{\infty} a_n n^{-s_j} - a_N } &= \Abs{ \sum_{n = N + 1}^{\infty} a_n \left(\frac{N}{n}\right)^{s_j} }\\
&\leq \sum_{n = N + 1}^{\infty} |a_n| \left(\frac{N}{n}\right)^{\sigma_j}\\
&= N^b \sum_{n = N + 1}^{\infty} |a_n| n^{-b} \left(\frac{N}{n}\right)^{\sigma_j - b}\\
&\leq N^b \sum_{n = N + 1}^{\infty} |a_n| n^{-b} \left(\frac{N}{N + 1}\right)^{\sigma_j - b} &\text{ for } \frac{N}{n} \leq \frac{N}{N+1}\\
&\leq \underbrace{N^b \left( \sum_{n = N + 1}^{\infty} |a_n| n^{-b} \right)}_{\text{constant}} \left(\frac{N}{N + 1}\right)^{\sigma_j - b}\\
&\rightarrow 0 \text{ as } j \rightarrow \infty
\end{align*}
because $\sigma_j \rightarrow \infty$.

Now to derive Theorem 6.25, first we note that we could assume the existence of $s_j$ such that $\Re s_j \rightarrow \infty$. For the other assumption where $s_j$ has a limit point in a common half plane, the two series are identical there and so we can simply pick another sequence.

Second, subtracting $f$ by $g$, we can assume $g = 0$ without loss of generality. In other words, it suffices to show that if there exists a sequence $s_j$ with $\Re s_j \rightarrow \infty$ and that $\Fhat(s_j) = 0$ for all $j$ then $f = 0$. We do this by induction:
\begin{itemize}
\item $f(1) = 0$: Follows from the limit formula for $N = 1$:
$$f(1) = \lim_{j \rightarrow \infty} \Fhat(s_j) = 0.$$
\item Induction: Assume that $f(1) = f(2) = ... = f(n - 1) = 0$. We show that $f(n) = 0$. By the limit formula for $N = n$:
\begin{align*}
f(n) &= \lim_{j \rightarrow \infty} n^{-s_j} \sum_{k = n}^{\infty} a_k k^{-s_j} \\
&= \lim_{j \rightarrow \infty} n^{-s_j} \sum_{k = 1}^{\infty} a_k k^{-s_j} &\text{by induction hypothesis}\\
&= \lim_{j \rightarrow \infty} n^{-s_j} \Fhat(s_j)\\
&= 0.
\end{align*}
\end{itemize}
So $f(n) = 0$ for all $n$.
\end{proof}

\textbf{Problem 6.23}: Let $\Fhat$ be a nonconstant D.s. with $\sigma_c(\Fhat) < \infty$ and let $\alpha$ be any fixed complex number. Show that there exists a half plane $\{s : \sigma > \sigma_0(\Fhat, \alpha) \}$ on which $\Fhat(s) \not= \alpha$.

\begin{proof}
Without loss of generality, we could assume $\alpha = 0$; otherwise, consider the Dirichlet series $\Fhat(s) - \alpha$. Assume that there is no such half plane, or equivalently $\sigma_0(\Fhat, 0)$. Then we could find a sequence $s_j \rightarrow \infty$ such that $\Fhat(s_j) = 0$. But then $\Fhat(s) = 0$ by Theorem 6.25 (or the previous problem), a contradiction.
\end{proof}

\unless\ifdefined\IsMainDocument
\end{document}
\fi
