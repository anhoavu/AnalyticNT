\unless\ifdefined\IsMainDocument
\documentclass[12pt]{article}
\usepackage{amsmath,amsthm,amssymb}
\newcommand{\Fhat}{\widehat{F}}
\newcommand{\A}{\mathcal{A}}
\newcommand{\Abs}[1]{\left| #1 \right|}
\begin{document}
\fi

\textbf{Problem 6.24}: Suppose $\varphi \in \A$, $\varphi(1) = 0$ and $f = \exp \varphi$. Show that if one of $f, \varphi$ is of polynomial growth then so is the other. Use these relations to give another proof of Lemma 6.30.

\begin{proof}
Let us show first that
\begin{enumerate}
\item The integral
\begin{align*}
\int_1^x \frac{x}{t} \log^{k-1}\left(\frac{x}{t}\right) \; dt
=& x \int_1^x (\log x - \log t)^{k-1} \; d(\log t)\\
=& x \int_0^{\log x} (\log x - u)^{k-1} \; du\\
=& x \left. \frac{-(\log x - u)^{k}}{k} \right|_{u=0}^{u=\log x}\\
=& \frac{x \log^k x}{k}.
\end{align*}

\item Define the function
$$s_k(n) = \sum_{d_1 ... d_k = n, \; d_i > 1 \; \forall i} 1$$ which counts the number of ways to write $n = d_1 ... d_k$ with $d_i > 1$ for all $i$. It is easy to see that
$$s_k = (1 - e_1)^{*k}.$$

Let $S_k$ be the summatory function of $s_k$. Then for any $k \geq 1$ and any $x \geq 1$,
$$S_k(x) \leq \frac{x \log^{k-1} x}{(k-1)!} =: f_k(x).$$

We prove this by induction.

\textbf{The base case} $k = 1$ is clear: $S_1(x) = N(x) - 1 \leq x = f_1(x)$.

\textbf{Induction}: Suppose that the inequality holds for $k$. We want to show that it holds for $k + 1$. One has
\begin{align*}
S_{k+1}(x) &= \int_{1^-}^x d S_k * (d N - \delta_1) &\text{from } s_{k+1} = s_k * (1 - e_1)\\
&= \int_{1^-}^x S_k(x/t) \; d (N - \delta_1) \\
&\leq \int_{1^-}^x S_k(x/t) \; dt &\text{by Lemma 3.11} \\
&\leq \int_{1^-}^x f_k(x/t) \; dt &\text{by induction hypothesis}\\
&= \int_1^x f_k(x/t) \; dt\\
&= f_{k+1}(x) &\text{as evaluated above}
\end{align*}
Note that the function $t \mapsto S_k(x/t)$ is nonnegative decreasing since $S_k$ is clearly increasing and $N(t) - 1 \leq t - 1$. Recall that $dt = d(t - 1)$ is continuous at 1 so we can replace $1^-$ by $1$.

\item Suppose that $g(1) = 0$ and $|g(n)| \leq A n^\alpha$ for some constants $A$ and $\alpha$. For any $k \geq 1$, we have
\begin{align*}
\Abs{ g^{*k}(n) } &= \Abs{ \sum_{d_1 ... d_k = n, \; d_i > 1 \; \forall i} g(d_1) \cdots g(d_k) } &\text{ since } g(1) = 0\\
&\leq \sum_{d_1 ... d_k = n, \; d_i > 1 \; \forall i} (Ad_1^\alpha \cdots Ad_k^\alpha)\\
&= \sum_{d_1 ... d_k = n, \; d_i > 1 \; \forall i} A^k n^\alpha\\
&= A^k n^\alpha s_k(n)\\
&\leq A^k n^\alpha S_k(n)\\
&\leq A^k n^\alpha \frac{n \log^{k-1} n}{(k-1)!}\\
&= \frac{A^k n^{\alpha + 1} \log^{k-1} n}{(k-1)!}.
\end{align*}
\end{enumerate}

Now we are ready to solve the main problem.
\begin{itemize}
\item Suppose that $\varphi$ is of polynomial growth. So $|\varphi(n)| \leq A n^\alpha$ for some $A$ and $\alpha$. Then for any $n \geq 3$, we have
\begin{align*}
\Abs{ f(n) } &= \Abs{ \sum_{k = 0}^{\infty} \frac{1}{k!} \varphi^{*k}(n) }\\
&\leq \sum_{k = 0}^{\infty} \frac{A^k n^{\alpha + 1} \log^{k-1} n}{k! (k-1)!} &\text{ by above inequality for } g = \varphi\\
&= n^{\alpha + 1} \sum_{k = 0}^{\infty} \frac{A^k \log^k n}{k!} &\text{ for } n \geq 3 \text{ we have } \log n \geq 1\\
&= n^{\alpha + 1} e^{A \log n}\\
&= n^{\alpha + A + 1}
\end{align*}
so $f$ is of polynomial growth.

\item For the other direction, suppose that $f$ is of polynomial growth so we can find $A$ and $\alpha$ such that $|f(n)| \leq A n^{\alpha}$. Note that $f - e_1$ satisfies the same inequality and vanishes at 0. For any $n \geq 3$:
\begin{align*}
\Abs{\varphi(n)} &= \Abs{ \sum_{k = 1}^{\infty} \frac{(-1)^{k-1}}{k} (f - e)^{*k} (n)}\\
&\leq \sum_{k = 1}^{\infty} \frac{1}{k} \frac{A^k n^{\alpha + 1} \log^{k-1} n}{(k-1)!} &\text{ by above inequality for } g = f - e_1\\
&\leq n^{\alpha + 1} \sum_{k = 0}^{\infty} \frac{A^k \log^k n}{k!}\\
&= n^{\alpha + A + 1}
\end{align*}
so $\varphi$ is of polynomial growth.
\end{itemize}

Finally, to derive Lemma 6.30, by assumption $f$ of polynomial growth, we have $\varphi = \log f$ is of polynomial growth. So is $-\varphi$ and subsequently $f^{*-1} = \exp(-\varphi)$ is of polynomial growth.
\end{proof}

\unless\ifdefined\IsMainDocument
\end{document}
\fi
