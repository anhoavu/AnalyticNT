\unless\ifdefined\IsMainDocument
\documentclass[12pt]{article}
\usepackage{amsmath,amsthm,amssymb}
\newcommand{\IntPart}[1]{\left[ #1 \right]}
\begin{document}
\fi

\textbf{Problem 4.3}: Show that for any positive integer $r$ there exists a number $c_r$ such that
$$\pi(x) \leq (x + c_r) \prod_{i=1}^{r} (1 - p_i^{-1})$$
for all positive $x$. Show that we may take $c_1 = 1$, $c_2 = 5$, $c_3 = 11$.

\begin{proof}
Let $N = \prod_{i = 1}^r p_i$ as in the proof of Lemma 4.3 and we recognize $\prod_{i=1}^{r} (1 - p_i^{-1})$ as $\varphi(N) / N$. So the inequality is equivalent to
$$\pi(x) \leq (x + c_r) \frac{\varphi(N)}{N} = x \frac{\varphi(N)}{N} + c_r \frac{\varphi(N)}{N}$$
and so the problem is tantamount to showing that
$$\pi(x) \leq x \frac{\varphi(N)}{N} + C_r$$
for some constant $C_r$.

Let $S$ be the set in the proof of Lemma 4.3 where we have
\begin{itemize}
\item $\pi(x) \leq r + S(x)$ and
\item $S(x) < S\left(\IntPart{ \frac{x + N}{N} } x\right) = \IntPart{ \frac{x + N}{N} } \varphi(N) \leq \left( \frac{x}{N} + 1 \right) \varphi(N) = x \frac{\varphi(N)}{N} + \varphi(N)$
\end{itemize}
and so $C_r = r + \varphi(N)$ works. Then
$$c_r = \frac{N C_r}{\varphi(N)} = \frac{N (r + \varphi(N))}{\varphi(N)} = N + \frac{N r}{\varphi(N)}.$$

Unfortunately, this choice of $c_r$ is larger for the remaining parts. So we make separate arguments for them
\begin{itemize}
\item $r = 1$ and $c_1 = 1$: The inequality then reads $\pi(x) \leq \frac12 (x + 1)$ which is evident considering primes are odd numbers, except for 2.
\item $r = 2$ and $c_2 = 5$: The inequality then reads $\pi(x) \leq \frac13 (x + 5) = \frac13 x + \frac53$. Now a prime $\geq 5$ can only be congruent to 1 or 5 modulo 6 and there are at most $\frac13x$ of those numbers. Adding $\frac53$ clearly accounts for 2 and 3.
\item $r = 3$ and $c_2 = 11$: The inequality then reads $\pi(x) \leq \frac4{15} (x + 11)$. Similar argument.
\end{itemize}
\end{proof}

\unless\ifdefined\IsMainDocument
\end{document}
\fi
