\unless\ifdefined\IsMainDocument
\documentclass[12pt]{article}
\usepackage{amsmath,amsthm,amssymb}

\begin{document}
\fi

\textbf{Problem 4.11}: Use the preceding method, with a suitable choice of truncation point, estimate
$$\lim_{x \rightarrow \infty} \left\{\sum_{n \leq x} \frac{\kappa(n)}{n} - \sum_{p \leq x} \frac{1}{p} \right\}$$
with an error smaller than 0.0001.

\begin{proof}
Recall on page 81 that
$$\lim_{x \rightarrow \infty} \left\{\sum_{n \leq x} \frac{\kappa(n)}{n} - \sum_{p \leq x} \frac{1}{p} \right\} = \sum_p \left\{ - \log (1-p^{-1}) - p^{-1} \right\}$$
so let us consider the function
$$f(x) := -\log (1-x^{-1}) - x^{-1}.$$
One has
$$f'(x) = - (1-x^{-1})^{-1} (x^{-2}) + x^{-2} = -(x^2 - x)^{-1} + x^{-2} = -\frac{1}{(x-1)x^2}$$
and
$$f''(x) = (x^2 - x)^{-2} (2x - 1) - 2x^{-3} = \frac{x (2x-1) - 2(x-1)^2}{(x - 1)^2 x^3} = \frac{3 x - 2}{(x - 1)^2 x^3}$$
so $f''(x) > 0$ for all $x \geq 1$ and $f$ satisfies the hypothesis of Lemma 4.12 on the interval $[1, \infty)$. (We have $f(x) \geq 0$ on this interval because $f'(x) < 0$ on $[1, \infty)$ so $f$ is decreasing and $f(x) \geq f(\infty) = 0$.)

By Lemma 4.12, as long as $N \geq 1$ we have $6N - 3 \geq 1$ so
\begin{align*}
\sum_{p > 6N - 3} f(p) &< \frac{1}{3} \int_{6N - 3}^{\infty} f(x) dx \\
&= \left. \frac{1}{3} (1 - x) \log (1-x^{-1}) \right|_{6N-3}^{\infty} &\text{integration by parts} \\
&= \frac{1}{3}\left\{ 1 - (4 - 6N) \log \left(1-\frac{1}{6N-3}\right) \right\} &\text{by L' Hospitale rule}
\end{align*}
so the problem becomes choosing $N$ so that the right hand side is less than the desired error. 

Put $y = \frac{1}{6N-3}$ for simplicity, we turn the problem into bounding
\begin{align*}
1 - (1 - y^{-1}) \log(1 - y) &= 1 + (1 - y^{-1}) \left(\sum_{j = 1}^{\infty} \frac{y^j}{j} \right)\\
&= 1 + \sum_{j = 1}^{\infty} \frac{y^j}{j} - \sum_{j = 1}^{\infty} \frac{y^{j-1}}{j}\\
&= 1 + \sum_{j = 1}^{\infty} \frac{y^j}{j} - \sum_{j = 0}^{\infty} \frac{y^j}{j + 1}\\
&= \sum_{j = 1}^{\infty} \frac{y^j}{j (j + 1)}\\
&< \sum_{j = 1}^{\infty} \frac{y}{j (j + 1)} &\text{if } 0 < y < 1\\
&= y
\end{align*}

So if we want the error to be smaller than $\epsilon = 0.0001$, we just need to choose $N$ such that $\frac{1}{3 (6N - 3)} < \epsilon$. In that case,
$$N > \frac{1}{18\epsilon} + \frac{1}{2} = 556.05...$$
so $N = 557$, hence summing up to $6N - 3 = 3339$, will do.

%We compute the anti-derivative
%\begin{align*}
%\int f(x) dx &= x f(x) - \int x f'(x) dx\\
%&= x f(x) + \int \frac{1}{x(x-1)} dx\\
%&= x f(x) + \int \left( \frac{1}{x-1} - \frac{1}{x} \right) dx\\
%&= - x \log (1-x^{-1}) + \underbrace{\log(x-1) - \log x}_{\log(1 - x^{-1})} + C\\
%&= (1 - x) \log (1-x^{-1}) + C
%\end{align*}

%Note that $(1 - x) \log (1-x^{-1})$ is of the form $\infty \cdot 0$ when $x \rightarrow \infty$. So we evaluate the limit with L' Hospitale rule:
%\begin{align*}
%\lim_{x \rightarrow \infty} (1 - x) \log (1-x^{-1}) &= \lim_{x \rightarrow \infty} \frac{\log (1-x^{-1})}{(1-x)^{-1}}\\
%&= \lim_{x \rightarrow \infty} \frac{(1-x^{-1})^{-1} x^{-2}}{(1-x)^{-2}}\\
%&= \lim_{x \rightarrow \infty} (1-x^{-1})\\
%&= 1
%\end{align*}

%For a direct proof that $1 - (1 - y^{-1}) \log(1 - y) < y$ if $0 < y < 1$, note that
%\begin{align*}
%1 - (1 - y^{-1}) \log(1 - y) < y &\iff 1 - y < (1 - y^{-1}) \log(1 - y)\\
%&\iff y < -\log(1 - y)\\
%&\iff e^y (1 - y) < 1\\
%&\iff 1 - y < e^{-y}\\
%&\iff \frac{1 - e^{-y}}{y} < 1
%\end{align*}
%which is true by Mean Value theorem: Recognize $1 = e^0$ so the left hand side is $e^\xi$ for some $-y < \xi < 0$ whence $e^\xi < 1$.
\end{proof}

\unless\ifdefined\IsMainDocument
\end{document}
\fi
