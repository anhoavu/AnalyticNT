\unless\ifdefined\IsMainDocument
\documentclass[12pt]{article}
\usepackage{amsmath,amsthm}
\newcommand{\V}{\mathcal{V}}
\begin{document}
\fi

\textbf{Problem 3.20}: If $F \in \V$ and $F(1) = 1$, then there exists a function $\varphi \in \V$ with $\varphi(1) = 0$ such that
$$dF = \exp d\varphi := \delta_1 + d\varphi + (d\varphi * d\varphi)/2! + \cdots$$
and $L \, dF = dF * L\, d\varphi$. Show that
$$\delta_1 + dt = \exp \left\{ \frac{1 - t^{-1}}{\log t} dt\right\}.$$

\begin{proof}
Note that $dF = \exp d\varphi$ implies
$$L \, dF = L \left( \delta_1 + d\varphi + \cdots \right) = L d\varphi + L(d\varphi * d\varphi / 2!) + \cdots = L d\varphi * dF.$$
But then we can determined uniquely
$$d\varphi = L^{-1} (L \, dF * (dF)^{*-1})$$
and so $\varphi$ is determined up to a constant. Similar to Theorem 2.20, we expect the equivalence of the three equations:
$$dF = \exp d\varphi \iff L \, dF = dF * L\, d\varphi \iff d\varphi = \log dF$$
with the last one being useful to derive $d\varphi$ from $dF$. (The prior formula for $d\varphi$ requires finding $dF^{*-1}$ which is not exactly easy in general.)

Thus, we only need to verify the equation $L \, dF = dF * L\, d\varphi$ holds for $dF = \delta_1 + dt$ and $d\varphi = \frac{1 - t^{-1}}{\log t} dt = L^{-1} (1 - t^{-1}) dt$:
\begin{align*}
L \, dF &= L (\delta_1 + dt)\\
&= L \delta_1 + L \, dt\\
&= L \, dt\\
dF * L\, d\varphi &= (\delta_1 + dt) * (1 - t^{-1}) dt\\
&= (\delta_1 + dt) * (dt - \delta_1 + \delta_1 - t^{-1} dt)\\
&= (\delta_1 + dt) * (dt - \delta_1) + (\delta_1 + dt) * (\delta_1 - t^{-1} dt)\\
&= dt * dt - \delta_1 * \delta_1 + \delta_1\\
&= dt * dt\\
&= L \, dt
\end{align*}
The fact that $L \delta_1 = 0$ comes from applying $L$ to both sides of $\delta_1 = \delta_1 * \delta_1$.

For any $a > 1$, we have
\begin{align*}
\int_a^x L^{-1} dF &= \int_a^x \log^{-1} t \, dF\\
&= \left. \frac{F}{\log t}\right|_a^x - \int_a^x F \, d(\log^{-1} t)\\
&= \left. \frac{F}{\log t}\right|_a^x - \int_a^x F \cdot \left(\frac{-1}{\log^2 t} t^{-1} dt \right)\\
&= \left. \frac{F}{\log t}\right|_a^x + \int_a^x \; \frac{F}{t \log^2 t} dt
\end{align*}
Unfortunately, we cannot simplify $L^{-1} dF$ to a nice formula.
\end{proof}

\textbf{Problem 3.21}: Show that if $F \in \V$ and $F(x) = \int_1^x dF * t^{-1} dt$ for all $x \geq 1$, then $F = 0$. 

\begin{proof}
First, we clearly have $F(1) = 0$ and so we find
$$F(x) = F(x) - F(1) = \int_1^x \, dF$$
which implies
$$\int_1^x \, dF = \int_1^x dF * t^{-1} dt$$
or equivalently
$$\int_1^x dF * (\delta_1 - t^{-1} dt) = 0$$
for all $x \geq 1$. But that means
$$dF * (\delta_1 - t^{-1} dt) = 0$$
and so
$$dF = 0$$
by convoluting both sides with the inverse $\delta_1 + dt$ of $\delta_1 - t^{-1} dt$. So $F$ is a constant for $x \geq 1$ and that constant has been determined by $F(1) = 0$.
\end{proof}

\unless\ifdefined\IsMainDocument
\end{document}
\fi
