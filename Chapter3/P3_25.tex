\unless\ifdefined\IsMainDocument
\documentclass[12pt]{article}
\usepackage{amsmath,amsthm,amssymb}

\begin{document}
\fi

\textbf{Problem 3.25}: Suppose that $a$ is a fixed positive integer.
\begin{enumerate}
\item Show that for $\Re s > 1$
$$\sum_{(n, a) = 1} \frac{n \mu^2(n)}{\varphi(n) n^s} = \sum_{\ell = 1}^\infty \frac{1}{\ell^s} \sum_{m = 1}^\infty \frac{c_m}{m^s}$$
where
$$\sum_{m = 1}^\infty \frac{c_m}{m^s} = \prod_{p | a} (1 - p^{-s}) \prod_{p \nmid a} \left( 1 + \frac{1}{(p-1)p^s} - \frac{p}{(p-1)p^{2s}}\right)$$
and the series $\sum c_m m^{-s}$ converges absolutely for $\Re s > 1/2$.

\item Show that
$$\sum_{n \leq x, \; (n, a) = 1} \frac{n \mu^2(n)}{\phi(n)} = \frac{\varphi(a)}{a} x + O(x^{1/2 + \epsilon}).$$

\item Show that
$$\sum_{n \leq x, \; (n, a) = 1} \frac{\mu^2(n)}{\varphi(n)} = \frac{\varphi(a)}{a} \log x + K(a) + O(x^{-1/2 + \epsilon}).$$
where $K(a)$ is a number depending on $a$.
\end{enumerate}

\begin{proof}
For an arithmetic function $h$, we will write $h^\Sigma$ its summatory function and $dh$ for $dh^\Sigma$.
Define the multiplicative functions
$$f(n) := \begin{cases}
\frac{n \mu^2(n)}{\varphi(n) n^s} &\text{if } (n, a) = 1,\\
0 &\text{if }(n, a) > 1
\end{cases}$$
and $g = f * \mu$ so $f = 1 * g$ and
$$g(p^k) = f(p^k) - f(p^{k-1}) = \begin{cases}
0 &\text{if } p | a, k > 1 \text{ or } p \nmid a, k > 2,\\
-1 &\text{if } p | a, k = 1,\\
\frac{1}{p-1} &\text{if } p \nmid a, k = 1,\\
-\frac{p}{p-1} &\text{if } p \nmid a, k = 2.
\end{cases}$$
Let $K$ be the support of $g$. It can be identified with the set of triples of positive integers $(d, \ell, m)$ such that
\begin{itemize}
\item $d, \ell, m$ square-free and pairwise relatively prime,
\item $d | a$,
\item $(\ell, a) = (m, a) = 1$.
\end{itemize}
under the identification $(d, \ell, m) \leftrightarrow d \ell m^2$.

\begin{enumerate}
\item Let $H_s(x) = x^s$ and we realize the product of $x^s$ with the LHS as
$$\sum_{n = 1}^{\infty} \left(\frac{x}{n}\right)^s f(n) = \int_{1^-}^x dH_s * df = \int_{1^-}^x dH_s * dN * dg = \int_{1^-}^x d \widehat{H}_s * dg$$
where
\begin{align*}
\widehat{H}_s^*(x) &= \int_{1^-}^x dH_s * dN = \int_{1^-}^x H_s(x/t) dN(t)\\
&= \sum_{n \leq x} (x/t)^s\\
&= x^s\left( \frac{x^{1-s}}{1-s} + \zeta(s) + O(x^{-s})\right)\\
&= \zeta(s) \; x^s + \underbrace{\frac{x}{1 - s} + O(1)}_{O(x)}.
\end{align*}

We have
\begin{align*}
g^\Sigma_v(x) &= |g|^\Sigma(x)\\
&= \sum_{(d, \ell, m) \in K_{\leq x}} |g(d) g(\ell) g(m^2)|\\
&= \sum_{(d, \ell, m) \in K_{\leq x}} \frac{1}{\varphi(\ell)} \frac{m}{\varphi(m)}\\
&= \sum_{d | a, d \in Q_1} \sum_{m \leq \sqrt{x/d}, m \in Q_a} \left(\sum_{\ell \leq x/dm^2, \ell \in Q_{am}} \frac{1}{\varphi(\ell)} \right) \frac{m}{\varphi(m)}
\end{align*}
where to simplify our notation, let
$$Q_a = \{m \text{ square free} : (m, a) = 1\}.$$

The sum over $d | a$ is a finite sum consisting of $2^{\omega(a)}$ terms with the maximal term is the one for $d = 1$. Therefore,
$$g^\Sigma_v(x) \leq 2^{\omega(a)} \sum_{m \leq \sqrt{x}, m \in Q_a} \left(\sum_{\ell \leq x/m^2, \ell \in Q_{am}} \frac{1}{\varphi(\ell)} \right) \frac{m}{\varphi(m)}$$
Using a simple inequality that
$$\varphi(m) \geq \sqrt{\frac m 2}$$
if $m$ is square-free (from $p - 1 \geq \sqrt{p}$ for all prime $p \geq 3$), we can bound
$$\sum_{\ell \leq x/m^2, \ell \in Q_{am}} \frac{1}{\varphi(\ell)} \leq \sqrt{2} \sum_{\ell \leq x/m^2} \ell^{-1/2} = O((x/m^2)^{1/2}) = O(x^{1/2} / m)$$
by Lemma 3.13 whence for some constant $C$ in the last $O$ notation,
$$g^\Sigma_v(x) \leq 2^{\omega(a)} \sum_{m \leq \sqrt{x}} C \frac{x^{1/2}}{m} \sqrt{2} m^{1/2} = x^{1/2} O(x^{1/4}) = O(x^{3/4})$$
by Lemma 3.13 again. Since $3/4 < 1$, we can now apply the stability theorem to get
$$x^s \cdot LHS = x^s \zeta(s) \left(\int_{1^-}^{\infty} t^{-s} dg\right) + O(x)$$
whence dividing both sides by $x^s$ and taking $x \rightarrow \infty$, we find the desired identity as
$$\int_{1^-}^{\infty} t^{-s} dg = \sum_{m = 1}^{\infty} \frac{c_m}{m^s} = \prod_{p | a} (...) \prod_{p \nmid a} (...)$$
using the computation for $g(p^k)$ at the beginning.

To show that $\sum c_m m^{-s}$ converges absolutely for $\Re s > 1/2$, we observe from the proof of the stability theorem (in particular, equation (3.19) in the book) that if $g^\Sigma_v(x) = O(x^\tau)$ then 
$$\sum_{m > x}^{\infty} \frac{|c_m|}{|m^s|} = \int_{x}^{\infty} |t^{-s}| \cdot d g^\Sigma_v = O(x^{\tau - \Re s})$$
converges to 0 as long as $\tau - \Re s < 0$ or equivalently, $\Re s > \tau$. To put it another way, suppose that $s$ is fixed with $\Re s = \frac12 + \epsilon$ for $\epsilon > 0$ and to show that $\sum c_m m^{-s}$ converges absolutely, we just have to show that $g^\Sigma_v(x) = O(x^\tau)$ for some $\tau < \frac12 + \epsilon$. This is tantamount to proving that
$$g^\Sigma_v(x) = O(x^{\frac12 + \epsilon})$$
for any $\epsilon > 0$.

To do this, recognize that the simple inequality $\varphi(m) \geq \sqrt{\frac m 2}$, which comes from $p - 1 \geq \sqrt{p}$ for all prime $p \geq 3$, can be improved rather easily: From
$$p - 1 = O(p^{1-\epsilon})$$
for any $0 < \epsilon < 1$, we have for all square-free $m$,
$$\varphi(m) \geq C_\epsilon^{-1} m^{1 - \epsilon}$$
for some constant $C_\epsilon$ depending on $\epsilon$ (to take care of the initial primes where $p - 1 < p^{1-\epsilon}$ since $p - 1 > p^{1-\epsilon}$ eventually). Just like we did previously, we then find
$$\sum_{\ell \leq x/m^2, \ell \in Q_{am}} \frac{1}{\varphi(\ell)} \leq C_\epsilon \sum_{\ell \leq x/m^2} \ell^{1 - \epsilon} = O((x/m^2)^{\epsilon}) = O(x^\epsilon/m^{2\epsilon})$$
by Lemma 3.13 whence if $C$ be the constant in the last $O$-notation,
$$g^\Sigma_v(x) \leq 2^{\omega(a)} \sum_{m \leq \sqrt{x}} C \frac{x^\epsilon}{m^{2\epsilon}} C_\epsilon m^\epsilon = O\left(x^{\epsilon} \sum_{m \leq \sqrt{x}} m^{-\epsilon}\right) = O(x^\epsilon (\sqrt x)^{1-\epsilon}) = O(x^{\frac 1 2 + \frac \epsilon 2}).$$
And we are done.

\item For this part, first note that
$$LHS = \int_{1^-}^{x} dN * dg$$
and we can apply the stability theorem right away with $N = x + O(1)$ and $g^\Sigma_v(x) = O(x^{\frac 12 + \epsilon})$ just proved to get
$$LHS = x \int_{1^-}^{\infty} t^{-1} dg + O(x^{\frac 12 + \epsilon}).$$

It remains to see that
$$\int_{1^-}^{\infty} t^{-1} dg = \frac{\varphi(a)}{a}.$$
Knowing that the series $\sum c_m m^{-s}$ converges absolutely for $\Re s > 1/2$, hence for $s = 1$, allows us to comnpute the integral using the Euler product:
$$\prod_{p|a}(1 - p^{-1}) \prod_{p \nmid a} \underbrace{\left( 1 + \frac{1}{(p-1)p} - \frac{p}{(p-1)p^2} \right)}_{1} = \prod_{p|a}(1 - p^{-1}) = \frac{\varphi(a)}{a}.$$

\item Observe that the LHS is exactly the LHS from part 1 if $s = 1$. So the same approach should work with appropriate modification to the estimation of $\widehat{H}_1$ that relies on $\Re s > 1$. Now $d H_1(t) = \delta_1 + dt$ and
\begin{align*}
\widehat{H}_1(x) &= \int_{1^-}^x (x/t) dN\\
&= \sum_{n \leq x} \frac{x}{t}\\
&= x( \log x + \gamma + O(x^{-1}))\\
&= x (\log x + \gamma) + O(1)
\end{align*}
by Lemma 3.13. So the stability theorem yields
$$x \cdot LHS = x (c_1 + c_0(\log x + \gamma)) + O(x^{1/2 + \epsilon})$$
where
$$c_\ell = \int_{1^-}^\infty (-\log t)^{-\ell} t^{-1} dg$$
In particular, $c_0 = \varphi(a)/a$ from the computation in part 2 and so
$$LHS = \frac{\varphi(a)}{a} \log x + \left\{ \frac{\varphi(a)}{a} \gamma + c_1 \right\} + O(x^{-1/2 + \epsilon}).$$
\end{enumerate}
\end{proof}

\unless\ifdefined\IsMainDocument
\end{document}
\fi
