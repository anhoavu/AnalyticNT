\let\IsMainDocument\relax

\documentclass[12pt]{amsbook}
\usepackage{amsmath,amsthm,amssymb,hyperref,cancel}
\usepackage[a4paper,margin=2cm]{geometry}

\title{Analytic Number Theory Study Notes}
\author{An Hoa Vu}

\DeclareMathOperator*{\Res}{Res}
\newcommand{\fnz}{\text{fnz}}

\newcommand{\Z}{\mathbb{Z}}
\newcommand{\R}{\mathbb{R}}
\newcommand{\C}{\mathbb{C}}

\newcommand{\IntPart}[1]{\left[ #1 \right]}
\newcommand{\Abs}[1]{\left| #1 \right|}

\renewcommand{\O}[1]{O\left( #1 \right)}

\begin{document}

\maketitle

\chapter{Introduction}

\unless\ifdefined\IsMainDocument
\documentclass{article}
\usepackage{amsmath,amsthm,amssymb}

\title{Chapter 1}
\author{An Hoa Vu}

\DeclareMathOperator*{\Res}{Res}
\newcommand{\Z}{\mathbb{Z}}

\begin{document}

\maketitle
\fi

\section{Fourier expansion of periodic function $x^2$}

Let $f(x) = x^2$ on $[-\pi, \pi]$ and extends to a periodic function of period $P = 2\pi$. Recall $f(x)$ has Fourier expansion
$$f(x) = \frac{a_0}{2} + \sum_{n = 1}^{\infty} a_n \, \cos nx + b_n \, \sin nx$$
where
\begin{align*}
a_n &= \frac{2}{P} \int_{-\pi}^{\pi} f(x) \; \cos nx \; dx\\
b_n &= \frac{2}{P} \int_{-\pi}^{\pi} f(x) \; \sin nx \; dx.
\end{align*}

In this case, we expect $b_n = 0$ for all $n$ since $f(x)$ is an even function. As for the $a_n$, we have
$$a_0 = \frac{1}{\pi} \int_{-\pi}^{\pi} x^2 dx = \frac{1}{2\pi} \left. \frac{x^3}{3} \right]_{-\pi}^{\pi} = \frac{1}{\pi} \frac{2\pi^3}{3} = \frac{2\pi^2}{3}$$
and for $n \geq 1$,
\begin{align*}
a_n &= \frac{1}{\pi} \int_{-\pi}^{\pi} x^2 \; \cos nx \; dx\\
&= \frac{1}{\pi} \left( \underbrace{\left. x^2 \frac{\sin nx}{n} \right]_{-\pi}^{\pi}}_{0} - \int_{-\pi}^{\pi} 2x  \frac{\sin nx}{n} \;  \; dx \right) \qquad \text{(int by parts)}\\
&= -\frac{2}{n \pi} \int_{-\pi}^{\pi} x  \sin nx \; dx\\
&=  -\frac{2}{n \pi} \left( \left. x \frac{-\cos nx}{n} \right]_{-\pi}^{\pi} - \underbrace{\int_{-\pi}^{\pi} \frac{-\cos nx}{n} \;  \; dx}_{0} \right) \qquad \text{(int by parts)}\\
&= -\frac{2}{n \pi} \left( \frac{- 2 \pi \cos n\pi}{n} \right)\\
&= \frac{4 (-1)^n}{n^2}.
\end{align*}
So
$$f(x) = \frac{\pi^2}{3} + 4 \sum_{n = 1}^{\infty} (-1)^n n^{-2} \, \cos nx.$$

\section{Generalization of Theorem 1.8}

While solving problem 3.7, it appears that we can generalize the theorem as follow: Let $f$ be any arithmetic function and $F(x) = \sum_{n \leq x} f(n)$ be its summatory function. Assume that the limit
$$\alpha = \lim_{x \rightarrow \infty} \frac{F(x)}{x}$$
exists. Then
$$\lim_{s \rightarrow 1^+} (s - 1) \sum_{n = 1}^{\infty} \frac{f(n)}{n^s} = \alpha.$$

\section{Set without density}

For every subset $X$ of $Z = \Z^+ = \{1, 2, ...\}$, the set of all positive integers, we define the function
$$\zeta_X(s) = \sum_{n \in X} n^{-s} \qquad \text{ for all } s > 1.$$

Let
$$A = \bigcup_{j = 0}^{\infty} \{n : 4^j \leq n < 2 \cdot 4^j\}$$
then we show that
$$\lim_{s \rightarrow 1+} (s - 1) \zeta_A(s) = \frac{1}{2}.$$

(Note that this union is disjoint for if $[4^j, 2\cdot 4^j) \cap [4^k, 2 \cdot 4^k) \not= \emptyset$ for some $j < k$ then $4^j < 4^k < 2 \cdot 4^j$ but that means $j < k < j + \log_4 2 = j + 1/2$ which is impossible as $j, k$ are integers.)

If $X$ is a subset of the complex numbers and $f$ is a function, recall that $f(X)$ denotes the image of $X$.
Thus, the notation $mX + b$ where $m, b$ are integers has obvious meanings, namely
$$m X + b = \{mn + b \;:\; n \in X\}.$$
Examples: $2Z$ and $2Z - 1$ are the subsets of positive even (odd, respectively) integers.

The idea is to try squeezing the expression $(s - 1) \zeta_A(s)$ between two expressions that have the same limit given.
To do that, observe disjoint unions
\begin{align*}
Z &= \bigcup_{j = 0}^{\infty} \{n : 4^j \leq n < 4^{j+1}\}\\
2A &= \bigcup_{j = 0}^{\infty} \{n : 2 \cdot 4^j \leq n < 4^{j+1}, n \text{ even}\}\\
2A + 1 &= \bigcup_{j = 0}^{\infty} \{n : 2 \cdot 4^j \leq n < 4^{j+1}, n \text{ odd}\}
\end{align*}
and so we have a partition
$$Z = A \cup (2A) \cup (2A + 1).$$

\begin{itemize}
\item \emph{Lower bound}: It follows from the partition that
$$\zeta(s) = \zeta_Z(s) = \zeta_A(s) + \zeta_{2A}(s) + \zeta_{2A+1}(s).$$
By definition,
$$\zeta_{2A}(s) = 2^{-s} \zeta_A(s)$$
and
$$\zeta_{2A+1}(s) < \zeta_{2A}(s).$$
We thus get the inequality
\begin{align*}
\zeta(s) &< \zeta_A(s) + 2 \zeta_{2A}(s)\\
&< \zeta_A(s) + 2 (2^{-s} \zeta_A(s))\\
&< (1 + 2^{1-s}) \zeta_A(s)
\end{align*}
or
$$\frac{\zeta(s)}{1 + 2^{1-s}} < \zeta_A(s).$$

\item \emph{Upper bound}: On the other hand, we have
$$Z - 1 = (A - 1) \cup (2A - 1) \cup (2A).$$
Note that $Z - 1 = Z \cup \{0\}$ so we cannot take $\zeta_{Z - 1}$ straight away.
Discarding 0 from both sides, we get
\begin{align*}
Z &= [(A - 1) \backslash \{0\}] \cup (2A - 1) \cup (2A)\\
&= [(A \backslash \{1\}) - 1] \cup (2A - 1) \cup (2A)
\end{align*}
since $1 \in A$. Now
$$\zeta(s) = \zeta_{(A \backslash \{1\}) - 1}(s) + \zeta_{2A - 1}(s) + \zeta_{2A}(s)$$
and a similar argument applied: The inequalities
$$\zeta_{2A - 1}(s) > \zeta_{2A}(s)$$
and
$$\zeta_{(A \backslash \{1\}) - 1}(s) > \zeta_{A \backslash \{1\}}(s) = \zeta_A(s) - 1$$
lead to
\begin{align*}
\zeta(s) &> \zeta_A(s) - 1 + 2 \zeta_{2A}(s)\\
&> (1 + 2^{1-s}) \zeta_A(s) - 1
\end{align*}
or
$$\zeta_A(s) < \frac{\zeta(s) + 1}{1 + 2^{1-s}}.$$
\end{itemize}

Combining the two inequalities obtained
$$\frac{\zeta(s)}{1 + 2^{1-s}} < \zeta_A(s) < \frac{\zeta(s) + 1}{1 + 2^{1-s}}$$
and multiplying by $s - 1$, we get the desired squeeze
$$\frac{(s - 1) \zeta(s)}{1 + 2^{1-s}} < (s - 1) \zeta_A(s) < (s - 1) \frac{\zeta(s) + 1}{1 + 2^{1-s}}.$$
Using the fact that $\lim_{s \rightarrow 1+} (s - 1) \zeta(s) = 1$ and $\lim_{s \rightarrow 1+} (1 + 2^{1-s}) = 2$, we can see that the bounds both tends to $\frac{1}{2}$.

\section{Problems}

\textbf{Problem 1.1}: Show that
$$\sum_{p \text{ prime}} p^{-2} < \frac{1}{2}.$$

\begin{proof}
Let $p_n$ denotes the $n$-th prime number where $n \geq 1$ and
$$S_r = \sum_{n = 1}^{r} p_n^{-2}$$
where $r$ is either a positive integer or $r = \infty$.

We have an inequality
$$\frac{1}{p_n^2} < \frac{1}{p_{n-1}} - \frac{1}{p_n}$$
for all $n \geq 2$. That is because the right hand side equals
$$\frac{p_n - p_{n-1}}{p_n p_{n-1}}$$
and obviously $p_n - p_{n-1} \geq 1$ while $p_n^2 < p_n p_{n-1}$.

Thus, for every $r \geq 1$, we have
\begin{align*}
S_\infty &< \sum_{n=1}^{r} \frac{1}{p_n^2} + \sum_{n = r+1}^{\infty} \frac{1}{p_{n-1}} - \frac{1}{p_n} \\
&< S_r + \frac{1}{p_r}
\end{align*}
as the second series telescopes.

Observe that
$$\lim_{r \rightarrow \infty} S_r + \frac{1}{p_r} = S_\infty$$
because $\frac{1}{p_r} \rightarrow 0$.
So if $S_\infty < \frac{1}{2}$ then we must have $S_r + \frac{1}{p_r} < \frac{1}{2}$ for some $r$.
Experimenting with a calculator and we find that for $r = 7$,
\begin{align*}
S_7 + \frac{1}{p_7} &= \frac{1}{2^2} + \frac{1}{3^2} + \frac{1}{5^2} + \frac{1}{7^2} + \frac{1}{11^2} + \frac{1}{13^2} + \frac{1}{17^2} + \frac{1}{17}\\
&= 0.4979...
\end{align*}

\textbf{Remark}: Our argument showed that the sequence $S_r + p_r^{-1}$ is a decreasing sequence while $S_r$ is an increasing one. As a result of the bounds
$$S_r < S_\infty < S_r + \frac{1}{p_r},$$
we can get the limit as precise as we want (within $\epsilon$) via computing $S_r$ for $\frac{1}{p_r} < \epsilon$.

\textbf{Conjecture}: Observe that
$$\frac{1}{2} = \sum_{n=2}^{\infty} 2^{-n}$$
so I am led to conjecture if we actually have
$$p_n^{-2} \leq 2^{-(n+1)} \qquad \text{ or equivalently, } \qquad p_n \geq 2^{(n+1)/2}$$
for all $n$?

A heuristic is to try the Prime Number Theorem i.e. $\pi(x) \approx \frac{x}{\log x}$ where for any real number $x \geq 1$, $\pi(x)$ counts the number of primes in $[1, x]$. Putting it another way, $p_n$ is the smallest value of $x$ such that $\pi(x) = n$. By the PNT, we are interested in the least $x \geq n$ such that
$$\frac{x}{\log x} \geq n \qquad \text{ or equivalently, } \qquad x \geq n \log x.$$

The function $f(x) = x - n \log x$ has $f'(x) = 1 - \frac{n}{x} \geq 0$ whenever $x \geq n$ so $f(x)$ is an increasing function on $[n, +\infty]$. Observe that
\begin{align*}
f(2^{(n+1)/2}) &= 2^{(n+1)/2} - n \log(2^{(n+1)/2})\\
&= 2^{(n+1)/2} - n \frac{n + 1}{2} \log(2)\\
&\rightarrow +\infty
\end{align*}
so the conjecture is most likely false.

It is nevertheless an interesting question: What is an effective lower bound for primes? (There is an obvious bound of course: $p_n \geq 2n - 1$.)
\end{proof}

\textbf{Problem 1.2}: Show that
$$\prod_{j = n + 1}^{\infty} (1 - j^{-2}) = 1 - \frac{1}{n+1}$$
then give an upper bound for $\epsilon$ such that
$$\prod_p (1 - p^{-2}) = (1 - \epsilon) \prod_{p < 100} (1 - p^{-2}).$$

\begin{proof}
One has
\begin{align*}
\prod_{j = n + 1}^{\infty} (1 - j^{-2}) &= \prod_{j = n + 1}^{\infty} \frac{j^2 - 1}{j^2}\\
&= \prod_{j = n + 1}^{\infty} \frac{(j - 1)(j + 1)}{j^2}\\
&= \prod_{j = n + 1}^{\infty} \frac{j - 1}{j} \frac{j + 1}{j}
\end{align*}
which is a telescoping product since the next term in the product is
$$\frac{j}{j + 1} \frac{j+2}{j+1}$$
whose first factor cancels with the second factor of the previous term. As a result, only the first factor of the first term i.e.
$$\frac{n}{n + 1}$$
remains.

For the second part, observe
$$\prod_{p \geq 100} (1 - p^{-2}) = 1 - \epsilon$$
so we want to find lower bound for the left hand side. Evidently,
$$\prod_{p \geq 100} (1 - p^{-2}) \geq \prod_{j = p_0}^{\infty} (1 - j^{-2}) = 1 - \frac{1}{p_0}$$
where $p_0$ is the first prime $\geq 100$ because the RHS is a product of more numbers all of which are less than 1. So $\epsilon \leq \frac{1}{p_0}$.
\end{proof}

\textbf{Problem 1.3}: Prove $\zeta(2) = \pi^2/6$ by integrating the function $z^{-2} \cot \pi z$ along the perimeter of a rectangle with vertices $\pm (N + 1/2) \pm N i$, where $N$ is an integer allowed to tend toward $\infty$.

\begin{proof}
Let $R_N$ denotes the perimeter of the rectangle.

Recall residue theorem
$$\int_{R_N} f(z) dz = 2 \pi i \sum_{\text{poles } a} I(R_N, a) \cdot \Res_{z = a} f(z)$$
where $I(R_N, a)$ is the index (winding number) of $R_N$ around $a$ and $\Res_{z = a} f(z)$ is the residue of $f$ at $z = a$ i.e. the coefficient of $(z-a)^{-1}$ in the Laurent series for $f(z)$ at $z = a$. % (Follow from homology.)

The function $\sin \pi z$ has simple zeros at all integers since
$$\sin'(\pi k) = \pi \cos(\pi k) = (-1)^k \pi \not= 0.$$
So the function
$$f(z) = z^{-2} \cot \pi z = \frac{\cos \pi z}{z^2 \sin \pi z}$$
has simple poles at the non-zero integers and has a pole of order 3 at $z = 0$.

Therefore, for all integers $k \not= 0$, one finds
\begin{align*}
\Res_{z = k} \frac{\cos \pi z}{z^2 \sin \pi z} &= \lim_{z \rightarrow k} \; (z - k) \frac{\cos \pi z}{z^2 \sin \pi z}\\
&= \frac{\cos \pi k}{k^2} \cdot \lim_{z \rightarrow k} \; \frac{z - k}{\sin \pi z} & k \not= 0\\
&= \frac{(-1)^k}{k^2} \cdot \lim_{z \rightarrow k} \; \frac{z - k}{(-1)^k \sin \pi (z - k)} & \text{as } \sin(t \pm \pi) = - \sin t\\
&= \frac{1}{\pi k^2} &\text{from } \lim_{t \rightarrow 0} \frac{\sin t}{t} = 1
\end{align*}

It remains to assess
$$\Res_{z = 0} \frac{\cos \pi z}{z^2 \sin \pi z} = \Res_{z = 0} \frac{1}{\pi z^3} \left(\frac{\pi z \cos \pi z}{\sin \pi z}\right)$$
where we need to find the coefficient of $z^2$ of the holomorphic function $\frac{\pi z \cos \pi z}{\sin \pi z}$.

Recall
\begin{align*}
\cos z &= \sum_{n = 0}^{\infty} \frac{(-1)^n}{(2n)!} z^{2n} = 1 - \frac{z^2}{2!} + O(z^4)\\
\frac{\sin z}{z} &= \sum_{n = 0}^{\infty} \frac{(-1)^n}{(2n + 1)!} z^{2n} = 1 - \frac{z^2}{3!} + O(z^4)
\end{align*}
and so we find
$$\frac{z}{\sin z} = 1 + \frac{z^2}{3!} + O(z^4)$$
and
\begin{align*}
\frac{z \cos z}{\sin z} &= \left( 1 - \frac{z^2}{2!} + O(z^4) \right) \left( 1 + \frac{z^2}{3!} + O(z^4) \right)\\
&= 1 - \frac{z^2}{2} + \frac{z^2}{6} + O(z^4)\\
&= 1 - \frac{z^2}{3} + O(z^4)
\end{align*}
so
$$\frac{\pi z \cos \pi z}{\sin \pi z} = 1 - \frac{\pi^2 z^2}{3} + O(z^4)$$
and we conclude
$$\Res_{z = 0} \frac{\cos \pi z}{z^2 \sin \pi z} = \frac{-\pi^2/3}{\pi} = - \frac{\pi}{3}.$$

Going back to the residue theorem, we obtain
\begin{align*}
\int_{R_N} f(z) dz &= 2 \pi i \left(\Res_{z = 0} f(z) + \sum_{k = 1}^{N} (\Res_{z = k} f(z) + \Res_{z = -k} f(z)) \right)\\
&= 2 \pi i \left(-\frac{\pi}{3} + \sum_{k = 1}^{N} \frac{2}{\pi k^2} \right)
\end{align*}
and letting $N \rightarrow \infty$,
$$0 = 2 \pi i \left(-\frac{\pi}{3} + \sum_{k = 1}^{\infty} \frac{2}{\pi k^2} \right)$$
whence $\zeta(2) = \pi^2/6$ follows.
\end{proof}

\textbf{Problem 1.6}: Show that if $A$ is a set of positive integers for which $\sum_{n \in A} \frac{1}{n}$ converges then $A$ has density 0.

\begin{proof}
We follow the first few lines of computation in the proof of Theorem 8.1 (Dirichlet-Dedekind) that is valid for $s = 1$ as well:
\begin{align*}
\sum_{n \in A \cap [1, N]} \frac{1}{n} &= \sum_{n = 1}^{N} \frac{A(n) - A(n-1)}{n}\\
&= \sum_{n = 1}^{N - 1} A(n) \left(\frac{1}{n} - \frac{1}{n+1}\right) + \frac{A(N)}{N}\\
&= \sum_{n = 1}^{N - 1} \frac{A(n)}{n(n+1)} + \frac{A(N)}{N}
\end{align*}

By assumption, the LHS converges as $N \rightarrow \infty$ so to prove that $\frac{A(N)}{N}$ converges, we only have to show that
$$\sum_{n = 1}^{N - 1} \frac{A(n)}{n(n+1)}$$
does which is evident because it is an increasing sequence (each term in the sum is non-negative) and is bounded above by the limit of the LHS:
\begin{align*}
\sum_{n = 1}^{N - 1} \frac{A(n)}{n(n+1)} &\leq \sum_{n \in A \cap [1, N]} \frac{1}{n} &\text{since } \frac{A(N)}{N} \geq 0\\
&\leq \sum_{n \in A} \frac{1}{n}
\end{align*}

So we established convergence of $\frac{A(x)}{x}$ i.e. that $A$ has a density. The density can then be determined to be zero from Theorem 8.1. In particular, the density is given by
$$\lim_{s \rightarrow 1+} (s-1) \sum_{n \in A} n^{-s}$$
and notice that for all $s > 1$, we can squeeze
$$0 \leq (s-1) \sum_{n \in A} n^{-s} \leq (s - 1) \sum_{n \in A} n^{-1}$$
and observe that $\lim_{s \rightarrow 1+} (s - 1) \sum_{n \in A} n^{-1} = 0$ follows the given convergence.
\end{proof}

\unless\ifdefined\IsMainDocument
\end{document}
\fi


\chapter{Calculus of Arithmetic Functions}

\newcommand{\A}{\mathcal{A}}
\newcommand{\M}{\mathcal{M}}

\unless\ifdefined\IsMainDocument
\documentclass{article}
\usepackage{amsmath,amsthm}

\title{Chapter 2}
\author{An Hoa Vu}

\newcommand{\fnz}{\text{fnz}}
\newcommand{\A}{\mathcal{A}}
\newcommand{\M}{\mathcal{M}}

\begin{document}

\maketitle
\fi

\section{Problems}

\textbf{Problem 2.1}: Find all integers $n > 1$ such that $\prod_{d | n} d = n^2$.

\begin{proof}
We observe first that such $n$ cannot have more than 3 prime divisors. Otherwise, suppose that $P, Q, R$ are distinct maximal prime powers dividing $n$. Then the left hand side is divisible by $P Q R (PQ) (QR) (RP) (PQR) = P^3 Q^3 R^3$ whereas the maximal prime power for those primes is $P^2 Q^2 R^2$ on the right hand side. Now

\begin{itemize}
\item If $n$ has only 1 prime divisor i.e. $n = p^k$ is a prime power with $k > 0$ then the left hand side is $1 (p) (p^2) ... (p^k) = p^{k(k+1)/2}$ and
\begin{align*}
& p^{k(k+1)/2} = n^2\\
\iff & \frac{k(k+1)}{2} = 2k\\
\iff & k(k+1) = 4k\\
\iff & k(k - 3) = 0\\
\iff & k = 0 \text{ or } k = 3
\end{align*}
so $n = p^3$.

\item For the remaining case where $n = p^k q^r$ has 2 distinct prime divisor ($r, k > 0$), we find similarly that the left hand side is
\begin{align*}
\prod_{i = 0}^{k} \prod_{j = 0}^{r} p^i q^j
&= \prod_{i = 0}^{k} p^{i (r+1)} \prod_{j = 0}^{r}  q^j\\
&= \prod_{i = 0}^{k} p^{i (r+1)} q^{r(r+1)/2}\\
&= q^{(k+1) r(r+1)/2} \prod_{i = 0}^{k} p^{i (r+1)}\\
&= q^{kr(r+1)/2} p^{k(k+1)(r+1)/2}
\end{align*}
and that is equal to $n^2$ if and only if $\frac{(k+1)r(r+1)}{2} = 2r$ and $\frac{k(k+1)(r+1)}{2} = 2k$ or simply $(r + 1)(k + 1) = 4$ which only happens when $r = k = 1$.
\end{itemize}

So we conclude that all such $n$ are either of the form $p^3$ or $pq$.
\end{proof}

\textbf{Problem 2.2}: Find expressions for $(f*g)(n)$ when $n = p^k$, $n = pp'$ or $n = p p' p''$.

\begin{proof}
$$(f*g)(p^k) = \prod_{i=1}^{k} f(p^i) g(p^{k - i})$$

$$(f*g)(p p') = f(1) g(n) + f(p) g(p') + f(p') g(p) + f(n) g(1)$$

\begin{align*}
(f*g)(p p' p'') = f(1) g(n) &+ f(p) g(p' p'') + f(p') g(p p'') + f(p'') g(p p')\\
&+  f(p' p'') g(p)  + f(p p'') g(p')  +  f(p p') g(p'') + f(n) g(1)
\end{align*}
\end{proof}

\textbf{Problem 2.3}: Show that $\fnz(f * g) = \fnz(f) \fnz(g)$.

\begin{proof}

The case when $f = 0$ or $g = 0$ is obvious. So let assume $f \not= 0$ and $g \not= 0$ so $m = \fnz(f)$ and $n = \fnz(g)$ are positive integers. We want to show that
\begin{itemize}
\item $(f * g)(mn) \not= 0$.

One has
$$(f * g)(mn) = \sum_{ij = mn} f(i) g(j) = f(m) g(n)$$
for other term where $i \not= m$ vanishes: obviously if $i < m$ and when $i > m$, $j < n$ so $g(j) = 0$ instead.

\item $(f * g)(x) = 0$ for all $x < mn$.

Again, from
$$(f * g)(x) = \sum_{ij = x} f(i) g(j)$$
we see that if $x < mn$ then $ij < mn$ implies either $i < m$ or $j < n$.

\end{itemize}
\end{proof}

\textbf{Problem 2.4}: Find all solutions to the convolution equations $f * f = e$ and $f * f = f$.

\begin{proof}
For the first equation
$$(f * f)(1) = f(1) f(1) = e(1) = 1$$
so we must have $f(1) = \pm 1$. It follows by induction that $f(n) = 0$ for all $n > 1$:
\begin{itemize}
\item If $n$ is prime then
$$(f * f)(n) = 2 f(1) f(n)$$
and
$$e(n) = 0$$
so $f(n) = 0$. In particular, the base case $n = 2$ is true.
\item For induction, suppose $f(x) = 0$ for all $1 < x < n$. One has
$$(f * f)(n) = 2 f(1) f(n) + \sum_{n = i j, 1 < i, j < n} f(i) f(j) = 2 f(1) f(n)$$
since the sum is zero by induction hypothesis. Therefore, $f(n) = 0$.
\end{itemize}
So the only solutions to the equation $f * f = e$ is $\pm e$.

For the second equation, we again have
$$(f * f)(1) = f(1) f(1)$$
so $f(1) = 0$ or $f(1) = 1$.

\begin{itemize}
\item If $f(1) = 0$ then we again find $f(n) = 0$ for all $n$ primes and $f(n) = 0$ for all $n$ by induction.

\item If $f(1) = 1$ then we have $f = e$ by a similar argument. For example, if $n$ is prime then $2 f(1) f(n) = f(n)$ implies $f(n) = 0$.
\end{itemize}

So the only solutions to the equation $f * f = f$ is $f = 0$ or $f = e$.
\end{proof}

\textbf{Problem 2.6}: Express $\omega$ and $\Omega$ as $1 * f$ and $1 * F$ for suitable functions $f$ and $F$.

\begin{proof}
Recall
$$\omega(n) = \sum_{p | n} 1$$
counts the number of prime divisors of $n$ and
$$(1 * f)(n) = \sum_{d | n} f(d)$$
so one choice for $f$ would be
$$f(d) = \begin{cases}
1 &\text{if } d \text{ is prime},\\
0 &\text{otherwise}.
\end{cases}$$
In other words, $f$ is the prime characteristic function.

On the other hands, $\Omega(n)$ counts the number of prime divisors of $n$ with multiplicities and could be expressed as
$$\Omega(n) = \sum_{p^k | n, k > 0} 1$$
and a choice for $F$ would be
$$F(d) = \begin{cases}
1 &\text{if } d \text{ is a prime power},\\
0 &\text{otherwise}.
\end{cases}$$
\end{proof}

\textbf{Problem 2.8}: Let $f, g \in \A$, $f, g \not= 0$ and $(Lf) * g = f * (Lg)$. Show that $f = cg$ for some constant $c$.

\begin{proof}
It suffices to show that $\fnz(f) = \fnz(g)$ under the given assumption. For then,
\begin{itemize}
\item We deduce the contrapositive that if $F$ and $G$ be such that $(LF) * G = F * (LG)$ but $\fnz(F) \not= \fnz(G)$ then either $F = 0$ or $G = 0$.
\item Assuming we proved that $\fnz(f) = \fnz(g) = n$. To establish the conclusion of the problem, let $c = \frac{f(n)}{g(n)}$ then we find that $F = f - cg$ and $G = g$ satisfies the property
$$(L F) * G = F * (L G)$$
but $\fnz(F) \not= \fnz(G)$ so we must have $F = 0$ (by assumption $G \not= 0$) and so $f = cg$.
\end{itemize}

To show that $\fnz(f) = \fnz(g)$, let $m = \fnz(f)$ and $n = \fnz(g)$. We evaluate both sides of
$$(Lf) * g = f * (Lg)$$
at $mn$:
\begin{align*}
((Lf) * g) (mn) = \sum_{ij = mn} Lf(i) g(j) = Lf(m) g(n)\\
(f * (Lg)) (mn) = \sum_{ij = mn} f(i) Lg(j) = f(m) Lg(n)
\end{align*}
so we expect
$$\log m \; f(m) \; g(n) = f(m) \; \log n \; g(n)$$
whence $\log m = \log n$ given $f(m), g(n) \not= 0$ by assumption. This shows $m = n$, as desired.

As a remark, we recall that $Lf$ is like taking the derivative of $f$ so the equivalent problem in calculus would be if $f' g = f g'$ then $f = cg$. To solve it, we notice $f' g - f g' = 0$ is the numerator of $(f/g)'$ and so we have $(f/g)' = 0$ which implies $f/g$ is a constant. This problem could be solved the same way if $g$ is invertible with respect to convolution i.e. there exists $h$ such that $g * h = e$ and thus we can consider $L(f * h) = 0$.
\end{proof}

\textbf{Problem 2.12}: Let $\{f_i\}$ be a sequence of arithmetic functions, none of which is identically zero. Assume $f_1 * f_2 * ...$ converges to $f$. Then $f \not= 0$.

\begin{proof}
By our assumption that $f_1 * f_2 * ...$ converges, there can only be finitely many $i$ such that $f_i(1) = 0$. Let $I$ be the set of all such $i$ and
$$n = \prod_{i \in I} \fnz(f_i).$$

Observe that if $\nu > \max I$ then we have
$$n = \prod_{i = 1}^{\nu} \fnz(f_i).$$
for $\fnz(f_i) = 1$ if $i \not\in I$.

We claim that $f(n) \not= 0$. To do that, recall from the solution of earlier problem that
$$(F * G)\left(\fnz(F) \cdot \fnz(G)) = F(\fnz(F)\right) \cdot G(\fnz(G))$$
so by induction
\begin{align*}
(f_1 * f_2 * ... * f_\nu)(n) &= \prod_{i=1}^{\nu} f_i(\fnz(f_i))\\
&= \prod_{i \in I} f_i(\fnz(f_i)) \prod_{i=1; i \not\in I}^{\nu} f_i(1)
\end{align*}

It is now clear that: as $\nu \rightarrow \infty$, $\prod_{i \in I} f_i(\fnz(f_i)) \not= 0$ is fixed constant and whereas $\prod_{i=1; i \not\in I}^{\nu} f_i(1)$ are the partial products of $\prod_{\nu=1}^{\infty} f_\nu(1)$ which converges to a non-zero number by convergence assumption. Hence, $f(n) = \lim_{\nu \rightarrow \infty} (f_1 * f_2 * ... * f_\nu)(n) \not= 0$.
\end{proof}

\textbf{Problem 2.13}: Let $f \in \A_1$ satisfy $f * f = 1$. Show that $f$ is given by $\exp(\kappa/2)$.

\begin{proof}
Since $f \in \A_1$, we have $f = \exp \lambda$ for the unique $\lambda \in \A_0$ by Theorem 2.20. Then
$f * f = \exp(2 \lambda)$ and so $\exp(2 \lambda) = \exp(\kappa)$ due to uniqueness in Theorem 2.20 whence $\kappa = 2 \lambda$. Thus $f = \exp(\kappa/2)$.
\end{proof}

\textbf{Problem 2.14}: Let $f$ be as in preceeding problem. Then
$$f = (e + (1 - e))^{*1/2} = \sum_{j = 0}^{\infty} \binom{1/2}{j} (1 - e)^{*j}.$$

\begin{proof}
Let $g$ denotes the function on the right hand side. It is clear that $g \in \A_1$. We want to show that $g * g = 1$. To do that, we go back to the real analysis version of the problem i.e. when the arithmetic function $1$ is replaced by the function $x$ and $e$ is replaced by 1:
$$\sqrt{x} = (1 + (x - 1))^{1/2} = \sum_{j = 0}^{\infty} \binom{1/2}{j} (x - 1)^j$$
This can also be easily checked by explicitly finding the higher derivatives of $\sqrt{x}$:
$$\left. \frac{d^n}{dx^j} \sqrt{x} \right|_{x = 1} = \frac{1}{2} \left(-\frac{1}{2}\right) ... = j! \binom{1/2}{j}.$$

As a result, squaring both sides
$$x = 1 + (x - 1) = \sum_{n = 0}^{\infty} \sum_{i + j = n} \binom{1/2}{i} \binom{1/2}{j} (x - 1)^n$$
and we get the combinatorial identity
$$\sum_{i + j = n} \binom{1/2}{i} \binom{1/2}{j} = \begin{cases}1 &\text{if } n \leq 1, \\ 0 &\text{if } n > 1.\end{cases}$$

Now we can go back to the problem. One has
\begin{align*}
g * g &= \sum_{n = 0}^{\infty} \left[\sum_{i + j = n} \binom{1/2}{i} \binom{1/2}{j} \right] (1 - e)^{*n}\\
&= e + (1 - e) &\text{ thank to the identity}\\
&= 1
\end{align*}
\end{proof}

\textbf{Problem 2.15}: Show that an arithmetic function has at most two convolution square roots.

\begin{proof}
Suppose that $g$ and $h$ are convolution square roots of $f$. Then
$$(g - h) * (g + h) = g * g + g * h - h * g - h * h = 0$$
and since $A$ has no zero divisor, we must either have $g - h = 0$ or $g + h = 0$ i.e. either $h = g$ or $h = -g$. So there are at most two convolution square roots.
\end{proof}

\textbf{Problem 2.16}: Let $f \in \M$. Show that
$$f(n) f(m) = f((m, n)) f([m, n]).$$
Conversely, if $f \in \A_1$ and the above holds then $f \in \M$.

\begin{proof}
Recall that values of $f$ is determined on prime powers:
$$f\left(\prod p^v\right) = \prod f(p^v).$$

So let $m = \prod p^{k_p}$ and $n = \prod p^{r_p}$ be prime factorizations of $m$ and $n$. We then have prime factorization
\begin{align*}
(m, n) &= \prod p^{\min(k_p, r_p)}\\
[m, n] &= \prod p^{\max(k_p, r_p)}
\end{align*}
and so
\begin{align*}
LHS &= \prod f(p^{k_p}) f(p^{r_p}) \\
RHS &= \prod f(p^{\min(k_p, r_p)}) f(p^{\max(k_p, r_p)})
\end{align*}
and it is clear that
$$f(p^{k_p}) f(p^{r_p}) = f(p^{\min(k_p, r_p)}) f(p^{\max(k_p, r_p)})$$
for if $\min(k_p, r_p) = k_p$ then $\max(k_p, r_p) = r_p$ and vice versa; in other words,
$$\{\min(k_p, r_p), \max(k_p, r_p)\} = \{k_p, r_p\}$$
as multi-sets of two elements.

The converse is clear: In the special case $(m, n) = 1$, the right hand side of the equation reads
$$f((m, n)) f([m, n]) = f(1) f(mn) = f(mn)$$
so the equation says that $f(mn) = f(m) f(n)$.
\end{proof}

\textbf{Problem 2.17}: Let $f = \exp \lambda$ be multiplicative. Compute $\lambda(p)$ and $\lambda(p^2)$ in terms of values of $f$.

\begin{proof}
First, $\lambda(1) = 0$ and an argument similar to Lemma 2.11 shows that if $n > m$ then
$$(\lambda^{* n})(p^m) = 0.$$
By definition,
$$(\lambda^{* n})(p^m) = \sum_{m_1 + ... + m_n = m} \lambda(p^{m_1}) \cdots \lambda(p^{m_n})$$
and since $m < n$, at least one of the $m_i < 1$ so $m_i = 0$ and the factor $\lambda(p^{m_i}) = \lambda(1) = 0$. Note that this argument does not depends on further assumptions on $\lambda$ such as having support in prime powers.

So
\begin{align*}
(\exp \lambda) \, (p) &= (e + \lambda)(p)\\
&= \lambda(p)\\
(\exp \lambda) \, (p^2) &= \left( e + \lambda + \frac{\lambda * \lambda}{2!} \right) (p^2)\\
&= \lambda(p^2) + \frac{1}{2} \lambda(p)^2
\end{align*}

And we have
\begin{align*}
\lambda(p) &= f(p)\\
\lambda(p^2) &= f(p^2) - \frac{1}{2} f(p)^2
\end{align*}

As an alternative, we have
$$\lambda = \log f = \sum_{j=1}^{\infty} (-1)^{j-1} \frac{(f - e)^{*j}}{j}$$
and observe that $(f - e)(1) = 0$ so when evaluating $(f - e)^{*j}$ will vanish on prime powers less than $p^j$ by our earlier argument. Thus,
\begin{align*}
\lambda(p) &= \frac{(f - e)}{1} (p)\\
&= f(p) - e(p)\\
&= f(p)\\
\lambda(p^2) &= \left( \frac{(f - e)}{1} - \frac{(f - e)^{*2}}{2} \right) (p^2)\\
&= f(p^2) - \frac{1}{2} [(f - e)(p)]^2\\
&= f(p^2) - \frac{1}{2} f(p)^2
\end{align*}
\end{proof}

\textbf{Problem 2.18}: Show directly from the definition of $*$ that convolution of two multiplicative functions is multiplicative.

\begin{proof}
Let $f$ and $g$ be multiplicative and $m, n$ be an arbitrary pair of relatively prime positive integers. Let $h = f * g$. We want to show that
$$h(mn) = h(m) h(n)$$
which by definition is the same as showing
$$\sum_{d | mn} f(d) g(mn/d) = \left( \sum_{d_1 | m} f(d_1) \, g(m/d_1) \right) \left( \sum_{d_2 | n} f(d_2) \, g(n/d_2) \right).$$

One has
\begin{align*}
RHS &= \sum_{d_1 | m} \sum_{d_2 | n} f(d_1) \, g(m/d_1) \, f(d_2) \, g(n/d_2)\\
&= \sum_{d_1 | m} \sum_{d_2 | n} f(d_1) \, f(d_2) \, g(m/d_1) \, g(n/d_2)\\
&= \sum_{d_1 | m} \sum_{d_2 | n} f(d_1 d_2) \, g(mn/(d_1d_2))
\end{align*}
by multiplicativity of $f$ and $g$ and the fact that since $(m, n) = 1$ then $(d_1, d_2) = (m/d_1, n/d_2) = 1$. The last sum matches term-by-term with the one on the LHS where each $d | mn$ corresponds to $d_1 = (d,m)$ and $d_2 = (d,n)$; and conversely each pair $d_1 | m, d_2 | n$ yields $d = d_1 d_2 | mn$.
\end{proof}


\textbf{Problem 2.20}: Let $f \in \A_1$, $f * f = 1$. Show that $f$ is multiplicative.

\begin{proof}
From previous problem, $f = \exp(\kappa/2)$ and this follows immediately from Theorem 2.27.
\end{proof}

\textbf{Problem 2.22}: Show that a multiplicative function $f = \exp \lambda$ is completely multiplicative if and only if $\lambda(p^\alpha) = \lambda(p)^\alpha / \alpha$ for all $p$.

\begin{proof}
Recall that a multiplicative function $f$ is completely multiplicative if we further have
$$f(p^v) = f(p)^v$$
for all prime $p$ and all $v \geq 1$.

Let $g = f^{*-1}$ be convolution inverse of $f$. We have $g(1) = 1$ and recursively for every $v \geq 1$,
$$g(p^v) = -\sum_{i = 1}^{v} f(p^i) g(p^{v-i})$$
by Theorem 2.7 (or straight from equation $f * g = e$). Checking the first couple of $v$:
\begin{itemize}
\item $v = 1$:
$$g(p) = -f(p)$$

\item $v = 2$:
\begin{align*}
g(p^2) &= -f(p)g(p) - f(p^2) g(1)\\
&= f(p)^2 - f(p^2)\\
&= 0
\end{align*}
from the assumption that $f$ is completely multiplicative.

\item $v = 3$:
\begin{align*}
g(p^3) &= -f(p) \underbrace{g(p^2)}_{0} - f(p^2) g(p) - f(p^3) g(1)\\
&= f(p^2) f(p) - f(p^3)\\
&= 0
\end{align*}
again from the assumption that $f$ is completely multiplicative.
\end{itemize}
And this suggests that $g(p^v) = 0$ for all $v \geq 2$. Such a statement can be verified by induction: Suppose that it is true for all prime power $\leq p^v$.
\begin{align*}
g(p^{v+1}) &= -\sum_{i = 1}^{v+1} f(p^i) g(p^{v+1-i})\\
&= - f(p^v) g(p) - f(p^{v+1}) g(1)\\
&= 0
\end{align*}

Now back to the problem, theorem 2.20 shows that $f = \exp \lambda$ if and only if
$$L \lambda = L f * g.$$
Evaluating both sides at $p^v$:
\begin{align*}
LHS &= \log(p^v) \lambda(p^v)\\
&= v \log(p) \lambda(p^v)\\
RHS &= (L f * g) (p^v)\\
&= \sum_{i = 0}^{v} Lf(p^i) \; g(p^{v - i})\\
&= Lf(p^v) \; g(1) + Lf(p^{v-1}) \; g(p) &\text{as we showed } g(p^k) = 0 \text{ if } k \geq 2\\
&= \log(p^v) f(p^v) + \log(p^{v-1}) f(p^{v-1}) (-f(p)) &\text{ as showed } g(1) = 1, g(p) = -f(p)\\
&= v \; \log p \; f(p)^v - (v-1) \log p \; f(p^{v-1}) \; f(p)\\
&= \log p \; f(p)^v &\text{from $f$ completely multiplicative}
\end{align*}

We recall $\lambda(p) = f(p)$ from problem 2.17 and thus we find
$$v \log(p) \lambda(p^v) = \log(p) \lambda(p)^v$$
whence the conclusion follows.


For the converse, assuming $v \lambda(p^v) = \lambda(p)^v$ for all $v$. We want to reverse the above argument to show first that $g(p^v) = 0$ for all $v \geq 2$ and so $f(p^v) = f(p)^v$ by induction: Note that we still have $g(1) = 1$ and $g(p) = -f(p)$ as this does not even depend on multiplicativity of $f$.
\begin{itemize}
\item Base case $v = 2$: Evaluating $L\lambda = Lf * g$ at $p^2$, we get
\begin{align*}
2 \log(p) \lambda(p^2) &= Lf(p) g(p) + Lf(p^2) g(1)\\
\log p \; \lambda(p)^2 &= - \log p f(p) f(p) + 2 \log p f(p^2) &\text{by assumption on }\lambda\\
f(p)^2 &= - f(p)^2 + 2 f(p^2) &\text{ as }\lambda(p) = f(p)
\end{align*}
hence $f(p^2) = f(p)^2$.

\item Induction: Suppose $g(p^k) = 0$ and $f(p^k) = f(p)^k$ for all $2 \leq k  < v$. Evaluating $L\lambda = Lf * g$ at $p^v$ yields
\begin{align*}
v \log p \; \lambda(p^v) &= Lf(p^v) g(1) + Lf(p^{v-1}) g(p) + \underbrace{Lf(1)}_{0} g(p^v)\\
\log p \; \lambda(p)^v &= v \; \log p \; f(p^v) -  (v-1) \log p \; f(p^{v-1}) f(p) &\text{by assumption on }\lambda\\
\log p \; \lambda(p)^v &= v \; \log p \; f(p^v) - (v-1) \log p \; f(p)^v &\text{by induction hypothesis}
\end{align*}
and so we find $f(p^v) = f(p)^v$. It then follows that $g(p^v) = 0$.
\end{itemize}
\end{proof}

\textbf{Problem 2.23}: Let $f$ be any completely multiplicative function except $e$. Show that $f * f$ and $f^{*-1}$ are not completely multiplicative.

\begin{proof}
Since $f$ is multiplicative, we write $f = \exp \lambda$ so
\begin{align*}
f * f &= \exp (2 \lambda)\\
f^{*-1} &= \exp (-\lambda)
\end{align*}
and by problem 2.22
\begin{equation}
\lambda(p^v) = \frac{\lambda(p)^v}{v}
\label{eq:lambda}
\end{equation}
for all prime $p$ and all $v$.

\begin{itemize}
\item By problem 2.22, $f * f$ is completely multiplicative then
$$2\lambda(p^v) = \frac{[2\lambda(p)]^v}{v}$$
for all $v$. Combine this equation and \eqref{eq:lambda} in the special case $v = 2$, we get
$$2\frac{\lambda(p)^2}{2} = \frac{4\lambda(p)^2}{2}$$
whence $\lambda(p) = 0$ and \eqref{eq:lambda} then forces $\lambda(p^v) = 0$ for all $v$. And this works for all primes $p$ so $\lambda = 0$ and $f = e$; a contradiction.

\item Again, if $f^{*-1}$ is completely multiplicative then
$$-\lambda(p^v) = \frac{[-\lambda(p)]^v}{v}$$
for all $v$.
In particular, when $v = 2$, we see that $\lambda(p^2) = 0$ and so $\lambda(p) = 0$ as well whence $\lambda(p^v) = 0$ for all $v$. We reach the contradiction just like before.
\end{itemize}

As an alternative for $f^{*-1}$, from the proof of problem 2.22, we have
$$f^{*-1}(p) = -f(p)$$
and
$$f^{*-1}(p^v) = 0$$
for all $v \geq 2$. So clearly, $f^{*-1}$ cannot be completely multiplicative unless $f(p) = 0$ for all $p$. But then $f = e$, contradicting the assumption.
\end{proof}

\textbf{Problem 2.24}: Let $f \in \A_1$. Show that $f$ is completely multiplicative if and only if $f * f = f \cdot \tau$.

\begin{proof}
The forward implication: If $f$ is completely multiplicative then
$$f(\prod p^v) = \prod f(p)^v.$$
Thus, for any $n$, factorize $n = \prod p^{v_p}$ and one has
\begin{align*}
(f * f)(n) &= \sum_{d | n} f(d) f(n/d)\\
&= \sum_{d | n} f(n) &\text{complete multiplicativity of } f\\
&= \tau(n) f(n)
\end{align*}
since $\tau(n) = \sum_{d | n} 1 = $ number of divisors of $n$.

For the converse implication: Assume $f * f = f \cdot \tau$. Recall $\Omega(n)$ counts the number of prime divisors (with multiplicities) of $n$ i.e. if $n = \prod p^{v_p(n)}$ then $\Omega(n) = \sum v_p(n)$. (We shall use the notation $v_p(n)$ for $p$-adic valuation of $n$.)

We perform induction on the quantity $k = \Omega(m) + \Omega(n)$ that $f(mn) = f(m) f(n)$.

\begin{itemize}
\item Base case $k = 0$: Then $m = n = 1$ and it is clear from $f(1) = 1$.

\item Induction: Let $k \geq 1$ and suppose that $f(m n) = f(m) f(n)$ for every $m, n$ such that $\Omega(m) + \Omega(n) < k$. Note that this implies
$$f(x) = \prod f(p)^{v_p(x)}$$
for all $x$ such that $\Omega(x) < k$.

We show that this is true for all $m, n$ satisfying $\Omega(mn) = \Omega(m) + \Omega(n) = k$ as well. Let $m, n$ be such an arbitrary pair. The statement is obvious if $m = 1$ or $n = 1$ so let us assume that both $m, n > 1$ and we have $\Omega(m), \Omega(n) < k$ as well.

Evaluating both sides of $f * f = f \cdot \tau$ at $mn$, we get
\begin{align*}
(f * f)(mn) &= 2 f(mn) f(1) + \sum_{d | mn; 1 < d < mn} f(d) f(mn/d)\\
&= 2 f(mn) + \sum_{d | mn; 1 < d < mn} \prod f(p)^{v_p(d) + v_p(mn/d)}
\end{align*}
due to induction hypothesis for if $1 < d < mn$ then $\Omega(x) < k$ for $x = d$ and $x = mn/d$. We continue
\begin{align*}
(f * f)(mn) &= 2 f(mn) + \sum_{d | mn; 1 < d < mn} \prod f(p)^{v_p(m) + v_p(n)}\\
&= 2 f(mn) + (\tau(mn) - 2) \prod f(p)^{v_p(m) + v_p(n)}
\end{align*}
which equals $\tau(mn) f(mn)$ so we must have
$$f(mn) = \prod f(p)^{v_p(m) + v_p(n)}$$
unless $\tau(mn) = 2$ which could not happen because $\tau(x) = 2$ if and only if $x$ is a prime number and $mn$ is clearly not by assumption. This shows $f(mn) = f(m) f(n)$ by induction hypothesis given $\Omega(m), \Omega(n) < k$.
\end{itemize}
\end{proof}

\textbf{Problem 2.25}: Let $f \in \A_1$. Show that $f$ is completely multiplicative if and only if for each $g, h \in \A$, we have $f \cdot (g * h) = (f \cdot g) * (f \cdot h)$.

\begin{proof}
Forward implication: Assume $f$ is completely multiplicative and let $n$ be arbitrary. One has
\begin{align*}
RHS(n) &= \sum_{de = n} f(d) \; g(e) \; f(e) \; h(d)\\
&= \sum_{de = n} \underbrace{f(d) f(e)}_{f(n)} g(e) h(d) &\text{by complete multiplicativity}\\
&= f(n) \underbrace{\sum_{de = n} g(e) h(d)}_{(g * h)(n)}\\
&= LHS(n).
\end{align*}

For the converse: Assume the equation holds for all $g, h$. Then in particular for $g = h = 1$, one finds
$$f \cdot (1 * 1) = f * f$$
and then recall that $1 * 1 = \tau$ so we have $f * f = f \cdot \tau$ so $f$ is completely multiplicative by problem 2.24.
\end{proof}

\textbf{Problem 2.26}: Show that $|\mu|$ is multiplicative but not completely multiplicative. Express $|\mu|$ as $f_1 * f_2 * ...$ as in Lemma 2.26. Describe $f_i$ and find $\log |\mu|$.

\begin{proof}
Recall that $\mu = \mu_2 * \mu_3 * ...$ so $\mu$ is multiplicative by Lemma 2.26. Thus, $|\mu|$ is multiplicative because absolute value is multiplicative $|x| \cdot |y| = |x \cdot y|$. From lemma 2.26, we know that
\begin{align*}
f_i(m) &= \begin{cases}
|\mu|(m) &\text{if } m = p_i^k,\\
0 &\text{otherwise}.
\end{cases}\\
&= \begin{cases}
1 &\text{if } m = 1,\\
-1 &\text{if } m = p_i,\\
0 &\text{otherwise}.
\end{cases}\\
&= e + e_{p_i}
\end{align*}
where $p_i$ is the $i$-th prime number. Now,
\begin{align*}
\log f_i &= \sum_{j=1}^{\infty} \frac{(-1)^{j-1}}{j} (f_i - e)^{*j} &\text{ by Theorem 2.20}\\
&= \sum_{j=1}^{\infty} \frac{(-1)^{j-1}}{j} e_{p_i}^{*j}\\
&= \sum_{j=1}^{\infty} \frac{(-1)^{j-1}}{j} e_{p_i^j} &\text{ since } e_x * e_y = e_{xy}
\end{align*}
and so
$$\log |\mu| = \sum_{i=1}^{\infty} \log f_i = \sum_p \sum_{j=1}^{\infty} \frac{(-1)^{j-1}}{j} e_{p^j}.$$
\end{proof}

\textbf{Problem 2.27}: Liouville's $\lambda$ function is defined by $\lambda(n) = (-1)^{\Omega(n)}$. Show that $\lambda$ is completely multiplicative, find $\log \lambda$. What is the relationship between $\lambda$ and $|\mu|$?

\begin{proof}
It follows from obvious fact that $\Omega(mn) = \Omega(m) + \Omega(n)$ that $\lambda$ is completely multiplicative.

Let $\alpha = \log \lambda$ then we know that $\alpha$ has support in prime powers, $\alpha(p) = \lambda(p) = -1$ and that
$$\alpha(p^j) = \frac{\alpha(p)^j}{j} = \frac{(-1)^j}{j}$$
for all $j \geq 2$ by problem 2.22. So
$$\alpha = \sum_p \sum_{j = 1}^{\infty} \frac{(-1)^j}{j} e_{p^j}.$$

We can now recognize that $|\mu| = \exp(-\alpha)$ so $\lambda$ and $|\mu|$ are convolution inverse of each other i.e. $\lambda * |\mu| = e$.
\end{proof}

\textbf{Problem 2.28}: For $b$ positive integer, $f_b(n) = (n, b)$ is multiplicative.

\begin{proof}
Straightforward from property of gcd: One has $(m, b) (n, b) = (mn, b)$ if $(m, n) = 1$.
\end{proof}

\textbf{Problem 2.29}: Let $f = \exp \lambda$ where $\lambda \in \A_0$ and let $g$ be completely multiplicative. Show that $f \cdot g = \exp(\lambda \cdot g)$.

\begin{proof}
By problem 2.25 and induction, we notice that $(\lambda \cdot g)^{*j} = (g \cdot \lambda) * ... * (g \cdot \lambda) = g \cdot (\lambda * ... * \lambda) = g \cdot \lambda^{*j}$. So
\begin{align*}
RHS &= \sum_{j=0}^{\infty} \frac{(\lambda g)^{*j}}{j!}\\
&= \sum_{j=0}^{\infty} \frac{g \cdot \lambda^{*j}}{j!}\\
&= g \cdot \sum_{j=0}^{\infty} \frac{\lambda^{*j}}{j!}\\
&= g \cdot \exp \lambda\\
&= g \cdot f.
\end{align*}
\end{proof}

\textbf{Problem 2.30}: Define $g_b(n) = 1$ if $(n, b) = 1$ and $g_b(n) = 0$ if $(n, b) > 1$. Show that $g_b$ is completely multiplicative.

\begin{proof}
Clearly, if $g_b(mn) = 1$ then $(mn, b) = 1$ so $(m, b) = (n, b) = 1$. If $g_b(mn) = 0$ then $(mn, b) > 1$ so either $(m, b) > 1$ or $(n, b) > 1$.
\end{proof}

\renewcommand{\S}{\mathcal{S}}

\textbf{Problem 2.31}: Let $\S$ be the set of squares. Show that $\tau^2 = 1^{*4} * 1_\S^{*-1}$.

\begin{proof}
Notice that
$$1_\S\left(\prod p^{v_p}\right) = \prod 1_\S(p^{v_p})$$
so $1_\S$ is multiplicative. That is due to Unique Factorization Theorem: Comparing factorizations of $n = x^2$, we see that $n$ is a square if and only if $v_p(n)$ are all even; equivalently $1_\S(p^{v_p}) = 1$ for all $p$.

Thus, we could factorize $1_\S = f_2 * f_3 * ...$ where
$$f_p = e + e_{p^2} + e_{p^4} + ...$$
according to Lemma 2.26. Now,
$$f_p^{*-1} = e - e_{p^2} = (e - e_p)(e + e_p)$$
and so we see that
$$1_\S^{*-1} = \prod_p^* f_p^{*-1} = \underbrace{\prod_p^* (e - e_p)}_{\mu} * \underbrace{\prod_p^{*} (e + e_p)}_{|\mu|}$$
where $\prod^*$ is to highlight that this is an infinite convolution product.

Thus, showing $\tau^2 = 1^{*4} * 1_\S^{*-1}$ is equivalent to showing
$$\tau^2 = 1^{*4} * \mu * |\mu| = 1^{*3} * |\mu|$$
because $1 * \mu = e$. As both sides are multiplicative functions, it suffices to check that they agree on the prime powers:
$$\tau^2(p^\alpha) = (\alpha + 1)^2$$
and as $|\mu|(p^x) = 0 \quad \forall x \geq 2$,
\begin{align*}
(1^{*3} * |\mu|)(p^\alpha) &= 1^{*3}(p^\alpha) \cdot |\mu|(1) + 1^{*3}(p^{\alpha-1}) \cdot |\mu|(p) \\
&= 1^{*3}(p^\alpha) + 1^{*3}(p^{\alpha-1})\\
&= \frac{(\alpha + 1)(\alpha + 2)}{2} + \frac{\alpha(\alpha + 1)}{2}\\
&= (\alpha + 1)^2.
\end{align*}
\end{proof}

\unless\ifdefined\IsMainDocument
\end{document}
\fi


\chapter{Summatory Functions}

\newcommand{\V}{\mathcal{V}}

\unless\ifdefined\IsMainDocument
\documentclass{article}
\usepackage{amsmath,amsthm,amssymb}

\title{Chapter 3}
\author{An Hoa Vu}

\newcommand{\Abs}[1]{\left| #1 \right|}
\newcommand{\R}{\mathbb{R}}
\newcommand{\V}{\mathcal{V}}

\begin{document}

\maketitle
\fi

\section{Midpoint approximation of convex function}

We prove a generalization of the inequality used in the proof of Theorem 3.5 with the justification being just ``by midpoint approximation of convex function'':
$$m^{-2} < \int_{m - \frac 12}^{m + \frac 12} t^{-2} dt.$$

Suppose that $f$ is Riemann integrable and that $f$ is convex i.e. $f''(x) \geq 0$ on $(a, b)$. Then
$$(b - a) \cdot f\left(\frac{a+b}{2}\right) \leq \int_a^b f(t) dt.$$

\begin{proof}
The right hand side can be computed as the limit
$$R_{f,a,b}(N) = \Delta_N \sum_{i=0}^{N-1} f\left(a + i \Delta_N + \frac{\Delta_N}{2} \right) $$
as $N \rightarrow \infty$ where $\Delta_N = \frac{b - a}{N}$. Basically, we divide the interval $[a, b]$ into $N$ subinterval
$$[a + i \Delta_N, a + (i+1) \Delta_N]$$
of length $\Delta_N$ and take their midpoints to approximate the integral.

Since $f$ is convex on $(a, b)$, we have Jensen's inequality
$$\sum_{i = 1}^{n} f(x_i) \geq n f\left(\frac{\sum_{i=1}^{n} x_i}{n}\right)$$
for all $x_1, ..., x_n \in (a, b)$. Applying it to the case of $n = N$ and the $x_i = a + \frac{2i - 1}{2} \Delta_N$ which are clearly in $(a, b)$, we find
\begin{align*}
R_{f,a,b}(N) &\geq (b - a) \cdot f \left( \frac{\sum_{i=0}^{N-1} a + i \Delta_N + \frac{\Delta_N}{2} }{N} \right)\\
&= (b - a) \cdot f \left( \frac{N (a + \frac{\Delta_N}{2}) + \sum_{i=0}^{N-1} i \Delta_N }{N} \right)\\
&= (b - a) \cdot f \left( a + \frac{N \frac{\Delta_N}{2} + \frac{(N-1)N}{2} \Delta_N }{N} \right)\\
&= (b - a) \cdot f \left( a + N \frac{\Delta_N }{2} \right)\\
&= (b - a) \cdot f \left( \frac{a + b}{2} \right)
\end{align*}
so taking the limit yields the desired inequality.
\end{proof}

Note that we can reduce the assumption to a weaker condition
$$f(x) + f(y) \geq 2 f\left( \frac{x + y}{2} \right)$$
for all $x, y \in [a, b]$ instead of the stronger condition $f'' > 0$ which requires $f$ to have second derivative. Then in the proof, we can pair off the terms of $R_{f,a,b}(N)$ instead: for every $i$, one has
$$f\left(a + i \Delta_N + \frac{\Delta_N}{2} \right) + f\left(a + (N-1-i) \Delta_N + \frac{\Delta_N}{2} \right) \geq 2 \cdot f \left( \frac{a + b}{2} \right).$$

As an application, if $\alpha > 0$ then $f(t) = t^{-\alpha}$ is convex on $(0, \infty)$ because $f'(t) = -\alpha t^{-\alpha-1}$ and $f''(t) = \alpha(\alpha+1) t^{-\alpha-2} > 0$ for all $t > 0$. So in particular for all integer $m \geq 1$, we have
$$m^{-3} < \int_{m - \frac 12}^{m + \frac 12} t^{-3} dt$$
and
$$m^{-\frac 12} < \int_{m - \frac 12}^{m + \frac 12} t^{-\frac 12} dt.$$

\section{Variation of a function}

The book didn't give the definition for the variation of a function on $[a, b]$ but thank to
\begin{itemize}
\item its hints on the ``familiar'' properties of total variation function $F_v(x) = $ variation of $F$ on $[0, x]$ and
\item the fact that function of bounded variations $F$ are used as differential/integrator $dF$ to integrate other functions as in
$$\int_a^b \; g \cdot dF$$
similar to Riemann sum,
\end{itemize}
we can go back to the classical situation where $F(x) = x$ and Riemann's definition
$$\int_a^b \; f \; dx = \lim_{\Pi \in P[a,b]} \; \sum f(t_i) \underbrace{\Abs{x_{i+1} - x_i}}_{\text{length of }[x_i, x_{i+1}]}$$
where $P[a, b]$ is the set of all partitions of the interval $[a, b]$. (A partition is essentially a finite subset of $[a, b]$ given by the points $a = x_0 < ... < x_n = b$. Here, the limit is taken via a \emph{directed system} where the partitions are ordered by \emph{refinement relation}) so that we could derive the definition: (Verified in Apostol Chapter 6.)
The variation of a function $F$ on $[a, b]$ should serve the same purpose as producing some kind of \emph{measurement} (e.g. the \emph{length} $|b - a|$ when $F(x) = x$) for the interval $[a, b]$ and so it should be given by\footnote{Why sup? Because refinement increases the sum due to triangle inequality.}
$$\sup_{\Pi \in P[a, b]} \left\{ \sum \Abs{F(x_{i + 1}) - F(x_i)} \right\}.$$

Let us denote
$$V_F(\Pi) = \sum_{i=0}^{|\Pi| - 1} \Abs{ F(x_{i + 1}) - F(x_i) }$$
and
$$V_F(a, b) = \sup_{\Pi \in P[a, b]} V_F(\Pi)$$
where where $|\Pi|$ denotes the number of points in $\Pi$.

\section{Some properties of variation}

It is easy to check that
\begin{enumerate}
\item $V_{c F}(a, b) = \Abs{c} V_F(a, b)$

\item $V_F(a, b) \geq 0$

\item If $a \leq b \leq c$ then
$$V_F(a, b) + V_F(b, c) = V_F(a, c).$$
(Proof: Insert $b$ to any partition of $[a, c]$ to make a refinement.)

In particular,
$$V_F(a, b) \leq V_F(a, c).$$

So our intuition of $V_F(a, b)$ as measuring the length of $[a, b]$ makes sense.

\item The book was wrong about $F$ increasing then $F_v = F$. This only works if $F(0) = 0$. The correct formula is that if $F$ is increasing then
$$F_v = F - F(0).$$
Or more generally,
$$V_F(a, b) = F(b) - F(a).$$

By the same token, if $F$ is decreasing then
$$V_F(a, b) = F(a) - F(b).$$

\item One has
\begin{align*}
\lim_{x \rightarrow a^+} V_F(a, x) &= \lim_{x \rightarrow a^+} \Abs{ F(x) - F(a) }\\
\lim_{x \rightarrow b^-} V_F(x, b) &= \lim_{x \rightarrow b^-} \Abs{ F(b) - F(x) }
\end{align*}
assuming the limits that appeared exists.

\begin{proof}
By replacing $F$ with $F_1(x) = F(x - a)$ and $F_2(x) = F(b - x)$, the problem reduces to showing the first formula for $a = 0$. Since $F$ is fixed, we shall subsequently drop the subscript in $V_F$. Let
\begin{align*}
\alpha &= \lim_{x \rightarrow 0^+} V(0, x) = \inf \left\{V(0, x) \;|\; x > 0 \right\}\\
\beta &= \lim_{x \rightarrow 0^+} \Abs{ F(x) - F(0) }
\end{align*}

Evidently, $V(0, x) \geq \Abs{ F(x) - F(0) }$ for all $x > 0$ considering the trivial partition $\{0, x\}$ of $[0, x]$. So $\alpha \geq \beta$.

Assume $\alpha > \beta$. Then by definition of limit, there exists $\epsilon_0 > 0$ such that for all $\delta > 0$, there is some positive $x_\delta < \delta$ such that $$V(0, x_\delta) > \beta + \epsilon_0.$$
In particular, taking $\delta = \frac{1}{n}$ for all positive integer $n$ and we can find a decreasing sequence $x_n$ such that $x_n < \frac 1n$ and that
$$V(0, x_n) > \beta + \epsilon_0$$
for all $n$.

We shall show that $V(0, x_n)$ cannot be a Cauchy sequence, in particular for $\epsilon = \frac{\epsilon_0}{2}$, thus violating the assumption on the convergence of $V(0, x)$. That means we have to show that for any $N$, there is always an $m > N$ such that
$$V(0, x_N) - V(0, x_m) > \frac{\epsilon_0}{2}.$$
Notice that by our construction, $x_m < x_N$ and so the left hand side of the above inequality is $V(x_m, x_N)$.

Let $N$ be arbitrary. We make use of the fact that $V(0, x_N) > \beta + \epsilon_0$ to obtain a partition $\pi = (0, t_1, ..., t_k = x_N)$ of $[0, x_N]$ satisfying
$$V(\pi) > \beta + \epsilon_0$$
by definition of the supremum. Now take $m$ sufficiently large such that $x_m < t_1$, possible as $x_n \rightarrow 0$ and that
$$\beta - \frac{\epsilon_0}{2} < \Abs{ F(x_m) - F(0) } < \beta + \frac{\epsilon_0}{2}$$
from the limit definition of $\beta$. Then consider the refinement partition $\pi \cup \{x_m\} = (0, x_m, t_1, ..., x_N)$ of $[0, x_N]$ and the partition $\pi' = \pi \cup \{x_m\} - \{0\} = (x_m, t_1, ..., x_N)$ of $[x_m, x_N]$ by dropping the zero. We have
$$V(\pi \cup \{x_m\}) \geq V(\pi) > \beta + \epsilon_0$$
so
\begin{align*}
V(x_m, x_N) \geq V(\pi') &= V(\pi \cup \{x_m\}) - \Abs{ F(x_m) - F(0) }\\
&\geq \beta + \epsilon_0 - \left( \beta + \frac{\epsilon_0}{2} \right)\\
&\geq \frac{\epsilon_0}{2}.
\end{align*}
\end{proof}

\end{enumerate}

\section{Total variation function of $[x] - x + 1$}

Let
$$F(x) = \begin{cases}
[x] - x + 1 &\text{for } x \geq 1,\\
0 &\text{for }x < 1.
\end{cases}$$

Assuming that $F$ is of bounded variation; need to use the definition otherwise. We prove that $F_v(x) = [x] + x - 1$ for $x \geq 1$.

For any integer $n \geq 1$, observe that if $x \in [n, n+1)$ then $[x] = n$ and $F(x) = n - x + 1$ is a decreasing function so $-F$ is an increasing function and so
\begin{align*}
V_F(n, x) &= F(n) - F(x)\\
&= 1 - (n - x + 1)\\
&= x - n.
\end{align*}

To prove the formula for general $x > 1$, let $n = [x]$. By the additivity property of variation,
\begin{align*}
V_F(0, x) &= \sum_{i = 0}^{n - 1} V_F(i, i+1) + V_F(n, x)\\
&= 1 + \sum_{i = 1}^{n - 1} 2 + (x - n)\\
&= 1 + 2(n - 1) + x - n\\
&= x + n - 1\\
&= x + [x] - 1
\end{align*}
Here, we make use of the obvious fact that
$$V_F(0, 1) = F(1) - F(0) = 1$$
because $F$ is increasing on $[0, 1]$ while
\begin{align*}
V_F(i, i + 1) &= \lim_{x \rightarrow (i+1)^-} V_F(i, x) + V_F(x, i+1)\\
&= 2
\end{align*}
for all $i \geq 1$.

\section{Pollard's definition of integral}

The existence on right hand side of
$$\int_a^b g dF = \lim_{\Pi \in P[a, b]} \sum g(t_i) (F(x_{i+1}) - F(x_i))$$
means: There is a number $A$ such that for every $\epsilon > 0$, there exists a partition $\Pi_\epsilon$ such that for every $\Pi$ that refines $\Pi_\epsilon$ and any choice of $t_i \in [x_i, x_{i+1}]$ then
$$\Abs{ \sum g(t_i) (F(x_{i+1}) - F(x_i)) - A } < \epsilon.$$

\section{Properties of integral}

\begin{itemize}
\item Example 3.9: Straightforward from the definition: For any $\epsilon$, take the partition with $n - \delta, n + \delta$ around every integer in $[a, b]$.

\item Example 3.10: Should apply to more than just Riemann integrable functions.

\item The familiar inequality $\Abs{ \int_a^b f(t) dt } \leq \int_a^b \Abs{ f(t) } \, dt$ generalizes to
$$\Abs{ \int_a^b g dF } \leq \int_a^b \Abs{ g } \, dF_v$$
from the definition and triangle inequality. Back to the classical situation, we have $F(t) = t$ is increasing so $F_v = F$.
\end{itemize}

\section{Total variation of summatory functions}

The equation
$$F_v(t) = \int_1^x \Abs{ f(t) } dt$$
on page 52 applies to summatory function of an arithmetic function with a small change to the lower limit $1^-$ instead of 1. In other words, if $f$ is an arithmetic function and $F$ its summatory function then
$$F_v(t) = \sum_{n \leq t} \Abs{ f(n) }.$$
That is to say $F_v$ is the summatory function of $\Abs{ f }$.

\section{Integrator associated to convolution of arithmetic functions}

On page 54: If $F, G$ are summatory functions of the arithmetic functions $f, g$ then the function
$$H(x) = \int_{1^-}^x G(x/t) \, dF(t), \quad x \geq 1$$
and $H(x) = 0$ for $x < 1$ is the summatory function of $f * g$ (so $dH$ is the integrator corresponding to $f * g$).

\begin{proof}
For fixed $x \geq 1$, the function $g_x(t) = G(x/t)$ is continuous from the left at all integer $(1-\delta, x]$ for small $\delta$ because $G$ is continuous from the right. So
\begin{align*}
\int_{1-\delta}^x G(x/t) \, dF(t) &= \int_{1-\delta}^x g_x(t) \, dF(t)\\
&= \sum_{1 - \delta < n \leq x} g_x(n) f(n) &\text{by example 3.9}\\
&= \sum_{1 \leq n \leq x} G(x/n) f(n)\\
&= \sum_{1 \leq n \leq x} (f * g)(n) &\text{by Lemma 3.1}
\end{align*}
Thus $H(x)$ is the summatory function of $f * g$.
\end{proof}

\section{Integrator defined by $\varphi \, dG$}

The book frequently used integrators of the form $\varphi(t) \, dt$ such as $t^{-1} \, dt$ and $\log t \, dt$ without explaining how they arise from a function $F \in \V$ as in Definition 3.15. It turns out that the function $F$ should be given by the formula
$$F(x) = \int_{1^-}^x \varphi(t) \, dt$$
for all $x \geq 1$ and $F(x) = 0$ if $x < 1$ following the explanation on page 52. In this case, we can drop the $1^-$ using Lemma 3.8 because the function associated to $dt$, despite its ambiguity, is continuous at 1.

On second thought, we could have taken
$$F(x) = \int_a^x \varphi(t) \, dt$$
for any $a < 1$ as long as the function $\varphi$ is defined on $x \geq a$ and has all the nice properties in the lemma for
$$d(F + c) = dF$$
for any constant $c$ because they assign the same value to any half interval $(a, b]$. A change from $1^-$ to $a$ simply incurs adjusting by a constant $\int_a^{1^-} \varphi(t) \, dt$. The choice of $1^-$ simply makes it uniform regardless of the domain of $\varphi$ so we don't have to keep specifying the domain and automatically makes $F$ zero on $(-\infty, 1)$ so that $F \in \V$.

More generally, the integrator $\varphi \, dG$ arises from the function
$$F(x) = \int_{1^-}^x \varphi(t) \, dG$$
However, unlike the previous situation, we indeed want to take the limit $1^-$ here because $dG$ is not guaranteed to be continuous at 1. Again, we could have taken
$$F(x) = \int_{1 - \epsilon}^x \varphi(t) \, dG$$
as long as $G$ and $\varphi$ play nice on $(1 - \epsilon, x]$. See the comment after Lemma 3.8 on page 46.

In some situation, it is necessary to take the lower limit to be number $> 1$. For example: The integrator
$$\frac{1}{\log t} dt$$
is defined on $(1, \infty)$. The associated function should be
$$\int_a^x \frac{1}{\log t} dt$$
for any $a > 1$ appropriate to the problem. This is a function on $(1, \infty)$ only but of course we could extend by zero. Functions in $\V$ need not be defined on the entire $\R$, I believe.

Also observe that the use of Lemma 3.8 here to ensure that the function
$$F(x) = \int_a^x \varphi dG$$
is continuous from the right so that it could be an element of $\V$. Its total variation should be given by
$$F_v(x) = \int_a^x \Abs{ \varphi } \; dG_v,$$
similar to the total variation of summatory functions mentioned above.

\section{On example 3.22 about $dN * dt$}

On page 57, the book gave
$$H(x) = \int_{1^-}^x dN * dt = \int_{1^-}^x (\frac xt - 1) dN(t)$$
which doesn't look to make sense at first if we think of $dt$ as $dG$ where $G(t) = t$ from experience or from the suggestion of the notation because $H(x)$ is supposed to be
$$H(x) = \int_{1^-}^x G(x/t) dN(t).$$
However, the notation $dF$ only applies to $F \in \V$ and such $F$ zeros out on $(-\infty, 1)$. Therefore, here $dt$ is actually $dF$ where
$$F(x) = \begin{cases}
0 &\text{if } x < 1,\\
x - 1 &\text{if } x \geq 1,
\end{cases}$$
as given on page 52. And so the formula makes sense.

So a WARNING: \textbf{The notation $dt$ could mean two different things}, depending on the context:
\begin{itemize}
\item The usual $dt$ in Riemann-Stieltjes integral i.e. $dF_1$ where $F_1(t) = t$. This version can only occurs inside the integral $\int_a^b ... dt$.
\item The integrator $dF_2$ associated to the function $F_2(t) = \delta_1(t) \, (t - 1)$ per Definition 3.15. This one can stand alone and will usually cause confusion when used under the integral.
\end{itemize}
Both $F_1$ and $F_2$ are continuous so usually the second usage will give the same answer in the first: we obviously have $\int_a^b g \, dF_1 = \int_a^b g \, dF_2$ for all $b \geq a \geq 1$ and also $\int_{1^-}^x g \, dF_1 = \int_{1^-}^x g \, dF_2$ by Lemma 3.8.

To check the derivative at $x \not\in Z$, we compute
\begin{align*}
\lim_{\delta \rightarrow 0^+} \frac{H(x + \delta) - H(x)}{\delta} &= \lim \frac 1\delta \left( \int_{1^-}^{x+\delta} (\frac {x+\delta} t - 1) dN - \int_{1^-}^x (\frac xt - 1) dN \right)\\
&= \lim \frac 1\delta \left( \int_{x}^{x+\delta} (\frac {x+\delta} t - 1) dN + \int_{1^-}^x \frac \delta t dN \right)\\
&= \int_{1^-}^x \frac{dN}{t} + \lim \frac 1\delta \left( \int_{x}^{x+\delta} (\frac {x+\delta} t - 1) dN +  \right)
\end{align*}
As $\delta \rightarrow 0^+$,
$$\int_{x}^{x+\delta} (\frac {x+\delta} t - 1) dN = \sum_{x < n \leq x + \delta} \left\{\frac{x + \delta}{n} - 1\right\} = 0$$
by example 3.9 assuming $x \not\in Z$ if we recall $N$ is the summatory function of $1$. Therefore,
$$\lim_{\delta \rightarrow 0^+} \frac{H(x + \delta) - H(x)}{\delta} = \int_{1^-}^x \frac{dN}{t}.$$
The limit when $\delta \rightarrow 0^-$ is likewise computed.

\section{Operators on integrators}

First, let me mention a little fact that if the integrator $dK$ is defined via R-S integral
$$dK(E) = \int_E \varphi \; dF$$
then for any $\psi$, one has
$$\int_E \psi \; dK = \int_E \; \psi \; \varphi \; dF.$$
This is the more general version of Example 3.10 which is mostly used to convert $d f(t) = f'(t) dt$ in practice.

To check an equality of integrators, we only have to check on basic intervals $(a, b]$. For equation (3.15):
\begin{align*}
T^\alpha(dF * dG) ((a, b]) &= \int_a^b t^\alpha (dF * dG)\\
&= \int_{1^-}^b \int_{a/s}^{b/s} (st)^\alpha dG(t) dF(s) &\text{lemma 3.23 with } \varphi(t) = t^\alpha\\
(T^\alpha dF * T^\alpha dG)((a, b]) &= \int_a^b (T^\alpha dF) * (T^\alpha dG)\\
&= \int_{1^-}^b \int_{a/s}^{b/s} T^\alpha dG(t) \{T^\alpha dF(s)\} &\text{lemma 3.23 with } \varphi(t) = 1\\
&= \int_{1^-}^b \int_{a/s}^{b/s} t^\alpha \; dG(t) \; s^\alpha dF(s)
\end{align*}
so we have identity. For equation (3.16):
\begin{align*}
L(dF * dG) ((a, b]) &= \int_a^b \log t \; (dF * dG)\\
&= \int_{1^-}^b \int_{a/s}^{b/s} \log (st) \; dG(t) \; dF(s) &\text{lemma 3.23 with } \varphi(t) = \log t\\
&= \int_{1^-}^b \int_{a/s}^{b/s} (\log s + \log t) \; dG(t) \; dF(s)\\
L \, dF * dG ((a, b]) &= \int_a^b (L \, dF) * dG\\
&= \int_{1^-}^b \int_{a/s}^{b/s} dG(t) \, \{L \, dF(s)\} &\text{lemma 3.23 with } \varphi(t) = 1\\
&= \int_{1^-}^b \int_{a/s}^{b/s} dG(t) \; \log s \; dF(s)
\end{align*}
while
\begin{align*}
dF * L \, dG ((a, b]) &= \int_a^b dF * L \, dG\\
&= \int_{1^-}^b \int_{a/s}^{b/s} L\, dG(t) \, dF(s) &\text{lemma 3.23 with } \varphi(t) = 1\\
&= \int_{1^-}^b \int_{a/s}^{b/s} \log t \; dG(t) \; dF(s)
\end{align*}

\unless\ifdefined\IsMainDocument
\end{document}
\fi


\section{Problems}
\unless\ifdefined\IsMainDocument
\documentclass{article}
\usepackage{amsmath,amsthm}

\begin{document}
\fi

\textbf{Problem 3.1}: Show that
$$\sum_{n \leq x} \tau(n) = \sum_{m \leq x} [x/m]$$
$$\sum_{n \leq x} \sigma(n) = \frac12 \sum_{\ell \leq x} [x/\ell] [x/\ell + 1] = \sum_{m \leq x} m [x/m]$$

\begin{proof}
Straight-forward from Lemma 3.1 and the fact that $\tau = 1 * 1$ whereas $\sigma = 1 * T1$. (Notice that $T1$ is the identity function $T1(n) = n$.)
\end{proof}

\textbf{Problem 3.2}: Suppose $f, g \in A$ and $f, g \geq 0$. Let $h = f * g$ and $F, G, H$ be the respective summatory functions. Show that $H(x) \leq F(x) G(x)$ for all $x \geq 1$. Find condition on $f$ and/or $g$ for which equality holds (a) for a particular $x \geq 2$, (b) for all $x \geq 2$.

\begin{proof}
One has
\begin{align*}
H(x) &= \sum_{m \leq x} F(x/m) g(m) &\text{by Lemma 3.1}\\
&\leq \sum_{m \leq x} F(x) g(m) &\text{by assumption }f, g \geq 0: F(x/m) \leq F(x)\\
&= F(x) \sum_{m \leq x} g(m)\\
&= F(x) G(x)
\end{align*}

Equality can only occurs if the inequality on the second line is an equality; which can only happen if $F(x/m) g(m) = F(x) g(m)$, equivalently $[F(x) - F(x/m)] \cdot g(m) = 0$, for all $m \leq x$. In other words, for every $1 \leq m \leq x$, either $g(m) = 0$ or $f(n) = 0$ for all $\frac{x}{m} < n \leq x$. Combining the two we get a symmetric condition: For every $1 \leq n, m \leq x$ such that $n m > x$ then $f(n) g(m) = 0$.

So if equality holds for all $x \geq 2$, assume that $f(n) > 0$ for some $n \geq 2$. Then we must have $g(m) = 0$ for all $m \geq 2$ because otherwise, taking $x = mn - 1 \geq 2 \cdot 2 - 1 = 3$ and we find that the inequality does not hold for $x$ using the above condition:
$$x \geq 2n - 1 = n + n - 1 \geq n + 2 - 1 > n$$
and similarly $x > m$ while $mn > x$. So either $f(n) = 0$ for all $n \geq 2$ or $g(m) = 0$ for all $m \geq 2$.
\end{proof}

\textbf{Problem 3.3}: Show that $1 = |\mu| * 1_S$ where $S$ is the set of squares. Show
$$\sum_{n \leq x} \frac{1}{n} \leq \sum_{\ell = 1}^{\infty} \frac{1}{\ell^2} \sum_{m \leq x} \frac{|\mu(m)|}{m}$$
for all $x$. Deduce $\sum_{m = 1}^{\infty} \frac{|\mu(m)|}{m} = \infty$.

\begin{proof}
In previous problem, we already show $\mu * |\mu| * 1_S = e$ so $1 = |\mu| * 1_S$ on account of $\mu * 1 = e$. This implies
$$T^{-1} 1 = T^{-1} |\mu| * T^{-1} 1_S$$
because the $T^{\alpha}$ are homomorphisms. Applying the inequality of problem 3.2, we get
$$S_{T^{-1} 1}(x) \leq S_{T^{-1} 1_S}(x) \cdot S_{T^{-1} |\mu|}(x)$$
where $S_f(x)$ denotes the summatory function of $f$. By definitions,
\begin{align*}
S_{T^{-1} 1}(x) &= \sum_{n \leq x} \frac{1}{n}\\
S_{T^{-1} 1_S}(x) &= \sum_{n \leq x} \frac{1_S(n)}{n} \leq \sum_{\ell = 1}^{\infty} \frac{1}{\ell^2}\\
S_{T^{-1} |\mu|}(x) &= \sum_{m \leq x} \frac{|\mu(m)|}{m}
\end{align*}
so we deduce the desired inequality. Taking the limit as $x \rightarrow \infty$ and recall $\sum 1/n = \infty$ while $\sum 1/n^2 = \zeta(2)$ is finite, we see that $\sum |\mu(m)|/m = \infty$.
\end{proof}

\unless\ifdefined\IsMainDocument
\end{document}
\fi

\unless\ifdefined\IsMainDocument
\documentclass{article}
\usepackage{amsmath,amsthm}
\newcommand{\IntPart}[1]{\left[ #1 \right]}
\newcommand{\Abs}[1]{\left| #1 \right|}
\newcommand{\O}[1]{O\left( #1 \right)}
\begin{document}
\fi

\textbf{Problem 3.5}: Show that the number of cube free integers in $[1, x]$ is $x/\zeta(3) + \O{ x^{1/3} }$.

\begin{proof}
Let $C$ be the set of cube-free integers. We want to estimate the summatory function of $1_C$. Notice similar to $1_S$ (characteristic function of square-free integers), $1_C$ is multiplicative and can be written as infinite convolution
$$1_C = \prod^* (e + e_p + e_{p^2})$$
whence we recognize
$$1_C * \mu = \prod^* (e - e_p^3)$$
which could be recognize as the function $f$ where $f(n) = \mu(\sqrt[3]{n})$ if $n$ is a cube and 0 otherwise. So $1_C = 1 * f$ by Mobius inversion.

We proceed as in Theorem 3.5:
\begin{align*}
\sum_{n \leq x} 1_C(n) &= \sum_{m \leq \sqrt[3]{x}} [x/m^3] \; \mu(m)\\
&= \sum_{m \leq \sqrt[3]{x}} x/m^3 \; \mu(m) + \underbrace{\sum_{m \leq \sqrt[3]{x}} ([x/m^3] - x/m^3) \; \mu(m)}_{\O{ \sqrt[3]{x} }}\\
&= x \underbrace{\sum_{m = 1}^{\infty} \mu(m)/m^3}_{1/\zeta(3)} - x \sum_{m > \sqrt[3]{x}} \mu(m) m^{-3} + \O{ \sqrt[3]{n} }
\end{align*}
Again, by midpoint approximation of a convex function
$$m^{-3} < \int_{m-1/2}^{m+1/2} t^{-3} dt$$
so
\begin{align*}
\left|x \sum_{m > \sqrt[3]{x}} \mu(m) m^{-3}\right| &\leq x \sum_{m > \sqrt[3]{x}} m^{-3}\\
&\leq x \int_{\sqrt[3]{x} - 1/2}^{\infty} t^{-3} dt\\
&= x \frac{(\sqrt[3]{x} - 1/2)^{-2}}{2}\\
&= \O{ \sqrt[3]{x} }.
\end{align*}
\end{proof}

\textbf{Problem 3.6}: Show that for $x \geq 1$,
$$\sum_{n \leq x} \frac{\phi(n)}{n} = \sum_{n \leq x} \IntPart{\frac{x}{n}} \frac{\mu(n)}{n} = \frac{6}{\pi^2} x + \O{ \log ex }$$

\begin{proof}
Recall example 2.9 (page 19) that
$$\phi = T1 * \mu$$
so
$$T^{-1} \phi = 1 * T^{-1} \mu.$$

(For a direct proof, $\phi$ is multiplicative. So is $T^{-1}\phi$. Then it is easy to check that $T^{-1}\phi(p^n) = 1 - \frac{1}{p} = (1 * T^{-1}\mu)(p^n)$ for all $n$.)

This establishes the first formula, by Lemma 3.1. For the second, we follow the proof of Theorem 3.5:
$$\sum_{n \leq x} \IntPart{\frac{x}{n}} \frac{\mu(n)}{n} = \underbrace{\sum_{n \leq x} \frac{x}{n} \frac{\mu(n)}{n}}_{I} + \underbrace{\sum_{n \leq x} \left\{\IntPart{\frac{x}{n}} - \frac{x}{n} \right\} \frac{\mu(n)}{n}}_{J}.$$

On the one hand, one has
$$|J| \leq \sum_{2 \leq n \leq x} \frac{1}{n} \leq \int_1^x \frac{1}{t} dt = \log x$$
due to the trivial estimate for $\int_1^x \frac{1}{t} dt$ using the right Riemann sum for the partition of $[1, x]$ given by points $2, 3, ..., n$ and the fact that $1/t$ is decreasing on $[1, \infty)$. Formally, by Mean Value Theorem:
$$\frac{1}{n+1} < \int_n^{n+1} \frac{1}{t} dt < \frac{1}{n}.$$

On the other hand, the argument in Theorem 3.5 yields
$$\Abs{\sum_{n > x} \frac{\mu(n)}{n^2}} < \int_{x - \frac12}^{\infty} t^{-2} dt = \frac{1}{x - \frac12}$$
and so
$$I = \frac{x}{\zeta(2)} + O\left(\frac{x}{x - \frac12}\right) = \frac{x}{\zeta(2)} + \O{ 1 }.$$
\end{proof}

\textbf{Problem 3.7}: Show that if (i) $g \in A$, (ii) $\sum_{n=1}^{\infty} |g(n)|/n$ converges, and (iii) $f = 1 * g$, then
$$\lim_{N \rightarrow \infty} \frac{1}{N} \sum_{n=1}^{N} f(n) = \sum_{n=1}^{\infty} \frac{g(n)}{n}.$$

\begin{proof}
Let $F(x)$ denote the summatory function of $F$ so the expression under limit is basically $\frac{F(N)}{N}$. We have
\begin{align*}
\frac{F(N)}{N} &= \frac1N \left(\sum_{m \leq N} \IntPart{\frac{N}{m}} g(m) \right)\\
&= \frac1N \left(\sum_{m \leq N} \frac{N}{m} g(m) + \sum_{m \leq N} \left\{ \IntPart{\frac{N}{m}} - \frac{N}{m} \right\} g(m) \right)\\
&= \sum_{m \leq N} \frac{g(m)}{m} + \underbrace{\frac1N \sum_{m \leq N} \left\{ \IntPart{\frac{N}{m}} - \frac{N}{m} \right\} g(m)}_{R(N)}.
\end{align*}
so it suffices to show that
$$\lim_{N \rightarrow \infty} R(N) = 0.$$

Bounding by absolute values, the above could be accomplished by showing that
$$\lim_{N \rightarrow \infty} \frac1N \sum_{m=1}^{N} |g(m)| = 0.$$

Let $G(x) = \sum_{1 \leq m \leq x} |g(m)|$ be the summatory function of $|g|$ so the above equation becomes 
$$\lim_{N \rightarrow \infty} \frac{G(N)}{N} = 0.$$

The left hand side is a reminiscent of \emph{the density of a set}: If $g = 1_A$ then $G(x) = A(x)$ and showing the above assuming (ii) is the content of Problem 1.6. So we duplicate the approach:
\begin{align*}
\sum_{n = 1}^N \frac{|g(n)|}{n} &= \sum_{n = 1}^{N} \frac{G(n) - G(n-1)}{n}\\
&= \sum_{n = 1}^{N - 1} G(n) \left(\frac{1}{n} - \frac{1}{n+1}\right) + \frac{G(N)}{N}\\
&= \sum_{n = 1}^{N - 1} \frac{G(n)}{n(n+1)} + \frac{G(N)}{N}
\end{align*}

By assumption (ii), the LHS converges as $N \rightarrow \infty$ so
$$B_N = \sum_{n = 1}^{N - 1} \frac{G(n)}{n(n+1)}$$
must converge as well because it is an increasing sequence and is bounded above by the limit of the LHS because $\frac{G(N)}{N} \geq 0$. This proves $\frac{G(N)}{N}$ converges\footnote{Note that it is easy to show that $\frac{G(N)}{N}$ is bounded, for example, by observing that $\frac{G(N)}{N} = \sum \frac{|g(m)|}{N} \leq \sum \frac{|g(m)|}{m}$ but we cannot say anything about its monotonicity.}, say to $\alpha \geq 0$.

Note that this implies $\frac{G(x)}{x}$ converges to the same limit $\alpha$ because
$$\frac{G(x)}{x} = \frac{G([x])}{[x]} \cdot \frac{[x]}{x}$$
and clearly
$$\frac{[x]}{x} \rightarrow 1$$
from obvious squeeze
$$\frac{x - 1}{x} \leq \frac{[x]}{x} \leq 1.$$
A generalization to Dirichlet-Dedekind's Theorem 1.8 should solve the problem as well. But we shall give a direct argument here.

Suppose $\alpha > 0$. Take $\epsilon = \frac{\alpha}{2} > 0$ in the definition of limit and we find that $\frac{G(N)}{N} >  \frac{\alpha}{2}$ for large $N > N_0$ whence the sequence $B_N$ diverges:
$$\sum_{n = N_0}^{\infty} \frac{G(n)}{n(n+1)} > \sum_{n = N_0}^{\infty} \frac{\alpha}{2(n+1)} = \frac{\alpha}{2} \sum_{n = N_0 + 1}^{\infty} \frac{1}{n}.$$
Therefore, $\alpha = 0$.
\end{proof}

\textbf{Problem 3.8}: Show that $\sum_{n \leq x} \lambda(n)/n = \O{ 1 }$ by using (a) $\lambda * 1 = 1_S$, $S$ the set of squares, (b) $\lambda * |\mu| = e$.

\begin{proof}
From (a), we have
\begin{align*}
S(x) &= \sum_{n \leq x} 1_S(n)\\
&= \sum_{m \leq x} \lambda(m) \IntPart{\frac x m}\\
&= \sum_{m \leq x} \lambda(m) \frac x m + \sum_{m \leq x} \lambda(m) \left\{\IntPart{\frac x m} - \frac x m \right\}
\end{align*}
so
\begin{align*}
\Abs{\sum_{m \leq x} \frac{\lambda(m)}{m}} &= \Abs{\frac{S(x) - \sum_{m \leq x} \lambda(m) \left\{\IntPart{\frac x m} - \frac x m \right\}}{x}}\\
&\leq \frac{\Abs{S(x)} + \sum_{m \leq x} \Abs{\lambda(m)} \cdot \Abs{\IntPart{\frac x m} - \frac x m}}{x}\\
&\leq \frac{S(x) + x}{x}
\end{align*}
is clearly bounded if one recalls the density of the set of squares is zero, that is $\frac{S(x)}{x} \rightarrow 0$ as $x \rightarrow \infty$.

From (b), we get
$$1 = \sum_{m \leq x} \lambda(m) Q\left(\frac{x}{m}\right)$$
where we recall $Q(x) = \sum_{n \leq x} |\mu(n)|$ is the summatory function of $|\mu|$, which also counts the number of square-free numbers $\leq x$. Theorem 3.5 suggests us to rewrite the equation as
\begin{align*}
1 &= \sum_{m \leq x} \lambda(m) \frac{x}{m \zeta(2)} + \sum_{m \leq x} \lambda(m) \left\{Q\left(\frac{x}{m}\right) - \frac{x}{m \zeta(2)}\right\}\\
&= \frac{x}{\zeta(2)} \sum_{m \leq x} \frac{\lambda(m)}{m} + \sum_{m \leq x} \lambda(m) \left\{Q\left(\frac{x}{m}\right) - \frac{x}{m \zeta(2)}\right\}
\end{align*}
A simple rearrangement yields
$$\sum_{m \leq x} \frac{\lambda(m)}{m} = \frac{\zeta(2)}{x} \left( 1 - \sum_{m \leq x} \lambda(m) \left\{Q\left(\frac{x}{m}\right) - \frac{x}{m \zeta(2)}\right\} \right)$$
and we are ready to take absolute value
\begin{align*}
\Abs{\sum_{m \leq x} \frac{\lambda(m)}{m}} &\leq \frac{\zeta(2)}{x} \left(1 + \sum_{m \leq x} \Abs{\lambda(m)} \cdot \Abs{Q\left(\frac{x}{m}\right) - \frac{x}{m \zeta(2)}} \right)\\
&\leq \frac{\zeta(2)}{x} \left( 1 + \sum_{m \leq x} 3 \sqrt{\frac{x}{m}} \right) &\text{by Theorem 3.5}\\
&\leq \frac{\zeta(2)}{x} + \frac{3\zeta(2)}{\sqrt{x}} \left(\sum_{m \leq x} m^{-1/2}\right)
\end{align*}
so if we can show that
$$\sum_{m \leq x} m^{-1/2}$$
is $\O { \sqrt{x} }$ then we are done. This can be accomplished by noting that $t^{-1/2}$ is a convex function so by the argument in Theorem 3.5:
$$\sum_{m \leq x} m^{-1/2} \leq \int_{\frac 12}^{x + \frac 12} t^{-1/2} dt = \left. 2 \sqrt{t} \right|_{\frac 12}^{x + \frac 12} < 2 \sqrt{x + \frac 12}.$$
\end{proof}

\unless\ifdefined\IsMainDocument
\end{document}
\fi

\unless\ifdefined\IsMainDocument
\documentclass{article}
\usepackage{amsmath,amsthm}

\newcommand{\Abs}[1]{\left| #1 \right|}

\begin{document}
\fi

\textbf{Problem 3.9}: Estimate $\sum_{n \leq x} |\mu(n)| \log n$ using Theorem 3.5 and an integration by parts of an appropriate RS integral.

\begin{proof}
By example 3.9, let $g(x) = \log x$ and $F(x) = \sum_{n \leq x} |\mu(n)| = Q(x)$ in Theorem 3.5 then we have
$$\sum_{n \leq x} |\mu(n)| \log n = \int_1^x g \; dF.$$
(Note that $|\mu(1)| \log 1 = 0$ so the sum on the left starts from 2.)

Integration by parts gives
\begin{align*}
\int_1^x g \; dF &= \left. g(t) F(t) \right|_1^x - \int_1^x F \; dg\\
&= g(x) F(x) - \int_1^x F(t) \; d \log t\\
&= F(x) \log x - \int_1^x F(t) \; \frac{1}{t} dt &\text{ from Example 3.10}\\
\end{align*}
so using Theorem 3.5:
\begin{align*}
\Abs{ \int_1^x g \; dF - \frac{6x}{\pi} \log x - \underbrace{\int_1^x \frac{6t}{\pi} \frac{1}{t} dt}_{\frac{6}{\pi} (x - 1) } } &= \Abs{ \left(F(x) - \frac{6x}{\pi} \right)\log x - \int_1^x \left( F(t) - \frac{6t}{\pi} \right) \; \frac{1}{t} dt }\\
&\leq \Abs{ \left(F(x) - \frac{6x}{\pi} \right)\log x } + \int_1^x \Abs { F(t) - \frac{6t}{\pi} } \frac{1}{t} dt\\
&\leq 3 \sqrt{x} \log x + \int_1^x 3 \sqrt{t} \frac{1}{t} dt\\
&\leq 3 \sqrt{x} \log x + \left. 6 \sqrt{t} \right|_1^x\\
&\leq 3 \sqrt{x} \log x + 6 (\sqrt{x} - 1).
\end{align*}

Cleaning up:
$$\Abs{ \sum_{n \leq x} |\mu(n)| \log n - \frac{6}{\pi} x (\log x + 1) } \leq 3 \sqrt{x} ( \log x + 2 ).$$ 
\end{proof}

\textbf{Problem 3.10}: Using the estimate $Q(x) > x/2$ for all $x \geq 1$, show that
$$\sum_{n \leq x} |\mu(n)|/n > \frac 12 + \frac 12 \log x, \qquad x \geq 1.$$

\begin{proof}
We apply Lemma 3.11 for $g(x) = \frac{1}{x}$, $\varphi(x) = Q(x)$ and
$$F(x) = \begin{cases}
x/2 &\text{if } x \geq 1,\\
0 &\text{if } x < 1.
\end{cases}$$
It is obvious that these satisfy the assumptions of the lemma. Thus, we get
$$\int_{1^-}^x g \; dF \leq \int_{1^-}^x g \; d\varphi.$$

By Example 3.9, the RHS is exactly
$$\sum_{n \leq x} |\mu(n)|/n$$
whereas the LHS can be evaluated as:
\begin{align*}
\int_{1^-}^x g \; dF &= \lim_{\delta \rightarrow 0^+} \int_{1 - \delta}^x g d F\\
&= \lim_{\delta \rightarrow 0^+} \int_{1 - \delta}^1 g d F + \underbrace{\int_1^x \frac{1}{t} d \left(\frac{t}{2}\right)}_{\frac 12 \log x}
\end{align*}

To evaluate the remaining term, we use Lemma 3.8. Observe that $g$ and $F$ satisfies the assumptions of Theorem 3.7 with $a = \frac 12$ and $b = 2$ so let us set $\psi(x) = \int_{\frac 12}^x g dF$ then
\begin{align*}
\lim_{\delta \rightarrow 0^+} \int_{1 - \delta}^1 g d F &= \psi(1) - \psi(1-)\\
&= g(1) (F(1) - F(1^-)) &\text{by Lemma 3.8}\\
&= 1 \cdot \left(\frac12 - 0\right)\\
&= \frac 12
\end{align*}
and so we get the desired inequality.
\end{proof}

\textbf{Problem 3.11}: By iterating the method of the preceeding lemma, show that the series
$$1 - 2 \cdot 2^{-s} + 3^{-s} + 4^{-s} - 2 \cdot 5^{-s} + 6^{-s} + 7^{-s} - 2 \cdot 8^{-s} + 9^{-s} + ...$$
converges and is nonnegative for any real positive $s$.

\begin{proof}
I am assuming that the series is
$$\sum_{n=1}^{\infty} f(n) \cdot n^{-s}$$
where
$$f(n) = \begin{cases}
1 &\text{if } n \equiv 0, 1 \mod 3,\\
-2 &\text{if } n \equiv 2 \mod 3.
\end{cases}.$$

Our goal is to show that the series is Cauchy i.e. for every $\epsilon > 0$, there is an $N$ such that if $b > a > N$ then
$$\Abs{ \sum_{n=a + 1}^{b} f(n) \cdot n^{-s} } < \epsilon.$$

To do that, we go back to the integration by parts formula. Let $g(t) = t^{-s}$ and
$$F(t) = \sum_{n \leq t} f(n) = \begin{cases}
1 &\text{if } [t] \equiv 1 \mod 3,\\
0 &\text{if } [t] \equiv 0 \mod 3,\\
-1 &\text{if } [t] \equiv 2 \mod 3,
\end{cases}$$
then
\begin{align*}
\Abs{ \sum_{n=a + 1}^{b} f(n) \cdot n^{-s} } &= \Abs{ \int_a^b g \; dF }\\
&= \Abs{ \left. g(t) F(t) \right|_a^b - \int_a^b F \; dg } &\text{integration by parts}\\
&= \Abs{ \left. g(t) F(t) \right|_a^b - \int_a^b F \cdot (-s t^{-s-1}) \; dt } &\text{as } dg = g' dt\\
&\leq \Abs{g(b) F(b)} + \Abs{g(a) F(a)} + \int_a^b s |F| t^{-s-1} dt\\
&\leq b^{-s} + a^{-s} + \int_a^b s t^{-s-1} dt &\text{since } |F| \leq 1\\
%&\leq b^{-s} + a^{-s} + a^{-s} - b^{-s}\\
&= 2a^{-s} \leq 2N^{-s}
\end{align*}
so to bound the above by $\epsilon$, we can simply choose $N$ so that $2 N^{-s} < \epsilon$ which is possible as $N^{-s} \rightarrow 0$ as $N \rightarrow \infty$.

Now, to show that the limit is nonnegative, we just have to note that the sum of every three consecutive term is nonnegative because $g(t)$ is convex:
$$(3k+1)^{-s} + (3k+3)^{-s} \geq 2 \left( \frac{3k+1 + 3k+3}{2} \right)^{-s} = 2(3k+2)^{-s}.$$
\end{proof}

\unless\ifdefined\IsMainDocument
\end{document}
\fi

\unless\ifdefined\IsMainDocument
\documentclass[12pt]{article}
\usepackage{amsmath,amsthm}
\newcommand{\Abs}[1]{\left|#1\right|}
\begin{document}
\fi

\textbf{Problem 3.12}: Let $a_n = -1$ if $n$ is not a square and $a_n = 2 \sqrt{n} - 1$ if $n$ is a square. Prove that $\sum_{n=1}^{\infty} a_n / n = \gamma$.

\begin{proof}
Observe that $a_n = b_n - 1$ where
$$b_n = 2 \sqrt{n} \cdot 1_S(n) = \begin{cases}
0 &\text{if }n \text{ is not a square},\\
2\sqrt{n} &\text{if } n \text{ is a square}.
\end{cases}$$
So we can write
\begin{align*}
\sum_{n \leq x} \frac{a_n}{n} &= \sum_{n \leq x} \frac{b_n - 1}{n}\\
&= \sum_{n \leq x} \frac{b_n}{n} - \sum_{n \leq x} \frac{1}{n}\\
&= \sum_{m^2 \leq x} \frac{2 m}{m^2} - \sum_{n \leq x} \frac{1}{n}\\
&= 2 \sum_{m \leq \sqrt{x}} \frac{1}{m} - \sum_{n \leq x} \frac{1}{n}\\
&= 2 \left(\log \sqrt{x} + \gamma + \frac{\theta_{\sqrt{x}}}{\sqrt{x}} \right) - \left(\log x + \gamma + \frac{\theta_x}{\sqrt{x}} \right)
\end{align*}
by Lemma 3.13 where $\theta_{\sqrt{x}}$ and $\theta_x$ are some numbers with absolute values $\leq 1$. Simplifying, one has
$$\sum_{n \leq x} \frac{a_n}{n} = \gamma + \frac{\theta_{\sqrt{x}}}{\sqrt{x}} - \frac{\theta_x}{\sqrt{x}}$$
where the extra terms clearly tends to 0 as $x \rightarrow +\infty$.
\end{proof}

\textbf{Problem 3.13}: For each $k \in Z^+$ show that
$$\sum_{n \leq x} \log^k n = \int_1^x \log^k t \, dt + O(\log^k x).$$
The constant implied by the $O$-symbol is uniform with respect to $k$.

\begin{proof}
Note that the sum starts from $n = 2$. We use Euler summation formula for the function $f(t) = \log^k t$ and $c = 0$:
\begin{align*}
LHS &= \left. \int_1^x f(t) \, dt + (t - [t]) f(t) \, \right|_1^x + \int_1^x (t - [t]) \, f'(t) \, dt\\
&= \int_1^x \log^k t \, dt + (x - [x]) \log^k x + \int_1^x (t - [t]) \, k \log^{k-1} t \, \frac{dt}{t}
\end{align*}
so
\begin{align*}
\Abs{ LHS - \int_1^x \log^k t \, dt } &\leq \Abs{ (x - [x]) \log^k x } + \int_1^x \Abs{ (t - [t]) \, k \frac{\log^{k-1} t}{t} } \, dt\\
&\leq \log^k x + \int_1^x f'(t) dt &\text{ since } 0 \leq t - [t] < 1\\
&\leq 2 \log^k x.
\end{align*}
\end{proof}

\unless\ifdefined\IsMainDocument
\end{document}
\fi

\unless\ifdefined\IsMainDocument
\documentclass{article}
\usepackage{amsmath,amsthm,amssymb,hyperref}

\newcommand{\C}{\mathbb{C}}
\newcommand{\Abs}[1]{\left| #1 \right|}

\begin{document}
\fi

\textbf{Problem 3.14}: For each $k \in Z^+$ show that
$$\gamma_k := \lim_{N \rightarrow \infty} \left\{ \sum_{n=1}^{N} \frac{\log^k n}{n} -  \frac{\log^{k+1} N}{k+1} \right\}$$
exists and that $\gamma_k = O(k!)$. Assuming the fact (c.f. Theorem 8.1) that $\zeta(s) - 1/(s-1)$ is entire, show that $\gamma_k = O_R(R^{-k} k!)$ holds for any $R > 0$.

\begin{proof}
Let us denote
$$b_{k,N} = \sum_{n=1}^{N} \frac{\log^k n}{n} -  \frac{\log^{k+1} N}{k+1}.$$
Let $f(t) = \frac{\log^k t}{t}$ so we have by Euler summation formula where $c = 0$ (note that the term for $n = 1$ in the sum is zero so we can use the integral from 1):
\begin{align*}
\sum_{n=1}^{N} \frac{\log^k n}{n} &= \left. \int_1^N f(t) \, dt - (t - [t]) f(t) \right|_1^N + \int_1^N (t - [t]) \, df\\
&= \frac{\log^{k+1} N}{k+1} + \int_1^N (t - [t]) \, \frac{(k \log^{k-1} t) \cdot \frac 1t \cdot t - \log^k t}{t^2} \, dt
\end{align*}
and so
$$b_{k,N} = \int_1^N \frac{(t - [t]) (k - \log t) \log^{k-1} t}{t^2} \, dt.$$

To show the limit exists, we can show that the sequence is Cauchy. Let $\epsilon > 0$ be arbitrary and we want to construct $N$ such that for any $N_2 > N_1 > N$, we have
$$\Abs{b_{k,N_2} - b_{k,N_1}} = \Abs{ \int_{N_1}^{N_2} \frac{(t - [t]) (k - \log t) \log^{k-1} t}{t^2} \, dt } < \epsilon.$$

One has
\begin{align*}
LHS \leq&\int_{N_1}^{N_2} \Abs{ \frac{(t - [t]) (k - \log t) \log^{k-1} t}{t^2} } \, dt\\
=& \int_{N_1}^{N_2} \frac{(t - [t]) (\log t - k) \log^{k-1} t}{t^2} \, dt &\text{assuming } N_1, N_2 > e^k \Rightarrow \log t > k\\
\leq& \int_{N_1}^{N_2} \frac{(\log t - k) \log^{k-1} t}{t^2} \, dt & \text{for } 0 \leq t - [t] < 1\\
=& \int_{N_1}^{N_2} - f'(t) dt = f(N_1) - f(N_2)\\
\leq& f(N)
\end{align*}
assuming $N > e^k$ for then $f$ is decreasing and nonnegative on $[e^k, +\infty)$ so to bound it by $\epsilon$, we have to choose $N > e^k$ such that $f(N) < \epsilon$. That is clearly possible since $\log^k t$ grows slower than $t$ i.e. $f(t) \rightarrow 0$ as $t \rightarrow +\infty$ (by L'Hospital rule).

So we showed that the limit exists and that it could be given by the integral
$$\gamma_k = \int_1^\infty \frac{(t - [t]) (k - \log t) \log^{k-1} t}{t^2} \, dt.$$

To prove that $\gamma_k = O(k!)$, let us compute the closed form for the functions on $(0, \infty)$:
$$a_k(x) := \frac{1}{k!} \int_x^\infty \frac{\log^k t}{t^2} \, dt$$
for any $k \geq 0$:
\begin{align*}
a_k(x) &= \frac{1}{k!} \int_x^\infty \frac{1}{t} \log^k t \, d(\log t)\\
&= \frac{1}{k!} \int_x^\infty \frac{1}{t} \, d \left( \frac{ \log^{k+1} t }{k+1} \right)\\
&= \frac{1}{k!} \left. \frac{1}{t} \frac{ \log^{k+1} t }{k+1} \right|_x^{\infty} - \frac{1}{k!} \int_x^\infty \frac{ \log^{k+1} t }{k+1} d \left( \frac{1}{t} \right) &\text{integration by parts}\\
&= -\frac{\log^{k+1} x}{(k+1)! x} + \frac{1}{(k+1)!} \int_x^\infty \frac{ \log^{k+1} t }{t^2} \, dt &\text{as } d \left( \frac{1}{t} \right) = -\frac{1}{t^2} dt
\end{align*}
so the sequence of functions $a_k(x)$ satisfies the recursive relation
$$a_{k+1}(x) = a_k(x) + \frac{ \log^{k+1} x }{(k+1)! x}$$
with the initial condition
$$a_0(x) = \int_x^\infty \frac{1}{t^2} \, dt = \left. -\frac{1}{t} \right|_x^{\infty} = \frac{1}{x}$$
Therefore,
$$a_k(x) = \frac{1}{x} \sum_{j = 0}^k \frac{\log^j x}{j!} = \frac{P_k(\log x)}{x}.$$
where
$$P_n(X) := \sum_{j = 0}^{n} \frac{X^j}{j!}$$
are the Taylor polynomials at $X = 0$ for the function $h(X) = e^X$. In particular, we have
\begin{itemize}
\item $a_k(1) = 1$ for all $k$
\item If $x > 1$ then
$$a_k(x) = \frac{P_k(\log x)}{x} < \frac{e^{\log x}}{x} = 1$$
for then $\log x > 0$ and obviously $P_k(x) < e^x$ for all $x > 0$.
\end{itemize}

Now we estimate
\begin{align*}
\Abs{\gamma_k} =& \Abs{ \int_1^\infty \frac{(t - [t]) (k - \log t) \log^{k-1} t}{t^2} \, dt }\\
%\leq &\int_1^\infty \frac{(t - [t]) k \log^{k-1} t}{t^2} \, dt + \int_1^\infty \frac{(t - [t])  \log^k t}{t^2} \, dt\\
\leq &\underbrace{\int_1^\infty \frac{k \log^{k-1} t}{t^2} \, dt}_{k! a_{k-1}(1)} + \underbrace{\int_1^\infty \frac{\log^k t}{t^2} \, dt}_{k! a_k(1)}\\
= & 2k!
\end{align*}

The final bound $\gamma_k = O(R^{-k} k!)$ is a reminiscence of Cauchy estimate: If $f$ is a holomorphic function on the region containing the disc $|z - z_0| \leq R$ then the Cauchy integral formula gives
$$f^{(k)} (z) = \dfrac{k!}{2 \pi i} \int_C \dfrac{f(w)}{(w - z)^{k + 1}} dw$$
where $C$ is the circle $|z - z_0| = R$ oriented counter clockwise so we get as a simple bound
$$|f^{(k)}(z)| \leq \frac{k!}{2 \pi} 2 \pi R \frac{M}{R^{k+1}} = M R^{-k} k!$$
where $M = \max\{|f(w)| : w \in \C\}$.

So the problem is solved if we can show that $\gamma_k$ are higher derivatives\footnote{Not quite but close.} of some holomorphic function and one candidate for such a function should be none other than the one mentioned $\zeta(s) - 1 / (s - 1)$. We expect the Taylor expansion at $s = 1$ so for $s > 1$, we can write
\begin{align*}
\zeta(s) &= \sum_{n = 1}^{\infty} n^{-s}\\
&= \sum_{n = 1}^{\infty} n^{-1} (n^{1-s})\\
&= \sum_{n = 1}^{\infty} \frac{1}{n} \, \sum_{k = 0}^{\infty} \frac{((1 - s) \log n)^k}{k!} &\text{as } n^{1-s} = e^{(1-s)\log n}\\
&= \sum_{n = 1}^{\infty} \sum_{k = 0}^{\infty} \frac{(\log n)^k}{n} \frac{(1 - s)^k}{k!}
\end{align*}

If we could swap the two infinite series then we get
$$\zeta(s) = \sum_{k = 0}^{\infty} \left( \sum_{n = 1}^{\infty} \frac{\log^k n}{n} \right) \frac{(1 - s)^k}{k!}$$
which looks like the power series centered at 1 and something like the $\gamma_k$ appears as the coefficients. But of course, this cannot be true since $\zeta(s)$ has a simple pole at 1 so we are expecting a term $(s-1)^{-1}$ here. On top of that, the inner series does not converge.

To rectify the idea, we have to insert the comparison term to ensure convergence. Write the outer series as the limit to allow swapping
\begin{align*}
\zeta(s) &= \lim_{N \rightarrow \infty} \sum_{n = 1}^{N} \sum_{k = 0}^{\infty} \frac{(\log n)^k}{n} \frac{(1 - s)^k}{k!} \\
&= \lim_{N \rightarrow \infty} \sum_{k = 0}^{\infty} \sum_{n = 1}^{N} \frac{(\log n)^k}{n} \frac{(1 - s)^k}{k!} \\
&= \lim_{N \rightarrow \infty} \sum_{k = 0}^{\infty} b_{k,N} \frac{(1 - s)^k}{k!} + \frac{\log^{k+1} N}{k+1} \frac{(1 - s)^k}{k!}\\
&= \lim_{N \rightarrow \infty} \sum_{k = 0}^{\infty} b_{k,N} \frac{(1 - s)^k}{k!} + \lim_{N \rightarrow \infty} \sum_{k = 0}^{\infty} \frac{\log^{k+1} N}{k+1} \frac{(1 - s)^k}{k!}
\end{align*}

The second limit is easily resolved:
\begin{align*}
\lim_{N \rightarrow \infty} \sum_{k = 0}^{\infty} \frac{\log^{k+1} N}{k+1} \frac{(1 - s)^k}{k!} &= \frac{1}{1 - s} \cdot \lim_{N \rightarrow \infty} \sum_{k = 0}^{\infty} \frac{((1 - s) \log N)^{k+1}}{(k+1)!}\\
&= \frac{1}{1 - s} \cdot \lim_{N \rightarrow \infty} \sum_{k = 1}^{\infty} \frac{((1 - s) \log N)^{k}}{k!}\\
&= \frac{1}{1 - s} \cdot \lim_{N \rightarrow \infty} (e^{(1 - s) \log N} - 1)\\
&= \frac{1}{s - 1}
\end{align*}
since under the assumption that $s > 1$, $(1 - s) \log N \rightarrow -\infty$ as $N \rightarrow \infty$ so $(e^{(1 - s) \log N} - 1) \rightarrow -1$. That takes care of the pole at $s = 1$.

It remains to show that the limit $\lim_{N \rightarrow \infty}$ can be swapped with the series $\sum_{k=0}^{\infty}$ in the first term so that we get
$$\lim_{N \rightarrow \infty} \sum_{k = 0}^{\infty} b_{k,N} \frac{(1 - s)^k}{k!} = \sum_{k = 0}^{\infty} \gamma_k \frac{(1 - s)^k}{k!}$$
and so we acquires the Taylor series for $Z(s) = \zeta(s) - \frac{1}{s-1}$ that works for all $s > 1$. (I would not managed for all $s > 1$ but for all $1 < s < 2$ but that is good enough.) But then the knowledge that $Z(s)$ is entire allows us to conclude that
$$Z(s) = \sum_{k = 0}^{\infty} (-1)^k \gamma_k (s - 1)^k$$
holds for all $s \in \C$ and not just $s > 1$ due to uniqueness of power series expansion. As a result,
$$\gamma_k = (-1)^k Z^{(k)}(1)$$
and we solved the problem. So now we have to show that for every $\epsilon > 0$, we can choose $N_0$ so that for all $N > N_0$,
$$\Abs{\sum_{k=0}^{\infty} \left( b_{k,N} - \gamma_k \right) \frac{(1 - s)^k}{k!} } < \epsilon.$$
Note that given  $b_{k,0} \rightarrow \gamma_0$, it suffices to do so for the tail series
$$\Abs{\sum_{k=1}^{\infty} \left( b_{k,N} - \gamma_k \right) \frac{(1 - s)^k}{k!} } < \epsilon.$$

Recall that for all $k \geq 1$:
$$\frac{b_{k,N} - \gamma_k}{k!} = \int_N^\infty \varphi_k \, dt$$
where
$$\varphi_k(t) = \frac{t - [t]}{t^2} \left( \frac{\log^k t}{k!} - \frac{\log^{k-1} t}{(k-1)!} \right).$$

So using a similar trick of writing series as limit in order to swap the integral outside, we have for any fixed $s > 1$:
\begin{align*}
&\sum_{k=1}^{\infty} \left( b_{k,N} - \gamma_k \right) \frac{(1 - s)^k}{k!}\\
=& \lim_{K \rightarrow \infty} \sum_{k=1}^{K} (1 - s)^k  \int_N^\infty \varphi_k \, dt\\
=& \lim_{K \rightarrow \infty} \int_N^\infty \sum_{k=1}^{K} (1 - s)^k  \varphi_k \, dt\\
=& \lim_{K \rightarrow \infty} \int_N^\infty \sum_{k=1}^{K} (1 - s)^k  \frac{t - [t]}{t^2} \left( \frac{\log^k t}{k!} - \frac{\log^{k-1} t}{(k-1)!} \right) \, dt\\
=& \lim_{K \rightarrow \infty} \int_N^\infty \frac{t - [t]}{t^2} \left(\sum_{k=1}^{K} \frac{(1 - s)^k \log^k t}{k!} - (1 - s) \sum_{k=1}^{K} \frac{(1 - s)^{k-1} \log^{k-1} t}{(k-1)!} \right) \, dt\\
=& \lim_{K \rightarrow \infty} \int_N^\infty \frac{t - [t]}{t^2} \left( \underbrace{\sum_{k=1}^{K} \frac{(1 - s)^k \log^k t}{k!}}_{P_K((1-s)\log t) - 1} - (1 - s) \underbrace{\sum_{k=0}^{K-1} \frac{(1 - s)^{k} \log^k t}{k!}}_{P_{K-1}((1-s)\log t)} \right) \, dt.
\end{align*}

Now the intuition\footnote{Not actually true.} from the computation of the other limit applies: As $N \rightarrow \infty$, $\log t$ becomes very large and so $(1 - s) \log t \rightarrow -\infty$. So $P_K((1-s)\log t) \approx e^{(1-s) \log t} \rightarrow 0$ and the integral inside the limit is approximately
$$\int_N^\infty \frac{t - [t]}{t^2} (-1) dt$$
which can be made arbitrarily close to 0 as $N \rightarrow \infty$ because $\int_1^\infty \frac{t - [t]}{t^2} dt$ exists. To do it formally, we use the exact statement of Taylor's theorem which generalizes the Mean Value Theorem. It says that for every $X \in R$, there exists $\xi$ between 0 and $X$ such that
$$h(X) - P_n(X) = \frac{1}{(n+1)!} \; h^{(n+1)}(\xi) \; (X - 0)^{n+1} = \frac{e^{\xi} \; X^{n+1}}{(n+1)!}.$$
In particular, if $X < 0$ then $\xi < 0$ so $e^\xi < 1$ and we have a bound
$$|P_n(X)| \leq e^X + \frac{|X|^{n+1}}{(n+1)!}$$
for all $n$. The fact that it works for all $n$ is important.

In particular, apply the result to $X = (1 - s) \log t < 0$, we find
\begin{align*}
&\Abs{P_K((1-s)\log t) - 1 + (s - 1) P_{K-1}((1-s)\log t)}\\
\leq& \Abs{P_K((1-s)\log t)} + 1 + (s - 1) \Abs{P_{K-1}((1-s)\log t)}\\
\leq& \left\{ e^{(1-s) \log t} + \frac{((s-1)\log t)^{K+1}}{(K+1)!} \right\} + 1 + (s-1) \left\{ e^{(1-s) \log t} + \frac{((s-1)\log t)^{K}}{(K)!} \right\}\\
=& s \cdot e^{(1-s) \log t} + 1 + (s - 1)^{K+1} \left\{\frac{\log^{K+1} t}{(K+1)!} + \frac{\log^K t}{(K)!} \right\}\\
\leq& 2 + (s - 1)^{K+1} \left\{\frac{\log^{K+1} t}{(K+1)!} + \frac{\log^K t}{(K)!} \right\}
\end{align*}
\underline{if we choose $N$ large enough so that $e^{(1 - s) \log t} < \frac{1}{s}$}. Now for every $K$, we can bound
\begin{align*}
&\Abs{\int_N^\infty \frac{t - [t]}{t^2} \left\{ P_K((1-s)\log t) - 1 + (s - 1) P_{K-1}((1-s)\log t) \right\} \, dt}\\
\leq& \int_N^{\infty} \frac{t - [t]}{t^2} \left(2 + (s - 1)^{K+1} \left\{\frac{\log^{K+1} t}{(K+1)!} + \frac{\log^K t}{(K)!} \right\} \right) dt\\
= &2 \int_N^{\infty} \frac{t - [t]}{t^2} dt + (s - 1)^{K+1} \int_N^{\infty} \frac{1}{t^2} \left\{\frac{\log^{K+1} t}{(K+1)!} + \frac{\log^K t}{(K)!} \right\} dt\\
= &2 \int_N^{\infty} \frac{t - [t]}{t^2} dt + (s - 1)^{K+1} (\underbrace{a_{K+1}(N) + a_K(N)}_{\leq 2})\\
\leq & 2 \int_N^{\infty} \frac{t - [t]}{t^2} dt + 2 (s - 1)^{K + 1}
\end{align*}
So if we assume further that $s - 1 < 1$ then taking the limit $K \rightarrow \infty$ yields
$$\lim_{K \rightarrow \infty} \Abs{\int_N^\infty \frac{t - [t]}{t^2} \left\{ P_K((1-s)\log t) - 1 + (s - 1) P_{K-1}((1-s)\log t) \right\} \, dt} \leq 2 \int_N^{\infty} \frac{t - [t]}{t^2} dt$$
and \underline{the later can be made arbitrarily small by choosing $N$ large enough}.

To sum up, we showed that for every fixed $s$ such that $1 < s < 2$ then for any $\epsilon > 0$, if we choose $N$ large enough such that
\begin{itemize}
\item $e^{(1-s) \log t} < \frac{1}{s}$ for all $t > N$
\item the integral
$$\int_N^{\infty} (t - [t]) t^{-2} dt < \frac{\epsilon}{2}$$
\end{itemize}
then
$$\Abs{\sum_{k = 1}^{\infty} (b_{k,N} - \gamma_k) \frac{(1 - s)^k}{k!}} < \epsilon.$$

So the equation
$$Z(s) = \sum_{k=0}^{\infty} (-1)^k \gamma_k (s - 1)^k$$
holds for all $1 < s < 2$. Then since $Z(s)$ is known to be entire, this series works for all $s \in \C$.

\textbf{Note}: The numbers $\gamma_k$ are known in the literature as \href{https://en.wikipedia.org/wiki/Stieltjes\_constants}{Stieltjes' Constants}.
\end{proof}

\unless\ifdefined\IsMainDocument
\end{document}
\fi

\unless\ifdefined\IsMainDocument
\documentclass{article}
\usepackage{amsmath,amsthm}

\begin{document}
\fi

\textbf{Problem 3.l5}: Let $S$ denote a finite union of intervals of the form $(a, b]$ and $dA$ any integrator. Establish the following useful convolution identities:
\begin{enumerate}
\item $\delta_1 * dA = dA$
\item $dA * t^{-1} dt = A(t) t^{-1} dt$
\item $(dN * dN)(S) = \sum_{n \in S} \tau(n) =: d N_2(S)$
\item $(dN * dQ)(S) = \sum_{n \in S} 2^{\omega(n)}$
\item $dt * dt = \log t \, dt$
\item $dN * dM = \delta_1$
\item $(\delta_1 + dt) * (\delta_1 - t^{-1} dt) = \delta_1$
\item $\int_{1^-}^x (t^{-1}dN) * (t^{-1}dt) = \frac12 \log^2 x + \gamma \log x - \gamma_1 + O(x^{-1})$.
\end{enumerate}

Note: The problem did not define $Q$ and $M$ but they should be the summatory function of $|\mu|$ and $\mu$ respectively.

\begin{proof}
\begin{enumerate}
\item Remember $\delta_1 = d E_1$ where $E_1$ is the summatory function of $e_1$. Therefore, $\delta * dA = dH$ where for $x \geq 1$:
\begin{align*}
H(x) &= \int_{1^-}^x E_1(x/t) dA(t)\\
&= \int_{1^-}^x dA(t)\\
&= A(x) - A(1^-)
\end{align*}
so $dH = dA$. The second line is because if $0 < 1 - \delta \leq t \leq x$ then $1 \leq x/t$ so we have $E(x/t) = 1$.

\item Recalling our experience with example 3.22, we first need to understand what $t^{-1} dt$ is. Going back to page 52, we are led to $t^{-1} dt = d F$ where
$$F(x) = \int_1^x f(t) dt$$
where $f$ needs to be supported in $[1, \infty)$ so
$$f(t) = \begin{cases}
0 &\text{if } t < 1,\\
t^{-1} &\text{if } t \geq 1.
\end{cases}$$
In other words, $f(t) = t^{-1} \delta_1 (t)$. We easily recognize then that
$$F(x) = \begin{cases}
0 &\text{if } x < 1,\\
\log x &\text{if } x \geq 1.
\end{cases}$$
Note that $F$ is continuous. Now we have $dA * t^{-1} dt = d H$ where for $x \geq 1$:
\begin{align*}
H(x) &= \underbrace{F(1)}_{0} A(x) + \int_1^x A(x/t) \, dF &\text{by equation before (3.11)}\\
&= \int_1^x A(x/t) \, t^{-1} \, dt\\
&= \int_x^1 A(u) (x/u)^{-1} \, d (x/u) &\text{by substitution } u = x/t\\
&= \int_x^1 A(u) \frac ux \, \left( \frac{-x}{u^2} \right) du\\
&= \int_1^x A(u) u^{-1} du
\end{align*}
To identify $dH$ with $A(t) t^{-1} \, dt$, one goes back to page 52 again and observe that $A(t) t^{-1} \, dt$, by definition, is $dH$ with $H$ defined by the last equation.

\item Recall that $N$ is summatory function of $1$. So $dN * dN$ is $dH$ where $H$ is the summatory function of $1 * 1 = \tau$.

\item Similar to the previous part but $Q$ is summatory function of $|\mu|$ and $1 * 1 = 2^\omega$.

\item Let $F(t) = t - 1$ if $t \geq 1$ and 0 otherwise so $dt = dF$. One has $dt * dt = dH$ where for $x \geq 1$: 
\begin{align*}
H(x) &= F(1) F(x) + \int_1^x F(x/t) dF(t) &\text{by equation before (3.11)}\\
&= \int_1^x \left( \frac xt - 1 \right) dt\\
&= \left. (x \log t - t) \right|_1^x\\
&= x \log x - x - (x \log 1 - 1)\\
&= x \log x - x + 1
\end{align*}
and (note that $H$ is continuous at 1 as well), for every $x \geq 1$:
$$H'(x) = \log x + x x^{-1} - 1 = \log x$$
while $H'(x) = 0$ for $x < 1$, $H'$ is continuous so $dt * dt = \log t \, dt$.

\item Like 3 and 4, except now that $1 * \mu = e_1$.

\item Clearly $dA + dB = d(A + B)$ from the definition as they agree on the intervals. So $\delta_1 + dt = dF$ and $\delta_1 - t^{-1} dt = dG$ where
$$F(t) = 1 + (t - 1) = t$$
and
$$G(t) = 1 - \log t$$
and $F(t) = G(t) = 0$ for $t < 1$. For $x \geq 1$, we compute
\begin{align*}
H(x) &= \int_{1^-}^x G(x/t) \, dF(t)\\
&= F(1) G(x) + \int_1^x (1 - \log x + \log t) dt\\
&= 1 - \log x + (x - 1) (1 - \log x) + \int_1^x \log t \, dt\\
&= x (1 - \log x) + t (\log t)|_1^x - \int_1^x t \, t^{-1} dt &\text{integration by parts}\\
&= x (1 - \log x) + x \log x - (x - 1)\\
&= 1
\end{align*}
so $dH = \delta_1$.

\item At this point, we know that $t^{-1} dN = dA$ where for $x \geq 1$:
\begin{align*}
A(x) = \int_{1^-}^x t^{-1} dN
&= \sum_{1 \leq n \leq x} n^{-1}
\end{align*}
(While it would be more reasonable to take the integral from 1, we need to do $1^-$ here; otherwise, the RHS will be off by $\log x$. Unfortunately, the book does not tell what to do for $t^{-1} dN$ so we must take a guess here. Also, the notation $\int_a^b dK$ where $dK$ is an integrator has not been defined at this point. It seems on page 52, they should used $F(x) = \int_{1^-}^x f(t) dt$ which just happens to be the same as $\int_{1^-}^x f(t) dt$ due to the nature of $dt$.)

We know from part 2 that
$$t^{-1} dN * t^{-1} dt = A(t) \, t^{-1} dt$$
so
\begin{align*}
LHS &= \int_{1^-}^x A(t) t^{-1} dt\\
&= \int_{1^-}^x \left\{ \sum_{1 \leq n \leq t} n^{-1} \right\} t^{-1} dt\\
&= \int_{1^-}^x \left\{ \sum_{1 \leq n \leq x} \delta_n(t) n^{-1} \right\} t^{-1} dt\\
&= \sum_{1 \leq n \leq x} \int_{1^-}^x n^{-1} \delta_n(t) t^{-1} dt\\
&= \sum_{1 \leq n \leq x} n^{-1} \int_n^x t^{-1} dt\\
&= \sum_{1 \leq n \leq x} n^{-1} (\log x - \log n)\\
&= \log x \sum_{1 \leq n \leq x} n^{-1} - \sum_{1 \leq n \leq x} \frac{\log n}{n}
\end{align*}
Recall the Euler summation formulas
\begin{align*}
\sum_{1 \leq n \leq x} n^{-1} &= \log x + \gamma - \frac{x - [x]}{x} + \int_x^\infty (t - [t]) t^{-2} dt\\
\sum_{1 \leq n \leq x} \frac{\log n}{n} &= \frac{\log^2 x}{2} - (x - [x]) \frac{ \log x}{x} + \gamma_1 - \int_x^\infty (t - [t]) (1 - \log t) t^{-2} dt
\end{align*}
because
$$\gamma_1 = \int_1^\infty (t - [t]) (1 - \log t) t^{-2} dt.$$

So we continue:
\begin{align*}
LHS &= \log x \left\{ \log x + \gamma - \frac{x - [x]}{x} + \int_x^\infty (t - [t]) t^{-2} dt \right\} \\
&\qquad \qquad - \left\{ \frac{\log^2 x}{2} - (x - [x]) \frac{ \log x}{x} + \gamma_1 - \int_x^\infty (t - [t]) (1 - \log t) t^{-2} dt \right\}\\
&= \frac12 \log^2 x + \gamma \log x - \gamma_1 + \log x \int_x^\infty (t - [t]) t^{-2} dt \\
&\qquad \qquad + \int_x^\infty (t - [t]) (1 - \log t) t^{-2} dt
\end{align*}
and so it remains to show that
$$\int_x^\infty (t - [t]) (\log x + 1 - \log t) t^{-2} dt$$
is $O(x^{-1})$. Evidently,
$$0 \leq \int_x^\infty (t - [t]) t^{-2} dt \leq \int_x^\infty t^{-2} dt = \left.\frac{-1}{t} \right|_x^{\infty} = \frac{1}{x}$$
whereas
\begin{align*}
0 &> \int_x^\infty (t - [t]) (\log x - \log t) t^{-2} dt\\
&\geq - \int_x^\infty \log \left(\frac tx\right) t^{-2} dt\\
&= -\int_1^\infty (\log u) \; (ux)^{-2} d(ux) &\text{by substitution } u = t / x\\
&= -x^{-1} \underbrace{\int_1^\infty (\log u) \; u^{-2} du}_{\text{constant = 1}}.
\end{align*}
And we are done.
\end{enumerate}
\end{proof}

\unless\ifdefined\IsMainDocument
\end{document}
\fi

\unless\ifdefined\IsMainDocument
\documentclass[12pt]{article}
\usepackage{amsmath,amsthm}

\begin{document}
\fi

\textbf{Problem 3.17}: Verify that
$$(dt)^{*n} = L^{n-1} dt / (n-1)!$$
for any $n \in Z^+$. Hint: Show that
$$T^{-1} (dt)^{*n} = (T^{-1} dt)^{*n} = T^{-1} L^{n-1} dt / (n-1)!$$

\begin{proof}
We follow the hint. The first equality is due to $T^{-1} dF * T^{-1} dG = T^{-1} (dF * dG)$ and a simple induction. We prove the identity
$$(T^{-1} dt)^{*n} = T^{-1} L^{n-1} dt / (n-1)!$$
also by induction on $n$.
\begin{itemize}
\item Base case $n = 1$: Then the LHS and RHS are both $t^{-1} \, dt$.
\item Induction: Suppose that the identity is true for $n$. Then
\begin{align*}
(T^{-1} dt)^{*(n+1)} &= (T^{-1} dt)^{*n} * T^{-1} dt\\
&= \frac{T^{-1} L^{n-1} dt}{(n-1)!} * t^{-1} dt\\
&= \frac{\log^{n-1} t}{t (n-1)!} dt * t^{-1} dt\\
&= d \left( \frac{\log^n t}{n!} \right) * t^{-1} dt\\
&= \frac{\log^n t}{n!} \, t^{-1} \, dt\\
&= T^{-1} L^n dt / n!
\end{align*}
where we recall from previous problem 3.15 that
$$dA * \underbrace{t^{-1} \, dt}_{T^{-1} \, dt} = A(t) \, t^{-1} \, dt.$$
\end{itemize}
\end{proof}

\textbf{Problem 3.18}: Show that the convolution inverse of $\delta_1 + dt$ can be obtained formally from the series $\delta_1 - dt + (dt * dt) - (dt * dt * dt) + \cdots$.

\begin{proof}
We use the previous problem to simplify
\begin{align*}
\sum_{j = 1}^{\infty} (-1)^j \, dt^{*j} &= \delta_1 + \sum_{j = 1}^{\infty} (-1)^j \, \frac{L^{j-1} dt}{(j-1)!}\\
&= \delta_1 + \sum_{j = 1}^{\infty} (-1)^j \, \frac{\log^{j-1} t \, dt}{(j-1)!}\\
&= \delta_1 - \left( \sum_{j = 0}^{\infty} \frac{(- \log t)^j}{j!} \right) dt\\
&= \delta_1 - e^{- \log t} dt\\
&= \delta_1 - t^{-1} dt
\end{align*}
and we recall from 3.15 that
$$(\delta_1 + dt) * (\delta_1 - t^{-1} dt) = \delta_1.$$
\end{proof}

\textbf{Problem 3.19}: Find the convolution inverse of $\delta_1 + L \, dt$. Verify that the integrator you found is indeed the inverse.

\begin{proof}
We expect the inverse to be
$$\sum_{j = 0}^{\infty} (-L\, dt)^{*j} = \delta_1 - L \, dt + (L\,dt * L\,dt) - (L\,dt * L\,dt * L\,dt) + \cdots$$

The result of problem 3.17 for $n = 2$ yields
$$L \, dt = dt * dt$$
and so for all $j \geq 1$, we can simplify
$$(-L\, dt)^{*j} = (-1)^j (dt * dt)^{j} = (-1)^j dt^{*2j} = (-1)^j \frac{L^{2j - 1} dt}{(2j - 1)!}$$
by problem 3.17 again and we thus obtain
$$\delta_1 + \sum_{j = 1}^{\infty} (-1)^j \frac{L^{2j - 1} dt}{(2j - 1)!} = \delta_1 + \underbrace{\left( \sum_{j = 1}^{\infty} (-1)^j \frac{\log ^{2j - 1} t}{(2j - 1)!} \right)}_{-\sin (\log t)} dt.$$

Now let us check that
$$(\delta_1 + L \, dt) * (\delta_1 - \sin(\log t) dt) = \delta_1.$$

It suffices to check that they give the same answer for all half intervals $(1-\epsilon, x]$. The left hand side is
$$\delta_1 - \sin(\log t) \, dt + L \, dt - L\,dt * \sin(\log t) \, dt$$
so we first compute
\begin{align*}
\int_{1^-}^x L\,dt * \sin(\log t) \, dt &= \int_{1^-}^x \left\{ \int_{1/s}^{x/s} \log t \, dt \right\} \, \sin(\log s) \, ds\\
&= \int_1^x \left\{ \int_{1/s}^{x/s} \log t \, dt \right\} \, \sin(\log s) \, ds\\
&= \int_1^x \left\{ \frac{x}{s} \log \left(\frac{x}{s}\right) - \frac{x}{s} + 1 \right\} \, \sin(\log s) \, ds\\
&= \int_1^x \sin(\log s) \, ds + \int_1^x \frac{x \log x - x \log s - x}{s} \, \sin(\log s) \, ds\\
&= \int_1^x \sin(\log t) \, dt + \int_0^{\log x} (x \log x - x u - x) \, \sin u \, du
\end{align*}
by a change of variable $u = \log s$. On the second line, we used Lemma 3.8 noting that the function defining $ds$ is continuous at 1. To see the third line, recall $\log t \, dt = d H$ where $H(t) = 0$ if $t < 1$ and $H(t) = t \log t - t + 1$ if $t \geq 1$. So one has
\begin{align*}
\int_{1/s}^{x/s} \log t \, dt =& H\left(\frac{x}{s}\right) - H \left(\frac{1}{s}\right)\\
=& \begin{cases}
\frac{x}{s} \log \left(\frac{x}{s}\right) - \frac{x}{s} + 1 &\text{if } s > 1,\\
\frac{x}{s} \log \left(\frac{x}{s}\right) - \frac{x}{s} - \frac{1}{s} \log \left(\frac{1}{s}\right) + \frac{1}{s} &\text{if } s \leq 1.
\end{cases}
\end{align*}

We resolve
\begin{itemize}
\item the first integral by integration by parts:
\begin{align*}
\int_1^x \sin(\log t) \, dt &= t \sin(\log t)|_1^x - \int_1^x t \cos(\log t) t^{-1} \, dt\\
&= t \sin(\log t)|_1^x - \int_1^x \cos(\log t) \, dt\\
&= t \sin(\log t)|_1^x - \left\{t \cos(\log t)|_1^x + \int_1^x t \sin(\log t) t^{-1} \, dt \right\}
\end{align*}
so
\begin{align*}
2 \int_1^x \sin(\log t) \, dt &= t \sin(\log t)|_1^x - t \cos(\log t)|_1^x\\
&= x \sin \log x - x \cos \log x + 1
\end{align*}

\item and the second integral similarly:
\begin{align*}
&\int_0^{\log x} (x \log x - x u - x) \, \sin u \, du\\
=& (x \log x - x) \int_0^{\log x} \sin u \, du - x \int_0^{\log x} u \, \sin u \, du\\
=& (x \log x - x) (1 - \cos \log x) - x (-\log x \cos \log x + \sin \log x)\\
=& x \log x - x  - x \log x \cos \log x + x \cos \log x + x \log x \cos \log x - x \sin \log x\\
=& x \log x - x  + x \cos \log x - x \sin \log x
\end{align*}
\end{itemize}

So for $x \geq 1$, we see that
\begin{align*}
&\int_{1^-}^x \delta_1 - \sin(\log t) \, dt + L \, dt - L\,dt * \sin(\log t) \, dt\\
=& 1 - \int_1^x \sin(\log t) \, dt + \int_1^x \log t \, dt - \int_1^x L\,dt * \sin(\log t) \, dt\\
=& 1 - 2 \int_1^x \sin(\log t) \, dt + (x \log x - x + 1) \\
& \qquad - (x \log x - x  + x \cos \log x - x \sin \log x)\\
=& 1 - (x \sin \log x - x \cos \log x + 1) + (x \log x - x + 1) \\
& \qquad - (x \log x - x  + x \cos \log x - x \sin \log x)\\
=& 1 - x \sin \log x + x \cos \log x - 1 + x \log x - x + 1 \\
& \qquad - x \log x + x - x \cos \log x + x \sin \log x\\
=& 1
\end{align*}
which matches $\int_{1^-}^x \delta_1$.
\end{proof}

\unless\ifdefined\IsMainDocument
\end{document}
\fi

\unless\ifdefined\IsMainDocument
\documentclass[12pt]{article}
\usepackage{amsmath,amsthm}
\newcommand{\V}{\mathcal{V}}
\begin{document}
\fi

\textbf{Problem 3.20}: If $F \in \V$ and $F(1) = 1$, then there exists a function $\varphi \in \V$ with $\varphi(1) = 0$ such that
$$dF = \exp d\varphi := \delta_1 + d\varphi + (d\varphi * d\varphi)/2! + \cdots$$
and $L \, dF = dF * L\, d\varphi$. Show that
$$\delta_1 + dt = \exp \left\{ \frac{1 - t^{-1}}{\log t} dt\right\}.$$

\begin{proof}
Note that $dF = \exp d\varphi$ implies
$$L \, dF = L \left( \delta_1 + d\varphi + \cdots \right) = L d\varphi + L(d\varphi * d\varphi / 2!) + \cdots = L d\varphi * dF.$$
But then we can determined uniquely
$$d\varphi = L^{-1} (L \, dF * (dF)^{*-1})$$
and so $\varphi$ is determined up to a constant. Similar to Theorem 2.20, we expect the equivalence of the three equations:
$$dF = \exp d\varphi \iff L \, dF = dF * L\, d\varphi \iff d\varphi = \log dF$$
with the last one being useful to derive $d\varphi$ from $dF$. (The prior formula for $d\varphi$ requires finding $dF^{*-1}$ which is not exactly easy in general.)

Thus, we only need to verify the equation $L \, dF = dF * L\, d\varphi$ holds for $dF = \delta_1 + dt$ and $d\varphi = \frac{1 - t^{-1}}{\log t} dt = L^{-1} (1 - t^{-1}) dt$:
\begin{align*}
L \, dF &= L (\delta_1 + dt)\\
&= L \delta_1 + L \, dt\\
&= L \, dt\\
dF * L\, d\varphi &= (\delta_1 + dt) * (1 - t^{-1}) dt\\
&= (\delta_1 + dt) * (dt - \delta_1 + \delta_1 - t^{-1} dt)\\
&= (\delta_1 + dt) * (dt - \delta_1) + (\delta_1 + dt) * (\delta_1 - t^{-1} dt)\\
&= dt * dt - \delta_1 * \delta_1 + \delta_1\\
&= dt * dt\\
&= L \, dt
\end{align*}
The fact that $L \delta_1 = 0$ comes from applying $L$ to both sides of $\delta_1 = \delta_1 * \delta_1$.

For any $a > 1$, we have
\begin{align*}
\int_a^x L^{-1} dF &= \int_a^x \log^{-1} t \, dF\\
&= \left. \frac{F}{\log t}\right|_a^x - \int_a^x F \, d(\log^{-1} t)\\
&= \left. \frac{F}{\log t}\right|_a^x - \int_a^x F \cdot \left(\frac{-1}{\log^2 t} t^{-1} dt \right)\\
&= \left. \frac{F}{\log t}\right|_a^x + \int_a^x \; \frac{F}{t \log^2 t} dt
\end{align*}
Unfortunately, we cannot simplify $L^{-1} dF$ to a nice formula.
\end{proof}

\textbf{Problem 3.21}: Show that if $F \in \V$ and $F(x) = \int_1^x dF * t^{-1} dt$ for all $x \geq 1$, then $F = 0$. 

\begin{proof}
First, we clearly have $F(1) = 0$ and so we find
$$F(x) = F(x) - F(1) = \int_1^x \, dF$$
which implies
$$\int_1^x \, dF = \int_1^x dF * t^{-1} dt$$
or equivalently
$$\int_1^x dF * (\delta_1 - t^{-1} dt) = 0$$
for all $x \geq 1$. But that means
$$dF * (\delta_1 - t^{-1} dt) = 0$$
and so
$$dF = 0$$
by convoluting both sides with the inverse $\delta_1 + dt$ of $\delta_1 - t^{-1} dt$. So $F$ is a constant for $x \geq 1$ and that constant has been determined by $F(1) = 0$.
\end{proof}

\unless\ifdefined\IsMainDocument
\end{document}
\fi

\unless\ifdefined\IsMainDocument
\documentclass[12pt]{article}
\usepackage{amsmath,amsthm,amssymb}

\newcommand{\C}{\mathbb{C}}
\newcommand{\Abs}[1]{\left| #1 \right|}

\begin{document}
\fi

\textbf{Problem 3.22}: Let $F(x) \sim ax^b$ where $a, b \in \C$, $a \not= 0$, $\Re b > 0$. Show that
$$\int_1^x dF * \frac{dt}{t} \sim \frac{a}{b} x^b.$$

\begin{proof}
Recall $dF * \frac{dt}{t} = F(t) t^{-1} dt$. We have to show that for every $\epsilon > 0$, there exists $x_0$ sufficiently large such that for all $x > x_0$,
$$\Abs{ \frac{\int_1^x F(t) t^{-1} dt}{\frac{a}{b} x^b} - 1 } < \epsilon$$
or equivalently
$$\Abs{ \int_1^x (F(t) t^{-1} - a t^{b-1}) dt} < \epsilon \Abs{ \int_1^x a t^{b-1} dt }.$$

By assumption, we can choose $t_1$ so that $\Abs{ F(t) t^{-1} - a t^{b-1} } < \delta_1 \Abs{ a t^{b-1} }$  as long as $t > t_1$ to take care of $\int_{t_1}^x$. To make our bound work, we need to ensure that the remaining integral $\int_1^{t_1}$, which is a constant since $t_1$ is fixed now, can be bounded by $\delta_2 \Abs{\frac{a}{b} x^b}$ but that is easy under the assumption that $\Re b > 0$. Taking $\delta_1 = \delta_2 = \epsilon /2$ (as long as $\delta_1 + \delta_2 = \epsilon$) should work in our case.

So, we find $x_0$ by
\begin{enumerate}
\item First find $t_1$ such that
$$\Abs{ F(t) - a t^b } < \frac\epsilon{2} \Abs{ a t^b }$$
for all $t \geq t_1$ using the assumption $F(x) \sim ax^b$.
\item Then we find $x_0 > t_1$ such that
$$\Abs{ \int_1^{t_1} (F(t) t^{-1} - a t^{b-1}) dt } < \frac\epsilon2 \Abs{\frac{a}{b} x^b}$$
for all $x > x_0$ using the assumption $\Re b > 0$.
\end{enumerate}

Then for all $x > x_0 > t_1$ we have
\begin{align*}
&\Abs{ \int_1^x (F(t) t^{-1} - a t^{b-1}) dt}\\
\leq &\Abs{ \int_1^{t_1} (F(t) t^{-1} - a t^{b-1}) dt} + \Abs{ \int_{t_1}^x (F(t) t^{-1} - a t^{b-1}) dt}\\
\leq & \frac\epsilon2 \Abs{ \frac{a}{b} x^b } + \int_{t_1}^x \Abs{F(t) t^{-1} - a t^{b-1}} dt &\text{by our choices}\\
\leq & \frac\epsilon2 \Abs{ \frac{a}{b} x^b } + \int_{t_1}^x \frac\epsilon{2} |a| \cdot |t|^{b-1} dt\\
=& \epsilon \Abs{ \frac{a}{b} x^b }
\end{align*}
as required.
\end{proof}

\unless\ifdefined\IsMainDocument
\end{document}
\fi

\unless\ifdefined\IsMainDocument
\documentclass[12pt]{article}
\usepackage{amsmath,amsthm}

\begin{document}
\fi

\textbf{Problem 3.23}: Apply the stability theorem to establish the following relations:
\begin{enumerate}
\item $\sum_{n \leq x} \varphi(n) = \sum_{n \leq x} (T1 * \mu)(n) = x^2 / (2 \zeta(2)) + O(x \log ex)$.
\item Let $S = \{n^2 : n \in Z^+\}$. Then for some constant $c$
$$\sum_{n \leq x} (L1 * 1_S)(n) = \zeta(2) x \log x + c x + O(x^{1/2} \log ex)$$
\item Let $k \in Z^+$. Then $\sum_{n \leq x} (\log x/n)^k = k! x + O((\log ex)^k)$.
\item Prove that $\sum_{n \leq x} \tau^2(n) \sim \pi^{-2} x \log^3 x$.
\end{enumerate}

\begin{proof}
\begin{enumerate}
\item We recognize
$$\sum_{n \leq x} \varphi(n) = \sum_{n \leq x} (T1 * \mu)(n) = \int_{1^-}^x dF * dM$$
where $F$ and $M$ are summatory functions of $T1$ and $\mu$. Now,
$$F(x) = \sum_{n \leq x} n = \frac{[x]([x] + 1)}{2} = \frac{x^2}{2} + O(x)$$
while the total variation function
$$M_v(x) = \sum_{n \leq x} |\mu(n)| = O(x)$$
by Theorem 3.5. Note that $P(u) = \frac{1}{2}$ here is of degree $m = 0$ and $\theta = \tau = 1$, $K = H = 0$. So stability theorem yields the estimate $x^2 P^*(u) + O(x \log ex)$ where
$$P^*(u) = \frac12 \int_{1^-}^{\infty} t^{-2} dM = \sum_{m = 1}^{\infty} \frac{\mu(m)}{m^2} = \frac{1}{2\zeta(2)}.$$

\item Similar to the previous part, we estimate the summatory function
\begin{align*}
\sum_{n \leq x} L1(n) =& \sum_{n \leq x} \log n\\
=& \int_1^x \log t \, dt + O(\log x) &\text{by problem 3.13}\\
=& x \log x - x + 1 + O(\log x)\\
=& x (\log x - 1) + O(\log ex)
\end{align*}
and the total variation of the summatory function $G$ of $1_S$ is easily estimated as $O(x^{1/2})$. Thus we can apply the stability theorem in this case to $s = 1$, $P(u) = u - 1$, $\theta = 0 < \tau = \frac 12 < s = 1$, $K = 1$, $H = 0$ to get the estimate $x P^*(\log x) + O(x^{1/2} \log ex)$ for the given sum where
\begin{align*}
P^*(u) =& (u - 1)\int_{1-}^{\infty} t^{-1} dG + \int_{1^-}^\infty (-\log t) t^{-1} dG\\
=& (u - 1) \sum_{m = 1}^{\infty} \frac{1_S(m)}{m} + c\\
=& (u - 1) \zeta(2) + c
\end{align*}

\item With $N = [x]^+$ being the summatory function of $1$, we can recognize the sum as
$$\int_{1^-}^x (\log x / t)^k dN = \int_{1^-}^x dN * dG$$
since the first formula is the one used to define $dN * dG$ for $G(t) = \log^k t$ as in equation (3.10) for $F = N$.

Evidently, $N(x) = x + O(1)$ while $G_v(x) = \log^k x = O(\log^k ex)$ because $G$ is increasing. So stability theorem yields the estimate
$$x \underbrace{\int_{1^-}^\infty t^{-1} dG}_{k!} + O((\log ex)^k).$$
Note here $\tau = 0 < s = 1$ so we can apply the stability theorem.

\item We have $1 = |\mu| * 1_S$ in problem 3.1 so
$$\tau^2 = 1^{*4} * 1_S^{*-1} = 1^{*3} * |\mu|.$$
Thus,
$$\sum_{n \leq x} \tau^2(n) = \int_{1^-}^x dN^{*3} * dM$$
where $M$ is the summatory function of $|\mu|$.

One direction is that we would like to apply the stability theorem starting with
$$F_0(x) = M(x) = \frac{6x}{\pi^2} + O(x^{1/2})$$
according to Theorem 3.5 to estimate
$$F_1(x) = \int_{1^-}^x dF_0 * dN = \int_{1^-}^x dM * dN = \sum_{n \leq x} (|\mu| * 1)(n)$$
and then using it to estimate
$$F_2(x) = \int_{1^-}^x dF_1 * dN = \int_{1^-}^x dM * dN * dN$$
and so on. The key problem to this approach is that we cannot apply the stability theorem to $dF * dN$ if $F = x P(\log x) + O(...)$ i.e. $s = 1$ because $N_v = N = O(x)$ so $\tau = 1$ due to $N$ monotone and we need $\tau < \Re s$ for the theorem to work! On top of that, the stability theorem does not increase the degree of the polynomial so we will not end up with $\log^3 x$. But we could see the $\log^3 x$ is sort of introduced with the three $dN$ convolutions.

To make things precise, we claim that if
$$F(x) = x P(\log x) + O(x^\theta \log^K x)$$
where either $\theta < 1, K = 0$ or $\theta = 1, K < \deg P$ then
$$\int_{1^-}^x dF * dN = x P^*(\log x) + O(x \log^{\deg P} x)$$
where
$$P^*(u) = \sum_{\ell = 0}^{\deg P} \frac{(-1)^\ell}{\ell!} P^{(\ell)}(u) \frac{u^{\ell + 1}}{\ell + 1}$$
is a polynomial of 1 degree higher than $P$ (and so we maintain $\deg P < \deg P_1$ to continuously apply the result). To prove that, we basically follow the proof of the stability theorem and expand
\begin{align*}
\int_{1^-}^x dF * dN &= \int_{1^-}^x F\left(\frac x t\right) dN(t) \\
&= \int_{1^-}^x \left\{ \frac{x}{t} P\left(\log \frac x t\right) + O\left( \left(\frac x t \right)^\theta \log^K \left( \frac x t \right) \right) \right\} dN \\
&= x \underbrace{\int_{1^-}^x \frac{1}{t} P\left(\log \frac x t\right) dN}_{I} + \underbrace{\int_{1^-}^x O\left( \left(\frac x t \right)^\theta \log^K \left( \frac x t \right) \right) dN}_{J}
\end{align*}
To resolve the first integral, we used Taylor's theorem to get for any $a$
$$P(u) = \sum_{\ell=0}^{\deg P} P^{(\ell)}(a) \frac{(u - a)^\ell}{\ell!}$$
and in particular for $a = \log x$ to write
\begin{align*}
I &= \int_{1^-}^x \frac{1}{t} \sum_{\ell = 0}^{\deg P} P^{(\ell)}(\log x) \frac{(-\log t)^\ell}{\ell!} dN\\
&= \sum_{\ell = 0}^{\deg P} \frac{(-1)^\ell}{\ell!} P^{(\ell)}(\log x) \int_{1^-}^x \frac{(\log t)^\ell}{t} dN
\end{align*}
In the proof of the stability theorem, we split $\int_{1^-}^x$ as $\int_{1^-}^\infty - \int_x^\infty$ under the assumption of $\tau < \Re s$ to ensure convergence. Here, we deviate and use the result of problem 3.14 that
$$\int_{1^-}^x \frac{(\log t)^\ell}{t} dN = \sum_{m \leq x} \frac{\log^\ell m}{m} = \frac{\log^{\ell+1} x}{\ell + 1} + O(1)$$
and so
\begin{align*}
x I &= x \left( \sum_{\ell = 0}^{\deg P} \frac{(-1)^\ell}{\ell!} P^{(\ell)}(\log x) \frac{\log^{\ell + 1} x}{\ell + 1} \right) + O\left(x \underbrace{\sum_{\ell = 0}^{\deg P} \frac{(-1)^\ell}{\ell!} P^{(\ell)}(\log x)}_{P(\log x - 1)}\right)\\
&= x P^*(\log x) + O(x \log^{\deg P} x)\\
&= O(x \log^{\deg P + 1} x)
\end{align*}

The second integral is:
$$|J| = O\left( x^\theta \int_{1^-}^x t^{-\theta} \log^K(x/t) dN \right)$$
\begin{itemize}
\item If $\theta = 1$ then $|J|$ is $O(x \log^{K + 1} x)$ as the special case of $I$ where the polynomial $P(u) = u^K$. And so
$$I + J = x P^*(\log u) + \underbrace{O(x \log^{\deg P} x) + O(x \log^{K+1} x)}_{O(x \log^{\deg P} x)}$$
under the assumption that $K < \deg P$.

\item If $\theta < 1$ and $K = 0$ then
$$\int_{1^-}^x t^{-\theta} \log^K(x/t) dN = \int_{1^-}^x t^{-\theta} dN = \sum_{n \leq x} n^{-\theta}$$
and we know from Lemma 3.13 that this is $\frac{x^{1-\theta}}{1 - \theta} + \zeta(\theta) + O(x^{-\theta})$. So $|J|$ is $O(x)$ and $I + J$ has the prescribed form.
\end{itemize}
Finally, we observe that
\begin{align*}
(P^*(u))' &= \sum_{\ell = 0}^{\deg P} \frac{(-1)^\ell}{\ell!} \left( P^{(\ell)}(u) \frac{u^{\ell + 1}}{\ell + 1} \right)' \\
&= \sum_{\ell = 0}^{\deg P} \frac{(-1)^\ell}{\ell!} \left( P^{(\ell + 1)}(u) \frac{u^{\ell + 1}}{\ell + 1} + P^{(\ell)}(u) u^\ell \right)\\
&= -\sum_{\ell = 0}^{\deg P} P^{(\ell + 1)}(u) \frac{(-u)^{\ell + 1}}{(\ell + 1)!} + \underbrace{\sum_{\ell = 0}^{\deg P} P^{(\ell)}(u) \frac{(-u)^\ell}{\ell!}}_{P(0)}\\
&= - \underbrace{\sum_{\ell = 1}^{\deg P + 1} P^{(\ell)}(u) \frac{(-u)^{\ell}}{\ell!}}_{P(0) - P(u)} + P(0)\\
&= P(u)
\end{align*}
and $P*(0) = 0$ so $P^*$ can be uniquely identified as $\int P(u) du$ where we take 0 for constant in the indefinite integral.

Now we are ready to apply it: With $F_0(x) = \frac{6x}{\pi^2} + O(x^{1/2})$ and $P_0(u) = \frac{6}{\pi^2}$, we find
$$F_1(x) = \int_{1^-}^x dF_0 * dN = x P_1(\log x) + O(x)$$
where $P_1(u) = P_0^*(u) = \frac{6}{\pi^2} u$. This leads to
$$F_2(x) = \int_{1^-}^x dF_1 * dN = x P_2(\log x) + O(x \log x)$$
where $P_2(u) = P_1^*(u) = \frac{6}{\pi^2} \frac{u^2}{2} = \frac{3}{\pi^2} u^2$ and finally
$$F_3(x) = \int_{1^-}^x dF_2 * dN = x P_3(\log x) + O(x \log^2 x)$$
where $P_3(u) = P_2^*(u) = \frac{3}{\pi^2} \frac{u^3}{3} = \pi^2 u^3$. So $F_3(x) \sim \pi^{-2} x \log^3 x$.
\end{enumerate}
\end{proof}

\unless\ifdefined\IsMainDocument
\end{document}
\fi

\unless\ifdefined\IsMainDocument
\documentclass[12pt]{article}
\usepackage{amsmath,amsthm,amssymb}

\newcommand{\Abs}[1]{\left| #1 \right|}

\begin{document}
\fi

\textbf{Problem 3.24}: Let $M$ be a given non-empty set of integers, each exceeding 1. Let $S$ be the set of positive integer such that if $n = \prod p_i^{m_i}$ is the factorization of $n$ then none of the $m_i$ is in $M$. Let $S(x) := \#\{n \leq x : n \in S\}$. Prove that
$$S(x) = \alpha x + O(x^{1/2})$$
where
$$\alpha = \prod_p \underbrace{\left\{1 - \sum_{m \in M} (p^{-m} - p^{-m-1})\right\}}_{\alpha_p}.$$

\begin{proof}
Write
$$1_S = 1 * (1_S * \mu)$$
and we have
$$S(x) = \int_{1^-}^x dN * dG$$
where $G$ is the summatory function of $1_S * \mu$.

We are going to show that $G_v(x) = O(x^{1/2})$ and so together with $N(x) = x + O(1)$ we get by the stability theorem that
$$S(x) = \int_{1^-}^x dN * dG = c x + O(x^{1/2})$$
where
\begin{align*}
c = \int_{1^-}^\infty t^{-1} dG(t) &= \sum_{n = 1}^{\infty} \frac{(1_S * \mu)(n)}{n}\\
&= \prod_p \sum_{k = 0}^{\infty} \frac{(1_S * \mu)(p^k)}{p^k} &\text{ by multiplicativity}\\
&= \prod_p \left(1 + \sum_{k = 2}^{\infty} \frac{1_S(p^k) - 1_S(p^{k-1})}{p^k}\right)\\
&= \prod_p \left(1 - \sum_{k = 2}^{\infty} \frac{1 - 1_S(p^k)}{p^k} + \sum_{k = 1}^{\infty} \frac{1 - 1_S(p^k)}{p^{k+1}}\right)\\
&= \prod_p \left(1 - \sum_{k \in M} \frac{1}{p^k} + \sum_{k \in M} \frac{1}{p^{k+1}}\right)\\
&= \alpha.
\end{align*}

To show that $G_v(x) = O(x^{1/2})$, using the fact that $1_S * \mu$ is multiplicative, we can write
$$(1_S * \mu)(n) =  \prod_{p^k || n} (1_S(p^k) - 1_S(p^{k-1}))$$
so
$$\Abs{(1_S * \mu)(n)} = \begin{cases}
0 &\text{ if for some }p:  k, k - 1 \in M \text{ or } k, k - 1 \not\in M,\\
1 &\text{ if for all }p: \text{ exactly one of } k, k - 1 \in M.
\end{cases}$$
In particular, if $n$ has a prime divisor whose exponent in $n$ is 1 i.e. $p | n$ but $p^2 \nmid n$ then $\Abs{(1_S * \mu)(n)} = 0$. Therefore, we can bound
\begin{align*}
G_v(x) &= \sum_{n \leq x} \Abs{(1_S * \mu)(n)}\\
&\leq \#\{n \leq x : p^2 | n \text{ for all } p | n\}.
\end{align*}
Note that equality could happen, for example when $M$ is the set of all positive even numbers.

It remains to estimate $T(x)$ where
$$T := \{n \in Z^+ : p^2 | n \text{ for all } p | n\}.$$
Expressing $1_T$ as an infinite convolution over the primes component by Lemma 2.26, we find
$$1_T = 1_{SQ} * f$$
where $1_{SQ}$ is the indicator function for the set of squares
$$1_{SQ}(n) = \begin{cases}
1 &\text{if } n = m^2,\\
0 &\text{otherwise}
\end{cases}$$
whose summatory function is $[\sqrt{x}]$ and
$$f(n) = \begin{cases}
1 &\text{if } n = m^3 \text{ and } m \text{ is square-free},\\
0 &\text{otherwise}.
\end{cases}$$
We could check the equation directly by doing it on prime powers as both sides are multiplicative. If $k < 3$, $1_{SQ} * f(p^k) = 1_{SQ}(p^k)$ takes value 1 only when $k = 0$ or $k = 2$ and this matches $1_T(p^k)$. If $k \geq 3$, $1_{SQ} * f(p^k) = 1_{SQ}(p^k) + 1_{SQ}(p^{k-3})$ is always $1 = 1_T(p^k)$ because exactly one of $p^k$ or $p^{k-3}$ is a square due to $k, k - 3$ of different parities.

Finally, we could estimate
\begin{align*}
T(x) &= \sum_{n \leq x} 1_T(n)\\
&= \sum_{m \leq x} \left[\sqrt{\frac{x}{m}}\right] f(m) &\text{by Lemma 3.1}\\
&= \sum_{m \leq \sqrt[3]{x} \text{ square free}} \left[\sqrt{\frac{x}{m^3}}\right]\\
&\leq \sqrt{x} \sum_{m \leq \sqrt[3]{x}} m^{-3/2}\\
&= \sqrt{x} \left( \frac{(\sqrt[3]{x})^{1 - 3/2}}{1 - 3/2} + \zeta(3/2) + O((\sqrt[3]{x})^{-3/2}) \right) &\text{by Lemma 3.13}\\
&= O(\sqrt{x}).
\end{align*}
So $G_v(x) = O(\sqrt{x})$.

\textbf{Note}: My first idea was to express $1_S$ as the infinite convolution product
$$1_S = f_1 * f_2 * f_3 * \cdots$$
where
$$f_j = \sum_{m \not\in M} e_{p_j^m}$$
and $p_j$ is the $j$-th prime and then to apply the stability theorem repeatedly to approximate
$$S_\ell(x) = \sum_{n \leq x} (f_1 * f_2 * ... * f_\ell) (n)$$
and $S(x) = \lim_{\ell \rightarrow \infty} S_\ell(x)$.

Let $F_\ell$ be the summatory function of $f_\ell$. Then we have a recursive relation
$$S_{\ell+1}(x) = \int_{1^-}^x dS_\ell * dF_{\ell + 1}.$$ %$F_{\ell, v} = F_\ell$.
Since $F_\ell(x)$ counts the number of prime powers $p^k \leq x$ with $k \not\in M$ which is the same as counting the number of exponents $\leq \log_{p_\ell} x$ that is not in $M$ and so we can also express
$$F_\ell(x) = 1 + M^*(\log_{p_\ell} x) = 1 + M^*\left(\frac{\log x}{\log p_\ell}\right) = O(\log x)$$
where $M^*$ is the summatory function of the characteristic function of the complement $Z^+ - M$ of $M$. Unfortunately, that means we are in the situation where $s = \tau = 0$ and so cannot apply stability theorem! On the top of that, we cannot apply the stability infinitely as the constant in the error term $O(...)$ could blow up. The way we solve it is basically starting with $S_0 = N$ which is of the correct order (linear in $x$) instead of $S_1 = F_1$ and then each step, convolution with $dF^*_\ell$ where $F^*_\ell$ is the summatory function of $f_\ell * \mu_\ell$ to account for the initial $N = \sum 1$. We also do it in a finite number of applications (exactly one) instead of an infinite application of stability theorem.
\end{proof}

\unless\ifdefined\IsMainDocument
\end{document}
\fi

\unless\ifdefined\IsMainDocument
\documentclass[12pt]{article}
\usepackage{amsmath,amsthm,amssymb}

\begin{document}
\fi

\textbf{Problem 3.25}: Suppose that $a$ is a fixed positive integer.
\begin{enumerate}
\item Show that for $\Re s > 1$
$$\sum_{(n, a) = 1} \frac{n \mu^2(n)}{\varphi(n) n^s} = \sum_{\ell = 1}^\infty \frac{1}{\ell^s} \sum_{m = 1}^\infty \frac{c_m}{m^s}$$
where
$$\sum_{m = 1}^\infty \frac{c_m}{m^s} = \prod_{p | a} (1 - p^{-s}) \prod_{p \nmid a} \left( 1 + \frac{1}{(p-1)p^s} - \frac{p}{(p-1)p^{2s}}\right)$$
and the series $\sum c_m m^{-s}$ converges absolutely for $\Re s > 1/2$.

\item Show that
$$\sum_{n \leq x, \; (n, a) = 1} \frac{n \mu^2(n)}{\phi(n)} = \frac{\varphi(a)}{a} x + O(x^{1/2 + \epsilon}).$$

\item Show that
$$\sum_{n \leq x, \; (n, a) = 1} \frac{\mu^2(n)}{\varphi(n)} = \frac{\varphi(a)}{a} \log x + K(a) + O(x^{-1/2 + \epsilon}).$$
where $K(a)$ is a number depending on $a$.
\end{enumerate}

\begin{proof}
For an arithmetic function $h$, we will write $h^\Sigma$ its summatory function and $dh$ for $dh^\Sigma$.
Define the multiplicative functions
$$f(n) := \begin{cases}
\frac{n \mu^2(n)}{\varphi(n) n^s} &\text{if } (n, a) = 1,\\
0 &\text{if }(n, a) > 1
\end{cases}$$
and $g = f * \mu$ so $f = 1 * g$ and
$$g(p^k) = f(p^k) - f(p^{k-1}) = \begin{cases}
0 &\text{if } p | a, k > 1 \text{ or } p \nmid a, k > 2,\\
-1 &\text{if } p | a, k = 1,\\
\frac{1}{p-1} &\text{if } p \nmid a, k = 1,\\
-\frac{p}{p-1} &\text{if } p \nmid a, k = 2.
\end{cases}$$
Let $K$ be the support of $g$. It can be identified with the set of triples of positive integers $(d, \ell, m)$ such that
\begin{itemize}
\item $d, \ell, m$ square-free and pairwise relatively prime,
\item $d | a$,
\item $(\ell, a) = (m, a) = 1$.
\end{itemize}
under the identification $(d, \ell, m) \leftrightarrow d \ell m^2$.

\begin{enumerate}
\item Let $H_s(x) = x^s$ and we realize the product of $x^s$ with the LHS as
$$\sum_{n = 1}^{\infty} \left(\frac{x}{n}\right)^s f(n) = \int_{1^-}^x dH_s * df = \int_{1^-}^x dH_s * dN * dg = \int_{1^-}^x d \widehat{H}_s * dg$$
where
\begin{align*}
\widehat{H}_s^*(x) &= \int_{1^-}^x dH_s * dN = \int_{1^-}^x H_s(x/t) dN(t)\\
&= \sum_{n \leq x} (x/t)^s\\
&= x^s\left( \frac{x^{1-s}}{1-s} + \zeta(s) + O(x^{-s})\right)\\
&= \zeta(s) \; x^s + \underbrace{\frac{x}{1 - s} + O(1)}_{O(x)}.
\end{align*}

We have
\begin{align*}
g^\Sigma_v(x) &= |g|^\Sigma(x)\\
&= \sum_{(d, \ell, m) \in K_{\leq x}} |g(d) g(\ell) g(m^2)|\\
&= \sum_{(d, \ell, m) \in K_{\leq x}} \frac{1}{\varphi(\ell)} \frac{m}{\varphi(m)}\\
&= \sum_{d | a, d \in Q_1} \sum_{m \leq \sqrt{x/d}, m \in Q_a} \left(\sum_{\ell \leq x/dm^2, \ell \in Q_{am}} \frac{1}{\varphi(\ell)} \right) \frac{m}{\varphi(m)}
\end{align*}
where to simplify our notation, let
$$Q_a = \{m \text{ square free} : (m, a) = 1\}.$$

The sum over $d | a$ is a finite sum consisting of $2^{\omega(a)}$ terms with the maximal term is the one for $d = 1$. Therefore,
$$g^\Sigma_v(x) \leq 2^{\omega(a)} \sum_{m \leq \sqrt{x}, m \in Q_a} \left(\sum_{\ell \leq x/m^2, \ell \in Q_{am}} \frac{1}{\varphi(\ell)} \right) \frac{m}{\varphi(m)}$$
Using a simple inequality that
$$\varphi(m) \geq \sqrt{\frac m 2}$$
if $m$ is square-free (from $p - 1 \geq \sqrt{p}$ for all prime $p \geq 3$), we can bound
$$\sum_{\ell \leq x/m^2, \ell \in Q_{am}} \frac{1}{\varphi(\ell)} \leq \sqrt{2} \sum_{\ell \leq x/m^2} \ell^{-1/2} = O((x/m^2)^{1/2}) = O(x^{1/2} / m)$$
by Lemma 3.13 whence for some constant $C$ in the last $O$ notation,
$$g^\Sigma_v(x) \leq 2^{\omega(a)} \sum_{m \leq \sqrt{x}} C \frac{x^{1/2}}{m} \sqrt{2} m^{1/2} = x^{1/2} O(x^{1/4}) = O(x^{3/4})$$
by Lemma 3.13 again. Since $3/4 < 1$, we can now apply the stability theorem to get
$$x^s \cdot LHS = x^s \zeta(s) \left(\int_{1^-}^{\infty} t^{-s} dg\right) + O(x)$$
whence dividing both sides by $x^s$ and taking $x \rightarrow \infty$, we find the desired identity as
$$\int_{1^-}^{\infty} t^{-s} dg = \sum_{m = 1}^{\infty} \frac{c_m}{m^s} = \prod_{p | a} (...) \prod_{p \nmid a} (...)$$
using the computation for $g(p^k)$ at the beginning.

To show that $\sum c_m m^{-s}$ converges absolutely for $\Re s > 1/2$, we observe from the proof of the stability theorem (in particular, equation (3.19) in the book) that if $g^\Sigma_v(x) = O(x^\tau)$ then 
$$\sum_{m > x}^{\infty} \frac{|c_m|}{|m^s|} = \int_{x}^{\infty} |t^{-s}| \cdot d g^\Sigma_v = O(x^{\tau - \Re s})$$
converges to 0 as long as $\tau - \Re s < 0$ or equivalently, $\Re s > \tau$. To put it another way, suppose that $s$ is fixed with $\Re s = \frac12 + \epsilon$ for $\epsilon > 0$ and to show that $\sum c_m m^{-s}$ converges absolutely, we just have to show that $g^\Sigma_v(x) = O(x^\tau)$ for some $\tau < \frac12 + \epsilon$. This is tantamount to proving that
$$g^\Sigma_v(x) = O(x^{\frac12 + \epsilon})$$
for any $\epsilon > 0$.

To do this, recognize that the simple inequality $\varphi(m) \geq \sqrt{\frac m 2}$, which comes from $p - 1 \geq \sqrt{p}$ for all prime $p \geq 3$, can be improved rather easily: From
$$p - 1 = O(p^{1-\epsilon})$$
for any $0 < \epsilon < 1$, we have for all square-free $m$,
$$\varphi(m) \geq C_\epsilon^{-1} m^{1 - \epsilon}$$
for some constant $C_\epsilon$ depending on $\epsilon$ (to take care of the initial primes where $p - 1 < p^{1-\epsilon}$ since $p - 1 > p^{1-\epsilon}$ eventually). Just like we did previously, we then find
$$\sum_{\ell \leq x/m^2, \ell \in Q_{am}} \frac{1}{\varphi(\ell)} \leq C_\epsilon \sum_{\ell \leq x/m^2} \ell^{1 - \epsilon} = O((x/m^2)^{\epsilon}) = O(x^\epsilon/m^{2\epsilon})$$
by Lemma 3.13 whence if $C$ be the constant in the last $O$-notation,
$$g^\Sigma_v(x) \leq 2^{\omega(a)} \sum_{m \leq \sqrt{x}} C \frac{x^\epsilon}{m^{2\epsilon}} C_\epsilon m^\epsilon = O\left(x^{\epsilon} \sum_{m \leq \sqrt{x}} m^{-\epsilon}\right) = O(x^\epsilon (\sqrt x)^{1-\epsilon}) = O(x^{\frac 1 2 + \frac \epsilon 2}).$$
And we are done.

\item For this part, first note that
$$LHS = \int_{1^-}^{x} dN * dg$$
and we can apply the stability theorem right away with $N = x + O(1)$ and $g^\Sigma_v(x) = O(x^{\frac 12 + \epsilon})$ just proved to get
$$LHS = x \int_{1^-}^{\infty} t^{-1} dg + O(x^{\frac 12 + \epsilon}).$$

It remains to see that
$$\int_{1^-}^{\infty} t^{-1} dg = \frac{\varphi(a)}{a}.$$
Knowing that the series $\sum c_m m^{-s}$ converges absolutely for $\Re s > 1/2$, hence for $s = 1$, allows us to comnpute the integral using the Euler product:
$$\prod_{p|a}(1 - p^{-1}) \prod_{p \nmid a} \underbrace{\left( 1 + \frac{1}{(p-1)p} - \frac{p}{(p-1)p^2} \right)}_{1} = \prod_{p|a}(1 - p^{-1}) = \frac{\varphi(a)}{a}.$$

\item Observe that the LHS is exactly the LHS from part 1 if $s = 1$. So the same approach should work with appropriate modification to the estimation of $\widehat{H}_1$ that relies on $\Re s > 1$. Now $d H_1(t) = \delta_1 + dt$ and
\begin{align*}
\widehat{H}_1(x) &= \int_{1^-}^x (x/t) dN\\
&= \sum_{n \leq x} \frac{x}{t}\\
&= x( \log x + \gamma + O(x^{-1}))\\
&= x (\log x + \gamma) + O(1)
\end{align*}
by Lemma 3.13. So the stability theorem yields
$$x \cdot LHS = x (c_1 + c_0(\log x + \gamma)) + O(x^{1/2 + \epsilon})$$
where
$$c_\ell = \int_{1^-}^\infty (-\log t)^{-\ell} t^{-1} dg$$
In particular, $c_0 = \varphi(a)/a$ from the computation in part 2 and so
$$LHS = \frac{\varphi(a)}{a} \log x + \left\{ \frac{\varphi(a)}{a} \gamma + c_1 \right\} + O(x^{-1/2 + \epsilon}).$$
\end{enumerate}
\end{proof}

\unless\ifdefined\IsMainDocument
\end{document}
\fi

\unless\ifdefined\IsMainDocument
\documentclass[12pt]{article}
\usepackage{amsmath,amsthm}

\begin{document}
\fi

\textbf{Problem 3.26}: Let $\tau_2 = 1 * 1$ and $\tau_k = \tau_{k-1} * 1$ for any $k > 2$. Prove that
$$\sum_{n \leq x} \tau_k(n) = \frac{1}{(k-1)!} x \log^{k-1} x + O(x \log^{k-2} x).$$

\begin{proof}
Straight from my solution of 3.23 (4).
\end{proof}

\textbf{Problem 3.27}: Show that
$$\lim_{x \rightarrow \infty} \frac{1}{x} \sum_{n \leq x} \left( \frac{x}{n} - \left[ \frac{x}{n} \right] \right) = 1 - \gamma$$

\begin{proof}
From Lemma 3.13,
$$\frac{1}{x} \sum_{n \leq x} \frac{x}{n} = \log x + \gamma + O(x^{-1})$$
and from Corollary 3.32,
$$\frac{1}{x} \sum_{n \leq x} \left[\frac{x}{n}\right] = \frac{N_2(x)}{x} = \log x + (2\gamma - 1) + O(x^{-1/2})$$
and so their difference clearly goes to
$$\gamma - (2 \gamma - 1) = 1 - \gamma$$
as $x \rightarrow \infty$.
\end{proof}

\unless\ifdefined\IsMainDocument
\end{document}
\fi

\unless\ifdefined\IsMainDocument
\documentclass[12pt]{article}
\usepackage{amsmath,amsthm}

\begin{document}
\fi

\textbf{Problem 3.28}: Show that there exists constants $\alpha, \beta \in R$ such that
$$\sum_{n \leq x} (1 * 1 * 1)(n) = \frac{x \log^2 x}{2} + \alpha x \log x + \beta x + O(x^{2/3} \log ex).$$

\begin{proof}
By Theorem 3.31, we have
$$\sum_{n \leq x} (1 * 1 * 1)(n) = \int_{1^-}^x dN_2 * dN = I + J - N_2(y) N(z)$$
where $y, z \geq 1$ to be chosen with $yz = x$,
\begin{align*}
I &= \int_{1^-}^z N_2\left(\frac{x}{t} \right) dN(t) \\
&= \int_{1^-}^z \left\{\frac{x}{t} \log \left(\frac{x}{t}\right) + (2 \gamma - 1)\frac{x}{t} + O\left(\sqrt{\frac{x}{t}}\right) \right\} dN(t) \\
&= x \int_{1^-}^z \frac{(\log x + 2 \gamma - 1) - \log t}{t} dN(t) + O\left(\sqrt{x} \int_{1^-}^z t^{-1/2} dN(t)\right)\\
&= x \left[ \left. \frac{(\log x + 2 \gamma - 1) - \log t}{t} N(t) \right|_{1^-}^{z} - \int_{1^-}^z N(t) \; d\left\{ \frac{(\log x + 2 \gamma - 1) - \log t}{t} \right\} \right] + O(\sqrt{xz})\\
&= x \left[ \frac{(\log x + 2 \gamma - 1) - \log z}{z} N(z) - \int_{1^-}^z N(t) \; \left\{ \frac{\log t - \log x - 2 \gamma}{t^2} \right\} dt \right] + O(\sqrt{xz})\\
&= y (\log y + 2 \gamma - 1) N(z) - x \int_{1^-}^z (t + O(1)) \; \left\{ \frac{\log t - \log x - 2 \gamma}{t^2} \right\} dt + O(\sqrt{xz})\\
&= y (\log y + 2 \gamma - 1) N(z) - x \int_{1^-}^z \left\{ \frac{\log t - \log x - 2 \gamma}{t} \right\} dt + O\left(x \int_{1^-}^z \frac{2\gamma + \log x - \log t}{t^2} dt\right) + O(\sqrt{xz})\\
&= y (\log y + 2 \gamma - 1) N(z) - x \left( \left. \frac{\log^2 t}{2} - (\log x + 2 \gamma)\log t \right|_{1^-}^z \right) + O\left(x \frac{1 - 2 \gamma - \log x + \log z}{z} \right) + O(\sqrt{xz})\\
&= y (\log y + 2 \gamma - 1) N(z) - x \left(\frac{\log^2 z}{2} - (\log x + 2 \gamma)\log z  \right) + O\left(y \log y \right) + O(x y^{-1/2})\\
&= x \left(\log y + 2 \gamma - 1 - \frac{\log^2 z}{2} + \log x \log z + 2 \gamma \log z  \right) + O\left(y \log y \right) + O(x y^{-1/2})
\end{align*}
and
\begin{align*}
J &= \int_{1^-}^y N\left(\frac{x}{s}\right) dN_2(s) \\
&= \int_{1^-}^y \left( \frac{x}{s} + O(1) \right) dN_2(s) \\
&= x \int_{1^-}^y \frac{1}{s} dN_2(s) + O(N_2(y)) \\
&= x \left\{\left. \frac{N_2(s)}{s} \right|_{1^-}^{y} - \int_{1^-}^y N_2(s) d\left(\frac{1}{s}\right) \right\} + O(y \log y) \\
&= x \left\{ \frac{N_2(y)}{y} + \int_{1^-}^y N_2(s) s^{-2} ds \right\} + O(y \log y) \\
&= z N_2(y) + x \int_{1^-}^y \left\{ s^{-1} \log s + (2\gamma - 1)s^{-1} + O(s^{-3/2}) \right\} ds  + O(y \log y) \\
&= z N_2(y) + x \left. \left\{ \frac{\log^2 s}{2} + (2\gamma - 1) \log s + O(s^{-1/2}) \right\} \right|_{1^-}^y  + O(y \log y) \\
&= z N_2(y) + x \left\{ \frac{\log^2 y}{2} + (2\gamma - 1) \log y \right\} + O(x y^{-1/2}) + O(y \log y)
\end{align*}
(Note that when we write $s^{-2}$, we actually mean the function $s^{-2} \delta_1(s)$ i.e. $s^{-2}$ on $[1, \infty)$ and 0 on $(-\infty, 1)$. Likewise for $s^{-1}$.) 

The strategy is as described in problem 3.30: For $I$, we use the approximation for $N_2(x/t)$. Then integration by parts to turn $dN(t)$ into $N(t) [...] dt$ and then use approximation for $N(t) = t + O(1)$ to make everything into smooth functions. Ditto for $J$.

Combining these, one obtains
\begin{align*}
\sum_{n \leq x} (1 * 1 * 1)(n) &= x \left(\log y + 2 \gamma - 1 - \frac{\log^2 z}{2} + \log x \log z + 2 \gamma \log z  \right) + O\left(y \log y \right) + O(x y^{-1/2}) \\
&\qquad + z N_2(y) + x \left\{ \frac{\log^2 y}{2} + (2\gamma - 1) \log y \right\} + O(x y^{-1/2}) + O(y \log y) - N_2(y) N(z)\\
&= x \left(\log y + 2 \gamma - 1 - \frac{\log^2 z}{2} + \log x \log z + 2 \gamma \log z + \frac{\log^2 y}{2} + (2\gamma - 1) \log y \right) \\
&\qquad + O\left(y \log y \right) + O(x y^{-1/2})\\
&= x \left(2 \gamma - 1 \underbrace{- \frac{\log^2 z}{2} + \log^2 z + \log y \log z + \frac{\log^2 y}{2}}_{\frac 12 \log^2 x} + \underbrace{2 \gamma \log z + 2\gamma \log y}_{2\gamma \log x} \right) \\
&\qquad + O\left(y \log y \right) + O(x y^{-1/2})\\
&= \frac{x \log^2 x}{2} + 2 \gamma x \log x + (2 \gamma - 1) x + O(y \log y) + O(z y^{1/2})
\end{align*}

The optimal choice is when $z y^{1/2} = y$ in which case $z = y^{1/2}$ and so $y = x^{2/3}$ and $z = x^{1/3}$. With this choice, we have $O(y \log y) + O(z y^{1/2}) = O(x^{2/3} \log ex)$.
\end{proof}

\unless\ifdefined\IsMainDocument
\end{document}
\fi

\unless\ifdefined\IsMainDocument
\documentclass[12pt]{article}
\usepackage{amsmath,amsthm}

\begin{document}
\fi

\textbf{Problem 3.29}: Let $b > a > 0$. Show that the number of lattice points lying in the region 
$$R(a, b, x) := \{(s, t) : s, t > 0, s^a t^b \leq x\}$$
is
$$\zeta(a/b) x^{1/b} + \zeta(b/a) x^{1/a} + O(x^{1/(a+b)}).$$

\begin{proof}
Note that
$$R(a, b, x) = R(1, b/a, x^{1/a})$$
so it suffices to solve the problem when $a = 1$ and $b > 1$. The number of lattice point in $R(1, b, x) = \{s t^b \leq x\}$ is clearly given by
$$\sum_{n \leq x} [\sqrt[b]{x/n}] = \int_{1^-}^x N(\sqrt[b]{x/s}) \, dN(s) = \int_{1^-}^x dF * dN$$
where $F(t) = N(t^{1/b})$. Now we use Dirichlet hyperbola method to write the above as $I + J - K$ where
$$K = F(y) N(z) = y^{1/b} z + O(y^{1/b}) + O(z) + O(1),$$
\begin{align*}
I &= \int_{1^-}^z F(x/t) dN(t)\\
&= \int_{1^-}^z N(x^{1/b} / t^{1/b}) dN(t)\\
&= \sum_{n \leq z} N(x^{1/b} / n^{1/b})\\
&= \sum_{n \leq z} (x^{1/b} n^{-1/b} + O(1))\\
&= x^{1/b} \left( \frac{z^{1-1/b}}{1 - 1/b} + \zeta(1/b) + O(z^{-1/b}) \right) + O(z)\\
&= \zeta(1/b) x^{1/b} + \frac{b}{b - 1} y^{1/b} z + O(y^{1/b}) + O(z)
\end{align*}
and
\begin{align*}
J &= \int_{1^-}^y N(x/s) dF(s)\\
&= \int_{1^-}^y N(x/s) \, dN(s^{1/b})\\
&= \int_{1^-}^{y^{1/b}} N(x/u^b) \, dN(u) &\text{change of var } u = s^{1/b}\\
&= \sum_{n \leq y^{1/b}} N(x/n^b)\\
&= \sum_{n \leq y^{1/b}} (x/n^b + O(1))\\
&= x \left( \frac{y^{1/b(1-b)}}{1 - b} + \zeta(b) + O(y^{1/b(-b)}) \right) + O(y^{1/b})\\
&= \zeta(b) x - \frac{1}{1 - b} z y^{1/b} + O(z) + O(y^{1/b}).
\end{align*}

So we see that the number of lattice points in $R(1, b, x) = \{s t^b \leq x\}$ is $\zeta(b) x + \zeta(1/b) x^{1/b} + O(z) + O(y^{1/b}) + O(1)$: The terms with $z y^{1/b}$ in $I, J, K$ magically cancel each other. Choosing $z = x^{1/(b + 1)}$ and $y = x^{b/(b+1)}$ yields the result.
\end{proof}

\unless\ifdefined\IsMainDocument
\end{document}
\fi

\unless\ifdefined\IsMainDocument
\documentclass[12pt]{article}
\usepackage{amsmath,amsthm}
\renewcommand{\O}[1]{O\left( #1 \right)}
\begin{document}
\fi

\textbf{Problem 3.30}: Show that if
$$M(x) := \sum_{n \leq x} \mu(n) = \O{ x^{1/2 + \epsilon} }$$
holds for all $\epsilon > 0$ then
$$Q(x) - 6x/\pi^2 = \O{ x^{2/5+\epsilon} }$$
for every $\epsilon > 0$.

\begin{proof}
\newcommand{\Mhat}{\widehat{M}}
In the proof of Theorem 3.5, by writing $|\mu| = 1 * (\mu * |\mu|)$, we find that
$$Q(x) = \int_{1^-}^x d\Mhat * dN$$
where $\Mhat(t)$ is the summatory function of $\mu * |\mu|$. By Lemma 3.4,
$$\mu * |\mu| \; (n) = \begin{cases}
\mu(m) &\text{if } n = m^2,\\
0 &\text{if } n \text{ is not a square}
\end{cases}$$
so we have
$$\Mhat(t) = \sum_{m^2 \leq t} \mu(m) = M(\sqrt{t}) = \O{ t^{1/4 + \epsilon} }.$$

Break the last integral as $I + J - K$ by Dirichlet hyperbola method where $\Mhat(y) N(z) = \O{ y^{1/4 + \epsilon} z }$,
\begin{align*}
I &= \int_{1^-}^z \Mhat(x/t) \, dN(t) \\
&= \int_{1^-}^z \Mhat(x/t) \, dN(t) \\
&= \O{ \int_{1^-}^z (x/t)^{1/4 + \epsilon} \, dN(t) } &\text{since } N_v = N \\
&= \O{ x^{1/4 + \epsilon} \int_{1^-}^z t^{-(1/4 + \epsilon)} \, dN(t) } \\
&= \O{ x^{1/4 + \epsilon} z^{1-(1/4 + \epsilon)} } \\
&= \O{ y^{1/4 + \epsilon} z }
\end{align*}
and
\begin{align*}
J &= \int_{1^-}^y N(x/s) \; d M(\sqrt{s})\\
&= \int_{1^-}^{\sqrt y} N(x/t^2) \; d M(t) &\text{change of variable } t = \sqrt{s}\\
&= \int_{1^-}^{\sqrt y} (x/t^2 + \O{ 1 }) \; d M(t)\\
&= \int_{1^-}^{\sqrt y} x t^{-2} \; dM(t) + \O{ \sqrt y } &\text{as }M_v(t) = Q(t) = \O{ t }\\
&= x \left\{\frac{6}{\pi^2} - \int_{\sqrt y}^{\infty} t^{-2} \; dM(t)\right\} + \O{ \sqrt y }.
\end{align*}

It remains to assess the tail integral
$$R = \int_{\sqrt y}^{\infty} t^{-2} \; dM(t) = \sum_{n > \sqrt y} \frac{\mu(n)}{n^2}$$
better than we did in Theorem 3.5. We do so using the trick of Theorem 1.8, namely writing $\mu(n) = M(n) - M(n - 1)$:
\begin{align*}
R &= \sum_{n > \sqrt y} \frac{M(n) - M(n - 1)}{n^2}\\
&= \sum_{n > \sqrt y} \frac{M(n)}{n^2} -  \sum_{n > \sqrt y - 1} \frac{M(n)}{(n+1)^2}\\
&= \sum_{n > \sqrt y} M(n) \underbrace{\left(\frac{1}{n^2} - \frac{1}{(n+1)^2}\right)}_{= \frac{2n + 1}{n^2 (n+1)^2} = \O{ n^{-3} }} - \underbrace{\frac{M([\sqrt y])}{([\sqrt y] + 1)^2}}_{\O{ y^{1/4 + \epsilon - 1} }}\\
&= \O{ \sum_{n > \sqrt y} n^{1/2 + \epsilon - 3} } + \O{ y^{-3/4 + \epsilon} }\\
&= \O{ \int_{\sqrt y}^{\infty} t^{-5/2 + \epsilon} } + \O{ y^{-3/4 + \epsilon} } &\text{same idea as in Theorem 3.5}\\
&= \O{ (\sqrt y)^{-3/2 + \epsilon} } + \O{ y^{-3/4 + \epsilon} }\\
&= \O{ y^{-3/4 + \epsilon} }.
\end{align*}
So $x R = \O{ yz y^{-3/4 + \epsilon} } = \O{ y^{1/4 + \epsilon} z }$ as well. Combining all these computations,
$$Q(x) = I + J - K = 6x/\pi^2 + \O{ y^{1/4 + \epsilon} z } + \O{ \sqrt y }$$
and the optimal choice is to make $y^{1/4} z = y^{1/2}$ or $y = x^{4/5}$ and $z = x^{1/5}$, in which case we find $\O{ y^{1/4 + \epsilon} z } + \O{ \sqrt y } = \O{ x^{2/5 + \epsilon} }$.

\textbf{Epiphany}: I now realize that the method of writing $\mu(n) = M(n) - M(n-1)$, in essence, is trying to turn the integral
$$\int t^{-2} dM$$
with a hard-to-control integrator $dM$ (for one, it is not monotone and its total variation $M_v(t) = Q(t)$ is of a magnitude higher than $M$) into the integral
$$\int M(t) t^{-3} dt$$
in which not only the familiar integrator $dt$ is increasing but we can make use of the given better estimate for $M(t)$ without incuring the penalty of going into its total variation $M_v(t)$.

Now $t^{-3} dt$ is pretty much $d(t^{-2})$ and so what we did by writing $\mu(n) = M(n) - M(n-1)$ is nothing transforming $t^{-2} \; dM$ into $M(t) \; d(t^{-2})$ a.k.a. performing an \textbf{integration by parts} in disguise. With this understanding, we can redo the entire computation in an easier manner:
\begin{align*}
R &= \int_{\sqrt y}^\infty t^{-2} \; dM(t)\\
&= \left. \frac{M(t)}{t^2} \right|_{\sqrt y}^{\infty} - 2 \int_{\sqrt y}^\infty M(t) \; d(t^{-2})\\
&= \frac{M(\sqrt y)}{y} + 2 \int_{\sqrt y}^\infty M(t) t^{-3} \; dt\\
&= \O{ (\sqrt y)^{1/2 + \epsilon} y^{-1} } + \O{ \int_{\sqrt y}^\infty t^{1/2 + \epsilon - 3} \; dt }\\
&= \O{ y^{-3/4 + \epsilon} }.
\end{align*}

It appears that in these various applications of the hyperbola method, the way we deal with the two smaller integrals
$$\int_{1^-}^z F(x/t) dG(t) \quad \text{ and } \quad \int_{1^-}^z G(x/s) dF(s)$$
involves 3 steps:
\begin{itemize}
\item Use the known estimate for $F(x/t)$ and $G(x/s)$ by smooth functions of $t$ and $s$.
\item Perform integration by parts to switch the harder integrator $dG$ and $dF$ into $ds$ and $dt$.
\item Use the estimate again and resolve the final classical integrals; or make use of Lemma 3.13 and others.
\end{itemize}
\end{proof}

\unless\ifdefined\IsMainDocument
\end{document}
\fi


\chapter{The Distribution of Prime Numbers}

\newcommand{\val}[1]{\nu\left( #1 \right)}
\newcommand{\pval}[1]{\nu_p\left( #1 \right)}

\unless\ifdefined\IsMainDocument
\documentclass[12pt]{article}
\usepackage{amsmath,amsthm,amssymb}

\begin{document}
\fi

On page 72: Recall
$$\kappa(n) = \begin{cases}
1/j &\text{if } n = p^j,\\
0 &\text{if } n = 1 \text{ or } \omega(n) > 1.
\end{cases}$$
So
$$\sum_{n \leq x} \kappa(n) = \sum_{p^j \leq x} \frac{1}{j} = \sum_{j = 1}^{\infty} \frac{1}{j} \sum_{p^j \leq x} 1  = \sum_{j \leq x} \frac{1}{j} \pi(x^{1/j}).$$


On page 74: Let us verify that $L \, dN = d\psi * dN$ is a consequence of $L1 = \Lambda * 1$. To do so, we integrate
$$\int_{1^-}^x L \, dN = \int_{1^-}^x \log t \, dN(t) = \sum_{n \leq x} \log n$$
while $L1 = \Lambda * 1$ implies
$$\int_{1^-}^x d\psi * dN = \int_{1^-}^x d(L1^\Sigma) = \sum_{n \leq x} L1(n) = \sum_{n \leq x} \log n.$$

On page 80: To prove the second identity of Lemma 4.10, we use the left hand side of (4.3) for $R(x)$, namely
$$R(x) := \int_1^x t^{-1} d\psi = \sum_{n \leq x} \frac{\Lambda(n)}{n} = \sum_{n \leq x} \frac{\log(n) \, \kappa(n)}{n}$$
and so it plays out exactly as in the proof of the first identity:
$$\sum_{n \leq x} \frac{\kappa(n)}{n} = \int_{2^-}^x \log^{-1} t \; dR(t) = ...$$
We find $B_2$ is given by exactly the same expression as $B_1$ except that the $R(x)$ are different.

\unless\ifdefined\IsMainDocument
\end{document}
\fi


\section{Problems}
\unless\ifdefined\IsMainDocument
\documentclass[12pt]{article}
\usepackage{amsmath,amsthm,amssymb}
\newcommand{\Z}{\mathbb{Z}}
\begin{document}
\fi

\textbf{Problem 4.1}: Given $r \in \Z^+$, find some number $x_0 = x_0(r)$ such that $x > (\log x)^r$ for all $x \geq x_0$.

\begin{proof}
For $r = 1$, we just take $x_0(1) = 0$ since $x > \log x$ for all $x \geq 0$ as it is equivalent to $e^x > x$ for $x \geq 0$; which is evident considering the power series for $e^x$.

So let assume $r \geq 2$. The inequality is equivalent to $x^{1/r} > \log x$. The function
$$f(x) = x^{1/r} - \log x$$
is increasing on $[r^r, \infty)$ as its derivative
$$f'(x) = r^{-1} x^{1/r - 1} - x^{-1} = r^{-1} x^{-1} (x^{1/r} - r) \geq 0$$
on that interval. Check that
\begin{align*}
f(r^{r^\alpha}) &= (r^{r^\alpha})^{1/r} - \log(r^{r^\alpha})\\
&= r^{r^{\alpha - 1}} - r^\alpha \log r\\
&> r^{r^{\alpha - 1}} - r^{\alpha + 1} &\text{because } \log r < r\\
&\geq 0 &\text{if } r^{\alpha - 1} \geq \alpha + 1
\end{align*}
So we can choose $\alpha = 3$ i.e. take $x_0(r) = r^{r^3}$. Then for all $x \geq x_0(r) > r^r$, we have $f(x) \geq f(x_0(r)) > 0$ so the inequality is true. 
\end{proof}

\textbf{Problem 4.2}: Derive a lower bound for $\pi(x)$ from inequality (4.1).

\begin{proof}
Inequality (4.1) yields
$$[x] \leq (1 + \log_2 x)^{\pi(x)}$$
so after taking the logarithm on both sides, we deduce
$$\frac{\log [x]}{\log (1 + \log_2 x)} \leq \pi(x).$$
\end{proof}

\unless\ifdefined\IsMainDocument
\end{document}
\fi

\unless\ifdefined\IsMainDocument
\documentclass[12pt]{article}
\usepackage{amsmath,amsthm,amssymb}
\newcommand{\IntPart}[1]{\left[ #1 \right]}
\begin{document}
\fi

\textbf{Problem 4.3}: Show that for any positive integer $r$ there exists a number $c_r$ such that
$$\pi(x) \leq (x + c_r) \prod_{i=1}^{r} (1 - p_i^{-1})$$
for all positive $x$. Show that we may take $c_1 = 1$, $c_2 = 5$, $c_3 = 11$.

\begin{proof}
Let $N = \prod_{i = 1}^r p_i$ as in the proof of Lemma 4.3 and we recognize $\prod_{i=1}^{r} (1 - p_i^{-1})$ as $\varphi(N) / N$. So the inequality is equivalent to
$$\pi(x) \leq (x + c_r) \frac{\varphi(N)}{N} = x \frac{\varphi(N)}{N} + c_r \frac{\varphi(N)}{N}$$
and so the problem is tantamount to showing that
$$\pi(x) \leq x \frac{\varphi(N)}{N} + C_r$$
for some constant $C_r$.

Let $S$ be the set in the proof of Lemma 4.3 where we have
\begin{itemize}
\item $\pi(x) \leq r + S(x)$ and
\item $S(x) < S\left(\IntPart{ \frac{x + N}{N} } x\right) = \IntPart{ \frac{x + N}{N} } \varphi(N) \leq \left( \frac{x}{N} + 1 \right) \varphi(N) = x \frac{\varphi(N)}{N} + \varphi(N)$
\end{itemize}
and so $C_r = r + \varphi(N)$ works. Then
$$c_r = \frac{N C_r}{\varphi(N)} = \frac{N (r + \varphi(N))}{\varphi(N)} = N + \frac{N r}{\varphi(N)}.$$

Unfortunately, this choice of $c_r$ is larger for the remaining parts. So we make separate arguments for them
\begin{itemize}
\item $r = 1$ and $c_1 = 1$: The inequality then reads $\pi(x) \leq \frac12 (x + 1)$ which is evident considering primes are odd numbers, except for 2.
\item $r = 2$ and $c_2 = 5$: The inequality then reads $\pi(x) \leq \frac13 (x + 5) = \frac13 x + \frac53$. Now a prime $\geq 5$ can only be congruent to 1 or 5 modulo 6 and there are at most $\frac13x$ of those numbers. Adding $\frac53$ clearly accounts for 2 and 3.
\item $r = 3$ and $c_2 = 11$: The inequality then reads $\pi(x) \leq \frac4{15} (x + 11)$. Similar argument.
\end{itemize}
\end{proof}

\unless\ifdefined\IsMainDocument
\end{document}
\fi

\unless\ifdefined\IsMainDocument
\documentclass[12pt]{article}
\usepackage{amsmath,amsthm,amssymb}

\begin{document}
\fi

\textbf{Problem 4.4}: Show that $\psi(N)$ equals the least common multiple of the positive integers $1, 2, ..., N$.

\begin{proof}
Write the least common multiple in prime factorization $\ell = \prod_{i=1}^{r} p_i^{\alpha_i}$. We see that the $\alpha_i$ should be the maximum exponent for $p_i$ such that $p_i^{\alpha_i} \leq N$ so that $\ell$ is divisible by the number $p_i^{\alpha_i}$ amongst the numbers $1, 2, ..., N$. The choice of $\alpha_i$ being the maximum exponent suffices because any other number amongst $1, 2, ..., N$ cannot be divisible by higher power of $p_i$.

Clearly, $\psi(N) = \sum_{\text{max } p^\alpha \leq n} \alpha \log p = \log \ell$.
\end{proof}

\unless\ifdefined\IsMainDocument
\end{document}
\fi

\unless\ifdefined\IsMainDocument
\documentclass[12pt]{article}
\usepackage{amsmath,amsthm,amssymb}
\renewcommand{\O}[1]{O\left( #1 \right)}
\begin{document}
\fi

\textbf{Problem 4.5}: Let $\chi_{(2,\infty)}$ denote the indicator function of the interval $(2, \infty)$. Use the approximate inverse $\delta_1 - 2 \chi_{(2, \infty)}(t) t^{-1} dt$ to show that
$$\limsup_{x \rightarrow \infty} \psi(x) / x \leq 1 + \log 2$$

\begin{proof}
To simplify our notation, let $\chi = \chi_{(2,\infty)}$. We have
\begin{align*}
R(x) &= \int_{1^-}^x dN * (\delta_1 - 2 \chi(t) t^{-1} dt)\\
&= N(x) - 2 \underbrace{\int_{1^-}^x N(x/t) \chi(t) t^{-1} dt}_{0 \text{ if } x \leq 2}\\
&= N(x) - 2 \int_2^x N(x/t) t^{-1} dt &\text{for } x > 2\\
&= N(x) - 2 \int_1^{x/2} N(u) u^{-1} du &\text{by change of variable } u = x/t\\ %$$N(u) (x/u)^{-1} d(x/u) = N(u) (u / x) (-x/u^2) du = -N(u) u^{-1} du$$
&= N(x) - 2 \int_1^{x/2} \frac{N(u) - u}{u} du - 2 \int_1^{x/2} du\\
&= N(x) + 2 \int_1^{x/2} \frac{u - N(u)}{u} du - 2 \left( \frac{x}{2} - 1 \right)\\
&= N(x) - x + 2 + 2 \int_1^{x/2} \frac{u - N(u)}{u} du\\
&\geq 1
\end{align*}
by the same argument as in the proof of Lemma 4.5: the integral is increasing and $N(x) - x + 2$ is periodic of period 1 so $R(x + 1) > R(x)$ and thus the minimum value of $R$ occurs at some $x \in (1, 2)$ and we find $R(x) \geq R(2^-) = 1$ and so
\begin{align*}
\int_{1^-}^x d \psi * dN * (\delta_1 - 2 \chi(t) t^{-1} dt) &= \int_{1^-}^x d \psi * dR\\
&= \int_{1^-}^x R(x/t) d\psi\\
&\geq \int_{1^-}^x d\psi &\text{ as } \psi \text{ is increasing}\\
&= \psi(x).
\end{align*}

On the other hand, let
\begin{align*}
F(x) &:= \int_{1^-}^x L dN = x \log x - x + O(\log ex)
\end{align*}
and we have for $x > 2$:
\begin{align*}
&\int_{1^-}^x L dN * (\delta_1 - 2 \chi(t) t^{-1} dt)\\
=& \int_{1^-}^x L dN - \int_{1^-}^x L dN * 2 \chi(t) t^{-1} dt\\
=& F(x) - \int_{1^-}^x F(x/u) \cdot 2 \chi(u) u^{-1} du\\
=& F(x) - 2 \int_{2}^x F(x/u) u^{-1} du\\
=& F(x) - 2 \int_1^{x/2} F(s) s^{-1} ds &\text{change of variable } s = x/u\\
=& F(x) - 2 \int_1^{x/2} \{ s \log s - s + \O{ \log es } \} s^{-1} ds\\
=& F(x) - 2 \left\{ F(x/2) - (x/2) + \O{ \int_1^{x/2} \log^2(es) ds } \right\}\\
=& x \log x - x - 2 \left\{ (x/2) \log(x/2) - (x/2) - (x/2) \right\} + \O{ \log^2 ex }\\
=& x \log x + x - x \log(x/2) + \O{ \log^2 ex }\\
=& (1 + \log 2) \, x + \O{ \log^2 ex }.
\end{align*}

Combining the two inequalities and Chebyshev's identity $d\psi * dN = L dN$ and we derive the inequality.

\textbf{Remark}: While solving this problem, I discover that we have to be very mindful when performing a change of variable.

The rule that always work is that if $s(u)$ is a strictly increasing\footnote{Can be improved to monotone but requires some proper care to the endpoints.} (hence one-to-one) continuous function then
$$\int_{a}^{b} g(s(u)) \, dF(s(u)) = \int_{s(a)}^{s(b)} g(t) \, dF(t).$$
This formulation\footnote{The alternative formula is replacing $s$ by $s^{-1}$ i.e.
$$\int_a^b g(t) \, dF(t) = \int_{s(a)}^{s(b)} g(s^{-1}(u)) \, dF(s^{-1}(u))$$
as it means we are replacing $u = s(t)$.} is usually used when the integrand can be recognized as a composition as in
$$\int_a^b (\log^2 u + \log u + 1) \, d(\log u)$$
whence one wants to put $t = \log u$ to turn the integral into $\int_{\log a}^{\log b} (t^2 + t + 1) dt$.

\begin{proof}
This comes straight from the definition of integral: Let
$$A = \int_{a}^{b} g(s(u)) \, dF(s(u))$$
and $\epsilon > 0$ be arbitrary. We want to find a partition $\Pi_\epsilon$ of $[s(a), s(b)]$ such that $|S(g, F, \Pi, T) - A| < \epsilon$ for all $\Pi$ that refines $\Pi_\epsilon$ and all choices of intermediate points $T = \{ t_i : t_i \in [x_i, x_{i+1}] \}$. Let $\Pi_\epsilon^*$ be the partition of $[a, b]$ in the definition of $A$ and $\Pi_\epsilon = s(\Pi_\epsilon^*) = \{s(x_i) : x_i \in \Pi_\epsilon^*\}$ be the image of $\Pi_\epsilon^*$ under $s$. Now if $\Pi$ is a refinement of $\Pi_\epsilon$ then $s^{-1}(\Pi)$ is a refinement of $\Pi_\epsilon^*$ and $s^{-1}(T)$ is a set of intermediate points of $s^{-1}(\Pi)$ \emph{due to our assumption on $s$ strictly increasing} whence
$$|S(g, F, \Pi, T) - A| = |S(g \circ s, F \circ s, s^{-1}(\Pi), s^{-1}(T)) - A| < \epsilon.$$
\end{proof}

Here is the warning: In the context of the book,
$$2 \, dt \not= d(2t)$$
so as a consequence,
$$\int_a^b g(t) \, d(2t) = 2 \int_a^b g(t) \, dt$$
does not work. The problem is because in the book, the measure $dt$ is truncated by $[1, \infty)$. The intuition is the following:
\begin{itemize}
\item $dt$ is the measure that gives length (read: Lebesgue measure) to subsets $X$ of the real numbers after taking $X \cap [1, \infty)$. In other words, $dt( [a, b] )$ is the usual length of $[a, b] \cap [1, \infty)$. For example, $dt( [0, 1] ) = 0$ because $[0, 1] \cap [1, \infty) = [1, 1]$ is a single point even though the interval $[0, 1]$ has length 1.

\item $d(2t)$ measures the length of $X$ by first scaling $X$ by 2 and then taking the intersection with $[1, \infty)$. As an example, $d(2t)$ returns 1 for $X = [0, 1]$ because $2X$ is the interval $[0, 2]$ whose intersection with $[1, \infty)$ is the interval $[1, 2]$, which is of length 1.

\item Now it is clear that $2dt$ and $d(2t)$ give different measures to the same interval $[0, 1]$.
\end{itemize}

To see it clearly, recall that $dt = d G_0(t)$ where
$$G_0(t) = \begin{cases} t - 1 &\text{if } t \geq 1\\ 0 &\text{if } t < 1 \end{cases}$$
so $2dt = d G_1(t)$ where
$$G_1(t) = \int_{1^-}^{t} 2 dG_0(u) = 2 G_0(t) = \begin{cases} 2(t - 1) &\text{if } t \geq 1\\ 0 &\text{if } t < 1 \end{cases}$$
while $d(2t) = dG_2(t)$ where
$$G_2(t) = G_0(2t) = \begin{cases} 2t - 1 &\text{if } t \geq 1/2\\ 0 &\text{if } t < 1/2 .\end{cases}$$
Clearly $G_1(t) - G_2(t)$ is not a constant so $d G_1 \not= d G_2$.

In this problem, both change of variables $u = x/t$ in $dt$ and $s = x/u$ in $du$ are safe. In other words, we do have
$$d\left(\frac{x}{t}\right) = \frac{-x}{t^2} dt$$
\textbf{when integrating from $1$ to $x/2$}. To check: the function associated to $d\left(\frac{x}{t}\right)$ is
$$H_1(t) = G_0\left(\frac{x}{t}\right) = \begin{cases} x/t - 1 &\text{if } t \leq x, \\ 0 &\text{if } t > x\end{cases}$$
by the substitution rule described above whereas the function associated to the integrator $\frac{-x}{t^2} dt$, for $t \geq 1$ is
$$H_2(t) = \int_1^{t} \frac{-x}{u^2} du = x (1/t - 1) = x/t - x$$
and 0 for $t < 1$. So $H_1 - H_2$ is a constant, namely $x - 1$, on $[1, \frac{x}{2}]$.
\end{proof}

\unless\ifdefined\IsMainDocument
\end{document}
\fi

\unless\ifdefined\IsMainDocument
\documentclass[12pt]{article}
\usepackage{amsmath,amsthm,amssymb}
\renewcommand{\O}[1]{O\left(#1\right)}
\begin{document}
\fi

\textbf{Problem 4.6}: Obtain an upper estimate for $\psi(x) - \psi(x/2)$ by using the approximate inverse $\delta_1 - 2\delta_2$. Use the inequality you have found to obtain an upper estimate for $\psi(x)$.

\begin{proof}
In the proof of Lemma 4.6, observe that
\begin{align*}
\int_1^x R(x/t) d\psi &= \sum_{1 < n \leq x} R(x/n) \Lambda(n)\\
&\geq \sum_{x/2 < n \leq x} R(x/n) \Lambda(n) &\text{since }R(x/n) \Lambda(n) \geq 0\\
&= \sum_{x/2 < n \leq x} \Lambda(n) &\text{since } 1 \leq x/n < 2 \text{ if } x/2 < n \leq x\\
&= \psi(x) - \psi(x/2)
\end{align*}
because $R(t) = N(t) - 2N(t/2) = 1$ if $1 \leq t < 2$. Thus, we obtain an upper bound
$$\psi(x) - \psi(x/2) \leq x \log 2 + \O{ \log ex }.$$

We then get
\begin{align*}
\psi(x) &= \sum_{n = 0}^{k} \psi(x/2^n) - \psi(x/2^{n+1}) &\text{where } k = [\log_2 x]\\
&\leq \sum_{n = 0}^{k} \frac{x}{2^n} \log 2 + \O{ \log\left( e\frac{x}{2^n} \right) }\\
&\leq 2 x (1 - (1/2)^{k+1}) \log 2 + \O{ (k + 1) \log ex }\\
&\leq x \log 4 + \O{ \log^2 ex }.
\end{align*}
\end{proof}

\unless\ifdefined\IsMainDocument
\end{document}
\fi

\unless\ifdefined\IsMainDocument
\documentclass[12pt]{article}
\usepackage{amsmath,amsthm,amssymb}
\newcommand{\Z}{\mathbb{Z}}
\newcommand{\IntPart}[1]{\left[ #1 \right]}
\newcommand{\val}[1]{\nu\left( #1 \right)}
\newcommand{\pval}[1]{\nu_p\left( #1 \right)}
\begin{document}
\fi

\textbf{Problem 4.7}: For $n \in \Z^+$ put $L(n) = \exp \psi(n)$.
\begin{enumerate}
\item For $k \in \Z^+$ show that
$$L(2k) \leq 2L(2k-1)$$
$$L(2k+1) \leq \binom{2k+1}{k+1} L(k+1) < 2^{2k} L(k+1)$$
\item Use (1) to show that $L(n) < 4^n$ i.e. $\psi(n) < n \log 4$ for $n \in \Z^+$.
\end{enumerate}

\begin{proof}
\begin{enumerate}
\item Evidently, $2L(2k-1)$ is divisible by $1, 2, ..., 2k - 1$ and is also divisible by $2k$ since $k | L(2k-1)$. Thus, $2L(2k-1)$ is a common multiple of $\{1, 2, ..., 2k\}$ and so it is $\geq L(2k)$ which is the least common multiple of those numbers.

To prove the second inequality, we show that the quotient
$$\frac{L(2k+1)}{L(k+1)} \text{ divides } \binom{2k+1}{k+1}$$
by first observing that the quotient is square-free. This is easy to see as its logarithm is:
$$\psi(2k+1) - \psi(k+1) = \sum_{k + 1 < p^\alpha \leq 2k+1} \log p$$
and any prime $p$ can only participate once in the sum: For we then have $2(k+1) < 2 p^\alpha$ and so $p^\alpha$ is the maximum power of $p$ not exceeding $2k+1$. But if $p$ is such a prime then it divides $\binom{2k+1}{k+1}$ by $p$-adic valuation:
$$\pval{\binom{2k+1}{k+1}} \geq \underbrace{\IntPart{\frac{2k+1}{p^\alpha}}}_{1} - \underbrace{\IntPart{\frac{k+1}{p^\alpha}}}_{0} - \underbrace{\IntPart{\frac{k}{p^\alpha}}}_{0} = 1.$$

The remaining inequality is simple:
\begin{align*}
2^{2k+1} &= (1 + 1)^{2k+1}\\
&= \sum_{j = 0}^{2k+1} \binom{2k + 1}{j}\\
&> \binom{2k+1}{k} + \binom{2k+1}{k+1}\\
&= 2 \binom{2k+1}{k+1}.
\end{align*}

\item Finally, we prove $L(n) < 4^n$ by induction. The base case $n = 1$ is evident. Suppose that the inequality is true for all $n < k$ and we show it is true for $n = k$.
\begin{itemize}
\item If $k = 2 \ell$ is even then
$$L(k) = L(2\ell) \leq 2 L(2\ell - 1) < 2 (4^{2\ell - 1}) < 4^{2\ell} = 4^k.$$
\item If $k = 2 \ell + 1$ is odd then
$$L(k) = L(2\ell+1) < 2^{2\ell} L(\ell + 1) < 4^{\ell} (4^{\ell + 1}) = 4^{2\ell + 1} = 4^k.$$
\end{itemize}
\end{enumerate}
\end{proof}

\unless\ifdefined\IsMainDocument
\end{document}
\fi

\unless\ifdefined\IsMainDocument
\documentclass[12pt]{article}
\usepackage{amsmath,amsthm,amssymb}
\newcommand{\Z}{\mathbb{Z}}
\newcommand{\IntPart}[1]{\left[ #1 \right]}
\newcommand{\val}[1]{\nu\left( #1 \right)}
\newcommand{\pval}[1]{\nu_p\left( #1 \right)}
\begin{document}
\fi

\textbf{Problem 4.8}: Let $L(\cdot)$ be as in the preceeding problem.
\begin{enumerate}
\item For $k \in \Z^+$ show that $L(2k)$ is divisible by $k \binom{2k}{k}$.
\item If $k \in \Z^+$ and $k \geq 4$, prove by induction that $k \binom{2k}{k} > 2^{2k}$.
\item Prove that $L(n) > 2^n$ for $n \geq 7$.
\end{enumerate}

\begin{proof}
\begin{enumerate}
\item For $n \in \Z^+$, we denote $\pval{n}$ the exponent of $p$ in the standard factorization of $n$. In other words, $\pval{n}$ is the number $\alpha$ such that $p^\alpha \;|\; n$ but $p^{\alpha+1} \nmid n$. Note that we then have $n = \prod_p p^{\pval{n}}$. We shall drop the subscript $p$ in the notation if $p$ is implicitly understood from the context.

Note that by unique factorization theorem, to prove that $m \;|\; n$, it suffices to prove that $\pval{m} \leq \pval{n}$ for all prime $p$. So fix an arbitrary $p$ and we show that
$$\val{L(2k)} \geq \val{k\binom{2k}{k}}.$$
Recall that the left hand side is $[\log_p (2k)]$ whereas the right hand side is
$$A := \val{k\binom{2k}{k}} = \val{\frac{(2k)!}{k!(k-1)!}} = \sum_{r = 1}^{\infty} \IntPart{ \frac{2k}{p^r} } - \IntPart{ \frac{k}{p^r} } - \IntPart{ \frac{k-1}{p^r} }$$
from the formula
$$\val{n!} = \sum_{r = 1}^{\infty} \IntPart{ \frac{n}{p^r} }$$
for any $n$. Thus to establish the inequality, we only need to show that
$$\IntPart{ \frac{2k}{p^r} } - \IntPart{ \frac{k}{p^r} } - \IntPart{ \frac{k-1}{p^r} } \leq 1$$
for all $r$ since the terms for $r > [\log_p(2k)]$ are all zero. Write $k - 1 = m p^r + s$ where $0 \leq s < p^r$ by division algorithm. Then $\IntPart{ \frac{k-1}{p^r} } = m$,
$$\IntPart{ \frac{k}{p^r} } = \begin{cases}
m + 1 &\text{if } s = p^r - 1,\\
m &\text{if } s < p^r - 1
\end{cases}$$
and
$$\IntPart{ \frac{2k}{p^r} } = \begin{cases}
2m + 1 &\text{if } s \geq p^r / 2,\\
2m &\text{if } s < p^r / 2.
\end{cases}$$
So the maximum possible value for
$$\IntPart{ \frac{2k}{p^r} } - \IntPart{ \frac{k}{p^r} } - \IntPart{ \frac{k-1}{p^r} } \leq (2m + 1) - (m) - (m) = 1.$$

\item For $k = 4$ we have $4 \binom{8}{4} = 4 \frac{8 \cdot 7 \cdot 6 \cdot 5}{4!} = 8 \cdot 35$ where as $2^{2 \cdot 4} = 2^8 = 8 \cdot 2^5 = 8 \cdot 32$.

Suppose that the inequality is true for $k$. We prove that it is true for $k + 1$: We have
\begin{align*}
(k+1) \binom{2(k+1)}{k+1} &= (k+1) \frac{2k+2}{k+1} \binom{2k+1}{k} &\text{as } \binom{n+1}{k+1} = \frac{n+1}{k+1} \binom{n}{k}\\
&= (2k+2) \frac{2k+1}{k+1} \binom{2k}{k} &\text{as } \binom{n+1}{k} = \frac{n+1}{n-k+1} \binom{n}{k}\\
&= 2 (2k+1) \binom{2k}{k}\\
&> 2 (2) (2^{2k}) = 2^{2(k+1)}.
\end{align*}

\item If $n = 2k \geq 7$ is even then $k \geq 4$, then obviously
$$L(n) = L(2k) \geq k\binom{2k}{k} > 2^{2k} = 2^n.$$
If $n = 2k - 1 \geq 7$ is odd then we use the previous problem
$$L(n) = L(2k-1) \geq \frac{L(2k)}{2} > 2^{2k - 1}.$$
\end{enumerate}
\end{proof}

\unless\ifdefined\IsMainDocument
\end{document}
\fi

\unless\ifdefined\IsMainDocument
\documentclass[12pt]{article}
\usepackage{amsmath,amsthm,amssymb}
\newcommand{\Abs}[1]{\left| #1 \right|}
\begin{document}
\fi

\textbf{Problem 4.9}: Assuming the truth of Lemma 4.8, give another proof that
$$\limsup_{x \rightarrow \infty} \frac{\psi(x)}{x} < \infty; \qquad \liminf_{x \rightarrow \infty} \frac{\psi(x)}{x} > 0.$$

\begin{proof}
Using integration by parts, we have
$$\int_1^x t^{-1} d\psi = \frac{\psi(x)}{x} + \int_1^x \psi(t) t^{-2} dt$$
so by Lemma 4.8 we get
$$\frac{\psi(x)}{x} = O(1)$$
which yields $\limsup < \infty$.

Now suppose that $C$ is the constant in the $O(1)$ of (4.3); that is
$$\Abs{ \int_1^x t^{-1} d\psi - \log x } < C$$
for all $x \geq 1$. Fix any $k > 1$ such that that $\log k > 2C$; for example, $k = e^{2|C| + 1}$. For any $x \geq 1$, apply (4.3) to $x$ and $kx$
$$\Abs{ \int_1^{kx} t^{-1} d\psi - \log (kx) } \leq C; \qquad \qquad \Abs{ \int_1^x t^{-1} d\psi - \log x } \leq C$$
to deduce that
$$\Abs{ \int_x^{kx} t^{-1} d\psi - \log k } \leq 2C$$
so
\begin{align*}
0 < \log k - 2C &\leq \int_x^{kx} t^{-1} d\psi\\
&\leq \int_x^{kx} x^{-1} d\psi &\text{as } \psi \text{ increasing and } t^{-1} \leq x^{-1}\\
&= x^{-1} \int_x^{kx} d\psi\\
&= \frac{\psi(kx) - \psi(x)}{x}\\
&< \frac{\psi(kx)}{x} &\text{because } \psi \geq 0.
\end{align*}

Thus we proved
$$\frac{\psi(kx)}{kx} \geq \underbrace{\frac{\log k - 2C}{k}}_{K} > 0$$
for all $x \geq 1$. That implies
$$\frac{\psi(x)}{x} \geq K$$
for all $x \geq k$ and that means the $\liminf$ is bounded from below by the constant $K > 0$.
\end{proof}

\unless\ifdefined\IsMainDocument
\end{document}
\fi

\unless\ifdefined\IsMainDocument
\documentclass[12pt]{article}
\usepackage{amsmath,amsthm,amssymb}

\begin{document}
\fi

\textbf{Problem 4.10}: Show that the integers in the interval $[1, x]$ have an average of $\log \log x + B_1$ distinct prime divisors.

\begin{proof}
\renewcommand{\P}{\mathcal{P}}
We have seen in problem 2.6 that $\omega = 1 * 1_\P$ where $1_\P$ is the indicator function of the set of primes. The average number of distinct prime divisor of integers in $[1, x]$ is given by
\begin{align*}
\frac{\sum_{n \leq x} \omega(n)}{N(x)} &= \frac{\sum_{n \leq x} (1 * 1_\P)(n)}{N(x)}\\
&= \frac{\sum_{p \leq x} N(x/p)}{N(x)}\\
&= \frac{x \sum_{p \leq x} p^{-1} + O(\pi(x))}{N(x)}\\
&= \frac{x}{N(x)} \sum_{p \leq x} p^{-1} + o(1) &\text{by Lemma 4.3}\\
&\sim \log \log x + B_1
\end{align*}
by Lemma 4.10.
\end{proof}

\unless\ifdefined\IsMainDocument
\end{document}
\fi

\unless\ifdefined\IsMainDocument
\documentclass[12pt]{article}
\usepackage{amsmath,amsthm,amssymb}

\begin{document}
\fi

\textbf{Problem 4.11}: Use the preceding method, with a suitable choice of truncation point, estimate
$$\lim_{x \rightarrow \infty} \left\{\sum_{n \leq x} \frac{\kappa(n)}{n} - \sum_{p \leq x} \frac{1}{p} \right\}$$
with an error smaller than 0.0001.

\begin{proof}
Recall on page 81 that
$$\lim_{x \rightarrow \infty} \left\{\sum_{n \leq x} \frac{\kappa(n)}{n} - \sum_{p \leq x} \frac{1}{p} \right\} = \sum_p \left\{ - \log (1-p^{-1}) - p^{-1} \right\}$$
so let us consider the function
$$f(x) := -\log (1-x^{-1}) - x^{-1}.$$
One has
$$f'(x) = - (1-x^{-1})^{-1} (x^{-2}) + x^{-2} = -(x^2 - x)^{-1} + x^{-2} = -\frac{1}{(x-1)x^2}$$
and
$$f''(x) = (x^2 - x)^{-2} (2x - 1) - 2x^{-3} = \frac{x (2x-1) - 2(x-1)^2}{(x - 1)^2 x^3} = \frac{3 x - 2}{(x - 1)^2 x^3}$$
so $f''(x) > 0$ for all $x \geq 1$ and $f$ satisfies the hypothesis of Lemma 4.12 on the interval $[1, \infty)$. (We have $f(x) \geq 0$ on this interval because $f'(x) < 0$ on $[1, \infty)$ so $f$ is decreasing and $f(x) \geq f(\infty) = 0$.)

By Lemma 4.12, as long as $N \geq 1$ we have $6N - 3 \geq 1$ so
\begin{align*}
\sum_{p > 6N - 3} f(p) &< \frac{1}{3} \int_{6N - 3}^{\infty} f(x) dx \\
&= \left. \frac{1}{3} (1 - x) \log (1-x^{-1}) \right|_{6N-3}^{\infty} &\text{integration by parts} \\
&= \frac{1}{3}\left\{ 1 - (4 - 6N) \log \left(1-\frac{1}{6N-3}\right) \right\} &\text{by L' Hospitale rule}
\end{align*}
so the problem becomes choosing $N$ so that the right hand side is less than the desired error. 

Put $y = \frac{1}{6N-3}$ for simplicity, we turn the problem into bounding
\begin{align*}
1 - (1 - y^{-1}) \log(1 - y) &= 1 + (1 - y^{-1}) \left(\sum_{j = 1}^{\infty} \frac{y^j}{j} \right)\\
&= 1 + \sum_{j = 1}^{\infty} \frac{y^j}{j} - \sum_{j = 1}^{\infty} \frac{y^{j-1}}{j}\\
&= 1 + \sum_{j = 1}^{\infty} \frac{y^j}{j} - \sum_{j = 0}^{\infty} \frac{y^j}{j + 1}\\
&= \sum_{j = 1}^{\infty} \frac{y^j}{j (j + 1)}\\
&< \sum_{j = 1}^{\infty} \frac{y}{j (j + 1)} &\text{if } 0 < y < 1\\
&= y
\end{align*}

So if we want the error to be smaller than $\epsilon = 0.0001$, we just need to choose $N$ such that $\frac{1}{3 (6N - 3)} < \epsilon$. In that case,
$$N > \frac{1}{18\epsilon} + \frac{1}{2} = 556.05...$$
so $N = 557$, hence summing up to $6N - 3 = 3339$, will do.

%We compute the anti-derivative
%\begin{align*}
%\int f(x) dx &= x f(x) - \int x f'(x) dx\\
%&= x f(x) + \int \frac{1}{x(x-1)} dx\\
%&= x f(x) + \int \left( \frac{1}{x-1} - \frac{1}{x} \right) dx\\
%&= - x \log (1-x^{-1}) + \underbrace{\log(x-1) - \log x}_{\log(1 - x^{-1})} + C\\
%&= (1 - x) \log (1-x^{-1}) + C
%\end{align*}

%Note that $(1 - x) \log (1-x^{-1})$ is of the form $\infty \cdot 0$ when $x \rightarrow \infty$. So we evaluate the limit with L' Hospitale rule:
%\begin{align*}
%\lim_{x \rightarrow \infty} (1 - x) \log (1-x^{-1}) &= \lim_{x \rightarrow \infty} \frac{\log (1-x^{-1})}{(1-x)^{-1}}\\
%&= \lim_{x \rightarrow \infty} \frac{(1-x^{-1})^{-1} x^{-2}}{(1-x)^{-2}}\\
%&= \lim_{x \rightarrow \infty} (1-x^{-1})\\
%&= 1
%\end{align*}

%For a direct proof that $1 - (1 - y^{-1}) \log(1 - y) < y$ if $0 < y < 1$, note that
%\begin{align*}
%1 - (1 - y^{-1}) \log(1 - y) < y &\iff 1 - y < (1 - y^{-1}) \log(1 - y)\\
%&\iff y < -\log(1 - y)\\
%&\iff e^y (1 - y) < 1\\
%&\iff 1 - y < e^{-y}\\
%&\iff \frac{1 - e^{-y}}{y} < 1
%\end{align*}
%which is true by Mean Value theorem: Recognize $1 = e^0$ so the left hand side is $e^\xi$ for some $-y < \xi < 0$ whence $e^\xi < 1$.
\end{proof}

\unless\ifdefined\IsMainDocument
\end{document}
\fi

\unless\ifdefined\IsMainDocument
\documentclass[12pt]{article}
\usepackage{amsmath,amsthm,amssymb}

\begin{document}
\fi

\textbf{Problem 4.12}: Prove that
$$\theta(x) = \sum_{n=1}^{\infty} \mu(n) \psi(x^{1/n}).$$

\begin{proof}
For fix $x$, we define the auxiliary functions
$$\theta_x(t) := \theta(x^{t/x}) \qquad \text{ and } \qquad \psi_x(t) := \psi(x^{t/x}).$$

We claim that $d\psi_x = d\theta_x * dN$, that is
$$\psi_x(t) = \int_{1^-}^t d\theta_x * dN.$$
To check:
\begin{align*}
\int_{1^-}^t d\theta_x * dN &= \int_{1^-}^t \theta_x(t/s) \; dN(s)\\
&= \sum_{n \leq t} \theta_x(t/n)\\
&= \sum_{n \leq t} \theta(x^{(t/n)/x})\\
&= \sum_{n \leq t} \theta(x^{(t/x)/n})\\
&= \psi(x^{t/x})\\
&= \psi_x(t).
\end{align*}
Here, we note that in the equation
$$\psi(y) = \sum_{n=1}^{\infty} \theta(y^{1/n}),$$
applied to $y = x^{t/x}$, the upper range for $n$ can be reduced from $\infty$ to $t$ because if $n > t$ then $x^{(t/x)/n} < x^{1/x} < 2$ whence $\theta = 0$. (Also note that $2^x > x$ for all $x$ so $2 > x^{1/x}$.)

So $d\psi_x * dM = d\theta_x * dN * dM = d\theta_x$ as $dN * dM = \delta_1$. As a result:
\begin{align*}
\theta(x) = \theta_x(x) &= \int_{1^-}^{x} d\theta_x(t)\\
&= \int_{1^-}^{x} d\psi_x * dM\\
&= \sum_{n \leq x} \psi_x(x/n) \mu(n)\\
&= \sum_{n \leq x} \mu(n) \psi(x^{1/n}).
\end{align*}
The range of summation can be adjusted from $x$ to $\infty$ since if $n > x$ then $x^{1/n} < x^{1/x} < 2$ and $\psi = 0$.
\end{proof}

\unless\ifdefined\IsMainDocument
\end{document}
\fi

\unless\ifdefined\IsMainDocument
\documentclass[12pt]{article}
\usepackage{amsmath,amsthm,amssymb}
\renewcommand{\O}[1]{O\left( #1 \right)}
\begin{document}
\fi

\textbf{Problem 4.13}: Prove that
$$\liminf_{n \rightarrow \infty} \frac{\varphi(n) \log \log n}{n} = e^{-\gamma}$$
You may assume that the constant $c$ of Lemma 4.11 is known to be $-\gamma$.

\begin{proof}
Let $n_k = p_1 p_2 ... p_k$ denote the product of the first $k$ primes. By lemma 4.13,
$$\frac{\varphi(n)}{n} \geq \frac{\varphi(n_k)}{n_k}$$
for all $1 \leq n < n_{k+1}$ and since $\log \log n$ is increasing, we have
$$\min\left\{ \frac{\varphi(n) \log \log n}{n} : n_k \leq n < n_{k+1} \right\} = \frac{\varphi(n_k) \log\log n_k}{n_k}.$$

Thus, for any $\ell$, we have
$$\inf\left\{ \frac{\varphi(n) \log \log n}{n} : n \geq n_\ell \right\} = \inf\left\{\frac{\varphi(n_k) \log\log n_k}{n_k} : k \geq \ell \right\}$$
and so by definition
$$\liminf_{n \rightarrow \infty} \frac{\varphi(n) \log \log n}{n} = \liminf_{k \rightarrow \infty} \frac{\varphi(n_k) \log\log n_k}{n_k}.$$

We are going to show that not only the second $\liminf$ above exists but the limit of the sequence also exists\footnote{Recall that if lim exists then $\lim = \limsup = \liminf$.} and
$$\lim_{k \rightarrow \infty} \frac{\varphi(n_k) \log\log n_k}{n_k} = e^{-\gamma}$$
which solves the problem.

By Lemma 4.11 plus the assumption on the constant, we have
$$\frac{\varphi(n_k)}{n_k} = \prod_{p \leq p_k} (1 - p^{-1}) \sim e^{-\gamma} / \log p_k$$
which yields
$$\lim_{k \rightarrow \infty} \frac{\varphi(n_k) \log p_k}{n_k} = e^{-\gamma}$$
To swap $\log \log n_k$ for $\log p_k$, we note that
\begin{align*}
\log \log n_k - \log p_k &= \log \theta(p_k) - \log p_k &\text{ as } \log n_k = \theta(p_k)\\
&= \log \frac{\theta(p_k)}{p_k}\\
&= \log \frac{\psi(p_k) + \O{ \sqrt{p_k} } }{p_k} &\text{by Lemma 4.14}\\
&= \O{1} &\text{by Lemma 4.5}
\end{align*}
is bounded. Thus, the difference
$$\frac{\varphi(n_k) \log \log n_k}{n_k} - \frac{\varphi(n_k) \log p_k}{n_k} = \O{ \frac{\varphi(n_k)}{n_k} } = \O{ \frac{e^{-\gamma}}{\log p_k} }$$
goes to 0 as $k \rightarrow \infty$ because $p_k \rightarrow \infty$.
\end{proof}

\unless\ifdefined\IsMainDocument
\end{document}
\fi


\chapter{An Elementary Proof of the PNT}

\unless\ifdefined\IsMainDocument
\documentclass[12pt]{article}
\usepackage{amsmath,amsthm,amssymb}
\newcommand{\Abs}[1]{\left| #1 \right|}
\begin{document}
\fi

We generalize Lemma 5.7 with a more flexible condition on $B_v(x)$: Let everything be as in Lemma 5.7 but with $B_v(x) = O(x^m)$, $A_1(x) = o(x)$,
$$\int_{1^-}^{\infty} t^{-1} \; dA_v(x) < \infty.$$
and suppose that there exists a function $C(x)$ such that
$$x^{m-1} \int_{C(x)}^x A_1(u) u^{-m-1} du = o(1).$$
Then we have the same conclusion.

The proof should proceed similarly and modified appropriately. Here, instead of picking a fixed $C$ in the hyperbola method, we let $C(x)$ varies. The conditions are selected to bound the three pieces.

\textbf{Notes}:
\begin{itemize}
\item Originally, I use the condition
$$\lim_{C \rightarrow \infty} \lim_{x \rightarrow \infty} x^{m-1} \int_{C}^x A_1(u) u^{-m-1} du = 0$$
but this usually is not true because
$$\int_{C}^x A_1(u) u^{-m-1} du = F(x) - F(C)$$
where $F$ is the anti-derivative and so the inner limit $\lim_{x \rightarrow \infty} x^{m-1} F(x) - x^{m-1} F(C)$ is usually $\infty$ for fixed $C$.
\item Unfortunately, while it fixes the core issue mentioned above, this reformulation is not helpful in general either because we now have to deal with another total variation function $A_v(x)$ so this is only helpful in the situation where say $A$ is monotone like $A(x) = \log x$. The total variation function is really hard to control because even if $A$ is bounded, $A_v(x)$ could still grow fast in $x$; as in the example $A(x) = x - [x]$ in Chapter 3.
\item So it seems that Lemma 5.7 is just right.
\item An observation is that the proof might have used a stricter inequality: We have
$$\Abs{ \int_a^b g \; dF } \leq \int_a^b |g \; dF|$$
by triangle inequality where the right hand side is defined to be
$$\lim \sum_i |g(t_i)| \cdot |F(x_{i+1}) - F(x_i)|.$$
(The limit is over all partition $(x_0, x_1, ..., x_k)$ of $[a, b]$ under refinement, just like in the definition of $\int_a^b g \; dF$.)

If we further use the inequality
$$|F(x_{i+1}) - F(x_i)| \leq V_F([x_i, x_{i+1}]) = F_v(x_{i+1}) - F_v(x_i)$$
to bound
$$\sum_i |g(t_i)| \cdot |F(x_{i+1}) - F(x_i)| \leq \sum_i |g(t_i)| \cdot (F_v(x_{i+1}) - F_v(x_i))$$
whose limit over the partitions can be recognized as
$$\int_a^b |g| \; d F_v$$
and thus
$$\Abs{ \int_a^b g \; dF } \leq \int_a^b |g \; dF| \leq \int_a^b |g| \; dF_v$$
which is what the book usually employed.

I hope we can get a better generalization if we take the better bound in the middle. However, there is another problem: We have no tool to manipulate the ``integrator'' $|g \; dF| = |g| \cdot |dF|$ in general practice, except for the ``inequality'' $|dF| \leq dF_v$ above.
\end{itemize}

A small fact that can be seen from Theorem 5.9: We have
\begin{align*}
\int_{1^-}^x \frac{x}{t} dM(t) &= \left. \frac{x M(t)}{t} \right|_{1^-}^x + x \int_{1^-}^x M(t) t^{-2} dt\\
&= M(x) + x \underbrace{\int_{1^-}^x M(t) t^{-2} dt}_{o(1)}
\end{align*}
To see that the integral is $o(1)$, dividing both sides by $x$ then recall that $\int_{1^-}^x t^{-1} dM(t) = o(1)$ by (5.13) and $M(x) = o(x)$ by (5.12) so $\frac{M(x)}{x} = o(1)$ as well. Hence, $\int_{1^-}^x M(t) t^{-2} dt = \frac{B(x) - M(x)}{x} = o(1)$. In other words,
$$\int_{1^-}^\infty M(t) t^{-2} dt = 0.$$

\unless\ifdefined\IsMainDocument
\end{document}
\fi


\section{Problems}
\unless\ifdefined\IsMainDocument
\documentclass[12pt]{article}
\usepackage{amsmath,amsthm,amssymb}

\begin{document}
\fi

\textbf{Problem 5.1}: Show that if $0 < \epsilon < 1$ then
$$0 \leq \psi(x + \epsilon x) - \psi(x) < 2 \epsilon x + O(x / \log x)$$
and thus for any fixed $\epsilon > 0$,
$$\limsup_{x \rightarrow \infty} \{\psi(x + \epsilon x) - \psi(x)\}/(\epsilon x) \leq 2.$$

\begin{proof}
We use Selberg's formula
\begin{align*}
\int_1^x Ld\psi + \psi * \psi &= 2 x \log x + O(x)\\
\int_1^{x + \epsilon x} Ld\psi + \psi * \psi &= 2 (x + \epsilon x) \log (x + \epsilon x) + O(x + \epsilon x)
\end{align*}
to obtain
\begin{align*}
\int_x^{x + \epsilon x} Ld\psi + \psi * \psi &= 2 (x + \epsilon x) \log (x + \epsilon x) - 2 x \log x + O(x)\\
&= 2 x \log (1 + \epsilon) + 2 \epsilon x \log (x + \epsilon x) + O(x)
\end{align*}

Observe that
\begin{align*}
\int_x^{x + \epsilon x} Ld\psi + \psi * \psi &\geq \int_x^{x + \epsilon x} Ld\psi &\text{ as } \Lambda * \Lambda \geq 0 \\
&\geq \int_x^{x + \epsilon x} \log x \; d\psi\\
&= \log x \; \{ \psi(x + \epsilon x) - \psi(x) \}
\end{align*}
so we have
\begin{align*}
\psi(x + \epsilon x) - \psi(x) &\leq \frac{2 x}{\log x} \log (1 + \epsilon) + 2 \epsilon x \frac{\log (x + \epsilon x)}{\log x} + O(x / \log x)\\
&< 2 \epsilon x \frac{\log x + \log (1 + \epsilon)}{\log x} + O(x / \log x)\\
&< 2 \epsilon x + O(x / \log x).
\end{align*}
\end{proof}

\unless\ifdefined\IsMainDocument
\end{document}
\fi

\unless\ifdefined\IsMainDocument
\documentclass[12pt]{article}
\usepackage{amsmath,amsthm,amssymb}

\begin{document}
\fi

\textbf{Problem 5.2}: Prove that $a + A = 2$.

\begin{proof}
We have the following rearrangement from equation (5.3) of Selberg's formula
$$\frac{1}{x \log x} \int_1^x \psi(x/t) d\psi = 2 - \frac{\psi(x)}{x} + O(1/\log x).$$
Taking $\limsup$ and $\liminf$ (it will be $x \rightarrow \infty$ throughout this problem so we will drop the notation) on both sides yield\footnote{If $F$ is continuous and monotonically decreasing then $\limsup F(\alpha(x)) = F(\liminf \alpha(x))$. Here, apply to $F(x) = 2 - x$. Also recall another property that $\limsup \alpha(x) + \beta(x) = \lim \alpha(x) + \limsup \beta(x)$ if $\lim \alpha(x)$ exists.}
\begin{align*}
\limsup \frac{1}{x \log x} \int_1^x \psi(x/t) d\psi &= 2 - a\\
\liminf \frac{1}{x \log x} \int_1^x \psi(x/t) d\psi &= 2 - A
\end{align*}
and so we just have to check that the left hand side of the first (second) equation is $A$ ($a$, respectively). But that is hard so we prove instead that the left hand side of the first equation is $\leq A$ while the left hand side of the second is $\geq a$. Together they yield $2 - a \leq A$ and $2 - A \geq a$ so $A + a = 2$.

Let $\epsilon > 0$ be arbitrary and let $S > 1$ be such that
$$\frac{\psi(s)}{s} < A + \epsilon$$
for all $s > S$ by definition of $\limsup$. We split the integral based on whether $x/t > S$:
\begin{align*}
\int_1^x \psi(x/t) d\psi &= \int_1^{x/S} \psi(x/t) d\psi + \int_{x/S}^x \psi(x/t) d\psi\\
&\leq \int_1^{x/S} (A + \epsilon) (x/t) d\psi + \int_{x/S}^x \psi(S) d\psi &\text{as }\psi \text{ is increasing}\\
&= (A + \epsilon) x \int_1^{x/S} t^{-1} d\psi + \psi(S) \{\psi(x) - \psi(x/S)\}\\
&= (A + \epsilon) x (\log x - \log S + O(1)) + O(x) &\text{by Lemma 4.8}
\end{align*}
so dividing $x \log x > 0$ and taking the $\limsup$, we find
$$\limsup \frac{1}{x \log x} \int_1^x \psi(x/t) d\psi \leq A + \epsilon.$$
As $\epsilon$ can be made arbitrarily small, we have $\limsup (...) \leq A$. By an analogous argument, we find
$$\liminf \frac{1}{x \log x} \int_1^x \psi(x/t) d\psi \geq a.$$
\end{proof}

\unless\ifdefined\IsMainDocument
\end{document}
\fi

\unless\ifdefined\IsMainDocument
\documentclass[12pt]{article}
\usepackage{amsmath,amsthm,amssymb}
\newcommand{\Abs}[1]{\left| #1 \right|}
\renewcommand{\O}[1]{O\left( #1 \right)}
\begin{document}
\fi

\textbf{Problem 5.3}: Let $f(n) \geq 0$ for $n = 1, 2, ...$. For $j = 0, 1, 2, ...$ define
$$F_j(x) = \sum_{n \leq x} f(n) (\log x/n)^j / j!.$$
For each $j \geq 1$, prove that $F_j(x) \sim x \iff F_0(x) \sim x$.

\begin{proof}
First, one has
$$F_0(x) = \sum_{n \leq x} f(n)$$
is the summatory function of $f$ and
$$F_j(x) =  \int_{1^-}^x \frac{(\log x/t)^j}{j!} \; dF_0(t)
= \int_{1^-}^x \frac{1}{j!} d(\log^j t) * dF_0(t)$$
for $j \geq 1$. Observe that
$$\frac{1}{j!} d(\log^j t) = \frac{1}{j!} j \log^{j-1} t \frac{dt}{t} = \frac{T^{-1} L^{j-1} dt}{(j-1)!}
= (T^{-1} dt)^{*j}$$
by problem 3.17. So
$$F_j(x) = \int_{1^-}^x (T^{-1} dt)^{*j} * dF_0(t) = \int_{1^-}^x (T^{-1} dt) * dF_{j-1}(t) = \int_{1^-}^x F_{j-1}(t) t^{-1} dt$$
by problem 3.15 (2).

We are going to show a general fact that if $G$ is nonnegative increasing and
$$H(x) = \int_{1^-}^x G(t) t^{-1} dt$$
then $G(x) \sim x \iff H(x) \sim x$.

The functions $F_j$ are clearly increasing and nonnegative by our assumption on $f \geq 0$. So applying this recursively will yield $F_j(x) \sim x \iff F_{j-1}(x) \sim x \iff ... \iff F_0(x) \sim x$.

Recall that $R(x) \sim x \iff R(x) - x + c = o(x)$ for any $R$. Therefore, let us turn the equation for $H$ into
$$H(x) - x + 1 = \int_{1^-}^x \frac{G(t) - t}{t} dt.$$

\begin{itemize}
\item The ``easier'' implication $G(x) \sim x \Rightarrow H(x) \sim x$: Assuming $G(x) \sim x$ then
$$\frac{G(t) - t}{t} = o(1)$$
so
$$\int_{1^-}^x \frac{G(t) - t}{t} dt = o(x).$$
So $H(x) - x + 1 = o(x)$ and $H(x) \sim x$ follows.

(In more details: Let $\epsilon > 0$ be arbitrary and $T$ be such that $\Abs{ \frac{G(t) - t}{t} } < \epsilon$ for all $t > T$. Then for $x > T$:
\begin{align*}
\Abs{ \int_{1^-}^x \frac{G(t) - t}{t} dt }
&\leq \int_{1^-}^x \Abs{ \frac{G(t) - t}{t} } dt\\
&= \left\{ \int_{1^-}^T  + \int_{T}^x \right\} \Abs{ \frac{G(t) - t}{t} } dt\\
&\leq \O{1} + \epsilon x
\end{align*}
so dividing by $x$ and as $\epsilon$ is arbitrary, we get $H(x) - x + 1 = o(x)$.)

\item The ``harder'' implication\footnote{This is sort of like tauberian method where we obtain information about $G$ from its average $H$.} $H(x) \sim x \Rightarrow G(x) \sim x$: We apply the method to prove Lemma 5.2 in the book.

Let $\epsilon \in (0, 1/2)$ be arbitrary and consider
$$I(x) = \{ H(x + \epsilon x) - ( x + \epsilon x) + 1 \} - \{ H(x) - x + 1\} = \int_x^{x + \epsilon x} \frac{G(t) - t}{t} dt$$
We have $I(x) = o(x)$ by assumption $H(x) \sim x$. The simple inequality from $G$ nonnegative increasing
$$\frac{G(t)}{t} \geq \frac{G(x)}{x + \epsilon x} \qquad \text{ for } x \leq t \leq x + \epsilon x$$
yields
\begin{align*}
I(x) &= \int_x^{x + \epsilon x} \frac{G(t) - t}{t} dt\\
&= \int_x^{x + \epsilon x} \frac{G(t)}{t} dt - \epsilon x\\
&\geq \int_x^{x + \epsilon x} \frac{G(x)}{x + \epsilon x} dt - \epsilon x\\
&= \frac{G(x)}{x + \epsilon x} \epsilon x - \epsilon x\\
&= \frac{\epsilon}{1 + \epsilon} G(x) - \epsilon x
\end{align*}
so we have
$$G(x) \leq \frac{o(x) + \epsilon x}{\epsilon / (1 + \epsilon)} = O(x).$$
Note that $\epsilon$ is fixed here. So $G(x) = O(x)$ since $G(x) \geq 0$.
Since $\frac{1}{1 + \epsilon} > 1 - \epsilon$ and $G(x) \geq 0$, we can continue 
\begin{align*}
I(x) &\geq (1 - \epsilon) G(x) \epsilon - \epsilon x &\\
&= \epsilon (G(x) - x) - \epsilon^2 G(x)
\end{align*}
so
$$G(x) - x \leq \frac{I(x)}{\epsilon} + \epsilon G(x) \leq o(x)/\epsilon + \epsilon K x$$
where $K$ is the bounding constant in $G(x) = O(x)$.

Analogously, using $\frac{1}{1-\epsilon} < 1 + 2\epsilon$, we have
\begin{align*}
\int_{x - \epsilon x}^{x} \frac{G(t) - t}{t} dt 
&\leq \int_{x - \epsilon x}^{x} \frac{G(x)}{x - \epsilon x} dt - \epsilon x\\
&\leq (1 + 2\epsilon) G(x) \epsilon - \epsilon x\\
&= \epsilon (G(x) - x) + 2 \epsilon^2 G(x)
\end{align*}
so
$$G(x) - x \geq o(x) / \epsilon - 2 K \epsilon x.$$

Combining the two bounds, we obtain
$$\frac{\Abs{ G(x) - x }}{x} \leq o(1) / \epsilon + 2 K \epsilon$$
so taking limsup (fixing $\epsilon$)
$$\limsup \frac{\Abs{ G(x) - x }}{x} \leq 2 K \epsilon.$$
As $\epsilon$ is arbitrary, the $\limsup = 0$ and so $G(x) \sim x$.
\end{itemize}
\end{proof}

\unless\ifdefined\IsMainDocument
\end{document}
\fi

\unless\ifdefined\IsMainDocument
\documentclass[12pt]{article}
\usepackage{amsmath,amsthm,amssymb}

\begin{document}
\fi

\textbf{Problem 5.4}: Show that equality holds in the last lemma.

\begin{proof}
We have
$$\frac{|R(x)|}{x} \leq \alpha + o(1)$$
by definition of $\alpha = \limsup |R(x)|/x$ so
$$\frac{1}{\log x} \int_1^x \frac{|R(u)|}{u^2} du \leq \frac{1}{\log x} \int_1^x \frac{\alpha + o(1)}{u} du = \alpha + o(1)$$
Taking $\limsup$ yields
$$\limsup \frac{1}{\log x} \int_1^x \frac{|R(u)|}{u^2} du \leq \alpha$$
which is the reverse inequality of the one in the lemma. So equality holds in the lemma.

(The mentality behind: Let $\epsilon > 0$ be arbitrary. Then there exists $N$ such that
$$\frac{|R(x)|}{x} < \alpha + \epsilon$$
for all $x > N$ by definition of $\limsup$. Then for $x > N$, we split the integral
\begin{align*}
\frac{1}{\log x} \int_1^x \frac{|R(u)|}{u^2} du &= \frac{1}{\log x} \left\{ \int_1^N + \int_N^x \right\} \frac{|R(u)|}{u^2} du\\
&\leq \frac{K}{\log x} + \frac{1}{\log x} \int_N^x \frac{\alpha + \epsilon}{u} du &\text{ where } K = \int_1^N |R(u)| u^{-2} du\\
&= \frac{K}{\log x} + \frac{(\alpha + \epsilon)(\log x - \log N)}{\log x}
\end{align*}
so taking $\limsup$ on both sides yields:
$$\limsup \frac{1}{\log x} \int_1^x \frac{|R(u)|}{u^2} du \leq \limsup \left( \frac{K}{\log x} + \frac{(\alpha + \epsilon)(\log x - \log N)}{\log x} \right) = \alpha + \epsilon.$$
This works for every $\epsilon$, so the left hand side is $\leq \alpha$.)
\end{proof}

\unless\ifdefined\IsMainDocument
\end{document}
\fi

\unless\ifdefined\IsMainDocument
\documentclass[12pt]{article}
\usepackage{amsmath,amsthm,amssymb}

\begin{document}
\fi

\textbf{Problem 5.5}: Give direct proofs of the following implications:
$$(5.12) \longrightarrow (5.10), \qquad (5.10) \longrightarrow (5.14), \qquad (5.13) \longrightarrow (5.14)$$

\begin{proof}
\begin{itemize}
\item $(5.12) \longrightarrow (5.10)$ i.e. $M(x) = o(x) \Rightarrow \psi(x) \sim x$: We use the formula $d\psi = LdN * dM$ from Chebyshev's identity. Subtracting $dN$ and then integrating yields
$$\psi(x) - N(x) = \int_{1^-}^x (LdN - dN * dN) * dM.$$
Now let
\begin{align*}
H(x) &:= \int_{1^-}^x LdN - dN * dN\\
&= \int_{1^-}^x \log t \; dN - \int_{1^-}^x dN * dN\\
&= \sum_{n \leq x} \log n - (x \log x + 2\gamma x - x + O(\sqrt x)) &\text{by Corollary 3.32}\\
&= \int_1^x \log t \; dt + O(\log x) - (...) &\text{by problem 3.13}\\
&= x \log x - x + 1 + O(\log x) - (x \log x + 2\gamma x - x + O(\sqrt x))\\
&= 2 \gamma x + O(\sqrt x)
\end{align*}
So assuming $M(x) = o(x)$, a simple application of Lemma 5.7 yields
$$\int_{1^-}^x d(H(t) - 2\gamma t) * dM = o(x)$$
and at the same time, we also have
$$\int_{1^-}^x d(2\gamma t) * dM = 2 \gamma \int_{1^-}^x \frac{x}{t} dM(t) = 2 \gamma x \int_{1^-}^x t^{-1} dM(t) = o(x)$$
by an application of Corollary 5.8. Thus,
$$\psi(x) = N(x) + \int_{1^-}^x dH * dM = x + o(x).$$

\item $(5.10) \longrightarrow (5.14)$ i.e. $\psi(x) \sim x \Rightarrow \int_1^x t^{-1} d\psi = \log x - \gamma + o(1)$: We express
$$x \int_1^x t^{-1} d\psi(t) = \int_1^x \{ d\psi * dN \;+\; (\delta_1 + dt - dN) * dt \;+\; (\delta_1 + dt - dN) * (d\psi - dt) \}$$
and assess the three integrals:
\begin{itemize}
\item By Chebyshev's identity $d\psi * dN = L dN$, we have
$$\int_1^x d\psi * dN = \int_1^x L dN = x \log x - x + \underbrace{1 + O(\log x)}_{o(x)}.$$

\item It is clear that
\begin{align*}
\int_1^x (\delta_1 + dt - dN) * dt &= \int_1^x \left( \frac{x}{t} - N\left(\frac{x}{t}\right) \right) dt\\
&= x \int_1^x t^{-1} dt - \int_1^x dN * dt\\
&= x \log x - \int_1^x (x/t - 1) dN\\
&= x \log x + N(x) - x \int_1^x t^{-1} dN\\
&= x \log x + N(x) - x (\log x + \gamma + O(x^{-1}))\\
&= N(x) - x \gamma + \underbrace{O(1)}_{= o(x)}
\end{align*}

\item Finally, the third piece
$$\int_1^x (\delta_1 + dt - dN) * (d\psi - dt) = o(x)$$
by Lemma 5.7 with $B(x) = \psi(x) - x = o(x)$ by our assumption (5.10), the total variation $B_v(x)$ is at most $\psi_v(x) + x_v = O(x)$ since both $\psi$ and $x \mapsto x$ are increasing and $|x - N(x)| \leq 1$ as exploited in Corollary 5.8.
\end{itemize}
So combining the three, we get
$$x \int_1^x t^{-1} d\psi(t) = x \log x - \gamma x + o(x)$$
which clearly implies (5.14).

\item $(5.13) \longrightarrow (5.14)$ i.e. $\sum_{n \leq x} \mu(n)/n = o(1) \Rightarrow \int_1^x t^{-1} d\psi = \log x - \gamma + o(1)$: We follow the proof of $(5.12) \longrightarrow (5.14)$ in Theorem 5.9 to write
\begin{align*}
x \int_1^x t^{-1} d\psi &= x \log x - \gamma x + \underbrace{\int_1^x (\delta_1 + dt) * (L dN - dt * dN + \gamma dN) * dM}_{I(x)}
\end{align*}
Since our assumption (5.13) is the same as
$$x \int_{1^-}^x t^{-1} dM = \int_{1^-}^x (\delta_1 + dt) * dM = o(x),$$
it suggests us to treat $I(x) = \int_{1^-}^x dA * dB$ where 
\begin{align*}
A(x) &= \int_{1^-}^x L dN - dt * dN + \gamma dN, \text{ and}\\
B(x) &= \int_{1^-}^x (\delta_1 + dt) * dM.
\end{align*}
Unfortunately, this idea does not work out because while it is easy to bound $|A| \leq O(\log x)$, it is \emph{not true} that $B_v(x) = O(x)$: We have
\begin{align*}
B_v(x) &= \int_{1^-}^x |(\delta_1 + dt) * dM|\\
&= \int_{1^-}^x \frac{x}{t} |dM|\\
&= \int_{1^-}^x \frac{x}{t} dQ(t)\\
&= Q(x) + x \int_{1^-}^x Q(t) t^{-2} dt\\
&= Q(x) + x \int_{1^-}^x \left( \frac{6t}{\pi^2} + O(\sqrt{t}) \right) t^{-2} dt\\
&= Q(x) + \frac{6x}{\pi^2} \int_{1^-}^x t^{-1} dt + O(x)\\
&= \frac{6}{\pi^2} x \log x + O(x).
\end{align*}
Note that it nevertheless follows from Lemma 3.16 that $B_v(x) = O(x^2)$.

In any case, it is a one-line proof to get (5.12) from (5.13).
\end{itemize}
\end{proof}

\unless\ifdefined\IsMainDocument
\end{document}
\fi

\unless\ifdefined\IsMainDocument
\documentclass[12pt]{article}
\usepackage{amsmath,amsthm,amssymb}

\begin{document}
\fi

\textbf{Problem 5.6}: Let $\lambda$ denote Liouville's function. Show that
$$\sum_{n \leq x} \lambda(n) = o(x) \iff M(x) = o(x)$$

\begin{proof}
\begin{itemize}
\item The reverse implication: Let $S$ be the set of squares then we recall that $1_S = \mu * |\mu|$ and $|\mu| * \lambda = e$ so $\lambda = \mu * 1_S$. We apply Lemma 5.7 with $B(x) = M(x)$ and
$$A(x) = \sum_{n \leq x} 1_S(n) = \sum_{m^2 \leq x} 1 = N(\sqrt{x}) \leq \sqrt{x}$$
Clearly for $A_1(x) = \sqrt{x}$ is increasing and we have
$$\int_1^\infty A_1(u) u^{-2} du = \int_1^\infty u^{-3/2} du = -2u^{-1/2}|_1^\infty = 2 < \infty.$$
So if $M(x) = o(x)$ then we have $\sum_{n \leq x} \lambda(n) = \int_{1^-}^{\infty} dA * dB = o(x).$

\item The forward implication: From $\lambda = \mu * 1_S$, we have $\mu = \lambda * 1_S^{*-1}$ where explicitly
$$1_S^{*-1}(n) = \begin{cases} \mu(\sqrt{n}) &\text{if } n \text{ is a square}, \\ 0 &\text{otherwise}.\end{cases}$$
We apply Lemma 5.7 again, this time with $B(x) = \sum_{n \leq x} \lambda(n)$ and
$$A(x) = \sum_{n \leq x} 1_S^{*-1}(n) = \sum_{m^2 \leq x} \mu(m) \leq N(\sqrt{x}) \leq \sqrt{x}.$$
The same argument applies: if $B(x) = o(x)$ then $M(x) = \int_{1^-}^x dA * dB = o(x)$.
\end{itemize}
\end{proof}

\unless\ifdefined\IsMainDocument
\end{document}
\fi

\unless\ifdefined\IsMainDocument
\documentclass[12pt]{article}
\usepackage{amsmath,amsthm,amssymb}

\begin{document}
\fi

\textbf{Problem 5.7}: Show that
$$\#\{n \leq x : \Omega(n) \text{ is even}\} \sim \frac{x}{2}.$$

\begin{proof}
Let
$$E(x) := \#\{n \leq x : \Omega(n) \text{ is even}\}$$
and
$$F(x) := \#\{n \leq x : \Omega(n) \text{ is odd}\}.$$

Clearly $E(x) + F(x) = N(x)$ and
$$E(x) - F(x) = \sum_{n \leq x} (-1)^{\Omega(n)} = \sum_{n \leq x} \lambda(n).$$
Thank to the previous problem and Theorem 5.9, we know that $E(x) - F(x) = o(x)$ so
$$E(x) = \frac{N(x) + o(x)}{2} \sim \frac{x}{2}.$$
\end{proof}

\unless\ifdefined\IsMainDocument
\end{document}
\fi

\input{Chapter5/P5_8.tex}
\unless\ifdefined\IsMainDocument
\documentclass[12pt]{article}
\usepackage{amsmath,amsthm,amssymb}

\begin{document}
\fi

\textbf{Problem 5.9}: Show that
$$Q(x) := \sum_{n \leq x} |\mu(n)| = 6 \pi^{-2} x + o(\sqrt{x}).$$

\begin{proof}
Write $|\mu| = 1 * f$ as in Lemma 3.4 (actually $f = 1_S^{*-1}$ as seen in problem 5.6) and note that the summatory function of $f$ is precisely $M(\sqrt{x})$. Therefore
\begin{align*}
Q(x) &= \int_{1^-}^x dN(t) * dM(\sqrt{t})\\
&= \int_{1^-}^x N\left(\frac x t\right) \; dM(\sqrt{t})\\
&= \int_{1^-}^{\sqrt x} N\left(\frac{x}{u^2}\right) \; dM(u) &\text{by change of variable } u = \sqrt{t}\\
&= \int_{1^-}^{\sqrt x} \frac{x}{u^2} \; dM(u) + \int_{1^-}^{\sqrt x} \left\{ N\left(\frac{x}{u^2}\right) - \frac{x}{u^2} \right\} \; dM(u) &\text{the usual trick}
\end{align*}

We can write $N\left(\frac{x}{u^2}\right) - \frac{x}{u^2} = N\left(\left\{\frac{\sqrt x}{u}\right\}^2\right) - \left\{\frac{\sqrt x}{u}\right\}^2$ to recognize the second integral
$$\int_{1^-}^{\sqrt x} \left\{ N\left(\frac{x}{u^2}\right) - \frac{x}{u^2} \right\} \; dM(u) = \int_{1^-}^{\sqrt x} dG * dM$$
where $G(t) = N(t^2) - t^2$. The same trick in Corollary 5.9 applies here: $|G(t)| \leq 1$ and $\int_{1^-}^{\infty} 1 \cdot u^{-2} du < \infty$ so $\int_{1^-}^{x} dG * dM = o(x)$ from the fact that $M(x) = o(x)$ by Theorem 5.9. This implies that the above integral is $o(\sqrt x)$. The first integral has the familiar treatment:
\begin{align*}
\int_{1^-}^{\sqrt x} \frac{x}{u^2} \; dM(u) &= x \left\{ \int_{1^-}^{\infty} - \int_{\sqrt x}^{\infty} \right\} \frac{1}{u^2} \; dM(u)\\
&= 6 \pi^{-2} x - x \int_{\sqrt x}^{\infty} \frac{1}{u^2} \; dM(u)\\
&= 6 \pi^{-2} x - x \left\{ \left. \frac{M(u)}{u^2} \right|_{\sqrt x}^{\infty} - \int_{\sqrt x}^{\infty} M(u) d(u^{-2}) \right\}\\
&= 6 \pi^{-2} x - x \left\{ -\frac{M(\sqrt x)}{x} + 2 \int_{\sqrt x}^{\infty} M(u) u^{-3} du \right\} &\text{ as } M(x) = o(x)\\
&= 6 \pi^{-2} x + M(\sqrt{x}) - 2 x \int_{\sqrt x}^{\infty} o(u^{-2}) du &\text{ as } M(x) = o(x)\\
&= 6 \pi^{-2} x + o(\sqrt{x}) &\text{ as } M(x) = o(x)
\end{align*}
and
$$\int_{\sqrt x}^{\infty} o(u^{-2}) du = o\left( \left. -\frac{1}{u} \right|_{\sqrt x}^{\infty} \right) = o(x^{-1/2}).$$
\end{proof}

\unless\ifdefined\IsMainDocument
\end{document}
\fi

\unless\ifdefined\IsMainDocument
\documentclass[12pt]{article}
\usepackage{amsmath,amsthm,amssymb}
\newcommand{\Abs}[1]{\left| #1 \right|}
\begin{document}
\fi

\textbf{Problem 5.10}: If $0 < c < 1$, let $f_c(x)$ denote the number of positive integers $n \leq x$ such that $[n^c]$ is prime. Use the PNT to prove that $f_c(x) \sim x/(c \log x)$.

\begin{proof}
Observe that the prime $p = [n^c] \leq n^c \leq x^c$ and that 
$$p = [n^c] \iff p \leq n^c < p + 1 \iff p^{1/c} \leq n < (p + 1)^{1/c}.$$

Note that if $x < n < (p + 1)^{1/c}$ then $p \leq x^c < n^c < p + 1 \leq x^c + 1$ so $x < n < (x^c + 1)^{1/c} = x + O(x^{1-c})$ by the Mean Value Theorem (see later). So there are at most $O(x^{1-c})$ integers in the interval $(x, (x^c + 1)^{1/c})$.

Therefore we have\footnote{The sets $\{n \leq x : p^{1/c} \leq n < (p + 1)^{1/c}\}$ are mutually disjoint. If $p < q$ are primes then $q \geq p + 1$. So if $q^{1/c} \leq n < (q + 1)^{1/c}$ then $n \geq (p + 1)^{1/c}$ so $n$ cannot be in the corresponding set for $p$. The converse is similarly seen. This also means the sets $\{n : x < n < (p+1)^{1/c}\}$ for $p < x^c$ are mutually disjoint and their union is a subset of $\{n : x < n < (x^c + 1)^{1/c}\}$.}
\begin{align*}
f_c(x) &= \sum_{p \leq x^c} \#\{n \leq x : p^{1/c} \leq n < (p + 1)^{1/c}\}\\
&= \sum_{p \leq x^c} \#\{ n : p^{1/c} \leq n < (p + 1)^{1/c} \} - \underbrace{ \sum_{p \leq x^c} \#\{ n : x < n < (p+1)^{1/c} \} }_{\leq \#\{n : x^c < n < (x^c + 1)^{1/c} \}} \\
&= \sum_{p \leq x^c} \sum_{p^{1/c} \leq n < (p + 1)^{1/c}} 1 + O(x^{1-c})\\
&= \sum_{p \leq x^c} \left( (p + 1)^{1/c} - p^{1/c} + O(1) \right) + O(x^{1-c})\\
&= \sum_{p \leq x^c} \left( (p + 1)^{1/c} - p^{1/c} \right) + O(\pi(x^c)) + O(x^{1-c})
\end{align*}
by the simple estimate
$$\{n : a \leq n < b\} = N(b) - N(a - 1) = b - a + O(1).$$
Note that by the PNT, $O(\pi(x^c)) = O(x^c / (c \log x)) = o(x / (c \log x))$ under the assumption that $0 < c < 1$.

We could recognize
$$\sum_{p \leq x^c} \left( (p + 1)^{1/c} - p^{1/c} \right) = \int_{2^-}^{x^c} ((t + 1)^{1/c} - t^{1/c}) \; d \pi.$$
(There is no prime $< 2$, right?) This is a nice integral as $\pi \geq 0$ is increasing so $\pi_v = \pi$, which makes absolute value bound easier.

Applying the Mean Value Theorem to the function $f(t) = t^{1/c}$, we have
$$(t + 1)^{1/c} - t^{1/c} = f(t+1) - f(t) = f'(\xi)$$
for some $\xi \in (t, t+1)$. For $0 < c < 1$, the function $f'(t) = \frac{1}{c} t^{1/c - 1}$ is increasing on $(0, \infty)$ because its derivative $f''(t) = \frac{1}{c} \left( \frac{1}{c} - 1\right) t^{1/c - 2} > 0$. Therefore, we can bound
$$f'(t) \leq f'(\xi) \leq f'(t + 1)$$
which shows that
$$(t + 1)^{1/c} - t^{1/c} \sim f'(t).$$
Note that this is how I know $(x^c + 1)^{1/c} = \underbrace{x}_{= (x^c)^{1/c}} + O(x^{1-c})$ for $f'(x^c) = \frac{1}{c} (x^c)^{1/c - 1} = \frac{1}{c} x^{1 - c}$.

This means
$$\int_{2^-}^{x^c} ((t + 1)^{1/c} - t^{1/c}) \; d \pi \sim \int_{2^-}^{x^c} f'(t) \; d \pi$$
if we write $(t + 1)^{1/c} - t^{1/c} = f'(t) + o(f'(t))$ and bound
$$\Abs{ \int_{2^-}^{x^c} ((t + 1)^{1/c} - t^{1/c} - f'(t)) \; d \pi } \leq \int_{2^-}^{x^c} \Abs{ (t + 1)^{1/c} - t^{1/c} - f'(t) } \; d \pi$$
thank to $\pi_v = \pi$. Then use the usual $\epsilon$ argument to swap the little $o$ outside.

The problem thus becomes estimating
\begin{align*}
\int_{2^-}^{x^c} f'(t) \; d \pi &= \int_{2^-}^{x^c} \frac{1}{c} t^{1/c - 1} \; d \pi(t)\\
&= \frac{1}{c} \int_{2^-}^{x} u^{1-c} \; d \pi(u^c) &\text{change of var } u = t^{1/c}\\
&= \frac{1}{c} \left( x^{1-c} \pi(x^c) - \int_{2^-}^{x} \pi(u^c)  d(u^{1-c}) \right)\\
&\sim \frac{1}{c} \left( x^{1-c} \frac{x^c}{c \log x} - \int_{2^-}^{x} \frac{u^c}{c \log u} \; (1-c) u^{-c} \; du \right) &\text{by PNT}\\
&= \frac{1}{c} \left( \frac{x}{c \log x} + \frac{c-1}{c} \int_{2}^{x} \frac{1}{\log u} du \right)\\
&\sim \frac{x}{c \log x}
\end{align*}
from the well-known knowledge that the function
$$\operatorname{Li}(x) = \int_2^x \frac{1}{\log t} dt \sim \frac{x}{\log x}$$
which can be seen from integrating by parts
$$\int_2^x \frac{1}{\log t} dt = \frac{x}{\log x} - \frac{2}{\log 2} + \int_2^x \frac{1}{\log^2 t} dt$$
and then observing that $\log^{-2} t = o(\log^{-1} t)$.
\end{proof}

\unless\ifdefined\IsMainDocument
\end{document}
\fi


\chapter{Dirichlet Series and Mellin Transforms}

\newcommand{\Fhat}{\widehat{F}}
\newcommand{\cconj}[1]{\overline{#1}}

\unless\ifdefined\IsMainDocument
\documentclass[12pt]{article}
\usepackage{amsmath,amsthm,amssymb,hyperref}
\newcommand{\Fhat}{\widehat{F}}
\newcommand{\Abs}[1]{\left| #1 \right|}
\begin{document}
\fi

\section{A partial converse to Lemma 6.4: Euler product implies convergence of series}

In Lemma 6.4, we show the Euler product formula
$$\sum f(n) = \prod (1 + f(p) + f(p^2) + \cdots)$$
assuming the series converges absolutely.

Let me prove a simple partial converse here: Suppose that $f \geq 0$ is multiplicative and that the infinite product $\prod (1 + f(p) + f(p^2) + \cdots)$ converges. Then the series $\sum f(n)$ converges absolutely (and hence, the Euler product above holds by the original Lemma 6.4).

By assumption $f \geq 0$, the sequence of partial sums for the series $\sum_{n \leq N} f(n)$ is an increasing sequence so to establish convergence, it suffices to show that it is bounded above and an obvious candidate for such bound is the limit of the infinite product.

To prove that, let $C_N$ be the finite set of all positive integers $n$ such that (i) every prime divisor of $n$ does not exceed $N$; and (ii) if $p^k | n$ then $k \leq N$. In other words, if $p_1, ..., p_r$ is the list of all primes $\leq N$ (i.e. $r = \pi(N)$) then
$$C_N = \{p_1^{\alpha_1} p_2^{\alpha_2} \cdots p_r^{\alpha_r} : 0 \leq \alpha_i \leq N \text{ for all } i\}.$$
It is clear that $C_N \supseteq \{1, 2, ..., N\}$. Therefore,
\begin{align*}
\sum_{n \leq N} f(n) &\leq \sum_{n \in C_N} f(n) \\
&= \prod_{p \leq N} (1 + f(p) + \cdots + f(p^N)) \\
&< \prod_{p \leq N} (1 + f(p) + f(p^2) + \cdots) \\
&< \prod_p (1 + f(p) + f(p^2) + \cdots).
\end{align*}
And so the series $\sum f(n)$ converges.

\section{On the simple logarithm bound in Example 6.8}

In Example 6.8, we have the simple bound
$$0 > \log\left(1 - \frac{1}{K}\right) > \frac{-1}{K - 1}, \qquad K > 2.$$

This inequality actually applies to all $K > 1$. Let me rewrite it as
$$0 > \log(1 - x) > -\frac{1}{x^{-1} - 1} = -\frac{x}{1 - x}, \qquad x \in (0, 1).$$

To prove that, recall the classical inequality
$$\log(1 + x) \leq x, \qquad \text{ for } x \geq -1$$
which could be proved by considering the function
$$f(x) = x - \log(1 + x)$$
defined on $(-1, \infty)$. Its derivative
$$f'(x) = 1 - \frac{1}{1 + x} = \frac{x}{1 + x}$$
is positive if $x > 0$ and is negative if $-1 < x < 0$. So $f(x) \geq f(0) = 0$ for all $x \in (-1, \infty)$.

Now back to our claimed inequality: We have
$$-\log(1 - x) = \log\left(\frac{1}{1 - x}\right) = \log\left(1 + \frac{x}{1 -  x}\right) \leq \frac{x}{1 - x}$$
as long as $\frac{x}{1 - x} \geq -1$ which is obvious if $x \in (0, 1)$.

\section{Uniform Convergence}

When we say $\Fhat$ converges uniformly on $U$, what we mean is: For any $\epsilon > 0$, there exists an $X_0$ such that for any $X > X_0$,
$$\Abs{\int_{1^-}^X x^{-s} dF(x) - \Fhat(s)} < \epsilon$$
for all $s \in U$.

In other words, the uniform convergence is for the $\lim_{X \rightarrow \infty}$ in the definition of $\Fhat$.

To see how this goes back to the \href{https://en.wikipedia.org/wiki/Uniform\_convergence}{classical definition of uniform convergence} of a sequence of functions $f_n \rightarrow f$, the ``sequence'' of functions in this case is
$$\Fhat_X(s) = \int_{1^-}^X x^{-s} dF(x).$$
So $\Fhat_X \rightarrow \Fhat$ uniformly, by the classical definition, means that for any $\epsilon > 0$, we can find a single $X_0(\epsilon)$ i.e. independent of $s$ such that
$$\Abs{ \Fhat_X(s) - \Fhat(s) } < \epsilon$$
as long as $X > X_0(\epsilon)$. See also the proof of Corollary 6.16.

Due to completeness of the real/complex numbers, the above definition has an equivalent Cauchy's version: $\Fhat_X \rightarrow \Fhat$ uniformly if and only if for any $\epsilon > 0$, we can find a single $X_0(\epsilon)$ such that for all $Y, Z > X_0(\epsilon)$, we have
$$\Abs{ \Fhat_Y(s) - \Fhat_Z(s) } < \epsilon.$$
This is the condition we used in the proof of Theorem 6.15. Also note in the proof of that Theorem, the reason we can write
$$\int_Y^Z x^{-s} dF(x) = \int_Y^Z x^{-(s-s_0)} d\psi(x)$$
is to first express
$$\int_Y^Z x^{-s} dF(x) = \int_Y^Z x^{-s_0} x^{-(s-s_0)} dF(x) = \int_Y^Z x^{-s_0} d\psi_0(x)$$
where 
$$\psi_0(y) = \int_{1^-}^y x^{-(s-s_0)} dF(x).$$
And then we observe that $\psi_0(y) - \psi(y) = K$ is a constant, namely $K = \int_{1^-}^{\infty} x^{-s_0} dF(x)$ by convergence assumption. This implies $\psi(y) = \psi_0(y) - K$ and so $d\psi = d\psi_0$.

\section{Weirstrass M test}

The classical statement is: Let $f_n$ be a sequence of functions $f_n: E \to \mathbb {C}$ and let $M_n$ be a sequence of positive real numbers such that $|f_n(x)|\leq M_{n}$ for all $x \in E$ and $n = 1, 2, 3, \ldots$. If $\sum M_n$ converges, then $\sum f_n$ converges uniformly on $E$.

In the proof of Lemma 6.13, we are using the obvious inequality
$$\Abs{ \int_Y^Z x^{-s} dF(x) } \leq \int_Y^Z x^{-b} dF_v$$
for all $s \in \{\sigma \geq b\}$ and the assumption $\int x^{-b} dF_v < \infty$ to deduce $\int x^{-s} dF(x)$ converges uniformly on that half plane.

\section{A consequence of Example 6.29}

We proved here that $\zeta(s) = \exp\left(\sum \kappa(n) n^{-s}\right)$ on $\{\Re s > 1\}$. This implies that $\zeta(s)$ has no zero on that halfplane.

\section{The inequality in the second proof of Lemma 6.30}

The inequality $n^\alpha \leq n^{\alpha + 1} - n^{\alpha}$ can be shown by noting that
$$n^\alpha \leq n^{\alpha + 1} - (n - 1)^{\alpha + 1} \iff n^\alpha + (n-1)^{\alpha + 1} \leq n^{\alpha + 1}$$
which easily follows from the trivial bound
$$(n-1)^{\alpha + 1} = (n-1) (n-1)^{\alpha} \leq (n-1) n^{\alpha}$$
whence
$$n^\alpha + (n-1)^{\alpha + 1} \leq n^{\alpha} + (n-1) n^{\alpha} = n^{\alpha + 1}.$$

\unless\ifdefined\IsMainDocument
\end{document}
\fi


\section{Problems}

\unless\ifdefined\IsMainDocument
\documentclass[12pt]{article}
\usepackage{amsmath,amsthm,amssymb}
\newcommand{\V}{\mathcal{V}}
\newcommand{\A}{\mathcal{A}}
\newcommand{\Fhat}{\widehat{F}}
\newcommand{\cconj}[1]{\overline{#1}}
\begin{document}
\fi

\textbf{Problem 6.1}: Suppose $F \in \V$ and is real valued. Prove that if $\Fhat(s)$ converges, then $\Fhat(\cconj{s})$ converges and equals $\cconj{\Fhat(s)}$.

\begin{proof}
\newcommand{\sbar}{\cconj{s}}
One has
\begin{align*}
\Fhat(\cconj{s}) &= \int x^{-\sbar} dF(x)\\
&= \cconj{\int \cconj{x^{-\sbar}} \; \cconj{dF(x)}} &\text{from the limit definition of integral}\\
&= \cconj{\int \cconj{x^{-\sbar}} \; dF(x)} &\text{by assumption } F \text{ real}\\
&= \cconj{\int x^{-s} \; dF(x)} = \cconj{\Fhat(s)}
\end{align*}
because if $s = \sigma + it$ then $\cconj{x^{-\sbar}} = \cconj{e^{-\sbar \log x}} = \cconj{e^{-(\sigma - i t) \log x}} = \cconj{e^{-\sigma \log x}} \; \cconj{e^{i t \log x}} = e^{-\sigma \log x} e^{- i t \log x} = e^{-s}$. Here we used the fact that $\cconj{e^{i\alpha}} = e^{-i\alpha}$ for real $\alpha$; which comes from an observation that if $|z| = z \; \cconj{z}= 1$ then $\cconj{z} = z^{-1}$.
\end{proof}

\textbf{Problem 6.2}: Find $f \in \A$ such that $\zeta(2s) = \sum f(n)n^{-s}$.

\begin{proof}
We have
$$\zeta(2s) = \sum n^{-2s} = \sum (n^2)^{-s}$$
so $f$ should be $1_S$ where $S$ is the set of squares.
\end{proof}

\unless\ifdefined\IsMainDocument
\end{document}
\fi

\unless\ifdefined\IsMainDocument
\documentclass[12pt]{article}
\usepackage{amsmath,amsthm,amssymb}

\begin{document}
\fi

\textbf{Problem 6.3}: Find the infinite product representation of $\sum |\mu(n)| n^{-s}$. Determine the region in which the representation is valid. Express the product as a quotient of zeta functions.

\begin{proof}
Obviously $|\mu|$ is multiplicative and $|\mu(n)| = O(1)$, just like $\mu$. So by Corollary 6.6 with $c = 0$, we find
$$\sum |\mu(n)| n^{-s} = \prod_p (1 + p^{-s})$$
valid for $\sigma > 1$. As
$$1 + p^{-s} = \frac{1 - p^{-2s}}{1 - p^{-s}},$$
we find
$$\prod_p (1 + p^{-s}) = \frac{\zeta(s)}{\zeta(2s)}.$$
\end{proof}

\textbf{Problem 6.4}: Show that there is a constant $k$ such that
$$\prod_{2 < p \leq x} (1 - 2/p) \sim k/\log^2 x, \qquad x \rightarrow \infty.$$

\begin{proof}
We consider the quotient
$$\frac{1 - 2/p}{(1 - 1/p)^2} = \frac{(p-2)/p}{(p-1)^2/p^2} = \frac{p(p-2)}{(p-1)^2} = 1 - \frac{1}{(p-1)^2}$$
and recall that
$$\prod_{2 < p \leq x} (1 - 1/p) \sim e^c / \log x$$
for some constant $c$, known to be $-\gamma$, by Lemma 4.11 so
$$\prod_{2 < p \leq x} (1 - 2/p) = \prod_{2 < p \leq x} (1 - p^{-1})^2 \prod_{2 < p \leq x} \left( 1 - \frac{1}{(p-1)^2} \right) \sim \frac{e^{2c} K}{\log^2 x}$$
where the constant $K$ is
$$K = \lim_{x \rightarrow \infty} \prod_{2 < p \leq x} \left( 1 - \frac{1}{(p-1)^2} \right) = \prod_p \left( 1 - \frac{1}{(p-1)^2} \right) = \sum_{n=1}^{\infty} \frac{\mu(n) n^2}{\varphi(n)^2} n^{-2}$$
exists by Corollary 6.6: the function $g(n) = \frac{\mu(n) n^2}{\varphi(n)^2}$ is multiplicative and $O(n)$ by the simple bound $\varphi(n) \geq \sqrt{n / 2}$ for square-free $n$.
\end{proof}

\unless\ifdefined\IsMainDocument
\end{document}
\fi

\unless\ifdefined\IsMainDocument
\documentclass[12pt]{article}
\usepackage{amsmath,amsthm,amssymb}

\begin{document}
\fi

\textbf{Problem 6.5}: Use the result of Example 6.8 to approximate
$$\prod_p \left(1 - \frac{2}{p^2 - 1} \right).$$

\begin{proof}
We factor out two $\zeta$-values:
\begin{align*}
I &:= \prod_p \left(1 - \frac{2}{p^2 - 1} \right)\\
&= \zeta(2)^{-2} \prod_p \left(1 - \frac{2}{p^2 - 1} \right) (1 - p^{-2})^{-2} &\text{as } -\frac{2}{p^2 - 1} \sim -2p^{-2}\\
&= \frac{36}{\pi^4} \prod_p \left(1 - \frac{3p^2 - 1}{(p^2 - 1)^3} \right)\\
&= \frac{36}{\pi^4} \zeta(4)^{-3} \prod_p \left(1 - \frac{3p^2 - 1}{(p^2 - 1)^3} \right) (1 - p^{-4})^{-3} &\text{ as } - \frac{3p^2 - 1}{(p^2 - 1)^3} \sim -3 p^{-4}\\
&= \frac{36}{\pi^4} \frac{90^3}{\pi^{12}} \prod_p \left(\frac{p^{16} (p^2 - 3) }{(p^2 - 1)^6 (p^2 + 1)^3} \right)\\
&= \frac{26,244,000}{\pi^{16}} \underbrace{\prod_p \left( 1 - g(p) \right)}_{J}
\end{align*}
where
\begin{align*}
g(p) &= 1 - \frac{ p^{16} (p^2 - 3) }{(p^2 - 1)^6 (p^2 + 1)^3} \\
&= \frac{8 p^{12} - 6 p^{10} - 6 p^8 + 8 p^6 - 3 p^2 + 1}{(p^2 - 1)^6 (p^2 + 1)^3}.
\end{align*}

Using the inequality
$$0 > \log(1 - K^{-1}) > -(K-1)^{-1}, \qquad \forall K > 2$$
in Example 6.8 to reduce the complexity (via getting rid of the logarithm) in Lemma 4.12, we have
\begin{align*}
0 &> \sum_{p > X} \log(1 - g(p))\\
&> - \sum_{p > X} \frac{1}{\frac{1}{g(p)} - 1}\\
&= - \sum_{p > X} \frac{g(p)}{1 - g(p)}\\
&= - \sum_{p > X} \frac{8 p^{12} - 6 p^{10} - 6 p^8 + 8 p^6 - 3 p^2 + 1}{p^{16} (p^2 - 3)}\\
&= - \sum_{p > X} \frac{8}{p^4 (p^2 - 3)} + \sum_{p > X} \frac{6 p^{10} +6 p^8 - 8 p^6 + 3 p^2 - 1}{p^{16} (p^2 - 3)}\\
&> - \sum_{p > X} \frac{8}{p^4 (p^2 - 3)}\\
&> - \sum_{p > X} \frac{1}{p^5} &\text{when } X > 7
\end{align*}
because then $p \geq 11$ and $p^2 - 3 \geq 8p$. % \iff (p - 4)^2 \geq 19 \iff p \geq 4 + \sqrt{19} = 8.35\cdots$.
The function $f(x) = x^{-5}$ is non-negative convex on $[0, \infty)$ and
\begin{align*}
\int_X^\infty f(x) dx = \left.\frac{x^{-4}}{-4} \right|_X^\infty = \frac{1}{4 X^4}
\end{align*}
which can be made $< 3 \epsilon$ by choosing $X > \sqrt[4]{\frac{1}{12\epsilon}}$. That means for $\epsilon = 10^{-6}$, taking $N = 4$ so that
$$X = 6N - 3 = 21 > \sqrt[4]{10^6 / 12} = 10 \sqrt[4]{100/12} = 16.99\cdots > 7$$
then
$$a = \sum_{p \leq X} \log(1 - g(p))$$
satisfies
$$|a - \log J| < \frac{1}{3} \int_X^\infty f(x) dx < \epsilon$$
by Lemma 4.12 whence (note that $J < 1$)
$$|e^a - J| < J (e^\epsilon - 1) < 2 \epsilon.$$

Note that % {p < 21} = {2, 3, 5, 7, 11, 13, 17, 19}
\begin{align*}
e^a &= \prod_{p \leq X} (1 - g(p)) \\
&= \frac{(\prod_{p \leq X} p)^{16} \prod_{p \leq X} (p^2 - 3)}{(\prod_{p \leq X} (p^2 - 1))^6 (\prod_{p \leq X} (p^2 + 1))^3} \\
&= \frac{9699690^{16} \cdot 12177858346368}{57789539942400^6 \cdot 141523538000000^3}\\
% &= \frac{356172372737609865916787176261675714823918430816027906960547594332402268815328155432773}{502973487059206069117596592324781827613978396665382251404157673603806003200000000000000}\\
&= 0.708133\cdots
\end{align*}
% In WolframAlpha:
% product_(i=1)^8 p_i          = 9699690
% product_(i=1)^8((p_i)^2 - 3) = 12177858346368
% product_(i=1)^8((p_i)^2 - 1) = 57789539942400
% product_(i=1)^8((p_i)^2 + 1) = 141523538000000
% (9699690^{16} * 12177858346368) / (57789539942400^6 * 141523538000000^3)
can be accurately computed as a rational number. With a rational approximation of $\frac{26,244,000}{\pi^{16}} \approx 0.291495642$ with error $< \epsilon$, we are guaranteed a rational approximation
$$0.291495642 \times 0.708133 = 0.206417683456386$$
to $I$ with error $\epsilon^2 + (e^a + 0.291495642) \epsilon < \epsilon$.

% Check that
% product_(i=1)^50(1 - 2/((p_i)^2 - 1)) = 0.20669326518...
\end{proof}

\unless\ifdefined\IsMainDocument
\end{document}
\fi

\unless\ifdefined\IsMainDocument
\documentclass[12pt]{article}
\usepackage{amsmath,amsthm,amssymb}
\newcommand{\Z}{\mathbb{Z}}
\begin{document}
\fi

\textbf{Problem 6.6}: Let $\varphi$ denote Euler's function, and for each $j \in \Z^+$ define $a_j = \#\{n \in \Z^+ : \varphi(n) = j\}$. Show that for $\sigma > 1$,
$$\sum_{j = 1}^{\infty} a_j j^{-s} = \sum_{n = 1}^{\infty} \varphi(n)^{-s} = \zeta(s) \prod_p \{ 1 + (p - 1)^{-s} - p^{-s} \}.$$

\begin{proof}
First, let us establish the absolute convergence of the series in the middle $\sum_{n = 1}^{\infty} \varphi(n)^{-s}$ when $\sigma > 1$. Observe that for any $\epsilon > 0$, there exists a constant $C$ (depending on $\epsilon$) such that $\varphi(n) \geq C n^{1 - \epsilon}$ for all $n$ because $p - 1 \geq p^{1-\epsilon}$ when $p$ is large enough. So we find that
$$\sum_{n = 1}^{\infty} |\varphi(n)^{-s}| \leq \sum_{n = 1}^{\infty} C^\sigma n^{(1 - \epsilon) \sigma} = C^\sigma \sum_{n = 1}^{\infty} n^{\sigma - \epsilon \sigma} < \infty$$
by picking the appropriate $\epsilon > 0$ such that $\sigma - \epsilon \sigma > 1$, equivalently $\epsilon < \frac{\sigma - 1}{\sigma}$ and that is clearly possible if $\sigma > 1$.

The first equality then comes from absolute convergence of the series in the middle. Recall that if $\sum a_n$ converges absolutely then any rearrangement i.e. $\sum a_{\sigma(n)}$ for any permutation $\sigma$ of $\Z^+$ converges to the same value. Thus, if $\sum_{n = 1}^{\infty} \varphi(n)^{-s}$ converges absolutely then we can permute the $n$ so that the ones with the same $\varphi$ are next to each other and that gives us $\sum_{j = 1}^{\infty} a_j j^{-s}$. It does not even matter the order in which the sum over $j$ is performed.

With absolute convergence comes
$$\sum_{n = 1}^{\infty} \varphi(n)^{-s} = \prod_p (1 + \varphi(p)^{-s} + \varphi(p^2)^{-s} + \cdots)$$
by Lemma 6.4 applied to the multiplicative function $\varphi(n)^{-s}$. It is easy to see that
\begin{align*}
1 + \varphi(p)^{-s} + \varphi(p^2)^{-s} + \cdots &= 1 + (p - 1)^{-s} + (p(p-1))^{-s} + \cdots\\
&= 1 + (p-1)^{-s} (1 + p^{-s} + p^{-2s} + \cdots)\\
&= 1 + (p-1)^{-s} (1 - p^{-s})^{-1}\\
&= (1 - p^{-s})^{-1} (1 - p^{-s} + (p-1)^{-s})
\end{align*}
and so the second equality follows.
\end{proof}

\textbf{Problem 6.7}: Show that
$$\prod_p \{ 1 + (p - 1)^{-1} - p^{-1} \} = \zeta(2) \zeta(3) / \zeta(6).$$

\begin{proof}
This should follows easily by matching the Euler's factors on both sides
$$1 + (p - 1)^{-1} - p^{-1} = \frac{(1 - p^{-2})^{-1} (1 - p^{-3})^{-1}}{(1 - p^{-6})^{-1}}.$$

Explicitly, we expand the right hand side of the above
\begin{align*}
\frac{(1 - p^{-2})^{-1} (1 - p^{-3})^{-1}}{(1 - p^{-6})^{-1}} &= \frac{1 - p^{-6}}{(1 - p^{-2}) (1 - p^{-3})}\\
&= \frac{1 + p^{-3}}{1 - p^{-2}}\\
&= \frac{(1 + p^{-1})(1 - p^{-1} + p^{-2})}{(1 - p^{-1})(1 + p^{-1})}\\
&= \frac{1 - p^{-1} + p^{-2}}{1 - p^{-1}}\\
&= \frac{p^2 - p + 1}{p(p-1)}
\end{align*}
and compare with the left hand side
\begin{align*}
1 + (p - 1)^{-1} - p^{-1} &= 1 + \frac{1}{p-1} - \frac{1}{p}\\
&= \frac{1}{p-1} + \frac{p - 1}{p}\\
&= \frac{p + (p-1)^2}{p(p-1)}\\
&= \frac{p^2 - p + 1}{p(p-1)}.
\end{align*}
And there is a match.
\end{proof}

\unless\ifdefined\IsMainDocument
\end{document}
\fi

\unless\ifdefined\IsMainDocument
\documentclass[12pt]{article}
\usepackage{amsmath,amsthm,amssymb}
\newcommand{\C}{\mathbb{C}}
\newcommand{\V}{\mathcal{V}}
\newcommand{\Fhat}{\widehat{F}}
\begin{document}
\fi

\textbf{Problem 6.8}: Assume that it is known that $\sum \mu(n) n^{-1-it}$ converges for some real $t$. Deduce from this the validity of the PNT.

\begin{proof}
By Theorem 6.9, the assumption implies $M(x) = o(x)$. Here, we are in case (1) where $\sigma = 1 > 0$. Clearly, this is equivalent to the PNT by Theorem 5.9.
\end{proof}

\textbf{Problem 6.9}: Let $F \in \V$ and assume that $\int x^{-s_0} dF(x)$ converges for some $s_0$ with $\Re s_0 > 0$. Prove that
$$\int x^{-s_0} dF(x) = s_0 \int x^{-s_0-1} F(x) dx.$$

\begin{proof}
By Theorem 6.9, we have $F(x) = o(x^\sigma)$ where $\sigma = \Re s_0$. For any $X > 1$, integration by parts gives
$$\int_{1^-}^X x^{-s_0} dF(x) = F(X) X^{-s_0} + s_0 \int_{1^-}^X x^{-s_0-1} F(x) dx$$
where taking limit as $X \rightarrow \infty$, taking into account $F(x) = o(x^{\sigma})$, yields the desired equation.
\end{proof}

\textbf{Problem 6.10}: Suppose $F(x) \sim x$. Find the precise set of points $s \in \C$ at which $\Fhat$ converges.

\begin{proof}
The example $F(x) = x$ where $\Fhat(s) = \int x^{-s} dx = \lim_{X \rightarrow \infty} \frac{X^{-s + 1}}{1 - s} - \frac{1}{1 - s}$ and $F(x) = N(x)$ where $\Fhat(s) = \zeta(s)$ shows that the set of convergence should be $\Re s > 1$.

Clearly, if $F(x) \sim x$ then $F(x) = x + o(x) = O(x)$ so Theorem 6.9 implies that $\Fhat$ converges for $\Re s > 1$.

Now assume that $\Fhat(s)$ converges and we show that $\Re s > 1$. If $\sigma \leq 0$ then either $F(x) = o(\log ex)$ or $F(x) = c + o(x^\sigma)$ by Theorem 6.9 and this clearly violates the assumption that $F(x) \sim x$. If $0 < \sigma \leq 1$ then by Theorem 6.9 $F(x) = o(x)$ once again violates the assumption $F(x) \sim x$. So $\sigma > 1$.
\end{proof}

\unless\ifdefined\IsMainDocument
\end{document}
\fi

\unless\ifdefined\IsMainDocument
\documentclass[12pt]{article}
\usepackage{amsmath,amsthm,amssymb}
\newcommand{\Fhat}{\widehat{F}}
\begin{document}
\fi

\textbf{Problem 6.11}: Find the abscissa of convergence of the following M.t.'s:
\begin{enumerate}
\item $\sum |\mu(n)| n^{-s}$
\item $\int x^{-s} d(x^3)$
\item $\int x^{-s} \cos(x^2) dx$
\item $\sum n^{-2s}$
\item $\sum_{n \geq 10} n^{\log \log n} n^{-s}$
\item $\sum_{n \geq 10} n^{-\log \log n} n^{-s}$
\end{enumerate}

\begin{proof}
\begin{enumerate}
\item $\sum |\mu(n)| n^{-s} = \int x^{-s} dQ(x)$ has $\sigma_c = 1$ because the exact growth rate of $Q(x)$ is known to be linear. As an alternative, we can use Problem 6.10 with $\frac{\pi^2}{6} Q(x) \sim x$.

\item Clearly $\int x^{-s} d(x^3)$ has $\sigma_c = 3$ because $F(x) = x^3$ has obvious growth rate.

\item At first sight, we can estimate $\int_{1^-}^x \cos(t^2) \; dt = O(x)$ as $\cos(t^2) = O(1)$. So $\sigma_c \leq 1$ i.e. the M.t. converges for all $s$ with $\Re s > 1$. But this is not the right answer. Making a change of variable $u = x^2$ to get rid of the $x^2$ under $\cos$, we have
\begin{align*}
\int x^{-s} \cos(x^2) dx &= \lim_{X \rightarrow \infty} \int_{1^-}^X x^{-s} \cos(x^2) dx\\
&= \lim_{X \rightarrow \infty} \int_{1^-}^{X^2} u^{-s/2} \cos u \; d(u^{1/2})\\
&= \frac{1}{2} \lim_{X \rightarrow \infty} \int_{1^-}^{X^2} u^{-s/2-1/2} \cos u \; du\\
&= \frac{1}{2} \lim_{X \rightarrow \infty} \int_{1^-}^X u^{-(s/2+1/2)} d(\sin u)\\
&= \frac{1}{2} \Fhat(s/2 + 1/2) &\text{ where } F(x) = \sin x
\end{align*}
It is apparent that $\sin x = O(1)$ so $\Fhat(s)$ converges as long as $\Re(s) > 0$. Also, $\Fhat(s)$ diverges for all $s$ with $\Re(s) < 0$ because $\sin x$ is not $c + o(x^{-\epsilon})$ for any $\epsilon > 0$. Thus, $\sigma_c(\Fhat) = 0$ so $\sigma_c = -1$ for $\int x^{-s} \cos(x^2) dx$. %$\Re(s/2 + 1/2) > 0 \iff 1/2 \Re(s) > -1/2 \iff \Re(s) > -1$

\item $\sum n^{-2s} = \zeta(2s)$ so $\sigma_c = 1/2$ thank to the knowledge that $\sigma_c = 1$ for $\zeta(s)$.

\item $\sum_{n \geq 10} n^{\log \log n} n^{-s} = \int x^{-s} \; dF(x)$ minus a constant. Here, $F$ is the summatory function of $n^{\log \log n}$ (note that for $n = 1$, $1^{\log 0} = 1^{-\infty} = 0$ and we take $t^{\log \log t} = 0$ for $t < 1$) i.e.
\begin{align*}
F(x) &= \int_{1^-}^x t^{\log \log t} dN(t)\\
&= N(x) x^{\log \log x} - \int_{1^-}^x N(t) \; d\left( t^{\log \log t} \right)\\
&= N(x) x^{\log \log x} - \int_{1^-}^x \frac{N(t)}{t} t^{\log \log t}(\log \log t + 1) dt\\
&\sim x^{1 + \log \log x}
\end{align*}
which clearly is not of polynomial growth. (I might be wrong about this estimate. Let us write a program to confirm this.) So $\sigma_c = +\infty$ i.e. the series can never converge.

We can also see this directly: Fix $s$. As $\log \log n \rightarrow \infty$, we have $\log \log n > 1 + \sigma$ for $n$ large enough, say $n > N$. This implies that the terms $n^{\log \log n - s}$ does not converges to 0; otherwise $|n^{\log \log n - s}| = n^{\log \log n - \sigma} \rightarrow 0$ which cannot happen because $n^{\log \log n - \sigma} > n$. Hence, the series does not converge for $s$. (If $\sum a_n$ converges then $a_n \rightarrow 0$ as $a_n$ are difference of consecutive partial sums.)

\item $\sum_{n \geq 10} n^{-\log \log n} n^{-s}$ should have $\sigma_c = -\infty$ i.e. converges for all $s$ on account that
$$\int_{1^-}^x t^{-\log \log t} dN(t) \sim x^{1 - \log \log x}$$
which is lower than any $x^k$.

Again, we can see this directly: For fix $s$, take $N$ large enough such that $\log \log n > 2 + \sigma$. Then
$$\sum_{n \geq N} |n^{-\log \log n} n^{-s}| = \sum_{n \geq N} |n^{-(\log \log n - s)}| < \sum_{n \geq N} n^2 < \infty$$
so not only the series converges, it converges absolutely.
\end{enumerate}
\end{proof}

\unless\ifdefined\IsMainDocument
\end{document}
\fi

\unless\ifdefined\IsMainDocument
\documentclass[12pt]{article}
\usepackage{amsmath,amsthm,amssymb}
\newcommand{\Abs}[1]{\left| #1 \right|}
\begin{document}
\fi

\textbf{Problem 6.12}: For each of the following M.t.'s, find the set of points on its line of convergence at which it converges:
\begin{enumerate}
\item $\int x^{-s} dx$
\item $\int x^{-s} \log(ex)^{-1} dx$
\item $\int x^{-s} \log(ex)^{-2} dx$
\end{enumerate}

\begin{proof}
\begin{enumerate}
\item $\int x^{-s} dx$: The line of convergence is $\sigma = 1$. For $t \not= 0$, we have
$$\int x^{-1+it} dx = \frac{1}{it} (\lim_{X \rightarrow \infty} X^{it} - 1)$$
does not converge since $X^it = e^{i t \log X}$ goes around the circle indefinitely. As for $t = 0$,
$$\int x^{-1+it} dx = \int x^{-1} dx = \lim_{X \rightarrow \infty} \log X$$
does not converge.

\item $\int x^{-s} \log(ex)^{-1} dx$: The function
$$\int_2^x \log(et)^{-1} dt \sim \frac{x}{\log ex}$$
is $o(x^\sigma)$ only for $\sigma > 1$. So $\sigma_c = 1$ and the line of convergence is $\sigma = 1$. For $s = -1$, one has
$$\int x^{-1} \log(ex)^{-1} dx = \int \frac{d\log x}{1 + \log x} = \lim_{X \rightarrow \infty} \log(1 + \log X)$$
which obviously diverges. For $s = 1 - it$ with $t \not= 0$, one has $x^{it} = e^{i t \log x} = \cos(t\log x) + i \sin(t\log x)$ so
\begin{align*}
\int x^{-1+it} \log(ex)^{-1} dx &= \int \frac{\cos(t\log x) + i \sin(t\log x)}{x \log(ex)} dx\\
&= \int \frac{\cos(t\log x)}{x \log(ex)} dx + i \int \frac{\sin(t\log x)}{x \log(ex)} dx\\
&= \int \frac{\cos u}{1 + \frac{u}{|t|}} \frac{du}{|t|} + i \frac{t}{|t|} \int \frac{\sin(u)}{1 + \frac{u}{|t|}} \frac{du}{|t|} &\text{ change } u = |t| \log x\\
&= \int \frac{\cos u}{|t| + u} du + i \frac{t}{|t|} \int \frac{\sin(u)}{|t| + u} du
\end{align*}
The use of absolute value is to make sure that $u \rightarrow +\infty$ as $x \rightarrow +\infty$. Note that the change of variable is only possible when $t \not= 0$ and the limit in $\int (...)$ is $\lim_{U \rightarrow \infty} \int_0^U (...)$ due to the change of variable. Put $a = |t|$. We have
$$\int_0^U \frac{\cos u}{a + u} du = \int_0^U \frac{1}{a + u} d\sin u = \left. \frac{\sin u}{a + u} \right|_0^U + \int_0^U \frac{\sin u}{(a + u)^2} du$$
converges as $U \rightarrow \infty$: It is clear for
$$\left. \frac{\sin u}{a + u} \right|_0^U = \frac{\sin U}{a + U} \rightarrow 0$$
whereas the later converges absolutely
\begin{align*}
\int_0^U \frac{|\sin u|}{(a + u)^2} du \leq \underbrace{\int_0^1 \frac{|\sin u|}{(a + u)^2} du}_{K} + \int_{1}^{U} \frac{1}{u^2} du = K + \left. \frac{-1}{u} \right|_{1}^{U} \rightarrow K + 1.
\end{align*}
So $\int x^{-s} \log(ex)^{-1} dx$ converges at every point on the line of convergence $\sigma = 1$, except for the real point $s = 1$.

\item $\int x^{-s} \log(ex)^{-2} dx$: We expect a similar analysis as the previous problem. Starting with
$$F(x) = \int_1^x \log(et)^{-2} dt \sim \frac{x}{\log^2 ex}$$
so we have the same line of convergence $\sigma = 1$. Note that $F$ is increasing so $F_v = F$. The difference now is that
\begin{align*}
\int |x^{-1+it}| d F_v(x) &= \int |x^{-1+it}| \log(et)^{-2} dt\\
&= \lim_{X \rightarrow \infty} \int_1^X \frac{d \log x}{(1 + \log x)^2}\\
&= \lim_{X \rightarrow \infty} 1 - \frac{1}{1 + \log X}\\
&= 1
\end{align*}
converges for all $t$. So not only $\int x^{-s} \log(ex)^{-2} dx$ converges at every point on the line of convergence $\sigma = 1$, it converges absolutely at those points as well.
\end{enumerate}
\end{proof}

\unless\ifdefined\IsMainDocument
\end{document}
\fi

\unless\ifdefined\IsMainDocument
\documentclass[12pt]{article}
\usepackage{amsmath,amsthm,amssymb}
\newcommand{\V}{\mathcal{V}}
\newcommand{\Fhat}{\widehat{F}}
\begin{document}
\fi

\textbf{Problem 6.13}: Let $F \in \V$ and $b > \sigma_a(\Fhat)$. Show that $\Fhat$ is bounded on $\{s : \sigma \geq b\}$.

\begin{proof}
By assumption on $b$, we have $A = \int x^{-b} dF_v$ converges. It then follows that for any $s$ with $\sigma \geq b$,
$$|\Fhat(s)| \leq \int x^{-\sigma} dF_v \leq \int x^{-b} dF_v = A$$
so $\Fhat$ is bounded on $\{s : \sigma \geq b\}$ by $A$.
\end{proof}

\textbf{Problem 6.14}: Suppose that $\sum a_n n^{-s}$ converges at $s = s_0$. Prove that $\sum |a_n| n^{-\sigma}$ converges if $\Re s > 1 + \Re s_0$ and conclude that $\sigma_a \leq 1 + \sigma_c$.

\begin{proof}
Let $A(x)$ denotes the summatory function of the $a_n$ and $\sigma_0 = \Re s_0$. By Theorem 6.9, the assumption that $\sum a_n n^{-s} = \int x^{-s} dA$ converges at $s = s_0$ implies that
\begin{itemize}
\item $A(x) = c + o(x^{\sigma_0})$ if $\sigma_0 \not= 0$ where $c$ is some constant ($c = 0$ if $\sigma_0 > 0$) so for $C$ large enough $|A(x) - c| \leq C x^{\sigma_0}$ for all $x$ and
\begin{align*}
A_v(x) &= \sum_{n \leq x} |a_n|\\
&= \sum_{n \leq x} |A(n) - A(n-1)|\\
&\leq \sum_{n \leq x} C (n^{\sigma_0} + (n-1)^{\sigma_0})\\
&\leq 2C \int_1^{x+1} t^{\sigma_0} dt \sim \frac{2C}{\sigma_0 + 1} x^{\sigma_0 + 1}
\end{align*}
since the function $t^{\sigma_0}$ is monotone and so $\int_{n-1}^n t^{\sigma_0} dt \geq (n-1)^{\sigma_0}$ or $n^{\sigma_0}$. (We could use Lemma 3.13 when $\sigma_0 < 0$.) So $A_v(x) = O(x^{\sigma+1})$ and that implies $\sum |a_n| n^{-\sigma} = \int |x^{-s}| dA_v$ converges by Theorem 6.9 again.

\item $A(x) = o(\log ex)$ if $\sigma_0 = 0$. We use a similar argument: Let $C$ be the constant $A(x) < C \log ex$ and then
\begin{align*}
A_v(x) &\leq \sum_{n \leq x} C (\log (en) + \log (e(n-1)))\\
&\leq 2C \int_{1^-}^{x+1} \log(et) dt &\text{for } \int_n^{n+1} \log(et) dt \geq \log(en)\\
&\leq 2C x \log (ex)
\end{align*}
So if $\Re s > 1 + \sigma_0 = 1$ then let $\delta = \sigma - 1 > 0$ then we find $A_v(x) = O(x^{1+\delta/2})$ whence $\sum |a_n| n^{-\sigma} = \int |x^{-s}| dA_v$ converges by Theorem 6.9 due to $\sigma > 1 + \delta/2$.
\end{itemize}
\end{proof}

\unless\ifdefined\IsMainDocument
\end{document}
\fi

\unless\ifdefined\IsMainDocument
\documentclass[12pt]{article}
\usepackage{amsmath,amsthm,amssymb}
\newcommand{\Fhat}{\widehat{F}}
\newcommand{\Abs}[1]{\left| #1 \right|}
\begin{document}
\fi

\textbf{Problem 6.15}: Use Theorem 6.15 to give another proof that the region of convergence of an M.t. is a half plane.

\begin{proof}
Basically, this is just due to the fact that as $\delta \rightarrow 0$, the sector $S_\delta$ becomes the half plane $\sigma > \Re s_0$.

To do it formally, assume the M.t. converges at $s_0$ and we show that it converges at all $s$ such that $\Re s > \Re s_0$ as in Lemma 6.11. Since $\Re s > \Re s_0$, the angle $\arg(s - s_0)$ made by the two points is between $(-\pi/2, \pi/2)$. So $\delta = \pi/2 - |\arg(s - s_0)| > 0$ and so we have a sector $S_{\delta/2}$ contains $s$ which implies convergence at $s$ by Theorem 6.15.
\end{proof}

\textbf{Problem 6.16}: Let $F$, $\delta$, $S_\delta$ be as in Theorem 6.15. Show that $\lim \Fhat(s) = F(1)$, where $s \rightarrow \infty$ in the sector $S_\delta$.

\begin{proof}
In the sector $S_\delta$ we have uniform convergence to swap the limit over $X$ and over $s$:
\begin{align*}
\lim_s \Fhat(s) &= \lim_s \lim_{X \rightarrow \infty} \int_{1^-}^X x^{-s} dF(x)\\
&= \lim_{X \rightarrow \infty} \lim_s \int_{1^-}^X x^{-s} dF(x) &\text{by uniform convergence}\\
&= \lim_{X \rightarrow \infty} F(1)\\
&= F(1)
\end{align*}
To explain why the inner limit $\lim_s \int_{1^-}^X x^{-s} dF(x)$ is constant $F(1)$, we observe that
$$\int_{1^-}^X x^{-s} dF(x) = F(1) - \underbrace{F(1^-)}_{0} + \int_1^X x^{-s} dF(x)$$
by Lemma 3.8. Let $\epsilon > 0$ be arbitrary. Since $x^{-s} \rightarrow 0$ as $s \rightarrow \infty$ as long as $x > 1$, we have $x^{-s} < \frac{\epsilon}{F_v(X) - F_v(1)}$ for all $s$ sufficiently large\footnote{In other words, there exists some $N$ such that the inequality holds for all $s \in S_\delta$ such that $|s| > N$.} and all $1 < x < X$ whence
\begin{align*}
\Abs{ \int_1^X x^{-s} dF(x) } &\leq \int_1^X \frac{\epsilon}{F_v(X) - F_v(1)} dF_v(x)\\
&= \frac{\epsilon}{F_v(X) - F_v(1)} (F_v(X) - F_v(1))\\
&= \epsilon.
\end{align*}
So we just showed
$$\int_1^X x^{-s} dF(x) \rightarrow 0, \qquad s \rightarrow \infty \text{ in } S_\delta$$
by definition of limit.
\end{proof}

\textbf{Problem 6.17}: Let $F(x) := \int_1^x \cos(\log u) du$ for $x \geq 1$. Show that the integral defining $\Fhat$ converges, but not uniformly, in the open half plane $\{s : \sigma > 1\}$.

\begin{proof}
We have
\begin{align*}
F(x) &= \int_1^x \cos(\log u) \; du\\
&= \int_0^{\log x} \cos(t) \; d(e^t) &\text{substitute } t = \log u\\
&= \left. \frac{1}{2} e^t (\sin t + \cos t) \right|_0^{\log x} &\text{integration by parts}\\
&= \frac{1}{2} x (\sin \log x + \cos \log x) - \frac{1}{2}
\end{align*}
is clearly $O(x)$. So $\Fhat$ converges on $\{\sigma > 1\}$ by Theorem 6.9. As for uniform convergence, we want to check
\begin{align*}
\int_Y^Z x^{-s} \cos(\log x) \; dx &= \int_Y^Z x^{-(s-1)} \cos(\log x) \; d(\log x)\\
&= \int_Y^Z e^{- (s-1) \log x} \cos(\log x) \; d(\log x)\\
&= \int_{\log Y}^{\log Z} e^{- (s-1) u} \cos(u) \; du\\
&= \left. \frac{e^{-(s-1)u} (\sin u - (s - 1) \cos u)}{(s-1)^2 + 1} \right|_{\log Y}^{\log Z}
\end{align*}
The trouble is apparent now: Given any fixed $Y, Z$ (no matter how large), the denominator $(s-1)^2 + 1 = (s - 1 - i) (s - 1 + i)$ could become arbitrarily close to zero as $s \rightarrow 1 \pm i$ whereas the two numerators $e^{-(s-1)u} (\sin u - (s - 1) \cos u)$ converges to non-zero values
$$e^{\mp i \log Z} (\sin \log Z - (\mp i) \cos \log Z)$$
and similarly for $Y$ so that the integral blows up. Here, note that the complex number $\sin \log Z - (\mp i) \cos \log Z = 0$ if and only if $\sin \log Z = \pm \cos \log Z = 0$ as those are the real and imaginary parts but $\sin$ and $\cos$ cannot be simultaneously zero. The fact that $e^{\mp i \log Z} \not= 0$ is obvious since the exponential function cannot take value 0.
\end{proof}

\unless\ifdefined\IsMainDocument
\end{document}
\fi

\unless\ifdefined\IsMainDocument
\documentclass[12pt]{article}
\usepackage{amsmath,amsthm,amssymb}
\newcommand{\R}{\mathbb{R}}
\newcommand{\Fhat}{\widehat{F}}
\newcommand{\Abs}[1]{\left| #1 \right|}
\begin{document}
\fi

\textbf{Problem 6.18}: If the Dirichlet series $\sum c_n n^{-s}$ converges at $s = \sigma_0 + i t_0$, prove that the function defined by the series for $\Re s > \sigma_0$ cannot have a pole on the line $\Re s = \sigma_0$.

\begin{proof}
Let $F(x) = \sum_{n \leq x} c_n$. Without loss of generality, we could assume $s_0 = 0$.

In other words, it suffices to show that if $\sum c_n = \lim_{x \rightarrow \infty} F(x)$ converges then the function defined by $\sum c_n n^{-s}$ on the right half plane $H := \{s : \Re s > 0\}$ cannot have a pole on the imaginary axis.

If we manage to do that, the general statement follows by applying this simplification with $c_n n^{-(\sigma_0 + i t_0)}$ in place of the $c_n$. The technique of translating the argument $s$ by $s - (\sigma_0 + i t_0)$ is used the proof of Theorem 6.9 and 6.15.

Now let $c = F(\infty) = \sum c_n$ and assume that $\Fhat(s)$ has a pole of order $k$ at some point on the imaginary axis, say at $i t \not= 0$. Then the limit
$$\lim_{s \rightarrow i t} (s - i t)^k \Fhat(s), \qquad s \in H$$
exists and is non-zero. In this problem, we only need to care about the simpler limit on the horizontal line $\Im s = t$, namely
$$\lim_{\sigma \rightarrow 0^+} \sigma^k \Fhat(\sigma + i t)  \text{ exists and } \not= 0.$$

Using equation (6.5) in the proof\footnote{We are in the case $\alpha = 0$ i.e. $F(x) - c = O(1)$ there. Since $F(x) \rightarrow c$, for $X$ large enough $|F(x) - c| < 1$ for all $x \geq X$. Then we have an upper bound $|F(x) - c| \leq 1 + \sum_{n \leq X} |F(n) - c|$ for all $x$.} of Theorem 6.9
$$\Fhat(s) = c + s \int_1^\infty x^{-s-1} (F(x) - c) dx, \qquad \forall s \in H,$$
we obtain
\begin{align*}
\lim_{\sigma \rightarrow 0^+} \sigma^k \Fhat(\sigma + i t)
&= \lim_{\sigma \rightarrow 0^+} c \sigma^k + \sigma^k (\sigma + i t) \int_1^\infty x^{-\sigma - i t - 1} (F(x) - c) dx\\
&= (it) \lim_{\sigma \rightarrow 0^+} \underbrace{\sigma^k \int_1^\infty x^{-\sigma - i t - 1} (F(x) - c) dx}_{H_\infty(\sigma)}
\end{align*}
where for $X \in [1, \infty]$, we define the functions
$$H_X(\sigma) = \sigma^k \int_1^X x^{-\sigma - i t -1} (F(x) - c) dx$$
on $(0, 1)$.

We claim that $H_X$ converges uniformly to $H_\infty$ as $X \rightarrow \infty$: Let $\epsilon > 0$ be arbitrary and simply pick $X > 1$ such that $|F(x) - c| < \epsilon$ for all $x > X$. Then for any $Y > X$, we find $\Abs{ H_Y(\sigma) - H_\infty(\sigma) } < \epsilon$ for all $\sigma \in \R^+$ because
\begin{align*}
\Abs{ H_Y(\sigma) - H_\infty(\sigma) } &= \Abs{ \sigma^k \int_Y^\infty x^{-\sigma - i t -1} (F(x) - c) dx } \\
&\leq \sigma^k \int_Y^\infty x^{-\sigma-1} |F(x) - c| dx \\
&\leq \epsilon \sigma^k \int_Y^\infty x^{-\sigma-1} dx \\
&= \epsilon \sigma^{k-1} Y^{-\sigma} \\
&\leq \epsilon
\end{align*}
as long as $\sigma \in (0, 1)$. Here since $Y > X > 1$, the exponential function $t \mapsto Y^t$ is increasing on $\R$ and so $Y^{-\sigma} < Y^0 = 1$.

Uniform convergence allows us to swap the two limits over $\sigma$ and $X$ and get a contradiction
\begin{align*}
0 \not= \lim_{\sigma \rightarrow 0^+} H_\infty(\sigma) &= \lim_{\sigma \rightarrow 0^+} \lim_{X \rightarrow \infty} H_X(\sigma) \\
&= \lim_{X \rightarrow \infty} \lim_{\sigma \rightarrow 0^+} H_X(\sigma) \\
&= \lim_{X \rightarrow \infty} \lim_{\sigma \rightarrow 0^+} \sigma^k \int_1^X x^{-\sigma - i t -1} (F(x) - c) dx \\
&= \lim_{X \rightarrow \infty} \lim_{\sigma \rightarrow 0^+} \sigma^k \left(\left.\frac{(F(x) - c)x^{-\sigma-it}}{-\sigma - it}\right|_1^X - \sum_{1 < n \leq X} \frac{c_n n^{-\sigma - i t}}{-\sigma - it} \right) \\
&= 0
\end{align*}
since the expression in parentheses converges as $\sigma \rightarrow 0^+$.

\textbf{Remark}: The previous problem showed that this property only works for Dirichlet series. The Mellin transform there has two simple poles on the line $\Re s = 1$ and converges otherwise.
\end{proof}

\textbf{Problem 6.19}: Use the preceding problem, show that
$$\sum_{n=1}^{\infty} \mu(n) n^{-\frac12 - it} \qquad \text{ and } \qquad \sum_{n=1}^{\infty} n^{-1 - it}$$
each diverge for all real $t$. (You may assume that the Riemann zeta function has zeros with real part $1/2$.)

\begin{proof}
Let $\eta(s) = \sum \mu(n) n^{-s}$ whenever the limit exists and assume $\eta(-\frac12 - it)$ converges for some real $t$. Then $\eta(s)$ cannot have a pole on $\Re s = 1/2$. We also have $\eta(s)$ converges for $\Re s > 1/2$ which implies analytic continuation of $\zeta(s)$ to the half plane $\Re s > 1/2$ with the the places where $\eta = 0$ removed. We have the identity
$$\zeta(s) \eta(s) = 1$$
on $\Re s > 1$. This identity still holds for the extended $\zeta$ function by identity theorem in complex analysis (view both sides as complex analytic functions). But then $\eta(s)$ has some pole on the line $\Re s = 1/2$ because $\zeta(s)$ has zeros on that line. We thus reached a contradiction. So $\eta(s)$ does not converge.

The second one is obvious: If $\sum_{n=1}^{\infty} n^{-1 - it}$ converges for some real $t$ then the function $\zeta(s) = \sum n^{-s}$ cannot have a pole on the line $\Re s = 1$. But $\zeta(s)$ does have a pole at $s = 1$; a contradiction. So $\sum_{n=1}^{\infty} n^{-1 - it}$ diverges for all real $t$.
\end{proof}

\unless\ifdefined\IsMainDocument
\end{document}
\fi

\unless\ifdefined\IsMainDocument
\documentclass[12pt]{article}
\usepackage{amsmath,amsthm,amssymb}

\begin{document}
\fi

\textbf{Problem 6.20}: Let $\lambda$ denote Liouville's function (c.f. Problem 2.27). Find the numerical value of $\sum_{n=1}^{\infty} \lambda(n)/n$, assuming that the series converges.

\begin{proof}
We have
$$\sum_{n=1}^{\infty} \lambda(n) n^{-s} = \frac{\zeta(2s)}{\zeta(s)}$$
so taking the limit as $s \rightarrow 1^+$ yields
$\sum_{n=1}^{\infty} \lambda(n)/n = 0$
for $\zeta(s)$ has a pole there.
\end{proof}

\unless\ifdefined\IsMainDocument
\end{document}
\fi

\unless\ifdefined\IsMainDocument
\documentclass[12pt]{article}
\usepackage{amsmath,amsthm,amssymb}
\newcommand{\Abs}[1]{\left| #1 \right|}
\begin{document}
\fi

\textbf{Problem 6.21}: Use the formula
$$\frac{1}{s - 1} - \frac{\zeta(s)}{s} = \int_1^\infty \frac{x - [x]}{x^{s+1}} dx \qquad (\Re s > 0)$$
(alternative form of (6.12)) to prove that $\zeta(s) \not= 0$ on the set
$$\{s : \Re s \geq 1/2, |\Im s| \leq \sqrt 3 / 2\}.$$
Hint: For $\Re s > 0$, show that
$$\Abs{ \frac{1}{s - 1} - \frac{\zeta(s)}{s} } < \frac{1}{2\sigma}.$$

\begin{proof}
Note that the formula comes from applying Euler's summation formula with $c = 0$ instead of $c = 1/2$ in (6.12).

To prove the hinted inequality, we have
$$\Abs{ \frac{1}{s - 1} - \frac{\zeta(s)}{s} } \leq \int_1^\infty \frac{x - [x]}{x^{\sigma + 1}} dx$$
and equality could occur when $s = \sigma$ is real.
%Note that
%$$\frac{1}{2} \int_1^\infty \frac{1}{x^{\sigma + 1}} dx.$$ % = \frac{1}{2} \left. \frac{x^{-\sigma}}{-\sigma} \right|_1^\infty.$$

We have
$$\frac{1}{\sigma} = \int_1^\infty \frac{1}{x^{\sigma + 1}} dx = \int_1^\infty \frac{x - [x]}{x^{\sigma + 1}} dx + \int_1^\infty \frac{1 - (x - [x])}{x^{\sigma + 1}} dx$$
and our intuition expects\footnote{The numerators in both integrals goes through the same values 0 to 1, but one covariates and the other contravariates with the denominators. If $a_1 < a_2 < ... < a_n$ and $b_1 < b_2 < ... < b_n$ then we have $\sum a_i b_i > \sum a_i b_{n-i}$. (Covariate gives larger sum product than contravariate.)}
$$\int_1^\infty \frac{x - [x]}{x^{\sigma + 1}} dx \leq \int_1^\infty \frac{1 - (x - [x])}{x^{\sigma + 1}} dx.$$

To formalize the idea, rewrite the integrals as infinite sums
\begin{align*}
\int_1^\infty \frac{x - [x]}{x^{\sigma + 1}} dx &= \sum_{n=1}^{\infty} \int_n^{n+1} \frac{x - n}{x^{\sigma + 1}} dx\\
&= \sum_{n=1}^{\infty} \int_0^1 \frac{t}{(n + t)^{\sigma + 1}} dt &\text{ change } t = x - n\\
\int_1^\infty \frac{1 - (x - [x])}{x^{\sigma + 1}} dx &= \sum_{n=1}^{\infty} \int_n^{n+1} \frac{n + 1 - x}{x^{\sigma + 1}} dx\\
&= \sum_{n=1}^{\infty} \int_0^1 \frac{t}{(n+1-t)^{\sigma + 1}} dt &\text{change } t = n + 1 - x
\end{align*}
and it is apparent that for any $n$:
$$\int_0^1 \frac{t}{(n + t)^{\sigma + 1}} dt \leq \int_0^1 \frac{t}{(n+1-t)^{\sigma + 1}} dt$$
because
\begin{align*}
&\int_0^1 \left( \frac{1}{(n+1-t)^{\sigma + 1}} - \frac{1}{(n + t)^{\sigma + 1}} \right) t \; dt\\
=& \int_0^{1/2} \left( \frac{1}{(n+1-t)^{\sigma + 1}} - \frac{1}{(n + t)^{\sigma + 1}} \right) t \; dt + \int_{1/2}^1 \left( \frac{1}{(n+1-t)^{\sigma + 1}} - \frac{1}{(n + t)^{\sigma + 1}} \right) t \; dt\\
=& \int_0^{1/2} \left( \frac{1}{(n+1-t)^{\sigma + 1}} - \frac{1}{(n + t)^{\sigma + 1}} \right) t \; dt + \int_{1/2}^0 \left( \frac{1}{(n+u)^{\sigma + 1}} - \frac{1}{(n + 1 - u)^{\sigma + 1}} \right) (1 - u) \; (-du) &\text{change } u = 1 - t\\
=& \int_0^{1/2} \left( \frac{1}{(n+1-t)^{\sigma + 1}} - \frac{1}{(n + t)^{\sigma + 1}} \right) t \; dt + \int_0^{1/2} \left( \frac{1}{(n+t)^{\sigma + 1}} - \frac{1}{(n + 1 - t)^{\sigma + 1}} \right) (1 - t) \; dt\\
=& \int_0^{1/2} \left( \frac{1}{(n+t)^{\sigma + 1}} - \frac{1}{(n + 1 - t)^{\sigma + 1}} \right) (1 - 2 t) \; dt\\
\geq& 0
\end{align*}
since the integrand
$$\left( \frac{1}{(n+t)^{\sigma + 1}} - \frac{1}{(n + 1 - t)^{\sigma + 1}} \right) (1 - 2 t) \geq 0$$
for all $t \in [0, 1/2]$ given $\sigma > 0$.

Now we show that $\zeta(s) \not = 0$ for $s$ in the given set. Suppose that $\zeta(s) = 0$ for some $s$ in that set. Then the inequality implies that
$$\Abs{\frac{1}{s - 1}} < \frac{1}{2\sigma}$$
which further implies
$$2\sigma < |s - 1|$$
so if we write $s = \sigma + i t$ then
$$4\sigma^2 < (\sigma - 1)^2 + t^2$$
so
$$t^2 > 3\sigma^2 + 2\sigma - 1 %= 3 (\sigma^2 + 2/3\sigma - 1/3)
= 3 \left(\sigma + \frac13\right)^2 - 4/3 \geq 3 \left(\frac12 + \frac13\right)^2 - \frac43 = \frac34.$$
But this contradicts the assumption that $|t| = |\Im s| \leq \sqrt 3 / 2$.
\end{proof}

\unless\ifdefined\IsMainDocument
\end{document}
\fi

\unless\ifdefined\IsMainDocument
\documentclass[12pt]{article}
\usepackage{amsmath,amsthm,amssymb}
\newcommand{\Fhat}{\widehat{F}}
\newcommand{\Abs}[1]{\left| #1 \right|}
\newcommand{\Z}{\mathbb{Z}}
\newcommand{\C}{\mathbb{C}}
\begin{document}
\fi

\textbf{Problem 6.22}: Let $\Fhat(s) = \sum a_n n^{-s}$ be a D.s. with $\sigma_c(\Fhat) < \infty$. Let $N \in \Z^+$ and let $s_1, s_2, ...$ be a sequence from $\C$ with $\Re s_j \rightarrow +\infty$ as $j \rightarrow \infty$. Show that
$$\lim_{j \rightarrow \infty} N^{s_j} \sum_{n = N}^{\infty} a_n n^{-s_j} = a_N.$$
Use this relation prove Theorem 6.25.

\begin{proof}
Note that
$$N^{s_j} \sum_{n = N}^{\infty} a_n n^{-s_j} = \sum_{n = N}^{\infty} a_n (N/n)^{s_j}$$
so if we have uniform convergence, we could then swap the limit and the series
$$\lim_{j \rightarrow \infty} \sum_{n = N}^{\infty} a_n (N/n)^{s_j} = \sum_{n = N}^{\infty} \lim_{j \rightarrow \infty} a_n (N/n)^{s_j} = a_N$$
for if $n > N$ then $N/n < 1$ and $(N/n)^{u} \rightarrow 0$ as $\Re u \rightarrow \infty$ so the only term that survives is the one for $n = N$. Without uniform convergence, we prove it directly from the definition. Recall $\sigma_a < \infty$ so we can fix an $b > \sigma_a$. Then for all $j$ such that $\sigma_j = \Re s_j > b$, we have
\begin{align*}
\Abs{ N^{s_j} \sum_{n = N}^{\infty} a_n n^{-s_j} - a_N } &= \Abs{ \sum_{n = N + 1}^{\infty} a_n \left(\frac{N}{n}\right)^{s_j} }\\
&\leq \sum_{n = N + 1}^{\infty} |a_n| \left(\frac{N}{n}\right)^{\sigma_j}\\
&= N^b \sum_{n = N + 1}^{\infty} |a_n| n^{-b} \left(\frac{N}{n}\right)^{\sigma_j - b}\\
&\leq N^b \sum_{n = N + 1}^{\infty} |a_n| n^{-b} \left(\frac{N}{N + 1}\right)^{\sigma_j - b} &\text{ for } \frac{N}{n} \leq \frac{N}{N+1}\\
&\leq \underbrace{N^b \left( \sum_{n = N + 1}^{\infty} |a_n| n^{-b} \right)}_{\text{constant}} \left(\frac{N}{N + 1}\right)^{\sigma_j - b}\\
&\rightarrow 0 \text{ as } j \rightarrow \infty
\end{align*}
because $\sigma_j \rightarrow \infty$.

Now to derive Theorem 6.25, first we note that we could assume the existence of $s_j$ such that $\Re s_j \rightarrow \infty$. For the other assumption where $s_j$ has a limit point in a common half plane, the two series are identical there and so we can simply pick another sequence.

Second, subtracting $f$ by $g$, we can assume $g = 0$ without loss of generality. In other words, it suffices to show that if there exists a sequence $s_j$ with $\Re s_j \rightarrow \infty$ and that $\Fhat(s_j) = 0$ for all $j$ then $f = 0$. We do this by induction:
\begin{itemize}
\item $f(1) = 0$: Follows from the limit formula for $N = 1$:
$$f(1) = \lim_{j \rightarrow \infty} \Fhat(s_j) = 0.$$
\item Induction: Assume that $f(1) = f(2) = ... = f(n - 1) = 0$. We show that $f(n) = 0$. By the limit formula for $N = n$:
\begin{align*}
f(n) &= \lim_{j \rightarrow \infty} n^{-s_j} \sum_{k = n}^{\infty} a_k k^{-s_j} \\
&= \lim_{j \rightarrow \infty} n^{-s_j} \sum_{k = 1}^{\infty} a_k k^{-s_j} &\text{by induction hypothesis}\\
&= \lim_{j \rightarrow \infty} n^{-s_j} \Fhat(s_j)\\
&= 0.
\end{align*}
\end{itemize}
So $f(n) = 0$ for all $n$.
\end{proof}

\textbf{Problem 6.23}: Let $\Fhat$ be a nonconstant D.s. with $\sigma_c(\Fhat) < \infty$ and let $\alpha$ be any fixed complex number. Show that there exists a half plane $\{s : \sigma > \sigma_0(\Fhat, \alpha) \}$ on which $\Fhat(s) \not= \alpha$.

\begin{proof}
Without loss of generality, we could assume $\alpha = 0$; otherwise, consider the Dirichlet series $\Fhat(s) - \alpha$. Assume that there is no such half plane, or equivalently $\sigma_0(\Fhat, 0)$. Then we could find a sequence $s_j \rightarrow \infty$ such that $\Fhat(s_j) = 0$. But then $\Fhat(s) = 0$ by Theorem 6.25 (or the previous problem), a contradiction.
\end{proof}

\unless\ifdefined\IsMainDocument
\end{document}
\fi

\unless\ifdefined\IsMainDocument
\documentclass[12pt]{article}
\usepackage{amsmath,amsthm,amssymb}
\newcommand{\Fhat}{\widehat{F}}
\newcommand{\A}{\mathcal{A}}
\newcommand{\Abs}[1]{\left| #1 \right|}
\begin{document}
\fi

\textbf{Problem 6.24}: Suppose $\varphi \in \A$, $\varphi(1) = 0$ and $f = \exp \varphi$. Show that if one of $f, \varphi$ is of polynomial growth then so is the other. Use these relations to give another proof of Lemma 6.30.

\begin{proof}
Let us show first that
\begin{enumerate}
\item The integral
\begin{align*}
\int_1^x \frac{x}{t} \log^{k-1}\left(\frac{x}{t}\right) \; dt
=& x \int_1^x (\log x - \log t)^{k-1} \; d(\log t)\\
=& x \int_0^{\log x} (\log x - u)^{k-1} \; du\\
=& x \left. \frac{-(\log x - u)^{k}}{k} \right|_{u=0}^{u=\log x}\\
=& \frac{x \log^k x}{k}.
\end{align*}

\item Define the function
$$s_k(n) = \sum_{d_1 ... d_k = n, \; d_i > 1 \; \forall i} 1$$ which counts the number of ways to write $n = d_1 ... d_k$ with $d_i > 1$ for all $i$. It is easy to see that
$$s_k = (1 - e_1)^{*k}.$$

Let $S_k$ be the summatory function of $s_k$. Then for any $k \geq 1$ and any $x \geq 1$,
$$S_k(x) \leq \frac{x \log^{k-1} x}{(k-1)!} =: f_k(x).$$

We prove this by induction.

\textbf{The base case} $k = 1$ is clear: $S_1(x) = N(x) - 1 \leq x = f_1(x)$.

\textbf{Induction}: Suppose that the inequality holds for $k$. We want to show that it holds for $k + 1$. One has
\begin{align*}
S_{k+1}(x) &= \int_{1^-}^x d S_k * (d N - \delta_1) &\text{from } s_{k+1} = s_k * (1 - e_1)\\
&= \int_{1^-}^x S_k(x/t) \; d (N - \delta_1) \\
&\leq \int_{1^-}^x S_k(x/t) \; dt &\text{by Lemma 3.11} \\
&\leq \int_{1^-}^x f_k(x/t) \; dt &\text{by induction hypothesis}\\
&= \int_1^x f_k(x/t) \; dt\\
&= f_{k+1}(x) &\text{as evaluated above}
\end{align*}
Note that the function $t \mapsto S_k(x/t)$ is nonnegative decreasing since $S_k$ is clearly increasing and $N(t) - 1 \leq t - 1$. Recall that $dt = d(t - 1)$ is continuous at 1 so we can replace $1^-$ by $1$.

\item Suppose that $g(1) = 0$ and $|g(n)| \leq A n^\alpha$ for some constants $A$ and $\alpha$. For any $k \geq 1$, we have
\begin{align*}
\Abs{ g^{*k}(n) } &= \Abs{ \sum_{d_1 ... d_k = n, \; d_i > 1 \; \forall i} g(d_1) \cdots g(d_k) } &\text{ since } g(1) = 0\\
&\leq \sum_{d_1 ... d_k = n, \; d_i > 1 \; \forall i} (Ad_1^\alpha \cdots Ad_k^\alpha)\\
&= \sum_{d_1 ... d_k = n, \; d_i > 1 \; \forall i} A^k n^\alpha\\
&= A^k n^\alpha s_k(n)\\
&\leq A^k n^\alpha S_k(n)\\
&\leq A^k n^\alpha \frac{n \log^{k-1} n}{(k-1)!}\\
&= \frac{A^k n^{\alpha + 1} \log^{k-1} n}{(k-1)!}.
\end{align*}
\end{enumerate}

Now we are ready to solve the main problem.
\begin{itemize}
\item Suppose that $\varphi$ is of polynomial growth. So $|\varphi(n)| \leq A n^\alpha$ for some $A$ and $\alpha$. Then for any $n \geq 3$, we have
\begin{align*}
\Abs{ f(n) } &= \Abs{ \sum_{k = 0}^{\infty} \frac{1}{k!} \varphi^{*k}(n) }\\
&\leq \sum_{k = 0}^{\infty} \frac{A^k n^{\alpha + 1} \log^{k-1} n}{k! (k-1)!} &\text{ by above inequality for } g = \varphi\\
&= n^{\alpha + 1} \sum_{k = 0}^{\infty} \frac{A^k \log^k n}{k!} &\text{ for } n \geq 3 \text{ we have } \log n \geq 1\\
&= n^{\alpha + 1} e^{A \log n}\\
&= n^{\alpha + A + 1}
\end{align*}
so $f$ is of polynomial growth.

\item For the other direction, suppose that $f$ is of polynomial growth so we can find $A$ and $\alpha$ such that $|f(n)| \leq A n^{\alpha}$. Note that $f - e_1$ satisfies the same inequality and vanishes at 0. For any $n \geq 3$:
\begin{align*}
\Abs{\varphi(n)} &= \Abs{ \sum_{k = 1}^{\infty} \frac{(-1)^{k-1}}{k} (f - e)^{*k} (n)}\\
&\leq \sum_{k = 1}^{\infty} \frac{1}{k} \frac{A^k n^{\alpha + 1} \log^{k-1} n}{(k-1)!} &\text{ by above inequality for } g = f - e_1\\
&\leq n^{\alpha + 1} \sum_{k = 0}^{\infty} \frac{A^k \log^k n}{k!}\\
&= n^{\alpha + A + 1}
\end{align*}
so $\varphi$ is of polynomial growth.
\end{itemize}

Finally, to derive Lemma 6.30, by assumption $f$ of polynomial growth, we have $\varphi = \log f$ is of polynomial growth. So is $-\varphi$ and subsequently $f^{*-1} = \exp(-\varphi)$ is of polynomial growth.
\end{proof}

\unless\ifdefined\IsMainDocument
\end{document}
\fi

\unless\ifdefined\IsMainDocument
\documentclass[12pt]{article}
\usepackage{amsmath,amsthm,amssymb}
\newcommand{\Fhat}{\widehat{F}}
\newcommand{\A}{\mathcal{A}}
\newcommand{\Abs}[1]{\left| #1 \right|}
\begin{document}
\fi

\textbf{Problem 6.25}: Suppose $f \in \A$, $f(1) = 1$ and $|f| \leq 1$. Use the equation $f * f^{*-1} = e_1$ to prove that $|f^{*-1}(n)| \leq n^2$ for all $n$. This result can be used to give still another proof of Lemma 6.30.

\begin{proof}
To simplify our notation, let $g = f^{*-1}$. To get our feeling, consider the case $n = p$ is prime, the equation $f * g = e_1$ gives
$$|g(p)| = |-f(p)| \leq 1 \leq p^2$$
and by induction, if $n = p^k$ is a prime power we have
$$|g(p^k)| = \Abs{ -\sum_{j = 0}^{k-1} f(p^{k-j}) g(p^j) } \leq \sum_{j = 0}^{k - 1} |g(p^j)| \leq \sum_{j = 0}^{k - 1} p^{2j} = \frac{p^{2k}-1}{p^2 - 1} \leq p^{2k}.$$
So the inequality is quite loose.

We prove that $|g(n)| \leq n^2$ by induction. Assume that this inequality is true for all $n < m$ and we prove that it is true for $n = m$. From $f * g = e_1$, we obtain
\begin{align*}
|g(m)| &= \Abs{ -\sum_{d | m, d < m} g(d) f(n/d) }\\
&\leq \sum_{d | m, d < m} |g(d)| &\text{by assumption } |f| \leq 1\\
&\leq \sum_{d | m, d < m} d^2 &\text{by induction hypothesis}\\
&= \left( \sum_{d | m} d^2 \right) - m^2\\
&= (T^2 1 * 1)(m) - m^2 &\text{note that } (T^2 1)(n) = n^2
\end{align*}

As both $T^2 1$ and 1 are multiplicative, $T^2 1 * 1$ is also multiplicative and we therefore have
\begin{align*}
(T^2 \; 1 * 1)(m) &= \prod_{p^k || m} (T^2 \; 1 * 1)(p^k)\\
&= \prod_{p^k || m} \left( \sum_{j = 0}^{k} p^{2j} \right)\\
&= \prod_{p^k || m} \frac{p^{2k+2} - 1}{p^2 - 1}\\
&= \prod_{p^k || m} p^{2k} \frac{p^{2k+2} - 1}{p^{2k}(p^2 - 1)}\\
&= m^2 \prod_{p^k || m} \left( 1 + \frac{1 - p^{-2k}}{p^2 - 1} \right)\\
&\leq m^2 \prod_{p^k || m} \left( 1 + \frac{1}{p^2 - 1} \right)\\
&\leq m^2 \underbrace{\prod_p \left( 1 + \frac{1}{p^2 - 1} \right)}_{1.64... < 2}
\end{align*}
So it is clear that $|g(m)| \leq m^2$.

To see how this implies Lemma 6.30, let us assume that $f$ is of polynomial growth. Then $|f(n)| \leq K n^\alpha$ for some constants $K, \alpha > 0$. We can assume $K = |f(1)|$ by increasing $\alpha$: If $|f(1)| < K$ then we can simply take $K = |f(1)|$; otherwise, we replace $\alpha$ by $\alpha + b$ for any $b > 0$ such that $2^b > \frac{K}{|f(1)|} > 1$.

Consider the function $g = \frac{1}{f(1)} T^{-\alpha} f$ so $|g| \leq 1$ and $g(1) = 1$. From $f = T^{\alpha} K g$, we get
$$e_1 = T^{\alpha} \underbrace{(K g * K^{-1} g^{*-1})}_{e_1} = \underbrace{(T^{\alpha} Kg)}_{f} * (T^{\alpha} K^{-1}g^{*-1})$$
and so
$$f^{*-1} = T^{\alpha} K^{-1} g^{*-1}$$
is of polynomial growth as long as $g^{*-1}$ is and that is the case here thank to the problem.
\end{proof}

\unless\ifdefined\IsMainDocument
\end{document}
\fi

\unless\ifdefined\IsMainDocument
\documentclass[12pt]{article}
\usepackage{amsmath,amsthm,amssymb}
\newcommand{\Fhat}{\widehat{F}}
\newcommand{\A}{\mathcal{A}}
\newcommand{\Abs}[1]{\left| #1 \right|}
\begin{document}
\fi

\textbf{Problem 6.26}: Let $f = \prod_{n=2}^{\infty} (e_1 - e_n)^{*-1}$ and let $F$ be the summatory function of $f$. Show that if $n \geq 2$ then $f(n)$ equals the number of representations of $n$ as a product $k_1 k_2 ... k_r$, where $r \geq 1$ and $2 \leq k_1 \leq k_2 \leq ... \leq k_r$. (For example $12 = 2 \cdot 6 = 3 \cdot 4 = 2 \cdot 2 \cdot 3$ and $f(12) = 4$.) Show that $\Fhat(s) = \prod_{n=2}^{\infty} (1 - n^{-s})^{-1}$ and $\sigma_c(\Fhat) = 1$. Conclude that $F(x) = O(x^{1 + \epsilon})$ for any positive number $\epsilon$.

\begin{proof}
\begin{enumerate}
\item  For any $N \geq 2$, let
$$f_N := \prod_{n=2}^{N} (e_1 - e_n)^{*-1}$$
and we show that $f_N(n)$ equals the number of representations of $n$ as a product $k_1 k_2 ... k_r$ where $r \geq 1$ and $2 \leq k_1 \leq k_2 \leq ... \leq k_r \leq N$. Then it is clear that $f(n) = f_N(n)$ for any $N \geq n$ since in any such factorization $n = k_1 ... k_r$, we must have $k_i \leq n$ as they are divisors of $n$ and so $f(n)$ has the prescribed description in the problem.

For $j \geq 2$, we have
$$(e_1 - e_j)^{*-1} = \sum_{m = 0}^{\infty} e_{j^m} = e_1 + e_j + e_{j^2} + ...$$
so $(e_1 - e_j)(d) = 1$ only when $d$ is a power of $j$ and is zero otherwise. Therefore,
\begin{align*}
f_N(n) &= \sum_{n = d_2 ... d_N} \prod_{j = 2}^N (e_1 - e_j)^{*-1}(d_j)\\
&= \sum_{n = 2^{\alpha_2} ... N^{\alpha_N}} 1
\end{align*}
counts the number of ways to write $n$ as the product $n = 2^{\alpha_2} ... N^{\alpha_N}$. Clearly, each such representation gives us a factorization $n = k_1 ... k_r$ where $r = \alpha_2 + ... + \alpha_N$ given by $k_1 = \cdots = k_{\alpha_2} = 2 < k_{\alpha_2 + 1} = \cdots = k_{\alpha_2 + \alpha_3} = \alpha_3 < \cdots$. In other words, the first $\alpha_2$ numbers are 2, the next $\alpha_3$ numbers are 3 and so on. Conversely, each factorization $n = k_1 ... k_r$ satisfying the conditions yields a representation $n = 2^{\alpha_2} ... N^{\alpha_N}$ where $\alpha_j$ is the number of indices $t$ where $k_t = j$. In our example, $12 = 2 \cdot 2 \cdot 3$ corresponds to $12 = 2^2 \cdot 3^1$.

\item In problem 6.24, we showed that for any fixed $r$,
$$\sum_{n = k_1 k_2 ... k_r, \; k_i \geq 2 \; \forall i} 1 \leq \frac{n \log^{r-1} n}{(r-1)!}$$
so for any $n$, we have
\begin{align*}
f(n) &= \sum_{r = 1}^{\infty} \sum_{n = k_1 k_2 ... k_r, \; 2 \leq k_1 \leq k_2 \leq ... \leq k_r} 1\\
&\leq \sum_{r = 1}^{\infty} \sum_{n = k_1 k_2 ... k_r, 2 \leq k_i \; \forall i} 1\\
&\leq \sum_{r = 1}^{\infty} \frac{n \log^{r-1} n}{(r-1)!}\\
&= n e^{\log n}\\
&= n^2
\end{align*}
which implies the summatory function $F(x)$ is $O(x^3)$. So $\Fhat(s)$ converges absolutely for all $s$ with $\Re s > 3$. So $\sigma_a(\Fhat) < +\infty$.

\item The key idea in the infinite product for $\Fhat$ is the fact that
$$\Fhat_N(s) = \prod_{n = 2}^{N} (1 - n^{-s})^{-1} \qquad (\Re s > 0)$$
by Theorem 6.2 so we expect $\Fhat(s) = \lim_{N \rightarrow \infty} \Fhat_N(s)$ since $f_N \rightarrow f$. Note that the Dirichlet series for $(e_1 - e_j)^{*-1}$ is $\sum_{m=1}^{\infty} j^{-m s}$ converges absolutely for $\Re s > 0$.

To actually prove it, at least for $\Re s > \sigma_a(\Fhat)$ where $\Fhat(s)$ converges absolutely, we imitate Lemma 6.4 and consider the difference
\begin{align*}
\Fhat(s) - \Fhat_N(s) &= \sum_{n=1}^{\infty} (f(n) - f_N(n)) \; n^{-s} \\
&= \sum_{n=N + 1}^{\infty} (f(n) - f_N(n)) \; n^{-s}
\end{align*}
since the difference $f(n) - f_N(n) = 0$ if $n \leq N$ and it counts the number of factorizations $n = k_1 ... k_r$ with $2 \leq k_1 \leq \cdots \leq k_r$ but $k_j > N$ for some $j$ (which implies $k_r > N$). The last series is the tail of an absolutely convergent series 
$$\Abs{ \sum_{n=N + 1}^{\infty} (f(n) - f_N(n)) \; n^{-s} } \leq \sum_{n = N + 1}^{\infty} f(n) \; n^{-\sigma}$$
so it goes to 0. Thus, fix $s$ with $\Re s > \sigma_a$ then for any $\epsilon > 0$, we simply choose $N$ large enough so that
$$\sum_{n = N + 1}^{\infty} f(n) \; n^{-\sigma} < \epsilon$$
by absolute convergence and that will ensure $|\Fhat(s) - \Fhat_m(s)| < \epsilon$ whenever $m \geq N$. This proves $\Fhat(s) = \lim_{N \rightarrow \infty} \Fhat_N(s)$ and thus, we get the product representation $\Re s > \sigma_a$.

\item To prove that $\sigma_c(\Fhat) = 1$, we first note that $\Fhat$ cannot converge at $s = 1$ because
$$\prod_{n = 2}^{\infty} (1 - n^{-s})^{-1} = \zeta(s) \prod_{\text{non-prime } n = 2}^{\infty} (1 - n^{-s})^{-1}$$
has a pole at 1. Note that each factor $(1 - n^{-s})^{-1} > 1$ if $s = \sigma > 0$. By Problem 6.18, $\Fhat$ does not converge at any point on the line $\sigma = 1$ and so $\sigma_c \geq 1$.

For any $\sigma > 1$, we consider the logarithm of the product
$$0 \leq -\sum_{n = 2}^{\infty} \log(1 - n^{-\sigma}) \leq \sum_{n = 2}^{\infty} \frac{1}{n^\sigma - 1}$$
where as exploited in Example 6.8, one has
$$0 > \log(1 - n^{-\sigma}) = \log\left(\frac{n^\sigma - 1}{n^\sigma}\right) = - \log\left(1 + \frac{1}{n^\sigma - 1}\right) \geq -\frac{1}{n^\sigma - 1}$$
for $n > 2$ and $\sigma > 1$ so
$$0 < -\log(1 - n^{-\sigma}) \leq \frac{1}{n^\sigma - 1} \leq \frac{1}{(n-1)^{\sigma}}$$
because $0 < (n-1)^{\sigma} \leq n^\sigma - 1 \iff 1 \leq n^\sigma - (n-1)^\sigma$ which is true by Mean Value Theorem: $n^\sigma - (n-1)^\sigma = \sigma \xi^{\sigma - 1} \geq 1$ for some $\xi \in (n-1, n)$ and $\sigma \xi^{\sigma - 1} \geq 1$ as $\xi \geq n - 1 \geq 2 - 1 = 1$ and $\sigma > 1$.

The last series converges absolutely which implies convergence of
$$-\sum_{n = 2}^{\infty} \log(1 - n^{-\sigma})$$
which estabishes convergence of $\Fhat(s)$ for $\Re s > 1$.
\end{enumerate}
\end{proof}

\unless\ifdefined\IsMainDocument
\end{document}
\fi

\unless\ifdefined\IsMainDocument
\documentclass[12pt]{article}
\usepackage{amsmath,amsthm,amssymb}
\newcommand{\Fhat}{\widehat{F}}
\begin{document}
\fi

\textbf{Problem 6.27}: Let $\Fhat(s) = \int_1^\infty x^{-s} (\log ex)^{-2} dx$. Observe that $\sigma_c = 1$, $F \uparrow$, and the integral defining $\Fhat(s)$ converges at $s = 1$. Is this consistent with Theorem 6.31? Explain.

\begin{proof}
Yes. The fact that the integral defining $\Fhat(s)$ converges at $s = 1$ does not imply $\Fhat(s)$ could be analytically continued to a region containing $s = 1$. In fact, it follows from Theorem 6.31 that $\Fhat(s)$ cannot extend beyond the line $\sigma = 1$.
\end{proof}

\unless\ifdefined\IsMainDocument
\end{document}
\fi


\chapter{Inversion Formulas}

\newcommand{\Ghat}{\widehat{G}}

\unless\ifdefined\IsMainDocument
\documentclass[12pt]{article}
\usepackage{amsmath,amsthm,amssymb}
\newcommand{\Fhat}{\widehat{F}}
\begin{document}
\fi

\section{Abelian vs Tauberian}

Going from $f$ to Dirichlet series is an abelian process: Think of assigning the weight of $n^{-s}$ to $f(n)$. When $s = 0$, the weights are all 1 and we have $\sum f(n) = \lim_{x \rightarrow \infty} F(x) = c$. That is the simple example on page 141.

For the ``converse'', we show that if $F$ is monotone and $\Fhat(\sigma) \rightarrow c$ as $\sigma \rightarrow 0^+$ then $F(x) \rightarrow c$ as $x \rightarrow \infty$.

The intuition is clear: Just like the abelian process, we think of
$$\Fhat(\sigma) = \sum f(n) n^{-\sigma}$$
as the weighted sum of $f(n)$ with decaying weights. For a fixed $\sigma$, the weight $n^{-\sigma}$ is close to 1 (in the sense of being within $\epsilon$ error) for the first $n < N_\sigma$ terms. So the partial sum $\sum_{n < N_\sigma} f(n) n^{-\sigma}$ is close to $F(N_\sigma)$. As $\sigma$ decreases to 0, $N_\sigma$ increases to infinity and $F(N_\sigma)$ is close to $c$ i.e. bounded by $c \pm \epsilon$. Since $F$ is increasing, the limit of the subsequence $F(N_\sigma)$ is the same as the limit $F(x)$ as $x \rightarrow \infty$.

To make things precise:
\begin{itemize}
\item We can assume $F$ is increasing without loss of generality since $\widehat{c F} = c \Fhat$ for any constant $c$. Then $F_v = F$.

\item It suffices to prove that $\lim_{x \rightarrow \infty} F(x)$ exists for then we can use the forward (abelian) direction to conclude that the limit is $c$.

\item Since $F$ is increasing, this can be achieved by proving that $F(x)$ is bounded.

\item Fix an $\epsilon \in (0, 1)$ and let $\delta > 0$ be such that $|\Fhat(\sigma) - c| < \epsilon$ for all $\sigma \in (0, \delta)$ by the definition of limit ($\delta$ depends on $\epsilon$). For any such $\sigma$, let $X_\sigma > 0$ be such that $0 < 1 - x^{-\sigma} < \epsilon$ for all $x \in (1, X_\sigma)$. (Despite the notation, $X_\sigma$ depends on both $\epsilon$ and $\sigma$.) Then
\begin{align*}
c + \epsilon > \Fhat(\sigma) &= \int_{1^-}^{X_\sigma} x^{-\sigma} dF + \int_{X_\sigma}^\infty x^{-\sigma} dF\\
&\geq \int_{1^-}^{X_\sigma} (1 - \epsilon) dF + \underbrace{\int_{X_\sigma}^\infty x^{-\sigma} dF_v}_{\geq 0} &\text{as } F \uparrow\\
&\geq (1 - \epsilon) F(X_\sigma)
\end{align*}
so we have
$$F(X_\sigma) \leq \frac{c + \epsilon}{1 - \epsilon}.$$

\item It is easy to work out that
\begin{align*}
1 - x^{-\sigma} < \epsilon &\iff 1 - \epsilon < x^{-\sigma}\\
&\iff \log(1 - \epsilon) < -\sigma \log x\\
&\iff -\log(1-\epsilon) > \sigma \log x\\
&\iff x < (1 - \epsilon)^{-1/\sigma}
\end{align*}
so $X_\sigma = (1 - \epsilon)^{-1/\sigma} = \left( \frac{1}{1 - \epsilon} \right)^{1/\sigma} \rightarrow \infty$ as $\sigma \rightarrow 0^+$.

\item To complete the proof that $F$ is bounded, pick our favourite $\epsilon$, say $\epsilon = 1/2$ and we show that
$$F(x) \leq \frac{c + \epsilon}{1 - \epsilon} = 2 c + 1$$
for all $x$. Let $x$ be arbitrary and let $\sigma \in (0, \delta)$ be large enough so that $X_\sigma = 2^{1/\sigma} > x$ (this is possible since $1/\sigma \rightarrow +\infty$ as $\sigma \rightarrow 0^+$). Then the above argument shows that $F(X_\sigma) \leq 2c + 1$ and since $F$ is increasing $F(x) \leq F(X_\sigma) \leq 2c + 1$.

Of course, this argument could be used to show that $F(x) \leq \frac{c + \epsilon}{1 - \epsilon}$ for all $x$ and all $\epsilon \in (0, 1)$. So it is true that $F(x) \leq c$.
\end{itemize}

\unless\ifdefined\IsMainDocument
\end{document}
\fi


\section{Problems}

\unless\ifdefined\IsMainDocument
\documentclass[12pt]{article}
\usepackage{amsmath,amsthm,amssymb}
\newcommand{\Fhat}{\widehat{F}}
\begin{document}
\fi

\textbf{Problem 7.1}: Give an example of a function $F$ for which the hypotheses of the Wiener-Ikehara theorem hold with $\sigma_c(\Fhat) = 1$ and $L = 0$. Verify that $F(x) = o(x)$.

\begin{proof}
We need $F$ real valued monotone nondecreasing and $\Fhat(s)$ extends to a continuous function on closed half plane $\{\sigma \geq 1\}$. The latter condition is the same as saying that the limit $\lim_{s \rightarrow s_0} \Fhat(s)$ exists for all $s_0$ on the line of convergence. We could use
$$F(x) = \int_{1^-}^x \frac{dx}{\log^2 ex} \sim \frac{x }{\log^2 ex}$$
in Problem 6.12 (c). We showed there that $\Fhat$ converges at every point on its line of convergence, hence continuous there by Theorem 6.15. Evidently $F(x) = o(x)$.
\end{proof}

\unless\ifdefined\IsMainDocument
\end{document}
\fi

\unless\ifdefined\IsMainDocument
\documentclass[12pt]{article}
\usepackage{amsmath,amsthm,amssymb,hyperref}

\begin{document}
\fi

\textbf{Problem 7.2}: Show that $\int_{-\infty}^{\infty} K_1(x) dx = \pi$ or obtain upper and lower numerical bounds for the integral.

\begin{proof}
We shall drop the subscript in $K_1$ as there can be no confusion. Recall that
$$K(x) = \left(\frac{\sin x}{x}\right)^2$$
by Lemma 7.4. Since $K$ is an even function, we have %(Also recall that $\cos 2x = 1 - 2 \sin^2 x$.)
$$\int_{-\infty}^{\infty} K(x) dx = 2 \int_0^{\infty} K(x) dx.$$

An integration by parts yields
\begin{align*}
\int_0^{\infty} K(x) dx &= \underbrace{\left. \frac{\sin^2 x}{x} \right|_0^{\infty}}_{0} - \int_0^{\infty} x \cdot \frac{(2 \sin x \cos x) \cdot x^2 - (\sin^2 x) \cdot 2 x}{x^4} dx \\
% &= - \int_{-\infty}^{\infty} \left( \frac{(2 \sin x \cos x) x - (\sin^2 x) 2}{x^2} \right) dx \\
&= - \int_0^{\infty} \left( \frac{\sin 2x}{x} - 2 K(x) \right) dx \\
&= 2 \int_0^{\infty} K(x) dx - \int_0^{\infty} \frac{\sin 2x}{x} dx 
\end{align*}
Here we recall that $\lim_{x \rightarrow 0} \frac{\sin x}{x} = \lim_{x \rightarrow 0} \frac{\cos x}{1} = 1$ by L'Hospital rule and so the result of evaluating $\frac{\sin^2 x}{x}$ at 0 is 0; as claimed in the first line. So
$$\int_0^{\infty} K(x) dx = \int_0^{\infty} \frac{\sin 2x}{x} dx = \int_0^{\infty} \frac{\sin x}{x} dx$$
by a simple change of variable $u = 2x$.

It would be easier to consider
$$\int_{-\infty}^{\infty} \frac{\sin x}{x} dx = 2 \int_0^{\infty} \frac{\sin x}{x} dx$$
because for any $R$, we have
$$\int_{-R}^{R} \frac{\cos x}{x} dx = 0$$
thank to $\frac{\cos x}{x}$ being an odd function. This leads to
$$\int_{-\infty}^{\infty} \frac{e^{ix}}{x} dx = \int_{-\infty}^{\infty} \frac{\cos x + i \sin x}{x} dx = i \int_{-\infty}^{\infty} \frac{\sin x}{x} dx$$
and the integrand $\frac{e^{ix}}{x}$ looks like something we can apply the residue theorem: Consider the path $C_{X_1, X_2, Y}$ made up of a large rectangle with vertices $-X_1, -X_1 + iY, X_2 + iY, X_2$ where $X_1, X_2, Y > 0$ with the segment $[-\delta, \delta]$ on the real axis replaced by a semicircle $S_\delta$ of radius $\delta$ on the lower half plane to avoid the singularity $0$, we find from residue theorem that
$$\int_{C_{X_1, X_2, Y}} \frac{e^{iz}}{z} dz = 2 \pi i e^{i \cdot 0} = 2 \pi i$$
given the winding number of $C_{X_1, X_2, Y}$ around 0 is clearly 1.
\begin{itemize}
\item On the vertical line segment $L$ from $X$ to $X + iY$ with $Y > 0$ using parametrization $z(t) = X + it$ for $t \in [0, Y]$ we have
$$\int_{L} \frac{e^{iz}}{z} dz = \int_0^Y \frac{e^{i(X + it)}}{X + it} i dt = i e^{iX} \int_0^Y \frac{e^{-it}}{X + it} dt$$
which can be bounded in absolute value by $\int_0^Y \frac{1}{|X|} dt = |\frac{Y}{X}|$ and so made arbitrarily small if $Y = o(X)$.

\item On the horizontal line segment $L$ from $-X_1 + iY$ to $X_2 + iY$ with parametrization $z(t) = t + iY$, we have
$$\int_{L} \frac{e^{iz}}{z} dz = \int_{-X_1}^{X_2} \frac{e^{i(t + iY)}}{t + iY} dt = e^{-Y} \int_{-X_1}^{X_2} \frac{e^{it}}{t + iY} dt$$
Again, in terms of absolute value $|t + iY| \geq |Y|$ so the above can be bound in absolute value by
$$e^{-Y} \int_{-X_1}^{X_2} \frac{1}{|Y|} dt = \frac{X_2 + X_1}{|Y| e^Y}$$
which goes to 0 as long as $Y \rightarrow \infty$ but $X_2 + X_1$ goes slower than $e^Y$. Combining with the above, the choice of $X_1 = X_2 = Y^2$ should work.

\item For the small semicircle $S_\delta$ centered at $0$ and radius $\delta$ on the lower half plane with natural parametrization $z(t) = \delta e^{it}$ where $t \in [\pi, 2\pi]$ we find
$$\int_{S_\delta} \frac{e^{iz}}{z} dz = i \int_\pi^{2\pi} e^{i \delta e^{it}} dt$$
because $dz = i z dt$. As $\delta \rightarrow 0$, the integrand $e^{i \delta e^{it}} = e^{i \delta \cos t - \delta \sin t} = e^{-\delta \sin t} e^{i \delta \cos t} \rightarrow 1$ uniformly and so
$$\lim_{\delta \rightarrow 0} i \int_\pi^{2\pi} e^{i \delta e^{it}} dt = i \int_\pi^{2\pi} dt = \pi i.$$
\end{itemize}

Combining all these, we derive
$$\int_{-\infty}^{\infty} \frac{e^{ix}}{x} dx = 2 \lim_{Y \rightarrow +\infty, \delta \rightarrow 0^+} \int_\delta^{Y^2} \frac{e^{ix}}{x} dx = 2 \pi i - \pi i = \pi i$$
which implies
$$\int_{-\infty}^{\infty} \frac{\sin x}{x} dx = \pi.$$

\textbf{Remark}: Sometimes ago, I think I came up with a way to compute this integral without complex analysis while preparing my notes for the modular forms seminar. But I forget how that works. In any case, a quick search yields \href{https://people.math.harvard.edu/~ctm/home/text/others/hardy/sinx/sinx.pdf}{Hardy's pleasant article} on \textit{The Mathematical Gazette}.
\end{proof}

\unless\ifdefined\IsMainDocument
\end{document}
\fi

\unless\ifdefined\IsMainDocument
\documentclass[12pt]{article}
\usepackage{amsmath,amsthm,amssymb}
\newcommand{\Fhat}{\widehat{F}}
\begin{document}
\fi

\textbf{Problem 7.3}: By exploiting the behavior of $e^{it\log n}$ for large values of $n$, show that $|\zeta(\rho + it)| < 2$ for all real $t \not= 0$.

\begin{proof}
From the series for $\zeta(s)$, it is clear that
$$|\zeta(\rho + it)| \leq \sum_{n=1}^{\infty} |n^{-\rho-it}| = \zeta(\rho) = 2$$
and equality can only occurs when $n^{-i t} = 1$ for all $n$; equivalently, $n^{i t} = e^{i t \log n} = 1$. Since $e^{i t \log n} = 1 \iff \frac{t \log n}{2\pi}$ is an integer, we find
$$\frac{1}{2\pi} (t \log(n + 1) - t \log n) = \frac{t}{2\pi} \log \frac{n+1}{n}$$
is an integer for all $n$. When $n$ becomes very large, this sequence of integers approaches 0 and so must be constant 0 from some point onward\footnote{Use $\epsilon = 1/2$ in the definition of limit.}. But that implies
$$\frac{t \log n}{2\pi} = \frac{t \log (n + 1)}{2\pi}$$
for all $n$ sufficiently large and that cannot happen unless $t = 0$.
\end{proof}

\textbf{Problem 7.4}: Let $\varphi$ denote Euler's function. Prove that as $y \rightarrow \infty$,
$$\#\{n: \varphi(n) \leq y\} \sim \frac{\zeta(2)\zeta(3)}{\zeta(6)} y.$$

\begin{proof}
Recall from Problem 6.6 that if we put $f(j) = \#\{n : \varphi(n) = j\}$ then
$$\sum f(j) j^{-s} = \sum_{n=1}^{\infty} \varphi(n)^{-s} = \zeta(s) \psi(s)$$
valid for $\sigma > 1$ where
$$\psi(s) = \prod_p (1 + (p-1)^{-s} - p^{-s})$$
converges at every point on the line of convergence $\sigma = 1$. The summatory function of $f$ is evidently
$$F(y) = \#\{n: \varphi(n) \leq y\}$$
real valued monotone non-decreasing. It is clear from the last formula that $\sigma_c(\Fhat) = 1$. From Problem 6.7
$$\psi(1) = L = \frac{\zeta(2)\zeta(3)}{\zeta(6)}$$
together the fact that $\zeta(s)$ is analytic on $\sigma > 0$ with a simple pole at $s = 1$, we get
$$\Fhat(s) - \frac{L}{s - 1} = \left(\zeta(s) - \frac{1}{s-1}\right) \psi(s) + \frac{\psi(s) - L}{s - 1}$$
extends to a continuous function on $\{\sigma \geq 1\}$. Thus, the Wiener-Ikehara theorem implies our desired assymptotic formula.
\end{proof}

\unless\ifdefined\IsMainDocument
\end{document}
\fi

\unless\ifdefined\IsMainDocument
\documentclass[12pt]{article}
\usepackage{amsmath,amsthm,amssymb}
\newcommand{\V}{\mathcal{V}}
\newcommand{\Fhat}{\widehat{F}}
\newcommand{\cconj}[1]{\overline{#1}}
\begin{document}
\fi

\textbf{Problem 7.5}: Let $F \in \V$ be defined by
$$F(x) = \int_1^x \{1 - \cos(\lambda_0 \log u)\} du \qquad (x \geq 1)$$
for some $\lambda_0 > 0$. Find $\Fhat$, $\limsup_{x \rightarrow \infty} F(x) / x$, and $\liminf_{x \rightarrow \infty} F(x) / x$.

\begin{proof}
We can compute $F(x)$ by integration by parts but the smarter way is to observe that
$$u^{i \lambda_0} = e^{i \lambda_0 \log u} = \cos(\lambda_0 \log u) + i \sin(\lambda_0 \log u)$$
so
\begin{align*}
\int_1^x \{\cos(\lambda_0 \log u) + i \sin(\lambda_0 \log u)\} du &= \int_1^x u^{i\lambda_0} du\\
&= \left. \frac{u^{i \lambda_0 + 1}}{i \lambda_0 + 1} \right|_1^x\\
&= \frac{x^{i \lambda_0 + 1} - 1}{i \lambda_0 + 1}\\
&= \frac{(x^{i \lambda_0 + 1} - 1) (1 - i \lambda_0)}{1 + \lambda_0^2}\\
&= \frac{x e^{i (\lambda_0 \log x + \beta)} }{\sqrt{ 1 + \lambda_0^2 } } - \frac{1 - i \lambda_0}{1 + \lambda_0^2}
\end{align*}
where we have expressed the complex number
$$1 - i \lambda_0 = \sqrt{1 + \lambda_0^2} \; e^{i \beta}$$
in polar form. We can then extract $\int_1^x \cos(\lambda_0 \log u) du$ as the real part of the last expression
$$\int_1^x \cos(\lambda_0 \log u) du = \frac{x \cos (\lambda_0 \log x + \beta) }{\sqrt{ 1 + \lambda_0^2 } } - \frac{1}{1 + \lambda_0^2}$$
and so
$$\frac{F(x)}{x} = 1 - \frac{1}{x} - \frac{\cos(\beta + \lambda_0 \log x)}{\sqrt{\lambda_0^2 + 1}}  + \frac{1}{x(\lambda_0^2 + 1)}.$$
This formula gives $\limsup = 1 + \frac{1}{\sqrt{\lambda_0^2 + 1}}$ and $\liminf = 1 - \frac{1}{\sqrt{\lambda_0^2 + 1}}$ as $\cos(\beta + \lambda_0 \log x)$ infinitely oscillates between $-1$ and 1 when $x \rightarrow \infty$ given $\lambda_0 \not= 0$.

Now to compute $\Fhat(s)$, we use a similar idea
\begin{align*}
\cos(\lambda_0 \log x) &= \Re(e^{i \lambda_0 \log x})\\
&= \frac{1}{2}(e^{i \lambda_0 \log x} + e^{-i \lambda_0 \log x}) &\text{ as } \Re(z) = \frac{z + \cconj{z}}{2}\\
&= \frac{x^{i\lambda_0} + x^{-i\lambda_0}}{2} &\text{ and } \cconj{z} = z^{-1} \text{ if } |z| = 1
\end{align*}
to get
\begin{align*}
\Fhat(s) &= \int x^{-s} \{1 - \cos(\lambda_0 \log x)\} dx\\
&= \int x^{-s} \left\{1 - \frac{x^{i\lambda_0} + x^{-i\lambda_0}}{2} \right\} dx\\
&= \lim_{X \rightarrow \infty} \int_1^X \left\{x^{-s} - \frac12 x^{-s + i\lambda_0} - \frac12 x^{-s-i\lambda_0} \right\} dx\\
&= \lim_{X \rightarrow \infty} \left. \left\{\frac{x^{1-s}}{1 - s} - \frac{x^{1 - s + i\lambda_0}}{2(1 - s + i\lambda_0)}  - \frac{x^{1-s-i\lambda_0}}{2(1 - s - i\lambda_0)} \right\} \right|_1^X\\
&= \frac{1}{s - 1} - \frac{1}{2(s - i\lambda_0 - 1)}  - \frac{1}{2(s + i\lambda_0 - 1)} &\text{ for } \sigma > 1
\end{align*}
(Note that $\sigma_c(\Fhat) = 1$ because we found that the correct order of growth of $F(x) = O(x)$.)

This problem illustrates that when $\Fhat(s)$ satisfies the hypothesis of Wiener-Ikehara theorem only in the strip $|t| < \lambda_0$ and we can only conclude that $F(x) = O(x)$; in fact, the assymptotic behavior does not exist.
\end{proof}

\unless\ifdefined\IsMainDocument
\end{document}
\fi

\unless\ifdefined\IsMainDocument
\documentclass[12pt]{article}
\usepackage{amsmath,amsthm,amssymb}
\newcommand{\Fhat}{\widehat{F}}
\begin{document}
\fi

\textbf{Problem 7.6}: Show that $M(x) = o(x)$ by applying the Wiener-Ikehara theorem to
$$\zeta(s) + \zeta^{-1}(s) = \int x^{-s} \{dN(x) + dM(x)\}.$$

\begin{proof}
Note that we would love to use the Wiener-Ikehara theorem to $\zeta^{-1}(s) = \int x^{-s} dM$. But we do not know its absissca of convergence. Hence, we use $dN + dM$.

Knowing that $\zeta(s)$ has no zero on the line of convergence $\sigma = 1$, hence $\zeta(s) \not= 0$ on $\sigma \geq 1$, we can deduce that $\zeta^{-1}(s)$ is continuous on the region $\{\sigma \geq 1\}$. This implies $\zeta(s) + \zeta^{-1}(s)$ has a simple pole at $s = 1$ and so $\sigma_c = 1$. Therefore,
$$\varphi(s) := \zeta(s) + \zeta^{-1}(s) - \frac{1}{s - 1}$$
is continuous on $\{\sigma \geq 1\}$ from the behavior of $\zeta(s)$. To satisfy the Wiener-Ikehara theorem, we still need $N(x) + M(x)$ is monotone nondecreasing but that is evident because it is the summatory function of $1 + \mu \geq 0$.

Thus, we get
$$N(x) + M(x) = x + o(x)$$
by Wiener-Ikehara theorem which easily implies
$$M(x) = x - N(x) + o(x) = o(x).$$
\end{proof}

\textbf{Problem 7.7}: Let $dN^{*1/2}$ denote the positive convolution square root of $dN$, and recall that $dt^{*1/2} = L^{-1/2} dt / \Gamma(1/2)$ (c.f. Example 6.3 and 7.27). Prove that
$$\int_{1^-}^x dN^{*1/2} * L^{-1/2} dt / \Gamma(1/2) \sim x.$$

\begin{proof}
Let
\begin{align*}
F(x) &:= \int_{1^-}^x dN^{*1/2} * L^{-1/2} dt / \Gamma(1/2)\\
&= \pi^{-1/2} \int_{1^-}^x \underbrace{\left( \int_1^{x/t} \log^{-1/2} u \; du \right)}_{\geq 0} dN^{*1/2} &\text{ as } \Gamma(1/2) = \sqrt{\pi}
\end{align*}

Observe that the function $\int_{1^-}^x dN^{*1/2}$ is monotone nondecreasing because it is the summatory function of the arithmetic function $\exp(\kappa/2) \geq 0$: It is clear from the definition of convolution that for any arithmetic functions $f, g \geq 0$, we have $f * g \geq 0$ so by induction $f^{*n} \geq 0$ if $f \geq 0$. Thus, $\exp(\nu) = \sum \nu^{*n}/n! \geq 0$ if $\nu \geq 0$. This is how we know $\exp(\kappa/2) \geq 0$.

So $F(x)$ is monotone nondecreasing by the definition of integral or by Lemma 3.13. Next, for $\sigma > 1$, we have
\begin{align*}
\Fhat(s) &= \underbrace{\left(\int x^{-s} dN^{*1/2} \right)}_{\sqrt{\zeta(s)}} \underbrace{\left( \int x^{-s} L^{-1/2} dt / \Gamma(1/2) \right)}_{\frac{1}{\sqrt{s-1}}} &\text{ by Theorem 6.2}\\
&= \sqrt{\frac{1}{s-1} \zeta(s)}\\
&= \frac{1}{s-1} \sqrt{1 + (s - 1) \psi(s)}
\end{align*}
where $\psi(s) = \zeta(s) - \frac{1}{s - 1}$ is an analytic function on $\{\sigma > 0\}$. This formula implies that $\Fhat(s)$ does not converge at $s = 1$ and hence, $\sigma_c(\Fhat) = 1$. (We should be more careful about the branch of square-root here.) The formula also implies that $\Fhat(s) - \frac{1}{s - 1}$ is an analytic function on $\{\sigma \geq 0\}$.

The estimate $F(x) \sim x$ now follows immediately from the Wiener-Ikehara theorem.
\end{proof}

\unless\ifdefined\IsMainDocument
\end{document}
\fi

\unless\ifdefined\IsMainDocument
\documentclass[12pt]{article}
\usepackage{amsmath,amsthm,amssymb}
\newcommand{\Fhat}{\widehat{F}}
\newcommand{\Abs}[1]{\left| #1 \right|}
\newcommand{\cconj}[1]{\overline{#1}}
\begin{document}
\fi

\textbf{Problem 7.9}: Let $c$ be a fixed positive number, and define a completely multiplicative function $f$ by setting $f(p) = 0$ for all primes $p \leq c$ and $f(p) = c$ for all $p > c$. Show that
$$\sum_{n \leq x} f(n) \sim \prod_{p \leq c} (1-p^{-1})^c \prod_{p > c} \left\{ \frac{(1 - p^{-1})^c}{1 - c p^{-1}} \right\} \frac{x(\log x)^{c - 1}}{\Gamma(c)}.$$

\begin{proof}
Let $F(x)$ denote the summatory function of $f$.

\textbf{Short version}: By complete multiplicativity, we expect an Euler product
$$\Fhat(s) = \prod_{p > c} (1 - c p^{-s})^{-1}.$$
According to the strategy in Example 6.8, we extract the zeta power $\zeta(s)^c$ from $\Fhat(s)$ and write
$$\Fhat(s) = \zeta(s)^c \; \xi(s)$$
where $\xi(s) = \Fhat(s) \zeta(s)^{-c}$ should have a lower abscissa of convergence. Then extracting the pole from $\zeta(s)$, we can write
$$\Fhat(s) = (s - 1)^{-c} \; \underbrace{((s - 1) \zeta(s))^c \; \xi(s)}_{\varphi(s)}$$
where $\varphi$ is analytic on the the closed half plane $\sigma \geq 1$. Applying the generalized Wiener-Ikehara theorem, we get the asymptotic estimate.

\textbf{Full version}: We work out the argument in full details.

\begin{itemize}
\item First, we prove that $\sigma_c(\Fhat) \leq 1$.

Define the half plane $H_a := \{s : \sigma > a\}$.

The simple bound $0 \leq f(n) < n$ implies that $F(x) = O(x^2)$ whence $\Fhat(s)$ converges absolutely on $H_2$ and we have the Euler product by Corollary 6.6.
To push absolute convergence to $H_1$, one can try to prove that $F(x) = O(x^{1+\epsilon})$ for any $\epsilon > 0$ but that is not going to be easy.
To get around the issue, we use our partial converse to Lemma 6.4: As $f(n) n^{-\sigma} > 0$ for all $n$, $\sum f(n) n^{-\sigma}$ converges whenever the infinite product $\prod_{p > c} (1 + c p^{-\sigma} + c^2 p^{-2\sigma} + \cdots)$ does.

To see that the infinite product for $\sigma > 1$ converges, we have to prove the equivalent fact that $\prod_{p > c} (1 - c p^{-\sigma})$ converges (to a non-zero limit). Clearly, the sequence of partial products $\prod_{c < p < N} (1 - c p^{-\sigma})$ is decreasing and bounded below by 0 for $0 < 1 - c p^{-\sigma} < 1$ so it has a limit. We just have to prove that the limit is not zero by proving that the logarithm
$$\sum_{p > c} \log(1 - c p^{-\sigma}) \not= -\infty$$
or simply that it is bounded below. Using the simple bound for $\log(1 - x)$ in Chapter 6, we obtain
\begin{align*}
\sum_{p > c} \log(1 - c p^{-\sigma}) &\geq \sum_{p > c} - \frac{1}{c^{-1} p^\sigma - 1}\\
&= - c \sum_{p > c} \frac{1}{p^\sigma - c}\\
&\geq -c \sum_{c < p \leq N} \frac{1}{p^\sigma - c} - c \sum_{p > N} \frac{1}{p^\sigma / 2} \\
&\geq -c \sum_{c < p \leq N} \frac{1}{p^\sigma - c} - 2 c \sum_{p > N} p^{-\sigma}
\end{align*}
where $N$ could be any number large enough so that $p^{\sigma} - c \geq p^{\sigma}/2$ for all $p > N$. The last series $\sum_{p > N} p^{-\sigma}$ is already known to converge for $\sigma > 1$ as it is bounded above by $\zeta(\sigma)$.

We have thus proved that $\Fhat(s)$ converges absolutely on $H_1$ and we have an Euler product there.

\item Next, we show that $\xi(s)$ has a lower abscissa of convergence.

Note that we expect $ \xi(s) = \Fhat(s) \zeta(s)^{-c}$ but we will not take this as its definition because at this point, $\Fhat(s)$ only exists for $\sigma > 1$. Instead, we shall later define it via the infinite product
\begin{align*}
\xi(s) &:= \prod_p (1 - p^{-s})^c \prod_{p > c} (1 - cp^{-s})^{-1}\\
&= \prod_{p \leq c} (1 - p^{-s})^c \prod_{p > c} \frac{(1 - p^{-s})^c}{1 - cp^{-s}}
\end{align*}
and obtains its analyticity on the larger region $H := H_{1/2}$ from that of another function $\eta(s)$.

Note that unlike series, an infinite product $\prod a_n$ needs not converge when $\prod |a_n|$ does; to wit, consider $a_n = (-1)^n$ or $a_n = e^{in}$.

We define $\xi(s)$ via its logarithm (omitting finitely many Euler factor)
$$\eta(s) := \sum_{p > N} c \log(1 - p^{-s}) - \log(1 - cp^{-s})$$
where $N > c$ is to be specified.

The Taylor series
$$\log(1 - z) = - \sum_{k = 1}^{\infty} \frac{z^k}{k}$$
converges for $|z| < 1$ by ratio test and define an analytic function on the unit disk extending the usual function $\log(1 - x)$ for real $x \in (0, 1)$.

So let $N > c$ is any number large enough so that both $c p^{-\sigma} < \frac12$ and $p^{-\sigma} < \frac12$ for any $p > N$ and $\sigma > 1/2$. This lets us use the above series and also make a future inequality holds.

We show that $\eta(s)$ converges and is analytic on $H$ so that
$$\xi(s) := e^{\eta(s)} \prod_{p \leq c} (1 - p^{-s})^c \prod_{c < p \leq N} \frac{(1 - p^{-s})^c}{1 - cp^{-s}}$$
is also.

Thank to our choice of $N$, we can use the series for logarithm:
\begin{align*}
\eta(s) &= \sum_{p > c} \sum_{k=1}^{\infty} \left( - c \frac{p^{-ks}}{k} + \frac{c^k p^{-ks}}{k} \right)\\
&= \sum_{p > c} \sum_{k=2}^{\infty} (c^k - c) \frac{p^{-ks}}{k}
\end{align*}
If $c \leq 1$ then $|c^k - c| = c - c^k \leq c$. If $c > 1$ then $|c^k - c| = c^k - c \leq c^k \leq c^{k+1}$. So in either case we find $|c^k - c| \leq c \cdot \max\{1, c^k\} = c d^k$ where $d := \max\{1, c\}$.

We have
\begin{align*}
\sum_{p > N} \sum_{k=2}^{\infty} \Abs{ (c^k - c) \frac{p^{-ks}}{k} } &= \sum_{p > N} \sum_{k=2}^{\infty} |c^k - c| \;   \frac{p^{-k\sigma}}{k}\\
&\leq \sum_{p > N} \sum_{k=2}^{\infty} c d^k \frac{p^{-k\sigma}}{k} \\
&= c \sum_{p > N} \sum_{k=2}^{\infty} \frac{(d p^{-\sigma})^k}{k} \\
&= c \sum_{p > N} - \log(1 - dp^{-\sigma}) - d p^{-\sigma} \\
&\leq \sum_{p > N} 2 d^2 p^{-2\sigma} \\
&< 2 c d^2 \zeta(2\sigma)
\end{align*}
so $\eta(s)$ converges absolutely for $\sigma > 1/2$. Here we have used the inequality
$$- \log(1 - x) - x \leq 2 x^2, \qquad x \in [0, 1/2)$$
which is another reason for our choice $N$ (so that $dp^{-\sigma} < \frac12$). This inequality comes from
$$- \log(1 - x) - x \leq \frac{x^2}{2(1-x)^2}$$
which I proved in Chapter 6.

The fact that $\eta(s)$ is analytic is not difficult to see: By the above bound, the series defining $\eta(s)$ converges uniformly on any closed half plane $\sigma \geq \delta$ for any fixed $\delta > 1/2$ (thank to convergence of $\zeta(2 \delta)$) so it follows by the well-known consequence of Morera's theorem that $\eta(s)$ is analytic on that half plane since it is the uniform limit of a sequence of analytic functions.

\item Now we can apply the generalized Wiener-Ikehara theorem: From absolute convergence of $\Fhat(s)$ on $H_1$, we have an Euler product which gives us the equation
$$\Fhat(s) = (s - 1)^{-c} \; ((s - 1) \zeta(s))^c \; \xi(s)$$
on $H_1$ where $\xi(s)$ is defined from $\eta(s)$ and is known to be analytic on $H$. This equation implies $\sigma_c(\Fhat) = 1$ because $\Fhat$ has a pole at 1 whence it cannot converge on the line $\sigma = 1$. Applying the theorem with $\varphi(s) = ((s - 1) \zeta(s))^c \; \xi(s)$ and $\psi(s) = 0$, we find the desired asymtotic formula
$$F(x) \sim \xi(1) \frac{x (\log x)^{c - 1}}{\Gamma(c)}$$
as the constant
$$\xi(1) = \prod_{p \leq c} (1-p^{-1})^c \prod_{p > c} \left\{ \frac{(1 - p^{-1})^c}{1 - c p^{-1}} \right\} \not= 0.$$
\end{itemize}
\end{proof}

\unless\ifdefined\IsMainDocument
\end{document}
\fi

\unless\ifdefined\IsMainDocument
\documentclass[12pt]{article}
\usepackage{amsmath,amsthm,amssymb}
\newcommand{\A}{\mathcal{A}}
\newcommand{\Fhat}{\widehat{F}}
\newcommand{\Ghat}{\widehat{G}}
\begin{document}
\fi

\textbf{Problem 7.10}: Let $g \in \A_1$ and $g * g = 1$. Find an assymptotic formula for the summatory function of $g$ (cf. $\S$2.4.2 and Example 6.3(4)).

\begin{proof}
Let $G$ be the summatory function of $g$. We already know that $G$ is monotone nondecreasing and on $\{\sigma > 1\}$
\begin{align*}
\Ghat(s) &= \sqrt{\zeta(s)} \\
&= (s - 1)^{-1/2} \sqrt{(s-1) \zeta(s)}
\end{align*}
so by Theorem 7.7 (note that $\varphi(s) = \sqrt{(s-1) \zeta(s)}$ is analytic and $\psi(s) = 0$ here), we obtain
$$G(x) \sim x (\log x)^{-1/2} / \Gamma(1/2) = \frac{x}{\sqrt{\pi \log x}}.$$
\end{proof}

\unless\ifdefined\IsMainDocument
\end{document}
\fi

\unless\ifdefined\IsMainDocument
\documentclass[12pt]{article}
\usepackage{amsmath,amsthm,amssymb}
\newcommand{\Fhat}{\widehat{F}}
\begin{document}
\fi

\textbf{Problem 7.11}: Using the identity
$$\zeta^4(s)/\zeta(2s) = \sum \tau^2(k) k^{-s}$$
and the generalized Wiener-Ikehara theorem, prove that
$$\sum_{k \leq x} \tau^2(k) \sim \pi^{-2} x \log^3 x.$$
(Cf. Problem 2.31 and 3.23, part (4).)

\begin{proof}
The summatory function of $\tau^2$ is obviously real valued monotone nondecreasing. The identity holds at least on $\sigma > 1$ thank to Theorem 6.2 and we deduce that $\sigma_c = 1$ using the pole of $\zeta$ function. Clearly
$$\zeta^4(s) / \zeta(2s) = (s-1)^{-4} \varphi(s)$$
where
$$\varphi(s) = \frac{((s-1) \zeta(s))^4}{\zeta(2s)}$$
is analytic on $\{\sigma > 1/2\}$. So the generalized Wiener-Ikehara theorem implies
$$\sum_{k \leq x} \tau^2(k) \sim \varphi(1) \frac{x \log^3 x}{\Gamma(4)}.$$
With $\varphi(1) = \frac{1}{\zeta(2)} = \frac{6}{\pi^2}$
and $\Gamma(4) = 3! = 6$, we get the claimed asymptotic formula.
\end{proof}

\unless\ifdefined\IsMainDocument
\end{document}
\fi

\unless\ifdefined\IsMainDocument
\documentclass[12pt]{article}
\usepackage{amsmath,amsthm,amssymb}
\newcommand{\A}{\mathcal{A}}
\newcommand{\Fhat}{\widehat{F}}
\newcommand{\Ghat}{\widehat{G}}
\begin{document}
\fi

\textbf{Problem 7.12}: (Generalized divisor function.) For $c$ a fixed real number, not 0 or a negative integer, define a multiplicative function $\tau_c$ by setting
$$\tau_c(p^j) = c (c + 1) \cdots (c + j - 1)/j!$$
for primes $p$ and positive integers $j$. Find an assymptotic formula for the associated summatory function. (Cf. Problem 3.26.)

\begin{proof}
Let $F_c$ be the summatory function of $\tau_c$.

\noindent \textbf{The case $c > 0$:} We consider the case $c > 0$ so that $\tau_c > 0$ and $F_c$ is monotone nondecreasing. Also observe that if $c < d$ then $\tau_c < \tau_d$ and so $F_c < F_d$. It is easy to check\footnote{Both are multiplicative and they agree on prime powers.} that $\tau_k = 1^{*k}$ for positive integer $k$ and we already have bound $F_k = O(x^{1 + \epsilon})$ for all positive integer $k$ (albeit not uniformly) thank to Problem 3.26. It follows that $F_c = O(F_{[c] + 1}) = O(x^{1 + \epsilon})$ for all $c > 0$ which implies $\sigma_c(F_c) = \sigma_a(F_c) \leq 1$. In other words, $\Fhat_c$ converges absolutely for $\sigma > 1$ and we have the Euler product there.

Notice that the function $h(x) = x^{-c}$ have higher derivatives
\begin{align*}
h_c^{(j)}(x) &= (-c) (-c - 1) ... (-c - j + 1) x^{- c - j}\\
&= (-1)^j c (c + 1) ... (c + j - 1) x^{- c - j}
\end{align*}
and thus Taylor expansion
$$h_c(x) = \sum_{j = 0}^{\infty} c (c + 1) ... (c + j - 1) \frac{(1 - x)^j}{j!}$$
which converges at least when $|x - 1| < 1$ (i.e. when $\lim_{j \rightarrow \infty} \frac{(c+j) |1-x|}{1 + j} < 1$ so ratio test applies).

Hence, we have
\begin{align*}
\Fhat_c(s) &= \prod_p \left(\sum_{j = 0}^{\infty} \frac{c (c + 1) \cdots (c + j - 1)}{j!} p^{-js}\right) & (\sigma > 1) \\
&= \prod_p h_c(1 - p^{-s})\\
&= \prod_p (1 - p^{-s})^{-c} \\
&= \zeta(s)^c \\
&= (s - 1)^{-c} \; ((s - 1)\zeta(s))^c
\end{align*}
so just like the previous problem: This formula shows that $\sigma_c(\Fhat_c) = 1$ (due to the pole of $\zeta$ and Problem 6.18) and $\Fhat_c$ satisfies the hypothesis of the generalized Wiener-Ikehara theorem with
$$\varphi(s) = ((s - 1)\zeta(s))^c$$
analytic on $\{ \sigma > 0 \}$. By the theorem, we have
$$F_c(x) = \sum_{n \leq x} \tau_c(n) \sim \frac{x (\log x)^{c - 1}}{\Gamma(c)}.$$

\textbf{Note}: If we do not want to use Problem 3.26, we can bootstrap the proof by solving this problem for $c = k$ being positive integers first: Theorem 6.2 gives us $\Fhat_k(s) = \zeta(s)^k$ for $\sigma > 1$ immediately thank to the representation $\tau_k = 1^{*k}$ even though we do not know $\sum_{n \leq x} \tau_k(n) = O(x^{1 + \epsilon})$. This implies $\sigma_c(\Fhat_k) = 1$ and we apply the generalized Wiener-Ikehara theorem just like the case for general $c > 0$.

\noindent \textbf{The case $c < 0$:} When $c < 0$, this argument does not work because the function $F_c$ is not monotone as $\tau_c(n)$ can oscillate between negative (when $n$ is a prime number and there are infinitely many such $n$) and positive values (when $n = p^2$, $\tau_c(p^2) = c^2 > 0$).

However, we can determine the right abscissa of convergence of $\Fhat_c(s)$ because of the assumption that $c$ is not 0 or a negative integer.

First, let us show that $\sigma_c(\Fhat_c) \leq 1$. It is clear that
$$|c + k| \leq |c| + k = -c + k$$
so
$$|\tau_c(p^j)| \leq \tau_{-c}(p^j)$$
for all prime $p$ and all integer $j \geq 0$. This implies
$$|\tau_c(n)| \leq \tau_{-c}(n)$$
for all integer $n$ so $|F_c(x)| \leq F_{-c}(x)$ and since we already know $F_{-c}(x) = O(x^{1 + \epsilon})$, it follows that $\Fhat_c(s)$ converes absolutely for $\sigma > 1$. So $\sigma_c \leq 1$. With absolute convergence, the Euler product formula above still holds and we also have
$$\Fhat_c(s) = \zeta(s)^c \qquad (\sigma > 1).$$

From the above equation, we can see that $\Fhat_c(s)$ has a zero of non--integral order $-c$ at 1. So let $N$ be any (positive) integer exceeding $-c$ then $\Fhat_c^{(N)}(s)$ has a pole at $s = 1$ (of order $N + c$) by apply repeatedly the observation that if $f(s)$ is analytic with a zero of order $\alpha > 0$ at $s_0$ where $\alpha \not= 1$ then $f'(s)$ has either a zero of order $\alpha - 1$ (if $\alpha > 1$) or a pole of order $1 - \alpha$ (if $\alpha < 1$)
\footnote{To see this observation, we can write $f(s) = (s - s_0)^\alpha g(s)$ with $g$ analytic and $g(s_0) \not= 0$. Then $f'(s) = \alpha (s - s_0)^{\alpha - 1} g(s) + (s - s_0)^\alpha g'(s) = (s - s_0)^{\alpha - 1} \{ \alpha g(s) + (s - s_0) g'(\alpha) \}$. The analytic function in the curly braces clearly does not vanish at $s_0$. We are also using the fact that if $f(s)$ has a pole of order $\alpha$ at $s_0$ then $f'(s)$ has a pole of order $\alpha + 1$ at $s_0$.}.
If $\sigma_c < 1$ then $\Fhat_c^{(N)}(s)$ is analytic at $s = 1$ by Theorem 6.20, a contradiction. So $\sigma_c = 1$.

\textbf{Reference}: The asymptotic formula even for complex value of $c$ was in the paper of A. Selberg, \textit{Note on a paper by L. G. Sathe}.
\end{proof}

\unless\ifdefined\IsMainDocument
\end{document}
\fi

\unless\ifdefined\IsMainDocument
\documentclass[12pt]{article}
\usepackage{amsmath,amsthm,amssymb}
\newcommand{\V}{\mathcal{V}}
\newcommand{\Abs}[1]{\left| #1 \right|}
\newcommand{\Fhat}{\widehat{F}}
\newcommand{\Ghat}{\widehat{G}}
\begin{document}
\fi

\textbf{Problem 7.13}: Suppose $F \in \V$ and $\sigma_c(\Fhat) < \alpha < \infty$. Let
$$I(T, \alpha) := \frac{1}{2T} \int_{-T}^T \Fhat(\alpha + it) dt.$$
Show that $\lim_{T \rightarrow \infty} I(T, \alpha)$ exists and equals $F(1)$. Hint. Show that
$$I(T, \alpha) = F(1) + \int_1^2 x^{-\alpha} \frac{\sin (T\log x)}{T \log x} dF(x) + \frac{\Ghat(\alpha + iT) - \Ghat(\alpha - iT)}{-2iT},$$
where $\Ghat(s) = \int_2^\infty x^{-s} (\log x)^{-1} dF(x).$

\begin{proof}
Note that $\Fhat(\alpha + it)$ converges uniformly on $[-T, T]$ by Theorem 6.15 so we can swap the limit in $\Fhat$ and the integral to get
\begin{align*}
I(T, \alpha) &= \frac{1}{2T} \int_{-T}^T \left( \lim_{X \rightarrow \infty} \int_{1^-}^X x^{-\alpha-it} dF \right) dt \\
&= \lim_{X \rightarrow \infty} \int_{1^-}^X \left( \frac{1}{2T} \int_{-T}^T x^{-\alpha-it}  dt \right) dF \\
&= \lim_{X \rightarrow \infty} \int_{1^-}^X x^{-\alpha} \left( \frac{1}{2T} \left. \frac{x^{-it}}{-i \log x} \right|_{-T}^T   \right) dF \\
&= \lim_{X \rightarrow \infty} \int_{1^-}^X \frac{x^{-\alpha}(x^{-iT} - x^{iT} )}{- 2 i T \log x} dF \\
&= \int_{1^-}^\infty \frac{x^{-\alpha-iT} - x^{-\alpha+iT}}{- 2 i T \log x} dF \\
&= F(1) \left\{ \lim_{x \rightarrow 1} \frac{x^{-\alpha-iT} - x^{-\alpha+iT}}{- 2 i T \log x} \right\} + \int_{1}^\infty \frac{x^{-\alpha-iT} - x^{-\alpha+iT}}{- 2 i T \log x} dF
\end{align*}
by Lemma 3.8. Observe that
\begin{align*}
\lim_{x \rightarrow 1} \frac{x^{-\alpha-iT} - x^{-\alpha+iT}}{- 2 i T \log x} &= \lim_{x \rightarrow 1} x^{-\alpha} \frac{\sin(T \log x)}{T \log x} \\
&= \lim_{x \rightarrow 1} \frac{\sin(T \log x)}{T \log x} &\text{since } \lim_{x \rightarrow 1} x^{-\alpha} = 1 \\
&= \lim_{h \rightarrow 1} \frac{\sin(h)}{h} \\
&= 1 &\text{by L'hospital rule}
\end{align*}
so it remains to show that
$$\lim_{T \rightarrow \infty} \int_{1}^\infty \frac{x^{-\alpha-iT} - x^{-\alpha+iT}}{- 2 i T \log x} dF = 0.$$

We would like to express
\begin{align*}
&\int_{1}^\infty \frac{x^{-\alpha-iT} - x^{-\alpha+iT}}{- 2 i T \log x} dF\\
=& - \frac{1}{2 i T} \left( \int_{1}^\infty x^{-\alpha-iT} \frac{1}{\log x} dF - \int_{1}^\infty x^{-\alpha+iT} \frac{1}{\log x} dF\right)\\
=& -\frac{\Ghat(\alpha + iT) - \Ghat(\alpha - iT)}{2iT}
\end{align*}
where $G(x) = \int_1^x \frac{1}{\log x} dF$ and then simply invoke Lemma 7.13 which gives $\Ghat(\alpha \pm iT) = o(T)$ to deduce that the limit is 0. The only issue with this argument is that $\log 1 = 0$ so we cannot define $G(x)$ like that.

To rectify the idea, we just have to split $\int_1^\infty = \int_1^2 + \int_2^\infty$. Then we could define
$$G(x) = \begin{cases} \int_2^x \frac{1}{\log x} dF &\text{if } x > 2 \\ 0 &\text{if } x \leq 2 \end{cases}$$
to settle $\lim_{T \rightarrow \infty} \int_2^\infty$. (It should be clear that $\sigma_c(G) \leq \sigma_c(F)$ so we can apply Lemma 7.13.)

To deal with the limit of
\begin{align*}
\int_1^2 \frac{x^{-\alpha-iT} - x^{-\alpha+iT}}{- 2 i T \log x} dF =& \int_1^2 x^{-\alpha} \frac{\sin(T \log x)}{T \log x} dF,
\end{align*}
we first recall that
$$\lim_{\delta \rightarrow 0^+} \int_1^{1 + \delta} x^{-\alpha} dF_v = 0$$
by Lemma 3.8. Let $\epsilon > 0$ be arbitrary then we can choose $\delta > 0$ be such that
$$0 < \int_1^{1 + \delta} x^{-\alpha} dF_v < \epsilon / 2.$$
We can bound
\begin{align*}
\Abs{ \int_1^{1 + \delta} x^{-\alpha} \frac{\sin(T \log x)}{T \log x} dF } \leq& \int_1^{1 + \delta} x^{-\alpha} \Abs{ \frac{\sin(T \log x)}{T \log x} } dF_v \\
\leq& \int_1^{1 + \delta} x^{-\alpha} dF_v
\end{align*}
because $\Abs{ \frac{\sin t}{t} } \leq 1$ for all $t$ and
\begin{align*}
\Abs{ \int_{1 + \delta}^2 x^{-\alpha} \frac{\sin(T \log x)}{T \log x} dF } \leq& \int_{1 + \delta}^2 x^{-\alpha} \Abs{ \frac{\sin(T \log x)}{T \log x} } dF_v\\
\leq& \int_{1 + \delta}^2 x^{-\alpha} \frac{1}{T \log(1 + \delta)} dF_v \\
=& \frac{1}{T \log(1 + \delta)} \int_{1 + \delta}^2 x^{-\alpha} dF_v \\
\leq& \frac{1}{T \log(1 + \delta)} \int_1^2 x^{-\alpha} dF_v 
\end{align*}
which we can bound by $\epsilon / 2$ if we pick $T$ large enough (note that $\delta$ is constant here).

\textbf{Note}: To check $\Abs{ \frac{\sin t}{t} } \leq 1$ for $t > 0$, we need to show that $-t \leq \sin t \leq t$. The derivative $(t \pm \sin t)' = 1 \pm \cos t \geq 0$ so $t \pm \sin t$ are both increasing functions and that means for $t \geq 0$, one has $t \pm \sin t \geq 0 \pm \sin 0 = 0$.
\end{proof}

\unless\ifdefined\IsMainDocument
\end{document}
\fi

\unless\ifdefined\IsMainDocument
\documentclass[12pt]{article}
\usepackage{amsmath,amsthm,amssymb}
\newcommand{\Fhat}{\widehat{F}}
\newcommand{\Z}{\mathbb{Z}}
\newcommand{\R}{\mathbb{R}}
\newcommand{\Abs}[1]{\left| #1 \right|}
\newcommand{\IntPart}[1]{\left[ #1 \right]}
\newcommand{\cconj}[1]{\overline{#1}}
\begin{document}
\fi

\textbf{Problem 7.14}: Discuss the reasons for assuming that $x > 1/2$ in most of the preceeding proof.

\begin{proof}
The equation
$$F(x) = \IntPart{ 1 + \frac{\log x}{\log 2} }$$
only works when $1 + \frac{\log x}{\log 2} \geq 0$ which is the same as $x \geq \frac{1}{2}$.
\end{proof}

\textbf{Problem 7.15}: Give a real variable proof that the Fourier series for $S(y)$ of Lemma 7.15 converges boundedly on $\R$ and uniformly away from integers. Hint. For $y$ near an integer $m$, use the inequality
$$\sin|2\pi n y| \leq 2\pi n |y - m|.$$
Elsewhere, use summation by parts.

\begin{proof}
For $\delta \in (0, 1/2)$, define
$$\R_\delta := \R \backslash \bigcup_{n \in \Z} B(n, \delta) = \bigcup_{n \in \Z} [n + \delta, n + 1 - \delta]$$
the set of real numbers that are at least $\delta$-away from the integers. For any real number $\rho \geq 0$, let
\begin{align*}
S_{N,\rho}(y) &:= \sum_{n=1}^N \frac{1}{\pi (n + \rho)} \sin 2\pi n y = \int_{1^-}^N \frac{1}{\pi (t + \rho)} d F_y(t)\\
C_{N,\rho}(y) &:= \sum_{n=1}^N \frac{1}{\pi (n + \rho)} \cos 2\pi n y = \int_{1^-}^N \frac{1}{\pi (t + \rho)} d G_y(t)
\end{align*}
where for any fixed $y \in \R$, let the summatory functions
\begin{align*}
F_y(x) := \sum_{n \leq x} \sin(2 \pi n y)\\
G_y(x) := \sum_{n \leq x} \cos(2 \pi n y).
\end{align*}
Note that our problem is to prove $S_N(y) := S_{N, 0}(y)$ converges boundedly on $\R$ and uniformly on $\R_\delta$.

Our first goal is to study the growth behavior of $F_y$ and $G_y$. To do so, we sum up the Euler's formulas\footnote{This should not count as complex variable method, right?}
$$e^{2 \pi y n i} = \cos(2 \pi y n) + i \sin(2\pi y n)$$
to get
$$\sum_{n \leq x} e^{2 \pi y n i} = G_y(x) + i F_y(x).$$
For $x \geq 1$ and $y \not\in \Z$, let $u = e^{2 \pi y i}$ and $m = [x]$ then $u \not= 1$ and the left hand side can be completely evaluated as
$$\sum_{n \leq x} e^{2 \pi y n i} = \sum_{n = 0}^{m} u^n - 1 = \frac{u^{m + 1} - 1}{u - 1} - 1.$$
Using the obvious inequality $|\Re z| \leq |z|$ and $|\Im z| \leq |z|$ and triangle inequality, we find
$$|F_y(x)| \leq \Abs{ \frac{u^{m + 1} - 1}{u - 1} - 1 } \leq 1 + \frac{2}{|u - 1|}$$
The same bound holds for $|G_y(x)|$. This inequality implies that both $|F_y(x)|$ and $|G_y(x)|$ are bounded above for all $x$ by
$$C_\delta := 1 + \frac{2}{|e^{2\pi i \delta} - 1|} \leq 1 + \frac{2}{\sin 2\pi\delta}.$$
for $y \in R_\delta$ because then $|u - 1| \geq |e^{2\pi i \delta} - 1|$. In other words, $F_y$ and $G_y$ are $O(1)$ uniformly for $y \in R_\delta$.

Performing integration by parts, we get
\begin{align}
S_{N,\rho}(y) &= \frac{F_y(N)}{\pi (N + \rho)} + \int_1^N F_y(t) \frac{1}{\pi (t + \rho)^2} dt \label{eq:SN_Int_Part}
\end{align}
which allows us to establishes uniform convergence and bounded convergence of $S_{N,\rho} \rightarrow S_\rho$ on $\R_\delta$: Bounded convergence is clear: For all $N$ and $y \in R_\delta$, we have
$$|S_{N,\rho}(y)| \leq \frac{\Abs{F_y(N)}}{\pi (N + \rho)} + \int_1^N \frac{|F_y(t)|}{\pi t^2} dt \leq \frac{2 C_\delta}{\pi}.$$
For uniform convergence, let $\epsilon > 0$ be arbitrary and let $N_0$ be large enough so that
$$\frac{C_\delta}{\pi N_0} < \frac{\epsilon}{3}$$
then for any $N, M > N_0$ and any $y \in \R_\delta$, we find
\begin{align*}
&\Abs{ S_{N,\rho}(y) - S_{M,\rho}(y) }\\
=& \Abs{ \frac{F_y(N)}{\pi (N + \rho)} + \int_1^N F_y(t) \frac{1}{\pi (t + \rho)^2} dt - \frac{F_y(M)}{\pi (M + \rho)} - \int_1^M F_y(t) \frac{1}{\pi (t + \rho)^2} dt } \\
\leq& \Abs{ \frac{F_y(N)}{\pi (N + \rho)} } + \Abs{ \frac{F_y(M)}{\pi (M + \rho)} } + \Abs{ \int_M^N F_y(t) \frac{1}{\pi (t + \rho)^2} dt } \\
\leq& \frac{C_\delta}{\pi (N + \rho)} + \frac{C_\delta}{\pi (M + \rho)} + \int_{N_0}^{\infty} \frac{C_\delta}{\pi t^2} dt \\
\leq& 3 \frac{C_\delta}{\pi N_0} \\
<& \epsilon.
\end{align*}
Note that we did not use any particular property of $F_y$ other than its uniform bound by $C_\delta$. So the argument above works for the functions $C_{N,\rho}(y)$ as well.

All that is left in the problem is to establish a uniform bound on the complement $\R - \R_\delta$ for $S_N(y) := S_{N, 0}(y)$. (Note that uniform boundedness on $\R$ does not apply to the function $C_{N, \rho}(y)$ since $C_{N, 0}(0) = \sum_{n = 1}^N \frac{1}{\pi n} \sim \frac{1}{\pi} \log N$.)

Fix $\delta = \frac14$ and let $y \in (0, \delta)$ be arbitrary. Then there is the smallest positive integer $M$ such that $y M \geq \delta$ by the Archimedean property. Our choice of $\delta$ ensures that $My \leq 1 - \delta$ as well for by definition, we must also have
$$(M - 1)y < \delta$$
so $My \leq \delta + y < 2 \delta = \frac{1}{2} < \frac34 = 1 - \delta$. (Note $\delta = \frac13$ also works.)

If $N \leq M - 1$ then using the inequality $0 \leq \sin x \leq x$ for small $x > 0$ which applies to $ry$ for $1 \leq r < M$ we get
$$0 < \sum_{r = 1}^{N} \frac{\sin 2\pi r y}{\pi r} \leq \sum_{r = 1}^{N} \frac{2\pi r y}{\pi r} = 2 N y \leq 2 (M - 1) y < 2 \delta$$
by minimality of $M$.

For $N \geq M$, we rearrange the terms in $S_N$ by their congruence modulo $M$:
\begin{align*}
S_N(y) &= S_{M-1}(y) + \sum_{r = 0}^{M - 1} \sum_{M \leq n \leq N, n \equiv r \bmod M} \frac{\sin 2\pi n y}{\pi n}\\
&= S_{M-1}(y) + \sum_{r = 0}^{M - 1} \sum_{1 \leq k \leq \frac{N-r}{M}} \frac{\sin 2\pi (M k + r) y}{\pi (M k + r)}
\end{align*}

The sum $S_{M - 1}(y) = \sum_{r = 1}^{M - 1} \frac{\sin 2\pi r y}{\pi r}$ already was bound by $2\delta$. For the remaining sum, we use the common trigonometric identities to expand
\begin{align*}
&S_N(y) - S_{M-1}(y)\\
=& \sum_{r = 0}^{M - 1} \sum_{1 \leq k \leq \frac{N-r}{M}} \frac{\sin 2\pi (M k + r) y}{\pi (M k + r)} \\
=& \sum_{r = 0}^{M - 1} \sum_{1 \leq k \leq \frac{N-r}{M}} \frac{\sin (2 \pi M k y) \cos (2\pi r y) + \cos (2 \pi M k y) \sin (2 \pi r y)}{\pi (M k + r)} \\
=& \sum_{r = 0}^{M - 1} \left\{ \cos (2\pi r y) \underbrace{\sum_{1 \leq k \leq \frac{N-r}{M}} \frac{\sin (2 \pi M k y)}{\pi M (k + \frac{r}{M})}}_{S_{[\frac{N-r}{M}], \frac{r}{M}}(My) / M} + \sin (2 \pi r y) \underbrace{\sum_{1 \leq k \leq \frac{N-r}{M}} \frac{\cos (2 \pi M k y)}{\pi M (k + \frac{r}{M})}}_{C_{[\frac{N-r}{M}], \frac{r}{M}}(My) / M} \right\}
\end{align*}

Now since $\delta \leq My \leq 1 - \delta$, we have 
$$\Abs{ S_{[\frac{N-r}{M}], \frac{r}{M}}(My) }, \Abs{ C_{[\frac{N-r}{M}], \frac{r}{M}}(My) } \leq \frac{2 C_\delta}{\pi}$$
so
\begin{align*}
\Abs{S_N(y) - S_{M - 1}(y)} 
\leq& \sum_{r = 0}^{M - 1} \left\{ \frac{2 C_\delta}{M \pi} + \frac{2C_\delta}{M \pi} \right\} = \frac{4C_\delta}{\pi}.
\end{align*}

This settles the case $y \in (0, \delta)$. For $y \in (1-\delta, 1)$, we use the fact that $S_N(1 - y) = -S_N(y)$ and apply the bound with $1 - y \in (0, \delta)$ in place of $y$. This gives us the bound
$$|S_N(y)| \leq \max\left\{2\delta, \frac{4C_\delta}{\pi}\right\}$$
for all $y \in (0, 1)$ and $N$. The bound for the remaining $y \in \R$ follows from periodicity of $S_N$.
\end{proof}

\unless\ifdefined\IsMainDocument
\end{document}
\fi

\unless\ifdefined\IsMainDocument
\documentclass[12pt]{article}
\usepackage{amsmath,amsthm,amssymb}

\begin{document}
\fi

\textbf{Problem 7.16}: Let $S(y)$ denote the sawtooth function and for $N = 1, 2, ...$, let $S_N(y)$ denote the $N$th partial sum of its Fourier series (7.22).
\begin{enumerate}
\item Show that $\int_0^1 \{S^2(y) - S_N^2(y)\} dy \rightarrow 0$ as $N \rightarrow \infty$.
\item By evaluating $\int_0^1 S^2(y) dy$ and $\int_0^1 S_N^2(y) dy$, give another proof of Theorem 1.5.
\end{enumerate}

\begin{proof}
\begin{enumerate}
\item Let $C$ be the uniform bound for $S_N(y)$ i.e.
$$|S_N(y)| \leq C, \qquad \text{ for all } y \in [0, 1].$$
by bounded convergence. Then
$$|S(y) + S_N(y)| \leq C + 1$$
and from the equation
$$S^2(y) - S_N^2(y) = (S(y) - S_N(y))(S(y) + S_N(y))$$
we have
$$|S^2(y) - S_N^2(y)| \leq (C + 1)^2$$
so $S_N^2(y) \rightarrow S^2(y)$ uniformly on $[\delta, 1 - \delta]$ for any $\delta \in (0, 1/2)$. This allows us to switch the limit over $N$ and the integral to get
$$\lim_{N \rightarrow \infty} \int_\delta^{1-\delta} \{S^2(y) - S_N^2(y)\} dy = \int_\delta^{1-\delta} \lim_{N \rightarrow \infty} \{S^2(y) - S_N^2(y)\} dy = 0.$$

To prove the full limit for $\int_0^1$, let $\epsilon > 0$ be arbitrary. We need to choose $N_0$ such that $|\int_0^1 ... dy| < \epsilon$ for all $N > N_0$. To do that, we first choose $\delta > 0$ such that $2 (C + 1)^2 \delta < \frac{\epsilon}{2}$ and then take $N_0$ such that
$$|S(y) - S_N(y)| < \frac{\epsilon}{2 (C + 1)}$$ 
for all $N > N_0$ and $y \in [\delta, 1-\delta]$ by uniform convergence. Note that the above inequality implies
$$|S^2(y) - S_N^2(y)| < \frac{\epsilon}{2}.$$

With this choice of $N_0$, we have for any $N > N_0$
\begin{align*}
|\int_0^1 \{S^2(y) - S_N^2(y)\} dy| &= |\int_0^\delta + \int_\delta^{1-\delta} + \int_{1-\delta}^1|\\
&\leq \left\{ \int_0^\delta + \int_{1-\delta}^1 \right\} (C + 1)^2 dy + \int_\delta^{1-\delta} \frac{\epsilon}{2} dy\\
&= 2\delta(C+1)^2 + (1 - 2\delta) \frac{\epsilon}{2}\\
&< \epsilon.
\end{align*}

\item One has
\begin{align*}
\int_0^1 S^2(y) dy &= \int_0^1 \left([y] - y + \frac12 \right)^2 dy\\
&= \int_0^1 \left(- y + \frac12 \right)^2 dy\\
&= \int_0^1 \left(y^2 - y + \frac14 \right) dy\\
&= \left.\frac{y^3}{3} - \frac{y^2}{2} + \frac14y \right|_0^1\\
&= \frac{1}{3} - \frac{1}{2} + \frac14\\
&= \frac{1}{12}
\end{align*}
and
\begin{align*}
\int_0^1 S_N^2(y) dy &= \int_0^1 \left( \sum_{n=1}^{N} \frac{1}{\pi n} \sin 2 \pi n y \right)^2 dy\\
&= \int_0^1 \left( \sum_{n=1}^{N} \sum_{m=1}^{N} \frac{1}{\pi n} \sin 2 \pi n y \frac{1}{\pi m} \sin 2 \pi m y \right) dy\\
&= \sum_{n=1}^{N} \sum_{m=1}^{N} \frac{1}{\pi^2 n m} \int_0^1 \sin (2 \pi n y) \sin(2\pi m y) \; dy\\
&= \sum_{n=1}^{N} \frac{1}{2 \pi^2 n^2}
\end{align*}
where we used the common trigonometric formula $2 \sin \alpha \sin \beta = \cos(\alpha - \beta) - \cos(\alpha + \beta)$ to resolve
\begin{align*}
\int_0^1 \sin (2 \pi n y) \sin(2\pi m y) \; dy &= \int_0^1 \frac{\cos (2 \pi (m - n) y) - \cos(2\pi (m + n) y)}{2} \; dy\\
&= \begin{cases}
1/2 &\text{if } m = n \not= 0,\\
-1/2 &\text{if } m = -n \not= 0,\\
0 &\text{otherwise.}
\end{cases}
\end{align*}
Thus, we have by part 1 that
$$\sum_{n=1}^{\infty} \frac{1}{2 \pi^2 n^2} = \frac{1}{12}$$
which is clearly the same as Theorem 1.5.
\end{enumerate}
\end{proof}

\unless\ifdefined\IsMainDocument
\end{document}
\fi

\unless\ifdefined\IsMainDocument
\documentclass[12pt]{article}
\usepackage{amsmath,amsthm,amssymb}
\newcommand{\Fhat}{\widehat{F}}
\newcommand{\Ghat}{\widehat{G}}
\newcommand{\R}{\mathbb{R}}
\begin{document}
\fi

\textbf{Problem 7.17}: Under the hypothesis of Theorem 7.16, prove that
$$\frac{1}{2\pi i} \int_{b - i \infty}^{b + i\infty} \frac{x^s \Fhat(s)}{s(s+1)} ds = \int_1^x \left(1 - \frac{t}{x}\right) dF(t), \qquad 1 < x < \infty.$$

\begin{proof}
The key strategy to get the denominator $s$ is to convolute $dF$ with $x^{-1} dx$ because the Mellin transform of $H(x) = \int_{1^-}^x x^{-1} dx$ is $\frac{1}{s}$. Let $dG = dH * dF$ then by Theorem 6.2 $\Ghat(s) = \Fhat(s) \widehat{H}(s) = \Fhat(s)/s$.

Thus, to get $s+1$ appears, we need to convolute with another integrator whose Mellin transform is $\frac{1}{s + 1}$. Observe that for any fix $\alpha$, the Mellin transform
$$\int x^{-s} \frac{dx}{x^\alpha} = \int x^{-s-\alpha} dx = \left. \frac{x^{-s-\alpha+1}}{-s-\alpha+1} \right|_{1^-}^\infty = \frac{1}{s + \alpha - 1}$$
converges absolutely for $\sigma > 1 - \alpha$. Taking $\alpha = 2$ will do in our case.

Now to imitate the proof of Theorem 7.16, consider the function
$$G(x) = \int_{1^-}^{x} dF(t) * t^{-2} dt$$
for $x \geq 1$ and $G(x) = 0$ for $x < 0$. Note that $t^{-2} dt = d\{ \delta_1(t)(1 - t^{-1}) \}$ so
$$G(x) = \int_{1^-}^x \delta_1(t) \left(1 - t/x \right) dF(t) = \int_1^x \left(1 - \frac{t}{x}\right) dF(t)$$
which is continuous because the function $t \mapsto \delta_1(t)(1 - t^{-1})$ is.

Then by Theorem 6.2 we have $\Ghat(s) = \frac{\Fhat(s)}{s+1}$ for $\sigma > \max\{\sigma_c(\Fhat), -1\}$. We apply Theorem 7.10 to $\Ghat$ to get
$$\lim_{T \rightarrow \infty} \frac{1}{2 \pi i} \int_{b - iT}^{b + iT} x^{-s} \frac{\Fhat(s)}{s + 1} \frac{ds}{s} = \frac{G(x+) + G(x-)}{2} = G(x)$$
for $b > \max\{\sigma_c(\Ghat), 0\}$ and so for $b > \max\{\sigma_c(\Fhat), 0\}$.

It remains to turn the symmetric limit to the asymmetric one. For that we need to show that
\begin{equation}
\lim_{T, T' \rightarrow \infty} \frac{1}{2 \pi i} \int_{b + iT}^{b + iT'} x^{-s} \frac{\Fhat(s)}{s(s + 1)} ds = 0. \label{eq:limit_int_xsFhatssp1_equals_0}
\end{equation}
Note that in this integral $|x^{-s}| = |x^{-b-it}| = x^{-b}$ is constant.

If $b > \sigma_a(\Fhat)$ then $|\Fhat(s)|$ is bounded and so $|\frac{\Fhat(s)}{s(s + 1)}| \leq \frac{C}{\max\{T + 1, T' + 1\}^2}$ for some constant $C$ and so
$$\int_{b + iT}^{b + iT'} |x^{-s} \frac{\Fhat(s)}{s(s + 1)}| ds \leq \frac{C x^{-b} |T' - T|}{\max\{T + 1, T' + 1\}^2} \rightarrow 0$$
as $T, T' \rightarrow \infty$. Note that we cannot simply use Lemma 7.13 here for $\Fhat(b + it) = o(|t|)$ since it does not provide enough control over the relationship between $T$ and $T'$.

If $b \leq \sigma_a(\Fhat)$, then we need to deform the integral by taking $b' > \sigma_a(\Fhat)$ and consider the rectangle $R_{T,T'}$ with vertices $b + iT$, $b' + iT$, $b + iT'$, $b' + iT'$. On the top and bottom sides of $R$, we use Lemma 7.13 to conclude that the integral tends to 0 as $T \rightarrow \infty$ (or $T' \rightarrow \infty$, respectively). The function $s \mapsto x^{-s} \frac{\Fhat(s)}{s(s + 1)}$ has no pole in $R$ by assumption on $\sigma > 0$ so $s \not= 0, -1$. So taking the limit
$$\int_{R_{T,T'}} x^{-s} \frac{\Fhat(s)}{s(s + 1)} ds = 0$$
we deduce \eqref{eq:limit_int_xsFhatssp1_equals_0}.
\end{proof}

We review in passing here that the meaning of the improper integral
$$\int_{-\infty}^{\infty} f(x) \, dx$$
is the limit
$$\lim_{\scriptsize\begin{array}{l} a \rightarrow -\infty\\ b \rightarrow \infty\end{array}} \int_a^b f(x) \, dx$$
which in general is not the same as
$$\lim_{a \rightarrow \infty} \int_{-a}^a f(x) \, dx.$$

To understand the ``simultaneous'' limit, we think of the function under limit as a two-variable function i.e. a function $\R^2 \rightarrow \R$
$$F(a, b) := \int_a^b f(x) \, dx$$
and the simultaneous limit can be think of as limit of function between metric spaces
$$\lim_{\scriptsize\begin{array}{l} a \rightarrow -\infty\\ b \rightarrow \infty\end{array}} \int_a^b f(x) \, dx = \lim_{(a, b) \rightarrow (-\infty, \infty)} F(a, b)$$

Unfortunately, $(-\infty, \infty)$ is not a point in $\R^2$; otherwise, we recall that if $X, Y$ are metric spaces and $g : X \rightarrow Y$ then $\lim_{x \rightarrow x_0} f(x) = y \in Y$ if for every $\epsilon > 0$, there exists $\delta > 0$ such that $d_Y (f(x), y) < \epsilon$ for all $x \in B_X(x_0, \delta) := \{x : d_X(x, x_0) < \delta\}$. To account for $(-\infty, \infty)$, we think of $\pm \infty$ as the limit of $\frac{1}{x}$ where $x \rightarrow 0^\pm$. So
$$\lim_{(a, b) \rightarrow (-\infty, \infty)} F(a, b) = \lim_{(a', b') \rightarrow (0^-, 0^+)} F\left(\frac{1}{a'}, \frac{1}{b'}\right)$$
which can be viewed as the limit $(a, b) \rightarrow (0, 0)$ in the second quadrant of $\R^2$. Thank to $(0, 0) \in \R^2$, the last limit can be defined explicitly via $\epsilon$-$\delta$ definition: For every $\epsilon > 0$, there exists $\delta > 0$ such that if $-\delta < a' < 0, 0 < b' < \delta$ then $|F\left(\frac{1}{a'}, \frac{1}{b'}\right) - L| < \epsilon$.

The conditions on $a', b'$ translates to $a < -\frac{1}{\delta}$ and $b > \frac{1}{\delta}$ and so we can directly define\footnote{This is how we define the one-variable $\lim_{x \rightarrow \infty} F(x)$ and the discrete limit of a sequence $\lim_{n \rightarrow \infty} a_n$ as well!} $\lim_{(a, b) \rightarrow (-\infty, \infty)} F(a, b) = L$ as: For any $\epsilon > 0$, there exists $N = N(\epsilon) > 0$ such that for any $a, b$ such that $a < -N$ and $b > N$, we have $|F(a, b) - L| < \epsilon$. Logically, this differs $\lim_{a \rightarrow \infty} F(-a, a) = L$ which only imposes $|F(a, a) - L| < \epsilon$.

For illustration: The series (ignoring $0$ here)
$$\sum_{n = -\infty}^{\infty} \frac{1}{n} = \lim_{m, k \rightarrow \infty} \sum_{n = -m, n \not= 0}^{k} \frac{1}{n}$$
does not converge for if we choose the special sequence of $(m, k)$ such as $k = m^2$ then
$$\sum_{n = -m, n \not= 0}^{m^2} \frac{1}{n} \sim \log(m^2) - \log(m) = \log(m)$$
diverges. However, the special sequence of $m = k$ yield
$$\lim_{m \rightarrow \infty} \sum_{n = -m, n \not= 0}^{m} \frac{1}{n} = 0.$$

\unless\ifdefined\IsMainDocument
\end{document}
\fi

\unless\ifdefined\IsMainDocument
\documentclass[12pt]{article}
\usepackage{amsmath,amsthm,amssymb}
\newcommand{\Abs}[1]{\left| #1 \right|}
\begin{document}
\fi

\textbf{Problem 7.18}: Suppose that the monotonicity condition of Lemma 7.18 is replaced by the difference estimate
$$|F(y) - F(x)| \leq (1 + y - x) x^{\alpha - 1}\varphi(x)$$
valid for $1 \leq x < y \leq 2x$. Here $\varphi$ is a monotone increasing function. Show first that $P(\log x) = O(\varphi(x))$; then show that
$$F(x) = x^\alpha Q(\log x) + O(\sqrt{x^\alpha \varphi(x) E(x)}) + O(x^{\alpha - 1} \varphi(x)).$$

\begin{proof}
Note that to show $P(\log x) = O(\varphi(x))$ is to show $\frac{|P(\log x)|}{\varphi(x)}$ is bounded above and that is the same as showing the reciprocal $\frac{\varphi(x)}{|P(\log x)|}$ is bounded below by a positive number.

We write
\begin{align}
G(y) - G(x) &= \int_x^y F(t) t^{-1} dt \notag\\
&= \int_x^y F(x) t^{-1} dt + \int_x^y (F(t) - F(x)) t^{-1} dt \notag \\
&= h F(x) + \int_x^y (F(t) - F(x)) t^{-1} dt
\label{eq:diffG}
\end{align}
where $h := h(x, y) = \log(y/x)$. The left hand side can be related to $P(\log x)$ so our goal is to find an upper bound for (the absolute value of) the right hand side in term of $\varphi(x)$.

For $1 \leq x < y \leq 2x$, we can bound the second integral
\begin{align}
\Abs{ \int_x^y (F(t) - F(x)) t^{-1} dt }
\leq& \int_x^y |F(t) - F(x)| t^{-1} dt \notag \\
\leq& \int_x^y (1 + t - x) x^{\alpha - 1}\varphi(x) \; t^{-1} dt \notag \\
=& x^{\alpha - 1} \varphi(x) \int_x^y (1 + t - x) t^{-1} dt \notag\\
=& x^{\alpha - 1} \varphi(x) \{ (1 - x) h + y - x \}. \label{ineq:bound_second_integral_Gdiff}
\end{align}

To get an upper bound for $|F(x)|$, let $x \geq 1$ be fixed and denote $k = k(x) := [\frac{\log x}{\log 2}]$ so $2^k \leq x < 2^{k+1}$ or equivalently, $1 \leq \frac{x}{2^k} < 2$ so we can apply the inequality for $x = 1$ and $y = \frac{x}{2^k}$ to get
$$\Abs{F\left(\frac{x}{2^k}\right) - F(1)} \leq \left(1 + \frac{x}{2^k} - 1\right) 1^{\alpha - 1} \varphi(1) < 2 \varphi(1).$$

For $r = 0, 1, ..., k - 1$, we have $x \geq 2^{r+1}$ so we can apply the given inequality to derive
\begin{align*}
\Abs{ F\left(\frac{x}{2^r}\right) - F\left(\frac{x}{2^{r+1}}\right) } \leq& \left(1 + \frac{x}{2^r} - \frac{x}{2^{r+1}}\right) \left(\frac{x}{2^{r+1}}\right)^{\alpha - 1} \varphi\left(\frac{x}{2^{r+1}}\right)\\
\leq& \left(1 + \frac{x}{2^{r+1}}\right) \frac{x^{\alpha - 1}}{2^{(r+1)(\alpha - 1)}} \varphi(x) &\text{ since } \varphi \uparrow\\
=& \varphi(x) (x^{\alpha - 1} 2^{-(\alpha - 1)(r+1)} + x^{\alpha} {2^{-\alpha (r+1)}})
\end{align*}

Combining these inequalities, we obtain the upper bound
\begin{align}
\Abs{ F(x) } \leq& |F(1)| + \Abs{F\left(\frac{x}{2^k}\right) - F(1)} + \sum_{r = 0}^{k-1} \Abs{ F\left(\frac{x}{2^r}\right) - F\left(\frac{x}{2^{r+1}}\right) } \notag\\
%%%
\leq & \underbrace{|F(1)| + 2\varphi(1)}_{C} + \sum_{r = 0}^{k-1} \varphi(x) (x^{\alpha - 1} 2^{-(\alpha - 1)(r+1)} + x^{\alpha} {2^{-\alpha (r+1)}}) \notag\\
%%%
= & C + \varphi(x) \left\{ \sum_{r = 0}^{k-1} x^{\alpha - 1} 2^{-(\alpha - 1)(r+1)} + \sum_{r = 0}^{k-1} x^{\alpha} {2^{-\alpha (r+1)}} \right\} \notag\\
%%%
= & C+ \varphi(x) \left\{ x^{\alpha - 1} 2^{-(\alpha - 1)} \sum_{r = 0}^{k-1} 2^{-(\alpha - 1) r} + x^{\alpha} 2^{-\alpha} \sum_{r = 0}^{k-1} 2^{-\alpha r} \right\} \notag\\
%%%
\leq & C + \varphi(x) \left\{ x^{\alpha - 1} B(x, \alpha)
+ \frac{x^{\alpha}}{2^{\alpha} - 1} \right\}
\label{ineq:upperbound_abs_F}
\end{align}
where
$$B(x, \alpha) := \begin{cases}
\dfrac{1}{2^{\alpha - 1} - 1} &\text{if } \alpha > 1,\\\\
\dfrac{x^{1-\alpha} - 1}{1 - 2^{\alpha-1}} &\text{if } 0 < \alpha < 1,\\\\
\dfrac{\log x}{\log 2} &\text{if } \alpha = 1.
\end{cases}$$

The final inequality \eqref{ineq:upperbound_abs_F} comes from the observation that
$$2^{-\alpha} \sum_{r = 0}^{k-1} 2^{-\alpha r} \leq 2^{-\alpha} \sum_{r = 0}^{\infty} 2^{-\alpha r} = \frac{x^{\alpha}}{2^{\alpha} - 1}$$
and
\begin{align*}
2^{-(\alpha - 1)} \sum_{r = 0}^{k-1} 2^{-(\alpha - 1) r} &= \begin{cases}
\dfrac{2^{-(\alpha - 1)k} - 1}{1 - 2^{\alpha-1}} &\text{if }\alpha \not= 1,\\\\
k &\text{if } \alpha = 1
\end{cases}\\
&\leq B(x, \alpha)
\end{align*}
for in case $0 < \alpha < 1$, $2^{\alpha - 1} < 1$ and $2^{-(\alpha - 1)k} = (2^k)^{1-\alpha} \leq x^{1-\alpha}$ while in case $\alpha > 1$, we just extend the summation to infinity.

Dividing both sides of \eqref{eq:diffG} by $x^\alpha |P(\log x)|$ assuming $x$ is sufficiently large so that $P(\log x) \not= 0$, we get
\begin{align*}
&\Abs{\frac{G(y) - G(x)}{x^\alpha P(\log x)}}\\
%%%
\leq& \Abs{\frac{hF(x)}{x^\alpha P(\log x)}} + \frac{\Abs{\int_x^y (F(t) - F(x)) t^{-1} dt}}{x^\alpha |P(\log x)|}\\
%%%
\leq& \frac{h (C + \varphi(x) \left\{ x^{\alpha - 1} B(x, \alpha) + \frac{x^{\alpha}}{2^{\alpha} - 1}\right\})}{x^\alpha |P(\log x)|} + \frac{x^{\alpha - 1} \varphi(x) \{ (1 - x) h + y - x \}}{x^\alpha |P(\log x)|} &\text{by } \eqref{ineq:bound_second_integral_Gdiff} \text{ and } \eqref{ineq:upperbound_abs_F}\\
%%%
=& \frac{C h}{x^\alpha |P(\log x)|} + \frac{\varphi(x)}{|P(\log x)|} \left\{ \frac{B(x, \alpha) h}{x}  + \frac{h}{2^{\alpha} - 1} + \frac{(1 - x) h + y - x}{x}\right\}
\end{align*}
while on the other hand, using the assumption on $G$
\begin{align*}
\Abs{\frac{G(y) - G(x)}{x^\alpha P(\log x)}} &= \Abs{\frac{y^\alpha P(\log y) + O(E(y)) - x^\alpha P(\log x) - O(E(x))}{x^\alpha P(\log x)}}\\
&= \Abs{\frac{y^\alpha P(\log y)}{x^\alpha P(\log x)} - 1 + \frac{O(E(y)) + O(E(x))}{x^\alpha P(\log x)}}
\end{align*}
so specializing $y = 2x$ (so $h = \log 2$) we get the inequality
\begin{align*}
&\Abs{2^\alpha \frac{P(\log x + \log 2)}{P(\log x)} - 1 + \frac{O(E(2x)) + O(E(x))}{x^\alpha P(\log x)}} \\
\leq &\frac{C \log 2}{x^\alpha |P(\log x)|} + \frac{\varphi(x)}{|P(\log x)|} \left\{ \frac{B(x, \alpha) \log 2}{x}  + \frac{\log 2}{2^{\alpha} - 1} + \frac{(1 - x) \log 2 + x}{x}\right\}
\end{align*}
which yields
\begin{align*}
&\frac{\Abs{2^\alpha \frac{P(\log x + \log 2)}{P(\log x)} - 1 + \frac{O(E(2x)) + O(E(x))}{x^\alpha P(\log x)}} - \frac{C \log 2}{x^\alpha |P(\log x)|} }{\frac{B(x, \alpha) \log 2}{x}  + \frac{\log 2}{2^{\alpha} - 1} + \frac{(1 - x) \log 2 + x}{x}}\leq \frac{\varphi(x)}{|P(\log x)|} 
\end{align*}
by a simple rearrangement.

As $x \rightarrow \infty$, we have $\frac{(1 - x) \log 2 + x}{x} \rightarrow 1 - \log 2, \frac{P(\log x + \log 2)}{P(\log x)} \rightarrow 1, \frac{B(x, \alpha) \log 2}{x} \rightarrow 0$ and $\frac{O(E(2x)) + O(E(x))}{x^\alpha P(\log x)} \rightarrow 0$ by assumption on $E$, namely $E(x) = o(x^\alpha P(\log x))$ and $E(2x) \leq K E(x)$ and so the left hand side has limit
$$L := \frac{2^\alpha - 1}{\frac{\log 2}{2^\alpha - 1} + 1 - \log 2} > 0.$$
Thus, for any $\epsilon > 0$, we have
$$\frac{\varphi(x)}{|P(\log x)|} \geq L - \epsilon$$
for all $x$ sufficiently large. This proves $P(\log x) = O(\varphi(x))$ as we can adjust the constant to account for the $x$ in a finite range.

To get the final formula for $F(x)$, we copy the proof of Lemma 7.18. As in the proof of that lemma, we put $f(u) = e^{\alpha u} P(u)$. There we find
\begin{itemize}
\item The difference
\begin{align*}
G(y) - G(x) &= f(\log y) - f(\log x) + O(E(x))\\
&= h f'(\log x) + \frac{h^2}{2} f''(\log z) + O(E(x))
\end{align*}
for some $z$ such that $x < z < y$ by Taylor's formula.

\item $f'(\log x) = x^\alpha Q(\log x)$ is the expected main term for $F(x)$.

\item $f''(u) = O(e^{\alpha u} P(u))$ for $u$ large enough so $f''(\log z) = O(z^\alpha P(\log z)) = O(y^\alpha P(\log y)) = O(e^h x^\alpha P(h + \log x))$.
\end{itemize}
So this time, dividing both sides of \eqref{eq:diffG} by $h$, we obtain from \eqref{ineq:bound_second_integral_Gdiff} that
$$\Abs{F(x) - \frac{G(y) - G(x)}{h}} \leq x^{\alpha - 1} \varphi(x) \left\{ 1 - x + \frac{y - x}{h} \right\}$$
which leads to
$$\Abs{F(x) - f'(\log x) + \frac{h}{2} f''(\log z) + O(h^{-1} E(x))} \leq x^{\alpha - 1} \varphi(x) \left\{ 1 - x + \frac{y - x}{h} \right\}$$
and we have the error bound
$$\Abs{F(x) - x^\alpha Q(\log x)} \leq \frac{h}{2} f''(\log z) + O(h^{-1} E(x)) + x^{\alpha - 1} \varphi(x) + x^{\alpha - 1} \varphi(x) \left\{\frac{y - x}{h} - x \right\}.$$

Now we choose $y$ such that
$$h = h(x) = \sqrt{\frac{E(x)}{x^\alpha \varphi(x)}}$$
i.e. $y = x e^{h(x)}$. By assumption $E(x) = o(x^\alpha P(\log x))$ and the proven fact that $P(\log x) = O(\varphi(x))$, we have $h \rightarrow 0$ as $x \rightarrow \infty$ so eventually $y \leq 2x$. So for $x$ sufficiently large, with this choice for $y$, we find
\begin{itemize}
\item the first error term
\begin{align*}
\frac{h}{2} f''(\log z) &= \frac{1}{2} \sqrt{\frac{E(x)}{x^\alpha \varphi(x)}} \; O(e^h x^\alpha P(h + \log x)) \\
&= \frac{1}{2} \sqrt{\frac{E(x)}{x^\alpha \varphi(x)}} \; O(x^\alpha P(\log x)) \\
&= O(\sqrt{E(x) x^\alpha \varphi(x)})
\end{align*}
because $h \rightarrow 0$ as $x \rightarrow \infty$ so $e^h \rightarrow 1$ and $\frac{P(h + \log x)}{P(\log x)} \rightarrow 1$ are eventually bounded and we have $O(e^h x^\alpha P(h + \log x)) = O(x^\alpha P(\log x))$. The second equality comes from $P(x) = O(\varphi(x))$.

\item the second and the third error terms are obvious
$$h^{-1} E(x) = \frac{E(x)}{\sqrt{\frac{E(x)}{x^\alpha \varphi(x)}}} = \sqrt{E(x) x^\alpha \varphi(x)}$$

\item finally, we use Taylor's theorem to write $e^h = 1 + h + \frac{h^2}{2} e^\xi$ for some $\xi \in (0, h)$ so $e^\xi \leq e^h$ to bound
$$\frac{y - x}{h} - x = \frac{e^h x - x}{h} - x = x\left( \frac{e^h - 1}{h} - 1 \right) \leq \frac{1}{2} h e^h x$$
so the final error term
$$x^{\alpha - 1} \varphi(x) \left\{\frac{y - x}{h} - x \right\} \leq \frac{e^h}{2} h x^\alpha \varphi(x) = \frac{e^h}{2} \sqrt{\frac{E(x)}{x^\alpha \varphi(x)}} x^\alpha \varphi(x)$$
is also $O(\sqrt{E(x) x^\alpha \varphi(x)})$ since $e^h \rightarrow 1$ as $x \rightarrow \infty$.
\end{itemize}

So the final formula for $F(x)$ is proven.
\end{proof}

\unless\ifdefined\IsMainDocument
\end{document}
\fi

\unless\ifdefined\IsMainDocument
\documentclass[12pt]{article}
\usepackage{amsmath,amsthm,amssymb,cancel}
\newcommand{\R}{\mathbb{R}}
\begin{document}
\fi

\textbf{Problem 7.19}: Show that $\sum_{n \leq x} T(1_2 * 1_2)(n) = O(x \log x)$ but does not have an asymptotic approximation of the form $c x \log x$. Hint. The sum can be expressed in closed form.

\begin{proof}
Note that
$$(1_2 * 1_2)(n) = \begin{cases}
m + 1 &\text{ if } n = 2^m,\\
0 &\text{ otherwise}.
\end{cases}$$
So
$$F(x) := \sum_{n \leq x} T(1_2 * 1_2) = \sum_{2^m \leq x} (m + 1) 2^m = \sum_{m = 0}^{k} (m + 1) 2^m$$
where $k = k(x) = [\frac{\log x}{\log 2}]$. For fix $k$, consider the function
$$f(x) = f_k(x) = \sum_{m = 0}^{k} (m + 1) x^m$$
we have
$$\int_0^x f(t) dt = \sum_{m = 0}^{k} x^{m+1} = x \sum_{m = 0}^{k} x^{m} = x \frac{x^{k+1} - 1}{x - 1} = \frac{x^{k+2} - x}{x - 1}$$
so
$$f(x) = \frac{d}{dx} \left(\frac{x^{k+2} - x}{x - 1}\right) = \frac{\{(k+2) x^{k+1} - 1\}(x - 1) - (x^{k+2} - x)}{(x - 1)^2}.$$

Thus, we have a closed form formula for
$$F(x) = f_k(2) = \frac{\{(k+2) 2^{k+1} - 1\}(2 - 1) - (2^{k+2} - 2)}{(2 - 1)^2} = k 2^{k+1} + 1$$
and it becomes apparent that $\frac{F(x)}{x \log x}$ has no limit as $x \rightarrow \infty$: If such a limit exists, taking the sequence $x_k = 2^k$ yields
$$\lim_{x \rightarrow \infty} \frac{F(x)}{x \log x} = \lim_{k \rightarrow \infty} \frac{k 2^{k+1} + 1}{k 2^k} = \lim_{k \rightarrow \infty} 2 + \frac{1}{k 2^k} = 2$$
whereas taking the sequence $x_k = 2^{k+1} - 1$ yields
\begin{align*}
\lim_{x \rightarrow \infty} \frac{F(x)}{x \log x} &= \lim_{k \rightarrow \infty} \frac{k 2^{k+1} + 1}{(2^{k+1} - 1) \log(2^{k+1} - 1)} \\
&= \lim_{k \rightarrow \infty} \frac{k}{(1 - 2^{-k-1}) \log(2^{k+1} - 1)} \\
&= \lim_{k \rightarrow \infty} \frac{k}{\log(2^{k+1} - 1)} \\
&= \frac{1}{\log 2} \left( \lim_{k \rightarrow \infty} \frac{k \log 2}{\log(2^{k+1} - 1)} \right)\\
&= \frac{1}{\log 2} \left( \lim_{k \rightarrow \infty} 1 - \frac{\log(2^{k+1} - 1) - k \log 2}{\log(2^{k+1} - 1)} \right) \\
&= \frac{1}{\log 2} - \lim_{k \rightarrow \infty} \frac{\log(2 - 2^{-k})}{\log(2^{k+1} - 1)} \\
&= \frac{1}{\log 2}.
\end{align*}
\end{proof}

\textbf{Problem 7.20}: \begin{enumerate}
\item Show by a lattice point counting argument that
$$\sum_{n \leq x} (1_2 * 1_3)(n) = \frac{\log^2 x}{2 \log 2 \log 3} + O(\log x).$$
Does this estimate lead to another proof of (7.28)?
\item It is known that actually the sharper estimate
$$\sum_{n \leq x} (1_2 * 1_3)(n) = \frac{\log^2 x}{2 \log 2 \log 3} + b \log x + o(\log x)$$
holds for a certain constant $b$. Does this result imply (7.28)?
\end{enumerate}

\begin{proof}
\begin{enumerate}
\item We have
$$G(x) := \sum_{n \leq x} (1_2 * 1_3)(n) = \sum_{2^a 3^b \leq x} 1$$
can be interpreted as the number of lattice points in the triangle
$$\{(a, b) \in \R^2 \;|\; a, b \geq 0 \text{ and } a \log 2 + b \log 3 \leq \log x\}$$
which could be approximated via its area
$$\frac{1}{2} \left(\frac{\log x}{\log 2}\right) \left(\frac{\log x}{\log 3}\right)$$
because this is a right triangle with sides of length $\frac{\log x}{\log 2}$ and $\frac{\log x}{\log 3}$.

The error is $O(\log x)$, the length of each side. This can be seen from a more general perspective of counting lattice point in the triangle $T$ with vertices $(0, 0)$, $(a, 0)$ and $(0, b)$ where $a, b > 0$: Let $T'$ be the triangle with vertices $(a, 0), (0, b)$ and $(a, b)$. Together the two triangles made up the rectangle whose number of lattice points is easily counted as $[a] \cdot [b]$. The difference in the number of lattice points between $T$ and $T'$ is at most $\max\{a, b\}$.

Then to estimate $\sum_{n \leq x} T(1_2 * 1_3)(n)$, we recognize it as $\int_{1^-}^x t \, dG(t)$ and integrate by part to apply the known estimate for $G(t)$:
\begin{align*}
\sum_{n \leq x} T(1_2 * 1_3)(n) &= \int_{1^-}^x t \, dG(t) \\
&= x G(x) - \int_1^x G(t) dt \\
&= x G(x) - \int_1^x \left\{ \frac{\log^2 t}{2 \log 2 \log 3} + O(\log t) \right\} dt \\
&= x G(x) - \frac{1}{2 \log 2 \log 3} \int_1^x \log^2 t dt + O(\int_1^x \log t dt )\\
&= x \left\{ \frac{\log^2 x}{2 \log 2 \log 3} + O(\log x) \right\} \\
&\qquad - \frac{1}{2 \log 2 \log 3} ( x \log^2 x - 2 x \log x + 2 x - 2) + O(x \log x)\\
&= \frac{x \log x}{\log 2 \log 3} + O(x \log x)\\
&= O(x \log x)
\end{align*}
so we do not gain anything.

\item In this case, redoing the above integration by parts argument from the third equation onward, we have
\begin{align*}
\sum_{n \leq x} T(1_2 * 1_3)(n) &= x G(x) - \int_1^x \left\{ \frac{\log^2 t}{2 \log 2 \log 3} + b \log t + o(\log t) \right\} dt \\
&= x \left\{ \frac{\log^2 x}{2 \log 2 \log 3} + b \log x + o(\log x) \right\} \\
&\qquad - \frac{1}{2 \log 2 \log 3} \left( x \log^2 x - 2 x\log x + 2 x - 2 \right) \\
&\qquad - b (x \log x - x + 1) + o(x \log x)\\
&= \cancel{\frac{x \log^2 x}{2 \log 2 \log 3}} + \cancel{b x \log x} + o(x\log x) \\
&\qquad - \cancel{\frac{x \log^2 x}{2 \log 2 \log 3}} + \frac{x\log x}{\log 2 \log 3} - \frac{x}{\log 2 \log 3} + \frac{1}{\log 2 \log 3} \\
&\qquad - \cancel{b x \log x} + bx - b + o(x \log x)\\
&= \frac{x\log x}{\log 2 \log 3} - \left(b - \frac{1}{\log 2 \log 3}\right) (x - 1) + o(x \log x)
\end{align*}
which clearly implies (7.28).
\end{enumerate}
\end{proof}

\unless\ifdefined\IsMainDocument
\end{document}
\fi


\end{document}
